\newpage
\section{Vorläufige Gliederung} \label{sec:Vorläufige Gliederung}

\begin{enumerate}[label=\textbf{\arabic*.}]
    \item \textbf{Einleitung (ca. 6-8 Seiten)}
    \begin{enumerate}[label=\textbf{1.\arabic*}]
        \item \textbf{Problemstellung und Motivation:} \\
        - Quantenbedrohung für aktuelle \ac{SSI}-Systeme \\
        - Herausforderungen in \ac{KRITIS} \\
        - Notwendigkeit kryptoagiler Lösungen
        \item \textbf{Zielsetzung und Forschungsfragen:} \\
        - Konkretisierung der vier Hauptforschungsfragen \\
        - Erwartete Beiträge zur Wissenschaft und Praxis
        \item \textbf{Methodisches Vorgehen:} \\
        - Design Science Research Ansatz \\
        - Abgrenzung zu realen \ac{KRITIS}-Umgebungen
        \item \textbf{Aufbau der Arbeit:} \\
        - Aufbau und Struktur der Masterarbeit
    \end{enumerate}
    \item \textbf{Theoretische Grundlagen (ca. 15-18 Seiten)}
    \begin{enumerate}[label=\textbf{2.\arabic*}]
        \item \textbf{Blockchain-Technologie für Identitätsmanagement:} \\
        - Permissioned vs. Permissionless Ledger \\
        - Smart Contracts für \ac{SSI} \\
        - Konsensverfahren und Skalierbarkeit
        \item \textbf{Self-Sovereign Identity:} \\
        - Grundprinzipien und Architekturkomponenten \\
        - \ac{DID} und \ac{VC} \\
        - Blockchain-basierte \ac{SSI}-Implementierungen
        \item \textbf{Post-Quantum Kryptografie:} \\
        - NIST-Standards (ML-KEM, ML-DSA, SLH-DSA) \\
        - Lattice-based und Hash-based Verfahren \\
        - Kryptoagilität und Migrationsstategien
        \item \textbf{Kritische Infrastrukturen und Compliance:} \\
        - \ac{KRITIS}-Anforderungen nach \ac{BSI} \\
        - Privacy by Design und DSGVO-Konformität \\
        - Regulatorische Rahmenbedingungen
    \end{enumerate}
    \item \textbf{Systemarchitektur und Design (ca. 12-15 Seiten)}
    \begin{enumerate}[label=\textbf{3.\arabic*}]
        \item \textbf{Anforderungsanalyse:} \\
        - Funktionale Anforderungen aus den Forschungsfragen \\
        - Nicht-funktionale Anforderungen (Performance, Sicherheit) \\
        - \ac{KRITIS}-spezifische Compliance-Anforderungen
        \item \textbf{Evaluation und Auswahl des \ac{SSI}-Frameworks:} \\
        - Kriterien für die Framework-Auswahl (Open Source, PQC-Kompatibilität) \\
        - Vergleich und Entscheidung (z.B. Hyperledger Indy/Aries) \\
        - Anpassungen und Erweiterungen für die ermittelten Anforderungen
        \item \textbf{Systemarchitektur:} \\
        - Gesamtarchitektur des \ac{SSI}-Prototyps auf Basis des Frameworks \\
        - Komponentendiagramm, Schnittstellen und Erweiterungspunkte\\
        - Integration von \ac{PQC}-Algorithmen
        \item \textbf{Kryptoagiles Design:} \\
        - Algorithmus-Abstraktion und Flexibilität \\
        - Update-Mechanismen für zukünftige \ac{PQC}-Standards
        \item \textbf{Blockchain-Integration:} \\
        - Auswahl der Blockchain-Plattform \\
        - Smart Contract Design für \ac{DID} und \ac{VC} \\
        - Konsensverfahren und Governance
    \end{enumerate}
    \item \textbf{Implementierung des Prototyps (ca. 12-15 Seiten)}
    \begin{enumerate}[label=\textbf{4.\arabic*}]
        \item \textbf{Technologie-Stack und Entwicklungsumgebung:} \\
        - Laborumgebung Setup (Hyper-V, Ubuntu 24.04 LTS, Docker, Github) \\
        - Entwicklungstools und Frameworks
        \item \textbf{SSI-Kern-Komponenten:} \\
        - Kryptografische Schlüsselverwaltung \\
        - Dezentrale Identifikatoren (\ac{DID}s) \\
        - Verifiable Credentials (\ac{VC}s) \\
        - Verifiable Presentations (\ac{VP}s) \\
        - Wallet (Agenten-Software) \\
        - Credential Issuer, Holder, Verifier (Rollenmodell)
        \pagebreak
        \item \textbf{Blockchain-Backend:} \\
        - Distributed Ledger (Blockchain/\ac{DLT}-Layer) \\
        - Konsensmechanismus \\
        - \ac{DID}-Registry / \ac{DID}-Storage \\
        - \ac{VC}-Schema- und Definitions-Registry \\
        - Revocation Registry \\
        - Zugriffs- und Berechtigungsmanagement (Governance Layer)
        \item \textbf{PQC-Algorithmen-Integration:} \\
        - Schlüsselerzeugung und -verwaltung (\ac{SSI} \& Blockchain) \\
        - Digitale Signaturen (\ac{SSI} \& Blockchain) \\
        - Verschlüsselung von Daten (SSI \& Blockchain) \\
        - Blockchain-Transaktionen \& Konsensmechanismus (Blockchain) \\
        - Smart Contracts (Blockchain) \\
        - Verifiable Credentials \& Präsentationen (\ac{SSI}) \\
        - \ac{DID}-Methoden und \ac{DID}-Dokumente (\ac{SSI})\\
        - Agenten-Software (Wallets, Clients) (\ac{SSI})\\
        - Credential Issuer, Holder, Verifier (\ac{SSI})\\
        - Umsetzung der Kryptoagilität (\ac{SSI} \& Blockchain)
    \end{enumerate}
    \item \textbf{Evaluation und Validierung (ca. 8-10 Seiten)}
    \begin{enumerate}[label=\textbf{5.\arabic*}]
        \item \textbf{Evaluationsmethodik:} \\
        - \ac{FEDS}-Framework Anwendung \\
        - Metriken und Bewertungskriterien
        \item \textbf{Funktionalitätstests:} \\
        - Identity Lifecycle Management Szenarien \\
        - Use Case Validierung
        \item \textbf{Performance-Analyse:} \\
        - Durchsatz und Latenz-Messungen \\
        - Speicher- und Rechenaufwand \\
        - Skalierbarkeitsanalyse
        \item \textbf{Sicherheitsbewertung:} \\
        - Kryptografische Stärke \\
        - Angriffsvektoren und Resilience \\
        - Compliance-Validierung
    \end{enumerate}
    \pagebreak
    \item \textbf{Ergebnisse und Diskussion (ca. 6-8 Seiten)}
    \begin{enumerate}[label=\textbf{6.\arabic*}]
        \item \textbf{Beantwortung der Forschungsfragen:} \\
        - FF1: Systemarchitektur \& Compliance \\
        - FF2: Algorithmenauswahl \& Sicherheitsbewertung \\
        - FF3: Performance \& Skalierbarkeit \\
        - FF4: Kryptografische Agilität
        \item \textbf{Kritische Reflexion:} \\
        - Limitationen der Laborumgebung \\
        - Übertragbarkeit auf reale \ac{KRITIS}-Umgebungen \\
        - Trade-offs zwischen Sicherheit und Performance
        \item \textbf{Wissenschaftliche und praktische Beiträge:} \\
        - Gestaltungsprinzipien für quantenresistente \ac{SSI} \\
        - Implementierungsrichtlinien
    \end{enumerate}
    \item \textbf{Fazit und Ausblick (ca. 4-5 Seiten)}
    \begin{enumerate}[label=\textbf{7.\arabic*}]
        \item \textbf{Zusammenfassung der Ergebnisse:} \\
        - Zentrale Erkenntnisse und Zielerreichung
        \item \textbf{Implikationen für Forschung und Praxis:} \\
        - Bedeutung und Anwendbarkeit der Resultate
        \item \textbf{Zukünftige Forschungsrichtungen:} \\
        - Offene Fragen und Forschungsperspektiven skizziert
    \end{enumerate}
\end{enumerate}