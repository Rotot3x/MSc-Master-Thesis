\newpage
\section{Methodik} \label{sec:Methodik}

\subsection{Systematische Literaturrecherche} \label{sec:Systematische Literaturrecherche}

Die systematische Literaturrecherche dieser Masterarbeit folgt ausgewählten Methoden der \ac{PRISMA} 2020 Richtlinien zur strukturierten Identifikation, Selektion, Bewertung und Synthese einschlägiger Studien \parencite[S. 1]{page_PRISMA2020Statementupdatedguidelinereportingsystematicreviews_2021}. Dieses Vorgehen bietet einen strukturierten Ansatz zur Durchführung und Dokumentation von Literaturrecherchen, welcher die Qualität und Vollständigkeit der Berichterstattung verbessert und einen transparenten und reproduzierbaren Prozess gewährleistet \parencite[S. 1]{page_PRISMA2020Statementupdatedguidelinereportingsystematicreviews_2021}, wodurch eine belastbare Grundlage für die Analyse der Forschungslücke (Kapitel~\ref{sec:Stand der Forschung und Identifikation der Forschungslücke}) und die Ableitung der Forschungsfragen (Kapitel~\ref{sec:Zielsetzung und Forschungsfragen}) dieser Masterarbeit geschaffen wird.

Die systematische Literaturrecherche wurde in zwei zeitlich getrennten Iterationen durchgeführt, um den dynamischen Charakter des Forschungsfeldes zu adressieren und die Aktualität der Wissensbasis sicherzustellen. Beide Iterationen sind in \ref{sec:Anhang_Dokumentation_der_systematischen_Literaturrecherche} ausführlich dokumentiert und dienen ausschließlich der Problemidentifikation, der Ergründung des Forschungsstands sowie der Ableitung der Forschungslücke und initialen Forschungsfragen.

Der konsolidierte Suchprozess in der Datenbank \gls{EBSCO} führt über beide Phasen hinweg zur Identifikation von insgesamt 95 potenziellen Quellen. Davon entfallen 61 Publikationen auf die erste Iteration im Mai 2025 (siehe \autoref{fig:PRISMA_Flussdiagramm_iteration1}) und 34 Publikationen auf die zweite Iteration im November 2025 (siehe \autoref{fig:PRISMA_Flussdiagramm_Iteration2}). Die methodische Konsistenz wird dabei durch die Anwendung identischer Suchstrategien und Selektionskriterien in beiden Zeiträumen sichergestellt.

Nach der Bereinigung von Duplikaten und der Anwendung der Ein- und Ausschlusskriterien verbleibt ein fokussierter Bestand an hochrelevanten Studien. Eine detaillierte Übersicht der identifizierten Quellen findet sich in \autoref{tab:quellenuebersicht_Iteration1} für die erste und \autoref{tab:quellenuebersicht_iteration2} für die zweite Phase. Die vollständige Aufschlüsselung der Selektionsschritte sowie die zugehörigen Flussdiagramme sind zudem gesammelt in \ref{sec:Anhang_Dokumentation_der_systematischen_Literaturrecherche} aufgeführt.

\subsection{Design Science Research} \label{sec:Design Science Research}

\ac{DSR} bildet den methodischen Rahmen dieser Masterarbeit und stellt neben der verhaltenswissenschaftlichen Forschung ein eigenständiges Forschungsparadigma dar \parencite[S. 75]{hevner_DesignScienceInformationsystemsresearch_2004}. Während verhaltenswissenschaftliche Ansätze Theorien zur Erklärung oder Vorhersage entwickeln, fokussiert \ac{DSR} auf die Erschaffung innovativer Artefakte zur Erweiterung menschlicher und organisatorischer Fähigkeiten \parencite[S. 75]{hevner_DesignScienceInformationsystemsresearch_2004}. Fundamental ist \ac{DSR} ein lösungsorientiertes Paradigma \parencite[S. 76]{hevner_DesignScienceInformationsystemsresearch_2004}, dessen Prinzip darin besteht, Wissen durch den Bau und die Anwendung eines Artefakts zu gewinnen \parencite[S. 82]{hevner_DesignScienceInformationsystemsresearch_2004}.

Das Information Systems Research Framework (\autoref{fig:hevner2004_framework}) nach \textcite[S. 80]{hevner_DesignScienceInformationsystemsresearch_2004} strukturiert den Forschungsprozess durch drei Hauptkomponenten.

\begin{figure}[H]
    \centering
    \includegraphics[width=\linewidth]{dsr_ISR_Framework.png}
    \caption{Information Systems Research Framework}
    \begin{flushleft}
    \textit{Anmerkung.} Aus \textcite[S. 80]{hevner_DesignScienceInformationsystemsresearch_2004}.
    \end{flushleft}
    \label{fig:hevner2004_framework}
\end{figure}

Die Environment-Komponente definiert den Problemraum \parencite[S. 108]{simon_SciencesArtificial_1996} mit Menschen, Organisationen und Technologien sowie den Geschäftsanforderungen \parencite[S. 7--11]{silver_InformationTechnologyInteractionModelFoundationMBACoreCourse_1995}. Für diese Arbeit bilden \ac{KRITIS} das Environment mit ihren Anforderungen an Post-Quantum-Sicherheit und selbstbestimmte Identitätsverwaltung. Die IS Research-Komponente umfasst Build- und Evaluate-Aktivitäten für Artefakte \parencite[S. 80]{hevner_DesignScienceInformationsystemsresearch_2004}. Hier erfolgt die Instanziierung und Evaluation eines \ac{PQC}-fähigen \ac{SSI}-Prototypen, der speziell für den Einsatz in \ac{KRITIS} konzipiert ist. Die Knowledge Base liefert Foundations (Theorien, Frameworks, Konstrukte) und Methodologies (Evaluationsmethoden) \parencite[S. 80]{hevner_DesignScienceInformationsystemsresearch_2004}. Sie umfasst für diese Arbeit \ac{SSI}-Standards, \ac{NIST}-\ac{PQC}-Algorithmen und \ac{KRITIS}-Anforderungen. Rigor wird durch angemessene Anwendung dieser Wissensbasis erreicht \parencite[S. 80]{hevner_DesignScienceInformationsystemsresearch_2004}.

\subsubsection{Zyklen}

\textcite[S. 88]{hevner_ThreeCycleViewDesignScienceResearch_2007} identifiziert drei eng verbundene Aktivitätszyklen, die in jedem \ac{DSR}-Projekt präsent sein müssen (\autoref{fig:3-cycle-model}).

\begin{figure}[H]
    \centering
    \includegraphics[width=\linewidth]{3-cycle.png}
    \caption{Design Science Research Zyklen}
    \begin{flushleft}
    \textit{Anmerkung.} Aus \textcite[S. 88]{hevner_ThreeCycleViewDesignScienceResearch_2007}.
    \end{flushleft}
    \label{fig:3-cycle-model}
\end{figure}

Der Relevance Cycle initiiert \ac{DSR} mit Anforderungen aus der Anwendungsdomäne und fordert Field Testing des Outputs \parencite[S. 88--89]{hevner_ThreeCycleViewDesignScienceResearch_2007}. In dieser Arbeit manifestiert sich dies durch die iterative Identifikation von \ac{KRITIS}-Anforderungen an Sicherheit, Compliance und Kryptoagilität in zwei Implementierungszyklen. Dazu gehören quantenresistente Kommunikations- und Signaturverfahren, die Einhaltung relevanter Regulierungsvorgaben sowie die Wahrung von Datenschutz und technischer Resilienz. Die entwickelten Artefakte werden kontinuierlich in Laborumgebungen getestet, wobei Erkenntnisse aus der ersten Iteration die Designziele der zweiten Iteration prägen. Der Rigor Cycle verbindet \ac{DSR}-Aktivitäten mit der Wissensbasis, um Innovation zu gewährleisten \parencite[S. 88--90]{hevner_ThreeCycleViewDesignScienceResearch_2007}. In dieser Arbeit operationalisiert dieser Zyklus die kontinuierliche Integration etablierter \ac{SSI}-Frameworks, \ac{NIST}-standardisierter \ac{PQC}-Algorithmen und \ac{KRITIS}-Standards durch wissenschaftlich fundierte Software-Engineering-Patterns, die \enquote{Defense in Depth} \parencite[S. 242--243]{alsaqour_DefenseDepthMultilayersecurity_2021} etablieren. Der Design Cycle strukturiert die Artefaktentwicklung als iterativen Prozess zwischen Konstruktion und Evaluation \parencite[S. 90--91]{hevner_ThreeCycleViewDesignScienceResearch_2007}. In dieser Arbeit erfolgt dies in zwei aufeinanderfolgenden Iterationen, wobei Evaluationsergebnisse als Designziele der Folge-Iteration operationalisiert werden.

\subsubsection{Artefakte}

\textcite[S. 255]{march_DesignNaturalScienceresearchinformationtechnology_1995} identifizieren vier Artefakttypen: Constructs, Models, Methods und Instantiations. Constructs stellen die Sprache bereit, in der Probleme und Lösungen definiert werden, und beeinflussen die Problemkonzeption \parencite[S. 78, 83]{hevner_DesignScienceInformationsystemsresearch_2004}. Models repräsentieren das Designproblem und den Lösungsraum, unterstützen das Problemverständnis und ermöglichen die Erkundung von Designentscheidungen \parencite[S. 78--79]{hevner_DesignScienceInformationsystemsresearch_2004}. Methods definieren Prozesse zur Problemlösung, von formalen Algorithmen bis zu Best-Practice-Beschreibungen \parencite[S. 79]{hevner_DesignScienceInformationsystemsresearch_2004}. Instantiations demonstrieren Machbarkeit durch Implementierung in einem funktionierenden System und liefern den Beweis durch Konstruktion \parencite[S. 79, 84]{hevner_DesignScienceInformationsystemsresearch_2004}.

Diese Arbeit entwickelt eine Instantiation in Form eines funktionsfähigen Prototypen eines blockchain-basierten \ac{SSI}-Systems mit integrierter \ac{PQC} für \ac{KRITIS}, welcher die technische Machbarkeit und praktische Anwendbarkeit des Ansatzes demonstriert.

\subsubsection{Richtlinien}

\textcite[S. 82]{hevner_DesignScienceInformationsystemsresearch_2004} formulieren sieben Richtlinien für \ac{DSR}, die jeweils in einer qualitätsvollen Forschungsarbeit adressiert werden sollten. Diese Richtlinien bilden einen strukturierten Rahmen zur Durchführung und Bewertung von Design-Science-Forschung und gewährleisten, dass sowohl wissenschaftliche Rigor als auch praktische Relevanz erreicht werden.

\pagebreak

\textbf{Guideline 1: Design as an Artifact} - Design-Science-Forschung muss ein brauchbares Artefakt produzieren \parencite[S. 82]{hevner_DesignScienceInformationsystemsresearch_2004}. Diese Arbeit erfüllt dies durch eine funktionsfähige Instantiation eines blockchain-basierten \ac{SSI}-Systems mit integrierter \ac{PQC}.

\textbf{Guideline 2: Problem Relevance} - Design-Science-Forschung zielt auf technologiebasierte Lösungen für relevante Geschäftsprobleme ab \parencite[S. 84--85]{hevner_DesignScienceInformationsystemsresearch_2004}. Die Relevanz ergibt sich aus der Quantenbedrohung für \ac{KRITIS}-Kryptografie und dem Bedarf an datenschutzfreundlichen Identitätslösungen.

\textbf{Guideline 3: Design Evaluation} - Die Nützlichkeit und Wirksamkeit eines Design-Artefakts müssen rigoros demonstriert werden \parencite[S. 85]{hevner_DesignScienceInformationsystemsresearch_2004}. Diese Arbeit validiert Funktionalität, Compliance und Kryptoagilität.

\textbf{Guideline 4: Research Contributions} - \ac{DSR} muss klare Beiträge zu Design-Artefakt, Foundations oder Methodologies liefern \parencite[S. 87]{hevner_DesignScienceInformationsystemsresearch_2004}. Der Beitrag liegt in der neuartigen Integration von \ac{PQC} in \ac{SSI}-Systemen für \ac{KRITIS}-Kontexte.

\textbf{Guideline 5: Research Rigor} - \ac{DSR} beruht auf rigoroser Anwendung von Methoden in Konstruktion und Evaluation \parencite[S. 87]{hevner_DesignScienceInformationsystemsresearch_2004}. Diese Arbeit nutzt für die Konstruktion etablierte \ac{SSI}-Frameworks, \ac{NIST}-standardisierte \ac{PQC}-Algorithmen, wissenschaftlich fundierte Software-Engineering-Patterns und das \ac{FEDS}-Framework für die Evaluation.

\textbf{Guideline 6: Design as a Search Process} - Die Suche nach einem effektiven Artefakt erfordert iterative Exploration unter Berücksichtigung der Problemumgebung \parencite[S. 87--88]{hevner_DesignScienceInformationsystemsresearch_2004}. Der Design Cycle dieser Arbeit manifestiert sich als Prozess mit zwei Iterationen, bei dem Erkenntnisse aus der ersten Iteration die Designziele der nachfolgenden Iteration prägen.

\textbf{Guideline 7: Communication of Research} - Design-Science-Forschung muss effektiv sowohl für technologie-orientierte als auch für management-orientierte Audiences präsentiert werden \parencite[S. 90]{hevner_DesignScienceInformationsystemsresearch_2004}. Diese Arbeit adressiert dies durch eine umfassende Dokumentation des entwickelten Prototypen, systematische Darstellung der Evaluationsergebnisse sowie die explizite Ableitung praktischer Implikationen für \ac{KRITIS}.

\pagebreak

\subsection{FEDS-Framework} \label{sec:FEDS-Framework}

Nach \textcite[S. 77]{venable_FEDSFrameworkEvaluationDesignScienceResearch_2016} ist die Evaluation von Design-Artefakten eine Schlüsselaktivität des \ac{DSR}, ohne die die Forschung auf der Ebene theoretischer Annahmen über die Utility eines Artefakts verbleibt, ohne Evidenz für dessen tatsächliche Funktionsfähigkeit zu liefern. Um diese Lücke zu schließen und Rigor sicherzustellen, folgt diese Arbeit dem \ac{FEDS}-Framework nach \textcite[S. 77--84]{venable_FEDSFrameworkEvaluationDesignScienceResearch_2016}, welches den Evaluationsprozess in vier iterative Schritte unterteilt, um die Strategie passgenau auf die spezifischen Projektrisiken abzustimmen.

Das \ac{FEDS}-Framework unterscheidet dabei fundamental zwischen zwei Dimensionen der Evaluation: der funktionalen Absicht und dem Paradigma \parencite[S. S. 77--84]{venable_FEDSFrameworkEvaluationDesignScienceResearch_2016}. Hinsichtlich der Absicht wird zwischen formativer Evaluation, die der kontinuierlichen Verbesserung während der Entwicklung dient, und summativer Evaluation, die eine abschließende Bewertung der Zielerreichung vornimmt, differenziert \parencite[S. S. 77--84]{venable_FEDSFrameworkEvaluationDesignScienceResearch_2016}. Bezüglich des Paradigmas unterscheidet das Framework zwischen Artificial Evaluation, die in kontrollierten Laborumgebungen stattfindet, und Naturalistic Evaluation, welche die Artefakte in realen organisatorischen Umfeldern unter echten Bedingungen prüft \parencite[S. S. 77--84]{venable_FEDSFrameworkEvaluationDesignScienceResearch_2016}. Im Folgenden wird der vierstufige FEDS-Prozess für die vorliegende Arbeit erläutert.

\subsubsection{Explikation der Evaluationsziele}
Der erste Schritt des Frameworks verlangt die Explikation der Evaluationsziele, um Konflikte zwischen konkurrierenden Anforderungen wie Rigor, Risikominimierung und Effizienz aufzulösen \parencite[S. 82--83]{venable_FEDSFrameworkEvaluationDesignScienceResearch_2016}. Für die Entwicklung eines blockchain-basierten \ac{SSI}-Protypen mit \ac{PQC} im \ac{KRITIS}-Kontext ergeben sich hieraus spezifische Prioritäten.

\textbf{Rigour (Efficacy vs. Effectiveness):} Das primäre Ziel dieser Arbeit ist der Nachweis der \enquote{Technical Feasibility} durch rigorose Demonstration der \enquote{Efficacy}. Dies bedeutet den rigorosen Beleg, dass das instanziierte Artefakt (die \ac{PQC}-Integration in \ac{SSI}) den beobachteten Effekt (sichere, quantenresistente Identitätsverifikation) kausal verursacht und nicht externe Störfaktoren \parencite[S. 82]{venable_FEDSFrameworkEvaluationDesignScienceResearch_2016}. Da es sich bei \ac{KRITIS}-Komponenten um sicherheitskritische Infrastruktur handelt, muss vor einem Feldtest (\enquote{Effectiveness} in realer Umgebung) zwingend die funktionale Korrektheit in einer kontrollierten Umgebung bewiesen werden, um \enquote{False Positives}, eine fälschliche Annahme der Sicherheit, auszuschließen \parencite[S. 79]{venable_FEDSFrameworkEvaluationDesignScienceResearch_2016}.

\pagebreak

\textbf{Risikoreduktion (Technical Risk):} Das Hauptrisiko dieser Arbeit ist technischer Natur. Es besteht die Unsicherheit, ob die \ac{PQC}-Algorithmen (ML-DSA, ML-KEM) technisch in bestehende \ac{SSI}-Frameworks (Hyperledger Aries) integriert werden können, ohne deren Kernfunktionen zu brechen. Soziale Risiken (z.B. Benutzerakzeptanz der Wallet-App) werden als nachrangig eingestuft und nicht evaluiert.

\textbf{Ausschluss von Effizienz-Zielen:} In Anlehnung an die Design-Science-Leitlinien wird explizit darauf hingewiesen, dass eine quantitative Performance-Evaluation (\enquote{Efficiency}) nicht Ziel dieser Arbeit ist. Die verwendeten \ac{PQC}-Referenzimplementierungen befinden sich in einem frühen Stadium, weshalb Laufzeitmessungen keine valide Aussagekraft für zukünftige produktive Systeme hätten \parencite[S. 3--5]{demir_PerformanceAnalysisIndustryDeploymentPostQuantumCryptographyAlgorithms_2025}.

\subsubsection{Wahl der Evaluationsstrategie}
Basierend auf den identifizierten Zielen und Risiken wird für diese Arbeit die Technical Risk \& Efficacy Strategy \parencite[S. 81]{venable_FEDSFrameworkEvaluationDesignScienceResearch_2016} gewählt, welche mit formativen, artifiziellen Tests beginnt und in einer summativen, artifiziellen Evaluation endet (\autoref{fig:feds_strategy}).

\begin{figure}[H]
    \centering
    \includegraphics[width=\linewidth]{FEDS_eval_strat}
    \caption{Gewählte Evaluationsstrategie im FEDS-Framework}
    \begin{flushleft}
    \textit{Anmerkung.} Eigene Darstellung in Anlehnung an \textcite[S. 80]{venable_FEDSFrameworkEvaluationDesignScienceResearch_2016}.
    \end{flushleft}
    \label{fig:feds_strategy}
\end{figure}

\pagebreak

Diese Strategie ist nach \textcite[S. 82]{venable_FEDSFrameworkEvaluationDesignScienceResearch_2016} dann indiziert, wenn das primäre Entwicklungsrisiko technischer Natur ist und Evaluationen in realen Umgebungen aus Sicherheits- oder Kostengründen nicht durchführbar sind.
Ein naturalisitischer Ansatz (\enquote{Human Risk \& Effectiveness}) wird verworfen, da der Zugriff auf reale \ac{KRITIS}-Netzwerke für experimentelle kryptografische Prototypen ethisch und regulatorisch nicht vertretbar ist und die Technologie noch nicht den Reifegrad für Endanwendertests besitzt.

\subsubsection{Bestimmung der zu evaluierenden Eigenschaften}
Der dritte Schritt definiert die konkreten Eigenschaften, die evaluiert werden sollen \parencite[S. 83--84]{venable_FEDSFrameworkEvaluationDesignScienceResearch_2016}. Für das entwickelte Artefakt leiten sich diese direkt aus den in Kapitel~\ref{sec:Anforderungsanalyse} definierten Anforderungen ab:

\begin{enumerate}
    \item \textbf{Fidelity (Funktionale Korrektheit):} Das System muss in der Lage sein, den vollständigen \ac{SSI}-Lebenszyklus (Issuance, Verification, Revocation) unter Verwendung von \ac{PQC}-Signaturen fehlerfrei durchzuführen. Es wird geprüft, ob die Artefakte (Agenten, Ledger, Wallet) spezifikationsgemäß interagieren.
    \item \textbf{\ac{KRITIS}-Compliance (Sicherheit):} Es wird evaluiert, ob die implementierten Mechanismen den regulatorischen Vorgaben entsprechen. Dies umfasst die strikte Durchsetzung von \ac{TLS} 1.3, die Verwendung hybrider Verfahren und die Netzwerksegmentierung.
    \item \textbf{Interoperabilität:} Die Fähigkeit des modifizierten Systems, Standard-\gls{DIDComm}-Nachrichten trotz der veränderten kryptografischen Payload zu verarbeiten, ist ein kritisches Kriterium für die Efficacy.
\end{enumerate}

\subsubsection{Design der individuellen Evaluationsepisoden}
\label{sec:Schritt4-Design der individuellen Evaluationsepisoden}

Der vierte Schritt umfasst das Design konkreter Evaluationsepisoden \parencite[S. 84]{venable_FEDSFrameworkEvaluationDesignScienceResearch_2016}. Für diese Arbeit wurden drei diskrete Episoden definiert, die parallel zu den Entwicklungszyklen verlaufen und in \autoref{tab:eval_episodes} dem jeweiligen Arbeitsfortschritt zugeordnet sind.

\pagebreak

\begin{longtable}{L{1.5cm}L{5cm}L{8.5cm}}
    \caption{Evaluationsepisoden nach dem FEDS-Framework}
    \label{tab:eval_episodes} \\
    \toprule
    \textbf{Episode} & \textbf{Phase \& Art} & \textbf{Fokus / Methode} \\
    \midrule
    \endfirsthead
    \multicolumn{3}{l}{\textit{Tabelle \thetable\ (Fortsetzung)}} \\
    \toprule
    \textbf{Episode} & \textbf{Phase \& Art} & \textbf{Fokus / Methode} \\
    \midrule
    \endhead
    \midrule
    \multicolumn{3}{r}{\textit{Fortsetzung auf nächster Seite}} \\
    \endfoot
    \bottomrule
    \multicolumn{3}{p{\linewidth}}{\textit{Anmerkung.} Eigene Darstellung der Evaluationsstrategie in Anlehnung an das \ac{FEDS}-Framework von \textcite{venable_FEDSFrameworkEvaluationDesignScienceResearch_2016}.} \\
    \endlastfoot
    1 & 
    Erste Iteration (Kapitel~\ref{sec:formative_evaluation_iteration1}) \newline 
    \textit{Formativ, Artifiziell} &
    Log-Analyse, Zertifikatsvalidierung, \ac{TLS}-Handshake-Analyse und Infrastruktur-Initialisierung in Form von White-Box Tests nach \textcite[S. 10]{myers_ArtSoftwareTesting_2012}. \\
    \midrule
    2 & 
    Zweite Iteration (Kapitel~\ref{sec:formative_evaluation_iteration2}) \newline
    \textit{Formativ, Artifiziell} &
    Integrationstests der Plugin-Architektur, \gls{DIDComm}-Verarbeitung und Validierung der \ac{PQC}-Signaturen in Form von White-Box Tests nach \textcite[S. 10]{myers_ArtSoftwareTesting_2012}. \\
    \midrule
    3 & 
    Finales Artefakt (Kapitel~\ref{sec:Summative Evaluation}) \newline
    \textit{Summativ, Artifiziell} &
    Requirement Tracing und Validierung der \ac{KRITIS}-Compliance am integrierten Gesamtsystem. \\
\end{longtable}

\textbf{Episode 1 (Formativ):} Diese Episode erfolgt parallel zur ersten Iteration und fokussiert sich auf die Validierung der \ac{TLS}. Da in dieser Phase fundamentale Infrastrukturkomponenten wie \glslink{Sidecar-Proxy}{Sidecar-Proxies} entwickelt werden, dient die formative Evaluation primär dazu, Designfehler so früh wie möglich zu identifizieren \parencite[S. 82--83]{venable_FEDSFrameworkEvaluationDesignScienceResearch_2016}. Methodisch erfolgt dies durch die Analyse von Handshake-Protokollen und Cipher-Suites.

\textbf{Episode 2 (Formativ):} In der zweiten Iteration verlagert sich der Fokus auf den Application-Layer. Hier wird formativ geprüft, ob die entwickelten Python-Plugins korrekt in den \ac{ACA-Py}-Core geladen werden und ob die erweiterte Kryptografiebibliothek liboqs korrekt angesprochen wird.

\textbf{Episode 3 (Summativ):} Die abschließende Evaluation in Kapitel~\ref{sec:Summative Evaluation} führt alle Komponenten zusammen. Sie dient dem rigorosen Nachweis, dass das Gesamtsystem die eingangs definierten Forschungsfragen beantwortet. Hierbei wird geprüft, ob die Efficacy des \ac{PQC}-\ac{SSI}-Prototypen für \ac{KRITIS}-Anwendungsfälle gegeben ist, ohne reale Risiken einzugehen.

\pagebreak

\subsection{DSRM Prozessmodell} \label{sec:DSRM Prozessmodell}

Zur Sicherstellung einer rigorosen methodischen Fundierung orientiert sich der Forschungsablauf dieser Arbeit am \ac{DSRM} Prozessmodell nach \textcite[S. 46]{peffers_DesignScienceResearchmethodologyinformationsystemsresearch_2007}, welches einen Rahmen für die Durchführung und Präsentation von \ac{DSR} bereitstellt um die wissenschaftliche Validität von Design-Artefakten zu gewährleisten.

Da die vorliegende Arbeit durch die spezifischen Bedrohungen des Quantencomputings für bestehende Infrastrukturen und die daraus resultierenden regulatorischen Anforderungen für \ac{KRITIS} motiviert ist (Kapitel~\ref{sec:Problemstellung und Motivation}), folgt das Forschungsvorhaben einer \enquote{Problem-Centered Initiation}. Dies entspricht dem klassischen Einstieg in den \ac{DSRM}-Prozess über die erste Phase (\autoref{fig:peffers_dsrm}) \parencite[S. 56]{peffers_DesignScienceResearchmethodologyinformationsystemsresearch_2007}.

Neben der \enquote{Objective-Centered Solution}, die durch einen Bedarf in der Industrie oder Forschung ausgelöst wird (Phase 2), definieren die \textcite[S. 56]{peffers_DesignScienceResearchmethodologyinformationsystemsresearch_2007} eine \enquote{Design \& Development Centered Initiation} auf Basis existierender Artefakte für explizite Problembereiche (Phase 3), sowie eine \enquote{Client-/Context-Initiated Solution}, welche auf der Beobachtung einer praktischen Lösung basiert (Phase 4), als drei weitere Einstiegspunkte.

\begin{figure}[H]
    \centering
    \includegraphics[width=\linewidth]{DSRM Process Model_red.png}
    \caption{DSRM Process Model}
    \begin{flushleft}
    \textit{Anmerkung.} Adaptiert aus \textcite[S. 54]{peffers_DesignScienceResearchmethodologyinformationsystemsresearch_2007}.
    \end{flushleft}
    \label{fig:peffers_dsrm}
\end{figure}

Die Operationalisierung der sechs Phasen des \ac{DSRM} beginnt mit der Phase \enquote{Problem Identification and Motivation} \parencite[S. 52--55]{peffers_DesignScienceResearchmethodologyinformationsystemsresearch_2007}. Die Definition des spezifischen Forschungsproblems und dessen Relevanz für \ac{KRITIS} wurde hierfür in Kapitel~\ref{sec:Problemstellung und Motivation} dargelegt. 

Darauf aufbauend werden in der Phase \enquote{Define the Objectives for a Solution} \parencite[S. 55]{peffers_DesignScienceResearchmethodologyinformationsystemsresearch_2007} aus der Problemanalyse konkrete Forschungsfragen (Kapitel~\ref{sec:Zielsetzung und Forschungsfragen}) und designrelevante Ziele für jede Iteration (Kapitel~\ref{sec:Designziele_Iteration_1} und Kapitel~\ref{sec:Designziele_Iteration_2}) abgeleitet, welche als Grundlage für die späteren Evaluationsphasen dienen.

Den Kern der Arbeit bildet die Phase \enquote{Design and Development} \parencite[S. 55]{peffers_DesignScienceResearchmethodologyinformationsystemsresearch_2007}, welche die Konzeption der Architektur sowie die technische Implementierung des \ac{PQC}-\ac{SSI}-Prototypen in beiden Iterationen umfasst (Kapitel~\ref{sec:Erste Iteration der Artefaktentwicklung} und \ref{sec:Zweite Iteration der Artefaktentwicklung}). Diese Phase operationalisiert die systematische Umsetzung der in den vorherigen Phasen definierten Anforderungen in ein funktionsfähiges Artefakt.

Die Eignung dieses Artefakts zur Problemlösung wird im Rahmen der Phase \enquote{Demonstration} \parencite[S. 55]{peffers_DesignScienceResearchmethodologyinformationsystemsresearch_2007} validiert. Diese erfolgt durch formative Evaluationsepisoden während der beiden Iterationen, welche schrittweise die Funktionsfähigkeit des Artefakts durch modulare Tests unter kontrollierten Bedingungen (Kapitel~\ref{sec:formative_evaluation_iteration1} und \ref{sec:formative_evaluation_iteration2}), bis hin zum vollständigen Prototypen (Kapitel~\ref{sec:Summative Evaluation}) nachweisen. 

Daran anschließend wird in der Phase \enquote{Evaluation} \parencite[S. 56]{peffers_DesignScienceResearchmethodologyinformationsystemsresearch_2007} strukturiert nach dem \ac{FEDS}-Framework (Kapitel~\ref{sec:FEDS-Framework}) in mehreren diskrete Evaluationsepisoden eine systematische Bewertung des Artefakts vorgenommen. Formative Episoden während beider Iterationen (Kapitel~\ref{sec:formative_evaluation_iteration1} und \ref{sec:formative_evaluation_iteration2}) adressieren technische Einzelfunktionen und Designfehler frühzeitig, die abschließende summative Evaluation in Kapitel~\ref{sec:Summative Evaluation} validiert systematisch die funktionale Korrektheit, \ac{KRITIS}-Compliance und Kryptoagilität des integrierten Gesamtsystems.

Den Abschluss bildet die Phase \enquote{Communication} \parencite[S. 56]{peffers_DesignScienceResearchmethodologyinformationsystemsresearch_2007}, in der die Ergebnisse, der Designprozess und die Evaluation durch die vorliegende schriftliche Ausarbeitung dokumentiert und der wissenschaftlichen Gemeinschaft zur Verfügung gestellt werden.

Obwohl das Modell sequenziell dargestellt ist, handelt es sich bei der Entwicklung in den Kapiteln~\ref{sec:Erste Iteration der Artefaktentwicklung} und \ref{sec:Zweite Iteration der Artefaktentwicklung} um einen iterativen Prozess, der Rücksprünge von der Evaluation zur Design-Phase erlaubt, um das Artefakt schrittweise zu verfeinern \parencite[S. 56]{peffers_DesignScienceResearchmethodologyinformationsystemsresearch_2007}.