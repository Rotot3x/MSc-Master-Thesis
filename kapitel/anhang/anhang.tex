\section{Systematische Literaturrecherche}\label{sec:Anhang_Systematische Literaturrecherche}

\begin{longtable}{L{0.5cm}L{4cm}L{1.5cm}L{7cm}}
    \caption[]{Bewertung der identifizierten Quellen hinsichtlich ihrer Relevanz}
    \label{tab:quellenbewertung} \\
    \toprule
    \textbf{Nr.} & \textbf{Quelle} & \textbf{Relevanz} & \textbf{Kommentar} \\
    \midrule
    \endfirsthead
    \multicolumn{4}{l}{\textit{Tabelle \thetable\ (Fortsetzung)}} \\
    \toprule
    \textbf{Nr.} & \textbf{Quelle} & \textbf{Relevanz} & \textbf{Kommentar} \\
    \midrule
    \endhead
    \midrule
    \multicolumn{4}{r}{\textit{Fortsetzung auf nächster Seite}} \\
    \endfoot
    \bottomrule
    \multicolumn{4}{p{\linewidth}}{\textit{Anmerkung.} Basierend auf den Abstracts aller in Spalte zwei unter \enquote{Quelle} aufgeführten Quellenangaben.} \\
    \endlastfoot
1 & Szymanski, T. H. (2024). A Quantum-Safe Software-Defined Deterministic Internet of Things (IoT) with Hardware-Enforced Cyber-Security for Critical Infrastructures. Information (2078-2489), 15(4), 173. \url{https://doi.org/10.3390/info15040173} & Hoch & Fokussiert auf die Entwicklung quantensicherer Kommunikations- und Sicherheitssysteme im Kontext von \ac{KRITIS} und Industrial IoT; adressiert explizit \ac{PQC} durch Einsatz quantensicherer Verschlüsselungsmechanismen und QKD-Netzwerke; behandelt hardwarebasierte Zugriffskontrollen, Zero Trust Architekturen und AI-gestützte Sicherheit, die für Resilienz und Sicherheitsanforderungen in \ac{KRITIS} maßgeblich sind; zwar steht \ac{SSI} und Blockchain-Technologie nicht im Mittelpunkt, jedoch zeigen die vorgestellten innovativen Konzepte und die experimentelle Validierung einen sehr hohen Anwendungs- und Erkenntniswert für den methodischen und technologischen Fortschritt in mindestens zwei zentralen Domänen (\ac{PQC}, \ac{KRITIS}); damit bietet der Beitrag substanzielle Impulse für den Schutz hochsensibler digitaler Infrastrukturen. \\
\midrule
2 & Nouma, S. E., \& Yavuz, A. A. (2024). Trustworthy and Efficient Digital Twins in Post-Quantum Era with Hybrid Hardware-Assisted Signatures. ACM Transactions on Multimedia Computing, Communications \& Applications, 20(6), 1–30. \url{https://doi.org/10.1145/3638250} & Hoch & Betont die Notwendigkeit verlässlicher und effizienter digitaler Signaturen im Kontext von Digital Twins, die primär auf IoT-Infrastrukturen für hochsensible Daten abzielen und damit eine wichtige Schnittmenge zu \ac{KRITIS} darstellen; adressiert explizit \ac{PQC} durch die Entwicklung und Umsetzung quantensicherer und hybrider Signaturlösungen—einschließlich Forward Security und Aggregation, die auch für Blockchain-basierte und dezentrale Identitätsanwendungen (insb. mit Skalierungsbedarf) hoch relevant sind; der methodische Fortschritt im Bereich hardwareunterstützter, ressourcenschonender Kryptografie bietet substanzielle Innovationsimpulse für sicherheitskritische Systeme mit beschränkten Ressourcen, wie sie für \ac{SSI}-Lösungen und die sichere Verwaltung digitaler Identitäten in \ac{KRITIS} essenziell sind; Blockchain-Technologie und \ac{SSI} werden nicht explizit vertieft, jedoch ist die Übertragbarkeit der vorgestellten Konzepte—insbesondere hybride und aggregierbare Signaturen—auf beide Domänen methodisch und praxisnah gegeben. \\
\midrule
3 & Sharif, A., Ranzi, M., Carbone, R., Sciarretta, G., Marino, F. A., \& Ranise, S. (2022). The eIDAS Regulation: A Survey of Technological Trends for European Electronic Identity Schemes. Applied Sciences (2076-3417), 12(24), 12679. \url{https://doi.org/10.3390/app122412679} & Hoch & Adressiert zentrale Entwicklungen und Herausforderungen europäischer elektronischer Identitätssysteme im Kontext der eIDAS-Regulierung; analysiert technologische Trends und ihre Auswirkungen auf Sicherheit, Datenschutz und Interoperabilität nationaler eID-Lösungen, was unmittelbar an die Domäne \ac{SSI} und deren regulatorisches Umfeld anschließt; behandelt aktuelle Technologiestandards wie OAuth 2.0, SAML und OpenID Connect, ohne explizit Blockchain- oder \ac{PQC}-Lösungen zu integrieren, beleuchtet jedoch die (in eIDAS 2.0 antizipierte) Entwicklung hin zu dezentralisierten Identitätsarchitekturen, die als Grundlage künftiger \ac{SSI}-Lösungen dienen; liefert wesentliche Erkenntnisse für die Ausgestaltung sicherer und interoperabler digitaler Identitäten in \ac{KRITIS} und gibt Impulse für die technologische und methodische Weiterentwicklung nationalübergreifender Identitätsverwaltung. \\
\midrule
4 & Alam, M., Hoffstein, J., \& Cambou, B. (2024). Privately Generated Key Pairs for Post Quantum Cryptography in a Distributed Network. Applied Sciences (2076-3417), 14(19), 8863. \url{https://doi.org/10.3390/app14198863} & Hoch & Fokussiert auf die praktische Erzeugung, Verteilung und Verifikation privat generierter post-quanten-sicherer Schlüsselpaaren in verteilten Netzwerken, mit expliziter Anwendung von Crystals-Dilithium als \ac{PQC}-Algorithmus; adressiert wesentlich die Domäne \ac{PQC} durch Integration und Umsetzung eines aktuellen Standards, was sowohl für die Sicherheit verteilter Infrastrukturen als auch für zukünftige Identitätslösungen (z.B. im Kontext von \ac{SSI}) zentral ist; Berücksichtigung von Multi-Faktor-Authentifizierung, Challenge-Response-Mechanismen und biometrielosen Verfahren bietet substanzielle methodische Impulse für die Entwicklung von sicheren, dezentralen Schlüsselmanagement- und Authentifizierungslösungen; direkte Einbindung in Blockchain- oder spezifische \ac{KRITIS} wird nicht explizit diskutiert, ist aufgrund des Protokoll- und \ac{PKI}-Fokus jedoch technisch anschlussfähig und für die Domänen \ac{SSI} und \ac{KRITIS} innovativ und relevant. \\
\midrule
5 & Radanliev, P. (2023). Review and Comparison of US, EU, and UK Regulations on Cyber Risk/Security of the Current Blockchain Technologies: Viewpoint from 2023. Review of Socionetwork Strategies, 17(2), 105–129. \url{https://doi.org/10.1007/s12626-023-00139-x} & Hoch & Vergleichende Analyse der US-, EU- und UK-Regulierung im Bereich Cyber-Risiken und Sicherheit aktueller Blockchain-Technologien, basierend auf dem Stand 2023; systematische Prüfung und Gegenüberstellung führender Standardwerke wie \ac{NIST} (US), ISO27001 (international), und neueren Regularien wie MiCA (EU) und CPMI-IOSCO, unter Einbeziehung technologieübergreifender Aspekte (u.a. \ac{PQC}, Cloud Security, IoT); liefert bedeutende Einblicke und unmittelbaren Anwendungsbezug für die Bewertung, Weiterentwicklung und Integration internationaler Cybersecurity-Standards in neue Blockchain-Projekte—insbesondere mit Blick auf die vier Domänen moderner Identitäts- und Sicherheitssysteme \\
\midrule
6 & Enaya, A., Fernando, X., \& Kashef, R. (2025). Survey of Blockchain-Based Applications for IoT. Applied Sciences (2076-3417), 15(8), 4562. \url{https://doi.org/10.3390/app15084562} & Mittel & Betrachtet zentrale Aspekte von Blockchain-Technologien und Sicherheit; Schwerpunkt auf IoT-Anwendungen und branchenspezifische Implementierungen; explizite Bezüge zu \ac{SSI}, \ac{PQC} und \ac{KRITIS} fehlen; breiter Überblick, jedoch geringere Spezifizität hinsichtlich der vier Domänen der Masterarbeit ( \ac{SSI}, Blockchain, \ac{PQC}, \ac{KRITIS}). \\
\midrule
7 & Siam, M. K., Saha, B., Hasan, M. M., Hossain Faruk, M. J., Anjum, N., Tahora, S., Siddika, A., \& Shahriar, H. (2025). Securing Decentralized Ecosystems: A Comprehensive Systematic Review of Blockchain Vulnerabilities, Attacks, and Countermeasures and Mitigation Strategies. Future Internet, 17(4), 183. \url{https://doi.org/10.3390/fi17040183} & Mittel & Systematische Analyse von Schwachstellen, Angriffsszenarien und Gegenmaßnahmen in Blockchain-Ökosystemen; konzentriert sich auf Sicherheitsaspekte von Blockchain-Technologien ohne spezifische Betrachtung von \ac{SSI}, \ac{PQC} oder \ac{KRITIS}; hoher inhaltlicher Wert für das allgemeine Verständnis von Blockchain-Sicherheit, jedoch eingeschränkte Anwendbarkeit auf alle vier Domänen der Masterarbeit. \\
\midrule
8 & Ramirez Lopez, L. J., \& Morillo Ledezma, G. G. (2025). Employing Blockchain, NFTs, and Digital Certificates for Unparalleled Authenticity and Data Protection in Source Code: A Systematic Review. Computers (2073-431X), 14(4), 131. \url{https://doi.org/10.3390/computers14040131} & Mittel & Fokussiert auf Blockchain-basierte Technologien zur Sicherung von Authentizität und Zugriffskontrolle, jedoch im Anwendungskontext akademischer Quellcode-Sicherheit; behandelt primär NFTs und digitale Zertifikate, mit begrenztem Bezug zu \ac{SSI} und ohne Einbeziehung von \ac{PQC}; adressiert die Domäne „Blockchain“ und Aspekte der Datensicherheit, weist jedoch eine geringe Relevanz für die Domänen \ac{SSI}, \ac{PQC} und \ac{KRITIS} im Kontext der Masterarbeit auf. \\
\midrule
9 & Sebestyen, H., Popescu, D. E., \& Zmaranda, R. D. (2025). A Literature Review on Security in the Internet of Things: Identifying and Analysing Critical Categories. Computers (2073-431X), 14(2), 61. \url{https://doi.org/10.3390/computers14020061} & Mittel & Umfassende Betrachtung aktueller Sicherheitsthemen und Identitätsmanagement im IoT-Kontext, unter Rückgriff auf neue Technologien wie Blockchain; Integrationspotenzial hinsichtlich Blockchain erkennbar, jedoch keine explizite Behandlung von \ac{SSI}, \ac{PQC} oder \ac{KRITIS}; bietet einen breiten Überblick zu technologischen Lösungen und Herausforderungen, adressiert jedoch nur partiell die Anforderungen und Innovationspotenziale der vier Domänen der Masterarbeit. \\
\midrule
10 & Nambundo, J. M., de Souza Martins Gomes, O., de Souza, A. D., \& Machado, R. C. S. (2025). Cybersecurity and Major Cyber Threats of Smart Meters: A Systematic Mapping Review. Energies (19961073), 18(6), 1445. \url{https://doi.org/10.3390/en18061445} & Mittel & Konzentriert sich auf Cybersecurity-Bedrohungen und Schwachstellen im Kontext von Smart Metern als Teil kritischer Infrastruktur; adressiert relevante Sicherheitsfragen in Bezug auf Energieversorgung, thematisch verwandt mit der Schutzbedarfsanalyse für \ac{KRITIS}; der Einsatz von \ac{SSI}, Blockchain-Technologien oder \ac{PQC} wird nicht explizit thematisiert; bietet wichtige Einblicke in Bedrohungsszenarien und Mitigationsstrategien für smarte Energiesysteme, bleibt jedoch im Hinblick auf innovative Identitäts- oder Kryptografielösungen und deren methodischer Integration in Smart Metering-Systeme unspezifisch. \\
\midrule
11 & Yuan, F., Huang, X., Zheng, L., Wang, L., Wang, Y., Yan, X., Gu, S., \& Peng, Y.(2025). The Evolution and Optimization Strategies of a PBFT Consensus Algorithm for Consortium Blockchains. Information (2078-2489), 16(4), 268. \url{https://doi.org/10.3390/info16040268} & Mittel & Fokussiert auf die Optimierung des PBFT-Konsensalgorithmus, was grundlegende Bedeutung für Leistung, Sicherheit und Zuverlässigkeit von Consortium Blockchains hat; leistet methodischen Beitrag zur technologischen Weiterentwicklung im Blockchain-Bereich, jedoch ohne explizite Einbindung von \ac{SSI}, \ac{PQC} oder direktem Bezug zu \ac{KRITIS}; relevante Erkenntnisse zur Verbesserung von Konsensmechanismen, die potenziell für skalierbare und sichere Blockchain-basierte Infrastrukturen adaptierbar sind, aber inhaltlich primär auf Konsensalgorithmen begrenzt. \\
\midrule
12 & Radanliev, P. (2024). Digital security by design. Security Journal, 37(4), 1640–1679. \url{https://doi.org/10.1057/s41284-024-00435-3} & Mittel & Bietet eine umfassende, technologieübergreifende Analyse aktueller Herausforderungen im Bereich digitale Sicherheit; adressiert relevante Zukunftsthemen wie AI, Blockchain und Quantencomputing im Kontext sich wandelnder Sicherheitsparadigmen; keine explizite Schwerpunktsetzung auf \ac{SSI}, \ac{PQC} oder spezifisch \ac{KRITIS}; sektorübergreifende Betrachtung liefert wertvolle Einblicke zu regulatorischen und praxisbezogenen Aspekten, bleibt jedoch in Bezug auf die integrative Anwendung innovativer Sicherheitslösungen innerhalb der vier Domänen der Masterarbeit unspezifisch. \\
\midrule
13 & Miller, T., Durlik, I., Kostecka, E., Sokołowska, S., Kozlovska, P., \& Zwolak, R. (2025). Artificial Intelligence in Maritime Cybersecurity: A Systematic Review of AI-Driven Threat Detection and Risk Mitigation Strategies. Electronics (2079-9292), 14(9), 1844. \url{https://doi.org/10.3390/electronics14091844} & Mittel & Fokus liegt auf der Anwendung von KI-gestützten Verfahren zur Erkennung und Minderung von Cyber-Bedrohungen im maritimen Sektor, adressiert damit relevante Aspekte kritischer Infrastrukturen; Schnittstellen zu Blockchain und quantenkryptographischen Ansätzen werden als Forschungsperspektiven genannt, ohne im Review zentrale methodische oder anwendungsbezogene Ausarbeitung zu bieten; \ac{SSI} wird nicht behandelt, der primäre Schwerpunkt liegt auf KI und Cybersecurity, wodurch methodische und technische Details zu \ac{SSI} und \ac{PQC} im Kontext der vier Domänen der Masterarbeit fehlen. \\
\midrule
14 & Atlam, H. F., Ekuri, N., Azad, M. A., \& Lallie, H. S. (2024). Blockchain Forensics: A Systematic Literature Review of Techniques, Applications, Challenges, and Future Directions. Electronics (2079-9292), 13(17), 3568. \url{https://doi.org/10.3390/electronics13173568} & Mittel & Umfassende Analyse von Blockchain-Technologien im digitalen Forensik-Kontext mit Schwerpunkt auf Untersuchungsmethodik und Anwendungsfeldern; adressiert insbesondere die Herausforderungen beim Nachweis und der Verfolgung von Aktivitäten auf Blockchain-Systemen und beleuchtet regulatorische Aspekte, jedoch ohne explizite Behandlung von \ac{SSI}, \ac{PQC} oder den speziellen Anforderungen kritischer Infrastrukturen; liefert wertvolle Einblicke in forensische Anwendungen der Blockchain, bleibt jedoch mit Blick auf innovative Identitäts- und Kryptografielösungen sowie den Schutz kritischer Infrastrukturen im Rahmen der Masterarbeit begrenzt anschlussfähig. \\
\midrule
15 & Yakubu, M. M., Fadzil B Hassan, M., Danyaro, K. U., Junejo, A. Z., Siraj, M., Yahaya, S., Adamu, S., \& Abdulsalam, K. (2024). A Systematic Literature Review on Blockchain Consensus Mechanisms’ Security: Applications and Open Challenges. Computer Systems Science \& Engineering, 48(6), 1437–1481. \url{https://doi.org/10.32604/csse.2024.054556} & Mittel & Umfassende systematische Analyse der Sicherheitsaspekte und Herausforderungen von Blockchain-Konsensmechanismen mit Fokus auf deren Integrität, Zuverlässigkeit und praktische Anwendungen; adressiert die Domäne „Blockchain“ in methodischer Tiefe und liefert wertvolle Erkenntnisse hinsichtlich Sicherheit, Skalierbarkeit und Energieeffizienz von Konsensprotokollen; explizite Bezüge zu \ac{SSI}, \ac{PQC} und spezifisch \ac{KRITIS} fehlen, sodass die Übertragbarkeit auf die weiteren drei Domänen des Masterarbeit-Themas begrenzt ist. \\
\midrule
16 & Oude Roelink, B., El, H. M., \& Sarmah, D. (2024). Systematic review: Comparing zk‐SNARK, zk‐STARK, and bulletproof protocols for privacy‐preserving authentication. Security \& Privacy, 7(5), 1–59. \url{https://doi.org/10.1002/spy2.401} & Mittel & Fokussiert auf den Vergleich und die Analyse moderner Zero-Knowledge-Protokolle (zk-SNARKs, zk-STARKs, Bulletproofs) mit hoher Relevanz für die Domänen Privacy und Blockchain, insbesondere im Kontext von Authentifizierung und Datenschutz; liefert methodische und performancebezogene Einblicke, jedoch ohne explizite Berücksichtigung von \ac{SSI} oder \ac{PQC}; Anbindung an \ac{KRITIS} nicht direkt gegeben, bietet jedoch Potenzial für Integration innovativer Datenschutztechnologien in Blockchain-basierte Identitäts- und Authentifizierungssysteme. \\
\midrule
17 & Cherbal, S., Zier, A., Hebal, S., Louail, L., \& Annane, B. (2024). Security in internet of things: a review on approaches based on blockchain, machine learning, cryptography, and quantum computing. Journal of Supercomputing, 80(3), 3738–3816. \url{https://doi.org/10.1007/s11227-023-05616-2} & Mittel & Umfassende Analyse sicherheitsrelevanter Technologien im IoT-Kontext, darunter Blockchain, Kryptografie und Quantencomputing, mit breitem Überblick zu Ansätzen und Herausforderungen; adressiert insbesondere die Domänen Blockchain und \ac{PQC} durch die Einbeziehung quantenkryptographischer und klassischer kryptographischer Lösungen, ohne explizite Behandlung von \ac{SSI} oder dem spezifischen Anwendungsfeld kritischer Infrastrukturen; liefert wertvolle Vergleiche und Taxonomien zu modernen Sicherheitsmechanismen im IoT, bleibt jedoch hinsichtlich der methodischen und domänenspezifischen Vertiefung für \ac{SSI} und \ac{KRITIS} begrenzt. \\
\midrule
18 & Jagarlamudi, G. K., Yazdinejad, A., Parizi, R. M., \& Pouriyeh, S. (2024). Exploring privacy measurement in federated learning. Journal of Supercomputing, 80(8), 10511–10551. \url{https://doi.org/10.1007/s11227-023-05846-4} & Mittel & Fokussiert auf die Analyse von Privacy-Maßnahmen und deren Messbarkeit im Kontext von föderiertem Lernen, mit hohem methodischen Wert zur Bewertung von Datenschutz und Sicherheitsmetriken; Bezüge zu Blockchain oder \ac{SSI} werden nicht explizit hergestellt, und \ac{PQC} ist nur als Zukunftsperspektive am Rand angedeutet; die behandelten Konzepte und resultierenden Erkenntnisse zu Privacy-Measurement-Methoden können für sichere, dezentrale Systeme (einschließlich kritischer Infrastrukturen) grundsätzlich relevant sein, liefern jedoch keine spezifische oder anwendungsbezogene Vertiefung in den vier Kernbereichen der Masterarbeit. \\
\midrule
19 & Alzoubi, Y. I., Gill, A., \& Mishra, A. (2022). A systematic review of the purposes of Blockchain and fog computing integration: classification and open issues. Journal of Cloud Computing (2192-113X), 11(1), 1–36. \url{https://doi.org/10.1186/s13677-022-00353-y} & Mittel & Systematische Analyse der Integration von Blockchain-Technologien mit Fog Computing zur Lösung sicherheitsrelevanter Herausforderungen in IoT-Anwendungen; adressiert zentrale Aspekte wie Sicherheit, Datenschutz, Zugriffs- und Vertrauensmanagement, jedoch ohne explizite Behandlung von \ac{SSI} oder konkreten Umsetzungen von \ac{PQC}; verweist auf bestehende regulatorische und technologische Herausforderungen durch aufkommende Technologien wie Quantencomputing, aber ohne methodische oder anwendungsbezogene Vertiefung in Bezug auf \ac{KRITIS} oder innovative Identitätskonzepte; bietet wertvolle Klassifikation und Überblick zu Blockchain-Anwendungen im Kontext verteilter Edge-Infrastrukturen, bleibt jedoch bezüglich der vier Kernbereiche der Masterarbeit auf die Domäne „Blockchain“ und allgemeine Sicherheitsaspekte beschränkt. \\
\midrule
20 & Kazmi, S. H. A., Hassan, R., Qamar, F., Nisar, K., \& Ibrahim, A. A. A. (2023). Security Concepts in Emerging 6G Communication: Threats, Countermeasures, Authentication Techniques and Research Directions. Symmetry (20738994), 15(6), 1147. \url{https://doi.org/10.3390/sym15061147} & Mittel & Umfassende Analyse sicherheitsrelevanter Konzepte, Bedrohungen und Authentifizierungsmethoden im Kontext der aufkommenden 6G-Kommunikation; adressiert innovative Technologien wie Künstliche Intelligenz, Quantencomputing und Föderiertes Lernen, wodurch potenzielle Schnittstellen zu \ac{PQC} und Sicherheitsanforderungen für \ac{KRITIS} bestehen; explizite Bezüge zu \ac{SSI} und Blockchain-Technologien fehlen, ebenso eine methodische Vertiefung für spezielle \ac{SSI}- oder Blockchain-basierte Authentifizierungslösungen; bietet wertvolle Einblicke in zukünftige Forschungsrichtungen und technologische Herausforderungen, jedoch begrenzte direkte Anwendbarkeit auf alle vier Kernbereiche der Masterarbeit. \\
\midrule
21 & Attkan, A., \& Ranga, V. (2022). Cyber-physical security for IoT networks: a comprehensive review on traditional, blockchain and artificial intelligence based key-security. Complex \& Intelligent Systems, 8(4), 3559–3591. \url{https://doi.org/10.1007/s40747-022-00667-z} & Mittel & Fokussiert auf Authentifizierung und Schlüsselmanagement in IoT-Netzen unter Einbeziehung klassischer, Blockchain-basierter und KI-gestützter Sicherheitsmechanismen; adressiert im Wesentlichen die Domäne \enquote{Blockchain} und bietet innovative Perspektiven zur dezentralen Verwaltung von Session-Keys sowie zu KI-basierten Angriffserkennungsmethoden im IoT-Kontext; \ac{SSI} und \ac{PQC} werden nicht explizit behandelt, ebenso fehlt die gezielte Anwendung auf \ac{KRITIS}; der umfassende Überblick zu Authentifizierung und Schlüsselmanagement bietet methodische Anschlussmöglichkeiten für \ac{SSI}-basierte Systeme oder kritische Infrastruktur, bleibt im Kern jedoch breit und technologieorientiert ohne vertiefte Ausarbeitung der vier Domänen der Masterarbeit. \\
\midrule
22 & Ray, P. P. (2023). Web3: A comprehensive review on background, technologies, applications, zero-trust architectures, challenges and future directions. Internet of Things \& Cyber Physical Systems, 3, 213–248. \url{https://doi.org/10.1016/j.iotcps.2023.05.003} & Mittel & Bietet einen breiten Überblick über die technologischen und gesellschaftlichen Grundlagen sowie Anwendungsfelder von Web3 und dezentralen Plattformen, wobei die Domäne Blockchain als zentrales Element systematisch behandelt wird; explizite Bezüge zu \ac{SSI} und Zero-Trust-Architekturen existieren, jedoch fehlt eine detaillierte methodische Analyse spezifischer \ac{SSI}-Lösungen oder Implementierungen, und \ac{PQC} wird nicht thematisiert; der Beitrag verweist auf innovative Identitätskonzepte, Anwendungsintegration und technische Herausforderungen, bleibt jedoch hinsichtlich der anwendungsbezogenen Vertiefung zu \ac{PQC} sowie spezifischen Schutzmaßnahmen für \ac{KRITIS} auf einer konzeptionellen Ebene. \\
\midrule
23 & Stach, C., Gritti, C., Bräcker, J., Behringer, M., \& Mitschang, B. (2022). Protecting Sensitive Data in the Information Age: State of the Art and Future Prospects. Future Internet, 14(11), 302. \url{https://doi.org/10.3390/fi14110302} & Mittel & Umfassende Analyse aktueller Privacy-Mechanismen im Kontext datengetriebener Smart Services, mit Fokus auf die praktische Umsetzung nutzerfreundlicher Datenschutzlösungen; adressiert zentrale Herausforderungen und praktische Einsatzfelder moderner Datenschutzverfahren, ohne jedoch explizit auf \ac{SSI}, Blockchain-Technologien, \ac{PQC} oder spezielle Schutzanforderungen kritischer Infrastrukturen einzugehen; liefert wertvolle Einblicke zu datenorientierten Schutzmechanismen und deren Limitationen, bleibt jedoch hinsichtlich der methodischen und domänenspezifischen Vertiefung für \ac{SSI}, \ac{PQC} und \ac{KRITIS} konzeptionell und generisch. \\
\midrule
24 & Farooq, M. S., Riaz, S., \& Alvi, A. (2023). Security and Privacy Issues in Software-Defined Networking (SDN): A Systematic Literature Review. Electronics (2079-9292), 12(14), 3077. \url{https://doi.org/10.3390/electronics12143077} & Mittel & Systematische Analyse von Sicherheits- und Datenschutzproblemen in Software-Defined Networks mit Schwerpunkt auf Schwachstellen, Angriffen und Sicherungsmechanismen entlang der verschiedenen Netzwerkebenen; adressiert grundlegende Herausforderungen für den Schutz moderner Netzwerkarchitekturen, insbesondere durch die Trennung von Steuer- und Datenebene—ein Aspekt, der für den Betrieb kritischer Infrastrukturen relevante Einblicke und Methoden liefert; explizite Bezüge zu Blockchain-Technologien, \ac{SSI} und \ac{PQC} fehlen, jedoch kann die vorgestellte Taxonomie und die Diskussion zukünftiger Forschungsrichtungen methodische Impulse für sichere Integrationskonzepte in verteilten, kritischen oder identitätsgetriebenen Architekturen bieten—die inhaltliche Tiefe und Anwendbarkeit bleibt jedoch primär auf SDN-spezifische Herausforderungen fokussiert. \\
\midrule
25 & Chanal, P. M., \& Kakkasageri, M. S. (2020). Security and Privacy in IoT: A Survey. Wireless Personal Communications, 115(2), 1667–1693. \url{https://doi.org/10.1007/s11277-020-07649-9} & Mittel & Bietet einen umfassenden Überblick über grundlegende Sicherheits- und Datenschutzherausforderungen im Kontext des Internet of Things mit Fokus auf ressourcenbeschränkte Geräte; adressiert Kernaspekte wie Vertraulichkeit, Integrität, Authentifizierung und Verfügbarkeit, ohne jedoch explizit auf \ac{SSI}, Blockchain-Technologien, \ac{PQC} oder spezifische Anforderungen kritischer Infrastrukturen einzugehen; liefert wertvolle konzeptionelle Grundlagen zu Sicherheitsanforderungen und -architekturen im IoT, bleibt jedoch bezüglich anwendungs- oder methodenspezifischer Vertiefung zu den vier Kernbereichen der Masterarbeit generisch. \\
\midrule
26 & Choudhary, A. (2024). Internet of Things: a comprehensive overview, architectures, applications, simulation tools, challenges and future directions. Discover Internet of Things, 4(1), 1–41. \url{https://doi.org/10.1007/s43926-024-00084-3} & Mittel & Umfassende Übersicht und Analyse der IoT-Architektur, Anwendungen und Herausforderungen; adressiert technologische, soziale und funktionale Aspekte des Internet of Things, mit generischer Betrachtung von Architekturen und Simulationsumgebungen; explizite Bezüge zu \ac{SSI}, Blockchain-Technologien, \ac{PQC} oder spezifischen Schutzanforderungen kritischer Infrastrukturen fehlen; liefert wertvolle Grundlagen für das technologische Umfeld, bleibt jedoch hinsichtlich der vier Domänen der Masterarbeit konzeptionell und unspezifisch. \\
\midrule
27 & Sikiru, I. A., Kora, A. D., Ezin, E. C., Imoize, A. L., \& Li, C.-T. (2024). Hybridization of Learning Techniques and Quantum Mechanism for IIoT Security: Applications, Challenges, and Prospects. Electronics (2079-9292), 13(21), 4153. \url{https://doi.org/10.3390/electronics13214153} & Mittel & Systematische Analyse hybrider Sicherheitsansätze in der Industrial IoT (IIoT), insbesondere durch die Kombination klassischer Lernverfahren und quantenmechanistischer Ansätze; adressiert relevante Herausforderungen und Perspektiven der IIoT-Sicherheit mit Berücksichtigung von Blockchain-Technologien und Quantum Mechanisms, wobei \ac{PQC} eher implizit thematisiert wird; explizite Vertiefung von \ac{SSI} und spezifische Anwendungen in \ac{KRITIS} fehlen, liefert jedoch Impulse für die Integration moderner Kryptografie- und Sicherheitsverfahren im industriellen Umfeld. \\
\midrule
28 & RadRadanliev, P. (2024). Artificial intelligence and quantum cryptography. Journal of Analytical Science \& Technology, 14, 1–17. \url{https://doi.org/10.1186/s40543-024-00416-6} & Mittel & Thematisiert den aktuellen Stand und die Zukunftsperspektiven an der Schnittstelle von künstlicher Intelligenz und quantenkryptografischen Verfahren, wobei insbesondere der Einfluss von AI-Methoden auf Effizienz und Robustheit kryptografischer Systeme sowie die Herausforderungen durch das \enquote{Quantum Threat}-Szenario im Zentrum stehen; explizite Bezüge zu \ac{SSI}, Blockchain-Technologien oder spezifischen Anwendungsfällen in \ac{KRITIS} fehlen; liefert wertvolle Impulse zu methodischen Innovationen in der \ac{PQC}, bleibt jedoch hinsichtlich der vier Domänen der Masterarbeit primär konzeptionell und technologisch fokussiert auf AI und Quantenkryptografie. \\
\midrule
29 & O’Donoghue, O., Vazirani, A. A., Brindley, D., \& Meinert, E. (2019). Design Choices and Trade-Offs in Health Care Blockchain Implementations: Systematic Review. Journal of Medical Internet Research, 21(5), e12426. \url{https://doi.org/10.2196/12426} & Mittel & Systematische Analyse von Architektur- und Design-Entscheidungen bei der Implementierung von Blockchain-Technologie im Kontext elektronischer Gesundheitsakten (EMR), mit Schwerpunkt auf sicherheitsrelevanten und skalierbaren Systemanforderungen sowie Trade-offs zwischen verschiedenen technischen, organisatorischen und anwendungsbezogenen Merkmalen; behandelt die Domäne Blockchain umfassend und liefert wertvolle Erkenntnisse über sicherheitsrelevante Kompromisse im Gesundheitswesen, adressiert jedoch weder \ac{SSI} noch \ac{PQC} oder \ac{KRITIS} explizit; die Untersuchung des Spannungsfelds zwischen Sicherheit, Skalierbarkeit und Datenmanagement bildet eine methodisch relevante Grundlage, bleibt aber in Bezug auf die vier zentralen Domänen der Masterarbeit auf anwendungsbezogene Blockchain-Implementierungen beschränkt. \\
\midrule
30 & Mulholland, J., Mosca, M., \& Braun, J. (2017). The Day the Cryptography Dies. IEEE Security \& Privacy, 15(4), 14–21. \url{https://doi.org/10.1109/MSP.2017.3151325} & Mittel & Überblickartige Darstellung der Auswirkungen von Quantencomputern auf bestehende kryptografische Verfahren; adressiert explizit die Bedrohung aktueller Sicherheitstechnologien (\ac{PQC}), jedoch ohne methodische oder technologische Vertiefung zu Blockchain, \ac{SSI} oder \ac{KRITIS}; bietet konzeptionelle Einblicke in Risikoszenarien und Bedrohungsmodelle, bleibt jedoch hinsichtlich innovativer Lösungsansätze oder spezifischer Anwendungsgebiete der vier Domänen der Masterarbeit unspezifisch. \\
\midrule
31 & G, C. A., \& Basarkod, P. I. (2024). A survey on blockchain security for electronic health record. Multimedia Tools and Applications: An Journal, 1–35. \url{https://doi.org/10.1007/s11042-024-19883-5} & Mittel & Fokus auf Blockchain-basierte Sicherheitslösungen für elektronische Gesundheitsakten (EHR) mit Einbindung von Deep-Learning-Methoden; adressiert primär die Domäne Blockchain durch Analyse von Datenschutz, Datensicherheit und Zugriffskontrolle im Gesundheitswesen, bietet wertvolle methodische Einblicke zur Anwendung verteilter Technologien im Bereich sensibler Daten; explizite Bezüge zu \ac{SSI}, \ac{PQC} und spezifisch \ac{KRITIS} außerhalb des Gesundheitssektors fehlen, wodurch die Anwendbarkeit auf alle vier Domänen der Masterarbeit beschränkt bleibt. \\
\midrule
32 & Batta, P., Ahuja, S., \& Kumar, A. (2024). Future Directions for Secure IoT Frameworks: Insights from Blockchain-Based Solutions: A Comprehensive Review and Future Analysis. Wireless Personal Communications: An International Journal, 139(3), 1749–1781. \url{https://doi.org/10.1007/s11277-024-11694-z} & Mittel & Systematische Untersuchung sicherer IoT-Frameworks unter Verwendung von Blockchain-Technologien; detaillierte Analyse verschiedener Algorithmen (u.a. Konsensmechanismen, \ac{RSA}, Hashing) und Plattformen (Ethereum, CoSMOS, Hyperledger Fabric), mit Fokus auf die Verbesserung von Sicherheit und Performance in IoT-Systemen; explizite Bezüge zu \ac{SSI}, \ac{PQC} und den besonderen Anforderungen kritischer Infrastrukturen fehlen; adressiert vor allem die Domäne „Blockchain“ und liefert grundlegende Einblicke zur Anwendung verteilter Sicherheitsmechanismen im IoT, bleibt jedoch hinsichtlich der vier Kerndomänen der Masterarbeit ( \ac{SSI}, Blockchain, \ac{PQC}, \ac{KRITIS}) methodisch und domänenspezifisch eingeschränkt. \\
\midrule
33 & Radanliev, P. (2024). Integrated cybersecurity for metaverse systems operating with artificial intelligence, blockchains, and cloud computing. Frontiers in Blockchain, 1–14. \url{https://doi.org/10.3389/fbloc.2024.1359130} & Mittel & Umfassende Analyse der Cybersicherheitslandschaft im Kontext integrierter Metaverse-Systeme unter Einbezug von Artificial Intelligence, Blockchain und Cloud Computing; adressiert zentrale Risikofelder, regulatorische Herausforderungen und die Rolle moderner Sicherheitstechnologien für die digitale Ökonomie, wobei insbesondere Blockchain in seiner Bedeutung für selbstverwaltete Systeme und Netzwerkgovernance diskutiert wird; explizite Vertiefungen zu \ac{SSI}, \ac{PQC} oder deren spezieller Anwendung im Umfeld kritischer Infrastrukturen fehlen, ebenso bleibt die methodische und technologische Anbindung an innovative Identitäts- und Kryptografieansätze im Rahmen der vier Domänen der Masterarbeit auf konzeptionelle Ausblicke beschränkt. \\
\midrule
34 & Hajian Berenjestanaki, M., Barzegar, H. R., El Ioini, N., \& Pahl, C. (2024). Blockchain-Based E-Voting Systems: A Technology Review. Electronics (2079-9292), 13(1), 17. \url{https://doi.org/10.3390/electronics13010017} & Mittel & Systematische Analyse von Blockchain-basierten E-Voting-Systemen mit Fokus auf Sicherheits-, Transparenz- und Integritätsaspekte; zentrale Bewertung technologischer Herausforderungen und zukünftiger Forschungsfragen, insbesondere zu Skalierbarkeit und Datenschutz – thematisch eng an die Domäne „Blockchain“ angelehnt; explizite Vertiefung von \ac{SSI}, \ac{PQC} oder die Anwendung in besonders schützenswerten \ac{KRITIS} fehlt, bietet jedoch methodische Ansätze und technische Perspektiven, die für die Entwicklung sicherer und vertrauenswürdiger Abstimmungssysteme in digitalen Infrastrukturen relevant sein können. \\
\midrule
35 & Pirbhulal, S., Chockalingam, S., Shukla, A., \& Abie, H. (2024). IoT cybersecurity in 5G and beyond: a systematic literature review. International Journal of Information Security, 23(4), 2827–2879. \url{https://doi.org/10.1007/s10207-024-00865-5} & Mittel & Systematische Literaturübersicht zu Cybersicherheitsaspekten in 5G- und Next-Generation-IoT-Umgebungen, insbesondere hinsichtlich Threats, Authentifizierung, Zugriffskontrolle, Netzwerk- und Anwendungsschicht sowie Herausforderungen durch Softwarisierung und Virtualisierung der Netze; adressiert methodisch den aktuellen Forschungsstand, evaluiert genutzte Validierungsansätze (Praxis, Simulation, Theorie) und liefert ein Kategorienschema für existierende Sicherheitsmechanismen und offene Forschungsfragen; explizite Bezüge zu \ac{SSI}, Blockchain-Technologien oder \ac{PQC} fehlen, ebenso eine gezielte Betrachtung von \ac{KRITIS}—die Branchenbeispiele (z. B. Healthcare, Energie) lassen eine indirekte Bedeutung für KRITIS erkennen, ohne diese jedoch methodisch zu vertiefen; methodische und technologische Tiefe für die vier Masterarbeitsdomänen beschränkt sich auf generische Cybersicherheitsbedrohungen und Lösungsansätze in modernen IoT/5G-Systemen. \\
\midrule
36 & Asif, M., Abrar, M., Salam, A., Amin, F., Ullah, F., Shah, S., \& AlSalman, H. (2025). Intelligent two-phase dual authentication framework for Internet of Medical Things. Scientific Reports, 15(1), 1–19. \url{https://doi.org/10.1038/s41598-024-84713-5} & Mittel & Fokus liegt auf der Entwicklung und Evaluierung eines intelligenten Zwei-Phasen-Authentifizierungsframeworks für die Internet of Medical Things (IoMT) mit Ziel der effizienten und sicheren Kommunikation sensibler Gesundheitsdaten; zentrale technische Ansätze umfassen ECDH für Schlüsselaustausch und AES-GCM für Datenverschlüsselung, wobei signifikante Verbesserungen in Effizienz und Sicherheit gegenüber klassischen Authentifizierungsmethoden nachgewiesen werden; explizite Bezüge zu \ac{SSI} und Blockchain-Technologien sowie \ac{PQC} fehlen vollständig, sodass methodische und technologische Innovationen in diesen Bereichen für den Rahmen der Masterarbeit unberücksichtigt bleiben; adressiert Schutzanforderungen im Bereich kritischer Infrastrukturen exemplarisch am Gesundheitswesen, bleibt jedoch in der Tiefe auf klassische kryptografische Verfahren und Authentifizierungsprozesse limitiert. \\
\midrule
37 & Marengo, A., \& Santamato, V. (2025). Quantum algorithms and complexity in healthcare applications: a systematic review with machine learning-optimized analysis. Frontiers in Computer Science, 1–30. \url{https://doi.org/10.3389/fcomp.2025.1584114} & Mittel & Systematische Übersicht zur Anwendung von Quantenalgorithmen und quanten-inspirierten Komplexitätsanalysen im Gesundheitswesen, mit zwei Schwerpunkten: (1) Quantum Computing für KI-basierte Analysen biomedizinischer Daten und (2) quantenkryptografische Protokolle zur Absicherung medizinischer Daten. Expliziter Bezug zur Domäne \ac{PQC} durch Analyse quantensicherer und blockchain-basierter Sicherheitsmechanismen im Healthcare-Kontext. Keine explizite Behandlung von \ac{SSI} oder dezidierten Blockchain-Architekturen außerhalb sicherheitsrelevanter Frameworks; Anwendung auf \ac{KRITIS} implizit durch den Fokus auf sichere medizinische Systeme, jedoch nicht technologieübergreifend vertieft. Insgesamt methodisch relevant für die Domäne \ac{PQC} und für Sicherheitsthemen im medizinisch-kritischen Sektor, für die vier Themenbereiche der Masterarbeit aber primär im Bereich quantensicherer Daten- und KI-Anwendungen anschlussfähig. \\
\midrule
38 & Ahakonye, L. A. C., Nwakanma, C. I., \& Kim, D.-S. (2024). Tides of Blockchain in IoT Cybersecurity. Sensors (14248220), 24(10), 3111. \url{https://doi.org/10.3390/s24103111} & Mittel & Umfassende Übersicht zu Anwendungsmöglichkeiten und Herausforderungen von Blockchain-Technologie im Bereich der IoT-Cybersicherheit, insbesondere in Verbindung mit KI-unterstützten Intrusion Detection Systemen; adressiert die Domäne „Blockchain“ grundlegend sowie deren Potenzial für Transparenz, Dezentralität und Unveränderlichkeit im IoT-Kontext; Integration von AI und Blockchain als Innovationstreiber für sichere und skalierbare IDS-Lösungen im IoT/IIoT; \ac{SSI} und \ac{PQC} werden nicht explizit behandelt; spezifische Anforderungen und Anwendungsfälle für \ac{KRITIS} werden nur indirekt adressiert; liefert wertvolle Einblicke und methodische Ansätze für die Weiterentwicklung sicherer IoT-Systeme, bleibt jedoch hinsichtlich der vier Domänen der Masterarbeit vorwiegend auf Blockchain und allgemeine Sicherheitsthemen im IoT fokussiert. \\
\midrule
39 & Nain, A., Sheikh, S., Shahid, M., \& Malik, R. (2024). Resource optimization in edge and SDN-based edge computing: a comprehensive study. Cluster Computing, 27(5), 5517–5545. \url{https://doi.org/10.1007/s10586-023-04256-8} & Niedrig & Umfassende systematische Analyse aktueller Optimierungsansätze für Ressourcenmanagement in Edge-Computing-Umgebungen, insbesondere unter Integration von Software-Defined Networking (SDN); adressiert zentrale Herausforderungen der effizienten Ressourcennutzung, Kontrollarchitekturen und Netzwerkprogrammierbarkeit, was insbesondere für leistungsfähige, latenzarme Anwendungen und Systemarchitekturen an der Netzwerkkante relevant ist; explizite Bezüge zu \ac{SSI}, Blockchain-Technologien, \ac{PQC} oder den besonderen Anforderungen kritischer Infrastrukturen fehlen; liefert dennoch wertvolle methodische Impulse für das Design verteilter, dynamischer Infrastrukturen, bleibt aber in Bezug auf die vier Domänen der Masterarbeit überwiegend allgemein und technologieorientiert. \\
\midrule
40 & Netinant, P., Saengsuwan, N., Rukhiran, M., \& Pukdesree, S. (2023). Enhancing Data Management Strategies with a Hybrid Layering Framework in Assessing Data Validation and High Availability Sustainability. Sustainability (2071-1050), 15(20), 15034. \url{https://doi.org/10.3390/su152015034} & Niedrig & Betrachtet Methoden zur nachhaltigen und hochverfügbaren Datenmigration, insbesondere durch ein hybrides Layering-Framework im Kontext von Data Management und Data Validation; adressiert primär Herausforderungen und Optimierungsstrategien im Bereich Datenkonsistenz, Datenintegrität und Verfügbarkeitsmanagement bei Migration und Transformation—relevant für unternehmensweite IT-Systeme und Logistikdaten; explizite Bezüge zu \ac{SSI}, Blockchain-Technologien, \ac{PQC} oder besonderen Anforderungen kritischer Infrastrukturen fehlen vollständig; liefert wertvolle Erkenntnisse zur Bewertung und Ausgestaltung von Datenmigrationsprozessen, bleibt jedoch hinsichtlich der vier Domänen der Masterarbeit methodisch und inhaltlich unberührt. \\
\midrule
41 & Trautman, L. J., Shackelford, S., Elzweig, B., \& Ormerod, P. (2024). Understanding Cyber Risk: Unpacking and Responding to Cyber Threats Facing the Public and Private Sectors. University of Miami Law Review, 78(3), 840–916. \url{https://repository.law.miami.edu/umlr/vol78/iss3/5/} & Niedrig & Umfassende Analyse aktueller Cyberbedrohungen und deren Auswirkungen auf öffentliche und private Sektoren, mit Fokus auf Angriffsszenarien (u.a. Ransomware, Cyberwarfare, Datenlecks), regulatorische und unternehmensbezogene Steuerungsmechanismen sowie das Zusammenspiel von Recht, Unternehmensführung und geopolitischen Risiken; adressiert zentrale Aspekte der Cyber-Risikobewertung und Reaktion auf digitale Angriffe, insbesondere aus administrativer und juristischer Perspektive; explizite Bezüge zu \ac{SSI}, Blockchain-Technologien, \ac{PQC} sowie spezifische Schutzmaßnahmen für \ac{KRITIS} fehlen; bietet wertvolle konzeptionelle Grundlagen im Bereich Cybersicherheit, Governance und Compliance, bleibt jedoch hinsichtlich der vier Schwerpunktdomänen der Masterarbeit in methodischer und technologischer Tiefe eingeschränkt. \\
\midrule
42 & Hendaoui, F., Ferchichi, A., Trabelsi, L., Meddeb, R., Ahmed, R., \& Khelifi, M. K. (2024). Advances in deep learning intrusion detection over encrypted data with privacy preservation: a systematic review. Cluster Computing, 27(7), 8683–8724. \url{https://doi.org/10.1007/s10586-024-04424-4} & Niedrig & Systematische Analyse der Fortschritte im Bereich Deep-Learning-basierter Intrusion Detection über verschlüsselte Daten mit Fokus auf Privacy-Preservation; behandelt innovative Ansätze zur Anomalieerkennung in verschlüsselten Datenströmen durch tiefe neuronale Netze, ohne auf Datenentschlüsselung angewiesen zu sein; explizite Bezüge zu \ac{SSI}, Blockchain-Technologien, \ac{PQC} oder dem Schutz kritischer Infrastrukturen fehlen; bietet wertvolle methodische und technologische Impulse zur sicheren Datenverarbeitung und Angriffserkennung in datenschutzorientierten Systemen, bleibt aber hinsichtlich der vier Kerndomänen der Masterarbeit überwiegend auf den Bereich Deep Learning und Privacy-Preserving IDS fokussiert \\
\midrule
43 & Akartuna, E. A., Johnson, S. D., \& Thornton, A. E. (2023). The money laundering and terrorist financing risks of new and disruptive technologies: a futures-oriented scoping review: The money laundering and terrorist financing risks of new and disruptive technologies: a futures-oriented scoping review. Security Journal, 36(4), 615–650. \url{https://doi.org/10.1057/s41284-022-00356-z} & Niedrig & Systematische Analyse von Geldwäsche- und Terrorismusfinanzierungsrisiken im Zusammenhang mit neuen und disruptiven Technologien, insbesondere Distributed-Ledger-Technologien (inkl. Kryptowährungen), neue Zahlungswege und FinTech; behandelt umfassend die Risiken, Methoden sowie betroffene Akteure und skizziert daraus resultierende Trends und politische Implikationen – mit klarem Bezug zur Domäne \enquote{Blockchain} und angrenzender regulatorischer Herausforderungen; explizite Vertiefungen zu \ac{SSI} oder \ac{PQC} fehlen, ebenso eine gezielte Betrachtung kritischer Infrastrukturen im engeren technischen Sinne; liefert wichtige Einblicke für die Risiko- und Bedrohungsanalyse im Kontext innovativer Finanztechnologien, bleibt aber hinsichtlich der methodischen Tiefe und direkten Anwendbarkeit für die vier Domänen der Masterarbeit beschränkt. \\
\midrule
44 & Zboril, M., \& Svatá, V. (2025). Performance comparison of cloud virtual machines. Journal of Systems \& Information Technology, 27(2), 197–213. \url{https://doi.org/10.1108/JSIT-02-2022-0040} & Niedrig & Thematischer Fokus liegt auf der vergleichenden Performancemessung von Cloud-basierten virtuellen Maschinen (VMs) bei AWS, Microsoft Azure und Google Cloud Platform, basierend auf Benchmark-Tests unter Linux; adressiert ausschließlich Infrastruktur- und Leistungsaspekte von Cloud-Diensten sowie Auswahlkriterien für IT-Betriebsmodelle; keine Verbindung zu den vier Domänen der Masterarbeit, da keine sicherheitsrelevanten, kryptografischen oder identitätsbezogenen Aspekte behandelt werden; relevante Erkenntnisse für Cloud-Infrastrukturmanagement und Benchmarking, jedoch methodisch und inhaltlich außerhalb des Kernbereichs der Masterarbeit. \\
\midrule
45 & Radanliev, P. (2024). The rise and fall of cryptocurrencies: defining the economic and social values of blockchain technologies, assessing the opportunities, and defining the financial and cybersecurity risks of the Metaverse. Financial Innovation, 10, 1–34. \url{https://doi.org/10.1186/s40854-023-00537-8} & Niedrig & Umfassende Analyse der wirtschaftlichen, sozialen und technologischen Aspekte von Blockchain-Technologien, insbesondere im Kontext von Kryptowährungen und deren Rolle im Metaverse; untersucht wirtschaftliche Chancen, Investitionsstrategien und Cybersecurity-Risiken mit interdisziplinärem Ansatz, inklusive Risikobewertung und maschinellem Lernen im Finanzsektor; explizite Bezüge zu \ac{SSI}, \ac{PQC} und dem Schutz kritischer Infrastrukturen fehlen, ebenso eine methodische oder technologische Vertiefung zu innovativen Identitäts- oder Kryptografielösungen—fokussiert primär auf ökonomische und anwendungsbezogene Fragestellungen der Blockchain im Finanz- und Metaverse-Umfeld. \\
\midrule
46 & Bunescu, L., \& Vârtei, A. M. (2024). Modern finance through quantum computing—A systematic literature review. PLoS ONE, 19(7), 1–22. \url{https://doi.org/10.1371/journal.pone.0304317} & Niedrig & Systematische Analyse des Einsatzes von Quantencomputing im Finanzsektor, mit Fokus auf Simulation, Optimierung und maschinelles Lernen; adressiert Kernaspekte der \ac{PQC} im Hinblick auf die transformative Wirkung quantenbasierter Technologien, ohne dabei explizit auf Blockchain-Technologien oder \ac{SSI} einzugehen; liefert wertvolle Einblicke in die methodische und anwendungsbezogene Entwicklung von Quantum Finance, bleibt jedoch hinsichtlich der vier thematischen Domänen der Masterarbeit ( \ac{SSI}, Blockchain, \ac{PQC}, \ac{KRITIS}) primär auf Finanzanwendungen und damit nur partiell anschlussfähig. \\
\midrule
47 & Alzoubi, Y. I., Mishra, A., \& Topcu, A. E. (2024). Research trends in deep learning and machine learning for cloud computing security. Artificial Intelligence Review: An International Science and Engineering Journal, 57(5). \url{https://doi.org/10.1007/s10462-024-10776-5} & Niedrig & Fokussiert auf den Einsatz von Deep-Learning- und Machine-Learning-Technologien zur Identifikation und Bewältigung von Cloud-Sicherheitsbedrohungen; adressiert zentrale Herausforderungen wie Anomalieerkennung, Security Automation und die Integration neuer Technologien, ohne jedoch explizit \ac{SSI}, Blockchain-Ansätze oder \ac{PQC} systematisch einzubinden; hebt methodische, datenschutzbezogene und regulatorische Fragestellungen hervor, die für den Schutz kritischer Infrastrukturen relevant sind, bleibt jedoch bezüglich der vier Kernbereiche der Masterarbeit hauptsächlich auf Cloud Security und AI-gestützte Verfahren konzentriert und bietet nur indirekte Anschlussmöglichkeiten für innovative Kryptografie- oder Identitätslösungen \\
\midrule
48 & Williamson, S. M., \& Prybutok, V. (2024). Balancing Privacy and Progress: A Review of Privacy Challenges, Systemic Oversight, and Patient Perceptions in AI-Driven Healthcare. Applied Sciences (2076-3417), 14(2), 675. \url{https://doi.org/10.3390/app14020675} & Niedrig & Kritische Analyse von Datenschutz-, Ethik- und Compliance-Herausforderungen in AI-gestützten Gesundheitssystemen, mit Fokus auf Differential Privacy und patientenzentrierte Datenverarbeitung; adressiert relevante technologische Ansätze wie Verschlüsselung und Differential Privacy sowie organisatorische und regulatorische Rahmenbedingungen—beinhaltet zudem die Herausforderungen bei der Integration von Blockchain-Technologien im healthcare-spezifischen Kontext und deren Vereinbarkeit mit der DSGVO, wodurch ein übergreifender Bezug zur Domäne Blockchain gegeben ist; \ac{SSI} und \ac{PQC} werden nicht explizit behandelt, ebenso steht die Anbindung an \ac{KRITIS} außerhalb des engeren Fokus; bietet wertvolle Erkenntnisse zu datenschutzgerechter Systemgestaltung und Interdisziplinarität im Gesundheitswesen, bleibt aber hinsichtlich methodischer Tiefe und anwendungsbezogener Integration in allen vier Kernbereichen der Masterarbeit überblicksartig und konzeptionell. \\
\midrule
49 & Tukur, M., Schneider, J., Househ, M., Dokoro, A. H., Ismail, U. I., Dawaki, M., \& Agus, M. (2023). The metaverse digital environments: a scoping review of the challenges, privacy and security issues. Frontiers in Big Data, 1–25. \url{https://doi.org/10.3389/fdata.2023.1301812} & Niedrig & Umfassende Übersicht zu Herausforderungen, Datenschutz- und Sicherheitsfragen bei der Entwicklung und Implementierung von Metaverse-Umgebungen, insbesondere infolge der pandemiebedingten Digitalisierungsschübe; adressiert wirtschaftliche, technische, ethische und soziale Herausforderungen, darunter Hard- und Softwarekosten, digitale Ungleichheit sowie Regelwerks- und Datenmanagement-Fragen; konkrete Analyse und Klassifikation von Policy-, Privacy- und Security-Problemen, mit Fokus auf privatsphärenbezogene Risiken und Governance-Anforderungen im Metaverse. Explizite Bezüge zu \ac{SSI}, Blockchain-Technologien und \ac{PQC} fehlen; \ac{KRITIS} werden nur implizit durch den Verweis auf digitale Spaltungen und gesellschaftliche Implikationen berührt. \\
\midrule
50 & Hanafi, B., Ali, M., \& Singh, D. (2025). Quantum algorithms for enhanced educational technologies. Discover Education, 4(1), 1–33. \url{https://doi.org/10.1007/s44217-025-00400-1} & Niedrig & Fokus auf die Potenziale und Herausforderungen von Quantum Computing und Quantenkryptografie in der Bildungsbranche, z. B. für personalisiertes Lernen und sichere Datenübertragung; Bezug zu \ac{PQC} nur anwendungsbezogen im Bildungskontext, ohne technische Tiefe oder Bezug zu \ac{SSI}, Blockchain oder \ac{KRITIS}; aus Sicht der vier Masterarbeitsdomänen methodisch und thematisch nur sehr eingeschränkt anschlussfähig. \\
\midrule
51 & Pillai, S. E. V. S., Nadella, G. S., Meduri, K., Priyadharsini, N. A., Bhuvanesh, A., \& Kumar, D. (2025). A walrus optimization-enhanced long short-term memory model for credit fraud detection in banking. International Journal of Information Technology: An Official Journal of Bharati Vidyapeeth’s Institute of Computer Applications and Management, 1–17. \url{https://doi.org/10.1007/s41870-025-02574-1} & Niedrig & Beschreibung einer innovativen Framework-Kombination aus Autoencoder, Long Short-Term Memory Netzwerken und Walrus Optimization Algorithm zur Verbesserung der Betrugserkennung im Bankensektor; konzentriert sich ausschließlich auf Machine-Learning-gestützte Analyse, Datenvorverarbeitung und Hyperparameteroptimierung zur Echtzeit-Erkennung betrügerischer Transaktionen in großen Datenmengen; keinerlei Behandlung oder Integration der vier Domänen der Masterarbeit; relevante methodische Beiträge beschränken sich auf KI-basierte Fraud Detection, ohne Anschlusspunkte zu den Kernthemen der Masterarbeit. \\
\midrule
52 & Priya, S. S., Vijayabhasker, R., \& Rajaram, A. (2025). Advanced Security and Efficiency Framework for Mobile Ad-Hoc Networks Using Adaptive Clustering and Optimization Techniques. Journal of Electrical Engineering \& Technology (19750102), 20(3), 1815–1826. \url{https://doi.org/10.1007/s42835-024-02119-9} & Niedrig & Fokus auf ein innovatives Sicherheits- und Effizienz-Framework für Mobile Ad-Hoc Networks (MANETs) durch adaptive Clusterbildung, AI-unterstützte Vertrauensbewertung und quantenresistente PUF-Authentifizierung; explizite Relevanz für \ac{PQC} durch QR-PUF-Komponente; \ac{SSI} und Blockchain werden nicht behandelt, ebenso fehlt eine gezielte Betrachtung kritischer Infrastrukturen; für die vier Kerndomänen der Masterarbeit somit vor allem im Kontext quantensicherer Validierung/mobiler Netzwerksicherheit anschlussfähig, ansonsten methodisch und domänenspezifisch eingeschränkt. \\
\midrule
53 & Berkani, A.-S., Moumen, H., Benharzallah, S., Yahiaoui, S., \& Bounceur, A. (2024). Blockchain Use Cases in the Sports Industry: A Systematic Review. International Journal of Networked \& Distributed Computing, 12(1), 17–40. \url{https://doi.org/10.1007/s44227-024-00022-3} & Niedrig & Fokus auf branchenspezifische Anwendungen der Blockchain-Technologie im Sportsektor (Athleten-Datenmanagement, Fandaten, NFT-Sammlerstücke); methodische und technologische Vertiefung im Hinblick auf \ac{SSI}, \ac{PQC} oder \ac{KRITIS} fehlt; relevante Erkenntnisse nur für den Bereich Blockchain-Anwendungsfälle in Sport und Entertainment, für die vier Domänen der Masterarbeit jedoch insgesamt wenig anschlussfähig. \\
\midrule
54 & 2023 PNS Annual Meeting - Copenhagen, 17-20 June 2023. (2023). Journal of the Peripheral Nervous System: JPNS, 28 Suppl 4, S3–S254. \url{https://doi.org/10.1111/jns.12585} & Niedrig & Konferenzband ohne Bezug zu \ac{SSI}, Blockchain oder \ac{PQC}. \\
\midrule
55 & Posters. (2017). FEBS Journal, 284, 102–403. \url{https://doi.org/10.1111/febs.14174} & Niedrig & Posterband, kein Bezug zu \ac{SSI}, Blockchain oder \ac{PQC}. \\
\midrule
56 & Annotated Listing of New Books. (2024). Journal of Economic Literature, 62(4), 1696–1750. \url{https://doi.org/10.1257/jel.62.4.1696} & Niedrig & Buchliste, kein Bezug zum Thema. \\
\midrule
57 & PNS Abstracts 2023. (2023). Journal of the Peripheral Nervous System, 28, S3–S254. \url{https://doi.org/10.1111/jns.12585} & Niedrig & Abstractband, kein Bezug zu SSI, Blockchain oder \ac{PQC}. \\
\midrule
58 & Elendu, C., Omeludike, E. K., Oloyede, P. O., Obidigbo, B. T., \& Omeludike, J. C. (2024). Legal implications for clinicians in cybersecurity incidents: A review. Medicine, 103(39), 1–26. \url{https://doi.org/10.1097/MD.0000000000039887} & Niedrig & Fokus auf die rechtlichen Implikationen von Cybersecurity-Vorfällen im Gesundheitswesen, insbesondere für klinisch tätige Personen; Betrachtung technologischer Entwicklungen (u. a. künstliche Intelligenz und Quantencomputing) sowie internationaler regulatorischer Unterschiede; praxisnahe Empfehlungen und Fallstudien zu Cybersecurity-Management, ethischen und juristischen Aspekten im Gesundheitssektor; keine explizite Behandlung von \ac{SSI}, Blockchain oder \ac{PQC}, \ac{KRITIS} werden durch den Gesundheitsbereich berührt, Schwerpunkt liegt jedoch auf juristischen und ethischen Fragestellungen. \\
\midrule
59 & Albshaier, L., Almarri, S., \& Hafizur Rahman, M. M. (2024). A Review of Blockchain’s Role in E-Commerce Transactions: Open Challenges, and Future Research Directions. Computers (2073-431X), 13(1), 27. \url{https://doi.org/10.3390/computers13010027} & Niedrig & Fokus auf die Anwendung von Blockchain-Technologien zur Verbesserung von Sicherheit, Transparenz und Betrugserkennung in E-Commerce-Transaktionen; betont die Rolle verteilter, unveränderlicher digitaler Ledger für den Schutz sensibler Kundendaten und die Stärkung des Vertrauens in Online-Plattformen; adressiert die Domäne „Blockchain“ vorrangig, ohne explizite Bezüge zu \ac{SSI} oder \ac{PQC}; \ac{KRITIS} werden nicht thematisiert, da der Schwerpunkt auf E-Commerce liegt; methodische und technologische Tiefe für die vier Domänen der Masterarbeit ist auf Blockchain-Anwendungen im Bereich E-Commerce beschränkt. \\
\midrule
60 & Reddy, R. C., Bhattacharjee, B., Mishra, D., \& Mandal, A. (2022). A systematic literature review towards a conceptual framework for enablers and barriers of an enterprise data science strategy. Information Systems \& e-Business Management, 20(1), 223–255. \url{https://doi.org/10.1007/s10257-022-00550-x} & Niedrig & Fokus liegt auf der systematischen Analyse von Erfolgsfaktoren und Hindernissen bei der unternehmensweiten Einführung von Data-Science-Strategien; methodische Entwicklung eines Enabler-Barrier-Frameworks für die erfolgreiche Umsetzung datengetriebener Projekte in Unternehmen; adressiert dabei organisatorische, technologische und strategische Aspekte der digitalen Transformation im breiten Kontext, jedoch ohne explizite Behandlung oder Integration von \ac{SSI}, Blockchain-Technologien, \ac{PQC} oder spezifischen Schutzanforderungen kritischer Infrastrukturen; liefert wertvolle Erkenntnisse zur Implementierung von Data Science im Unternehmensumfeld, ist für die vier Domänen der Masterarbeit methodisch und thematisch jedoch nicht anschlussfähig. \\
\midrule
61 & Kumar, Y., Marchena, J., Awlla, A. H., Li, J. J., \& Abdalla, H. B. (2024). The AI-Powered Evolution of Big Data. Applied Sciences (2076-3417), 14(22), 10176. \url{https://doi.org/10.3390/app142210176} & Niedrig & Fokus auf die Weiterentwicklung von Big-Data-Analyse und Management durch künstliche Intelligenz, mit Betonung neuer Rahmenwerke für die Charakterisierung und Handhabung großer, komplexer Datensätze. Der Beitrag stellt innovative AI-gestützte Tools (wie RAG-basierte Analyse-Bots/ChatGPT) zur Verbesserung der Datenanalyse vor und hebt methodologische Fortschritte im Bereich datengetriebene Entscheidungsunterstützung hervor. Keine explizite Behandlung oder Integration von \ac{SSI}, Blockchain-Technologien, \ac{PQC} oder besonderen Anforderungen kritischer Infrastrukturen; methodische und technologische Beiträge beschränken sich auf Big-Data-Management und AI-basierte Analytics, ohne Verknüpfung zu den vier zentralen Domänen der Masterarbeit. \\
\end{longtable}