\section{Systematische Literaturrecherche}
\label{sec:Anhang_Dokumentation der ersten Iteration der systematischen Literaturrecherche (Exposé)}

\subsection{Erste Iteration (Exposé)}

\fixme{Texte anpassen? + Kapitelüberschriften ergänzen? Anhang 1.1 etc.?}

Die systematische Literaturrecherche zum Stand der Forschung orientiert sich am \enquote{iterative Review-Ansatz} nach \textcite[S. 208--209]{brocke_StandingShouldersGiantsChallengesRecommendationsLiteratureSearchInformationSystemsResearch_2015}, der mit einer initialen Recherche startet und sich iterativ vertieft. Für die Struktur und Dokumentation sind ausgewählte Methoden der \ac{PRISMA} 2020 Richtlinien zugrunde gelegt (\autoref{tab:Ausgewählte Methoden der PRISMA 2020 Richtlinien}). \ac{PRISMA} gewährleistet hierbei einen transparenten, reproduzierbaren Prozess und verbessert die Berichtqualität \parencite[S. 1, 6]{page_PRISMA2020Statementupdatedguidelinereportingsystematicreviews_2021}.

\begin{longtable}{L{0.3\textwidth}L{0.7\textwidth}}
    \caption{Ausgewählte Methoden der PRISMA 2020 Richtlinien}
    \label{tab:Ausgewählte Methoden der PRISMA 2020 Richtlinien} \\
    \toprule
    \textbf{Methode} & \textbf{Beschreibung} \\
    \midrule
    \endfirsthead
    \multicolumn{2}{l}{\textit{Tabelle \thetable\ (Fortsetzung)}} \\
    \toprule
    \textbf{Methode} & \textbf{Beschreibung} \\
    \midrule
    \endhead
    \midrule
    \multicolumn{2}{r}{\textit{Fortsetzung auf nächster Seite}} \\
    \endfoot
    \bottomrule
    \multicolumn{2}{p{\linewidth}}{\textit{Anmerkung.} In Anlehnung an \textcite[S. 4]{page_PRISMA2020Statementupdatedguidelinereportingsystematicreviews_2021}.} \\
    \endlastfoot
    Ein- und Ausschlusskriterien & Ein- und Ausschlusskriterien für die Überprüfung. \\
    \midrule
    Suchstrategie & Vollständigen Suchstrategie für alle Datenbanken, Websites einschließlich aller verwendeten Filter. \\
    \midrule
    Selektionsprozess & Methoden an, die verwendet wurden, um zu entscheiden, ob eine Studie die Einschlusskriterien der Überprüfung erfüllt. \\
\end{longtable}


\begin{figure}[H]
    \centering
    \includegraphics[width=\paperwidth]{PRISMA_2020_flow_diagram_new_SRs_v1_ITERATION_1.png}
    \caption{PRISMA 2020 Flussdiagramm - erste Iteration}
    \begin{flushleft}
    \textit{Anmerkung.} In Anlehnung an \textcite[S. 5]{page_PRISMA2020Statementupdatedguidelinereportingsystematicreviews_2021}.
    \end{flushleft}
    \label{fig:PRISMA_Flussdiagramm_iteration1}
\end{figure}

\fixme{\autoref{fig:PRISMA_Flussdiagramm_iteration1} beschreibt den Prozess der systematischen Literaturrecherche in der ersten Iteration.}

\pagebreak

Die in \autoref{tab:einausschlusskriterien} dargestellten Ein- und Ausschlusskriterien gewährleisten eine transparente, nachvollziehbare und zielgerichtete Auswahl relevanter wissenschaftlicher Quellen.

\begin{longtable}{L{3cm}L{4cm}L{4cm}L{3cm}}
    \caption{Ein- und Ausschlusskriterien für die systematische Literaturrecherche}
    \label{tab:einausschlusskriterien} \\
    \toprule
    \textbf{Kategorie} & \textbf{Einschluss} & \textbf{Ausschluss} & \textbf{Begründung} \\
    \midrule
    \endfirsthead
    \multicolumn{4}{l}{\textit{Tabelle \thetable\ (Fortsetzung)}} \\
    \toprule
    \textbf{Kategorie} & \textbf{Einschluss} & \textbf{Ausschluss} & \textbf{Begründung} \\
    \midrule
    \endhead
    \midrule
    \multicolumn{4}{r}{\textit{Fortsetzung auf nächster Seite}} \\
    \endfoot
    \bottomrule
    \multicolumn{4}{p{\linewidth}}{\textit{Anmerkung.} Eigene Darstellung.} \\
    \endlastfoot
    Thematischer Fokus &
    \ac{SSI} und dezentrale Identitätslösungen;
    Blockchain-basierte Identitätsmanagementsysteme;
    \ac{PQC} und quantensichere Algorithmen;
    Sicherheit und Compliance in \ac{KRITIS};
    \ac{DSR}-Methodik in IT/Informationssystemen;
    Kryptoagilität und kryptografische Migration & 
    Identitätsmanagement ohne Bezug zu \ac{SSI} oder Blockchain;
    Klassische \ac{PKI} ohne \ac{PQC}-Bezug;
    Kryptografie ohne Post-Quantum-Relevanz;
    Arbeiten ohne Bezug zu \ac{KRITIS} oder ohne sicherheitskritischen Kontext;
    Nicht-\ac{DSR}-basierte Entwicklungsansätze & 
    Fokussierung auf die Forschungsfragen und relevante technologische, methodische und regulatorische Aspekte. \\
    \midrule
    Zeitrahmen & 2015 bis heute & Vor 2015 & Berücksichtigung aktueller technologischer Entwicklungen (Blockchain, \ac{PQC}, \ac{SSI}) und regulatorischer Anforderungen. \\
    \midrule
    Publikationstypen & 
    Peer-reviewed Journalartikel;
    Konferenzbeiträge anerkannter Fachgesellschaften (z. B. IEEE, ACM, IFIP); Preprints;
    Offizielle Standards und Empfehlungen (z. B. \ac{NIST}, W3C, \ac{BSI});
    Whitepaper etablierter Organisationen;
    Dissertationen und anerkannte Fachbücher & 
    Blogposts, Forenbeiträge, Marketingmaterial;
    Populärwissenschaftliche Artikel ohne wissenschaftliche Fundierung;
    Unveröffentlichte Manuskripte ohne Peer-Review;
    Seminar- und Abschlussarbeiten ohne wissenschaftliche Begutachtung & 
    Sicherstellung wissenschaftlicher Qualität, Nachvollziehbarkeit und Relevanz der Quellen für die Masterarbeit; Da das Forschungsthema aktuell noch sehr neu ist und die einschlägige Fachliteratur teilweise noch nicht den Peer-Review-Prozess durchlaufen hat, werden auch Preprints in die Analyse einbezogen. Preprints ermöglichen eine zeitnahe Verfügbarkeit aktueller Forschungsergebnisse, was insbesondere bei diesem innovativen von großer Bedeutung ist. \\
    \midrule
    Sprache & Deutsch; Englisch & Andere Sprachen als Deutsch und Englisch & Gewährleistung der Verständlichkeit und Zugänglichkeit für den deutsch- und englischsprachigen Forschungskontext. \\
    \midrule
    Zugänglichkeit & Verfügbare Volltexte & Nicht verfügbare Volltexte & Ermöglichung einer gründlichen Analyse und Bewertung der Inhalte. \\
\end{longtable}

\autoref{tab:suchstrategie} stellt ausgewählte methodische Schritte zur Entwicklung einer Suchstrategie nach \textcite[S. 532]{bramer_SystematicApproachSearchingefficientcompletemethoddevelopliteraturesearches_2018} dar, an denen sich diese Seminararbeit orientiert.

\begin{longtable}{L{0.1\textwidth}L{0.9\textwidth}}
    \caption{Überblick über die Entwicklung der Suchstrategie}
    \label{tab:suchstrategie} \\
    \toprule
    \textbf{Schritt} & \textbf{Beschreibung} \\
    \midrule
    \endfirsthead
    \multicolumn{2}{l}{\textit{Tabelle \thetable\ (Fortsetzung)}} \\
    \toprule
    \textbf{Methode} & \textbf{Beschreibung} \\
    \midrule
    \endhead
    \midrule
    \multicolumn{2}{r}{\textit{Fortsetzung auf nächster Seite}} \\
    \endfoot
    \bottomrule
    \multicolumn{2}{p{\linewidth}}{\textit{Anmerkung.} In Anlehnung an \textcite[S. 532]{bramer_SystematicApproachSearchingefficientcompletemethoddevelopliteraturesearches_2018}.} \\
    \endlastfoot
    1  & Identifikation relevanter Schlüsselkonzepte \\
    \midrule
    2  & Identifikation relevanter Keywords \\
    \midrule 
    3  & Erstellung einer strukturierten Suchanfrage mit Booleschen Operatoren \\
    \midrule
    4  & Auswahl geeigneter Datenbanken \\
    \midrule
    5  & Übersetzung der Suchanfrage für verschiedene Datenbanken \\
\end{longtable}

\paragraph*{Identifikation relevanter Schlüsselkonzepte}

Im ersten Schritt wird eine präzise Identifikation und Abgrenzung zentraler Schlüsselkonzepte vorgenommen (\autoref{tab:Abgrenzung zentraler Schlüsselkonzepte}).

\begin{longtable}{L{4cm}L{10cm}}
    \caption{Abgrenzung zentraler Schlüsselkonzepte}
    \label{tab:Abgrenzung zentraler Schlüsselkonzepte} \\
    \toprule
    \textbf{Schlüsselkonzept} & \textbf{Erläuterung} \\
    \midrule
    \endfirsthead
    \multicolumn{2}{l}{\textit{Tabelle \thetable\ (Fortsetzung)}} \\
    \toprule
    \textbf{Schlüsselkonzept} & \textbf{Erläuterung} \\
    \midrule
    \endhead
    \midrule
    \multicolumn{2}{r}{\textit{Fortsetzung auf nächster Seite}} \\
    \endfoot
    \bottomrule
    \multicolumn{2}{p{\linewidth}}{\textit{Anmerkung.} Basierend auf \textcite[S. 2]{solavagione_TransitionSelfSovereignIdentityPostQuantumCryptography_2025}, \textcite{nationalinstituteofstandardsandtechnologyus_ModulelatticebasedKeyencapsulationMechanismstandard_2024,nationalinstituteofstandardsandtechnologyus_ModulelatticebasedDigitalSignaturestandard_2024,nationalinstituteofstandardsandtechnologyus_StatelessHashbasedDigitalsignaturestandard_2024}, \textcite{bundesministeriumderjustiz_GesetzUeberBundesamtfuerSicherheitInformationstechnikBSIGesetzBSIG_2009}, \textcite{hevner_DesignScienceInformationsystemsresearch_2004}.} \\
    \endlastfoot
    Self-Sovereign Identity & 
    Das Paradigma der selbstbestimmten digitalen Identität, das Nutzenden die Kontrolle über ihre Identitätsdaten und deren Weitergabe ermöglicht, basierend auf dezentralen Technologien wie Blockchain und interoperablen Standards wie \ac{DID} ,\ac{VC} und \ac{VP} \parencite[S. 2]{solavagione_TransitionSelfSovereignIdentityPostQuantumCryptography_2025}. \\
    \midrule
    Blockchain-Technologie & 
    Die Nutzung von \ac{DLT} zur Sicherstellung von Integrität, Transparenz und Manipulationssicherheit im Identitätsmanagement, insbesondere im Kontext von \ac{SSI}-Systemen \parencite[S. 2]{solavagione_TransitionSelfSovereignIdentityPostQuantumCryptography_2025}. \\
    \midrule
    Post-Quantum Kryptografie & 
    Kryptografische Verfahren, die auch gegen Angriffe durch leistungsfähige Quantencomputer resistent sind, einschließlich aktueller Standardisierungsansätze (z. B. \ac{NIST} FIPS 203, 204, 205) \parencite{nationalinstituteofstandardsandtechnologyus_ModulelatticebasedKeyencapsulationMechanismstandard_2024,nationalinstituteofstandardsandtechnologyus_ModulelatticebasedDigitalSignaturestandard_2024,nationalinstituteofstandardsandtechnologyus_StatelessHashbasedDigitalsignaturestandard_2024} und Empfehlungen zur kryptoagilen Systemgestaltung. \\
    \midrule
    Kritische Infrastrukturen & 
    Sektoren und Systeme, deren Funktionsfähigkeit essenziell für das Gemeinwesen ist und die daher besonders hohe Anforderungen an Sicherheit, Compliance und Resilienz stellen \parencite{bundesministeriumderjustiz_GesetzUeberBundesamtfuerSicherheitInformationstechnikBSIGesetzBSIG_2009}. \\
    \midrule
    Design Science Research & 
    Die methodische Grundlage zur systematischen Entwicklung, Evaluation und wissenschaftlichen Fundierung innovativer IT-Artefakte nach \textcite{hevner_DesignScienceInformationsystemsresearch_2004} im Kontext der genannten Technologien und Anwendungsdomänen. \\
\end{longtable}

\paragraph*{Identifikation relevanter Keywords}

Basierend auf den zuvor definierten Schlüsselkonzepten wurden die wichtigsten Suchbegriffe in Deutsch und Englisch sowie die gängigen Akronyme zusammengestellt (\autoref{tab:keywords der schlüsselkonzepte}), um die technologische, methodische und regulatorische Breite der Recherche optimal abzudecken.

\begin{longtable}{L{4cm}L{10cm}}
    \caption{Keywords der Schlüsselkonzepte}
    \label{tab:keywords der schlüsselkonzepte} \\
    \toprule
    \textbf{Schlüsselkonzept} & \textbf{Keywords} \\
    \midrule
    \endfirsthead
    \multicolumn{2}{l}{\textit{Tabelle \thetable\ (Fortsetzung)}} \\
    \toprule
    \textbf{Schlüsselkonzept} & \textbf{Keywords} \\
    \midrule
    \endhead
    \midrule
    \multicolumn{2}{r}{\textit{Fortsetzung auf nächster Seite}} \\
    \endfoot
    \bottomrule
    \multicolumn{2}{p{\linewidth}}{\textit{Anmerkung.} Eigene Darstellung.} \\
    \endlastfoot
    Self-Sovereign Identity &
    Self-Sovereign Identity, \ac{SSI}, dezentrale Identität, digitale Identität, decentralized identity, user-controlled identity, identity wallet, Verifiable Credentials, \ac{VC}, Decentralized Identifiers, \ac{DID}, Identity Management, \ac{IdM}, Identity Access Management, \ac{IAM} \\
    \midrule
    Blockchain-Technologie &
    Blockchain, Distributed Ledger Technology, \ac{DLT}, Distributed Ledger, Smart Contract, Ethereum, Hyperledger, IOTA, Permissioned Ledger, Consensus Mechanism, On-Chain Identity \\
    \midrule
    Post-Quantum Kryptografie &
    Post-Quantum Cryptography, \ac{PQC}, quantensichere Verschlüsselung, quantum-resistant cryptography, Lattice-based Cryptography, Hash-based Signature, Code-based Cryptography, Multivariate Cryptography, Module-Lattice-based Digital Signature, ML-DSA, Module-Lattice-based Key Encapsulation Mechanism, ML-KEM, Stateless Hash-based Digital Signature, SLH-DSA, CRYSTALS-Dilithium, SPHINCS+, Kryptoagilität, Cryptographic Agility, Algorithm Agility, Migration Strategy, Cryptographic Migration, Flexible Cryptography Update \\ 
    \midrule
    Kritische Infrastrukturen &
    Kritische Infrastrukturen, KRITIS, Critical Infrastructure Protection, CIP, sector-specific security requirements, Compliance, Privacy by Design, Privacy by Default, Data Protection, Regulatory Requirements, Bundesamt für Sicherheit in der Informationstechnik, BSI, European Union Agency for Cybersecurity, ENISA, eIDAS \\ 
    \midrule
    Design Science Research &
    Design Science Research, \ac{DSR}, Design Science Research Methodology, DSRM, Artefact Development, Evaluation of Artefacts, Method-driven Development, Iterative Design Process, Research Methodology, Framework for Evaluation in Design Science, \ac{FEDS}, Preferred Reporting Items for Systematic Reviews and Meta-Analyses, \ac{PRISMA} \\
\end{longtable}

\pagebreak

\paragraph*{Erstellung einer strukturierten Suchanfrage mit Booleschen Operatoren}

Listing~\ref{lst:boolesche_suchanfrage} stellt eine strukturierte Suchstrategie mit Booleschen Operatoren auf Basis der identifizierten Keywords dar. Die Suchanfrage verbindet zentrale Schlüsselkonzepte und nutzt gezielt Synonyme und Akronyme zur Abdeckung verschiedener Terminologien und Schreibweisen.
\newline

\refstepcounter{manualListingCounter}
\label{lst:boolesche_suchanfrage}
\begin{lstlisting}[
caption={Listing \arabic{lstlisting}: Strukturierte Suchanfrage mit Booleschen Operatoren},
basicstyle=\small\ttfamily,
breaklines=true,
frame=single,
language=SQL
]
(
(
"self-sovereign identity" OR SSI OR "decentralized identity" OR "dezentrale Identität" OR "digitale Identität" OR "user-controlled identity" OR "identity wallet" OR "verifiable credentials" OR VC OR "decentralized identifiers" OR DID OR "identity management" OR IdM OR "identity access management" OR IAM
)
AND
(
blockchain OR "distributed ledger technology" OR DLT OR "distributed ledger" OR "smart contract" OR Ethereum OR Hyperledger OR IOTA OR "permissioned ledger" OR "consensus mechanism" OR "on-chain identity"
)
AND
(
"post-quantum cryptography" OR PQC OR "quantensichere Verschlüsselung" OR "quantum-resistant cryptography" OR "lattice-based cryptography" OR "hash-based signature" OR "code-based cryptography" OR "multivariate cryptography" OR "module-lattice-based digital signature" OR ML-DSA OR "module-lattice-based key encapsulation mechanism" OR ML-KEM OR "stateless hash-based digital signature" OR SLH-DSA OR CRYSTALS-Dilithium OR SPHINCS+ OR kryptoagilität OR "cryptographic agility" OR "algorithm agility" OR "migration strategy" OR "cryptographic migration" OR "flexible cryptography update"
)
AND
(
"kritische Infrastrukturen" OR KRITIS OR "critical infrastructure protection" OR CIP OR "sector-specific security requirements" OR compliance OR "privacy by design" OR "privacy by default" OR "data protection" OR "regulatory requirements" OR "bundesamt für sicherheit in der informationstechnik" OR BSI OR "european union agency for cybersecurity" OR ENISA OR eIDAS
)
AND
(
"design science research" OR DSR OR "design science research methodology" OR DSRM OR "artefact development" OR "evaluation of artefacts" OR "method-driven development" OR "iterative design process" OR "research methodology" OR "framework for evaluation in design science" OR FEDS OR "preferred reporting items for systematic reviews and meta-analyses" OR PRISMA
)
)
\end{lstlisting}

\paragraph*{Auswahl einer geeigneten Datenbank}

Die Wahl auf \gls{EBSCO} als Datenbank für die systematische Literaturrecherche resultiert aus der Existenz von Campuslizenzen der FOM für diese Datenbank.

\paragraph*{Übersetzung der Suchanfrage für EBSCO}

Die strukturierten Suchanfrage mit den Booleschen Operatoren kann für
EBSCO ohne Anpassungen übernommen werden. Das Ergebnis der EBSCO Suchanfrage
ist in \autoref{fig:EBSCO Ergebnis} dargestellt. Insgesamt wurden mit dieser Abfrage 61 Quellen identifiziert.

\begin{figure}[H]
    \centering
    \includegraphics[width=\paperwidth, height=\paperheight, keepaspectratio, angle=90]{EBSCO.png}
    \caption{Ergebnis der EBSCO Suchanfrage}
    \begin{flushleft}
    \textit{Anmerkung.} Eigene Darstellung.
    \end{flushleft}
    \label{fig:EBSCO Ergebnis}
\end{figure}

\paragraph*{Bewertung der Ergebnisse}

\autoref{tab:quellenuebersicht} stellt eine Übersicht der Bewertung der 61 identifizierten Quellen dar, welche vollständig in \ref{sec:Bewertung der identifizierten Quellen hinsichtlich ihrer Relevanz} aufzufinden ist.

Die Einstufung basiert auf Titel und Abstract in Bezug auf den thematischen Fokus der Arbeit. Hohe Relevanz erhalten Quellen mit klaren Beiträgen zu \ac{SSI}, \ac{PQC}, \ac{KRITIS} oder dezentralen Identitätsarchitekturen. Mittlere Relevanz wird Arbeiten zugeordnet, die angrenzende Technologien wie Blockchain-Sicherheit im \ac{IoT} oder digitale Forensik behandeln. Niedrige Relevanz erhalten Quellen zu allgemeinen Technologietrends ohne direkten Bezug zum Thema.

\begin{longtable}{L{1.5cm}L{11cm}L{1cm}}
    \caption[]{Übersicht der Bewertung der identifizierten Quellen hinsichtlich ihrer Relevanz}
    \label{tab:quellenuebersicht} \\
    \toprule
    \textbf{Nr.} & \textbf{Quelle} & \textbf{Relevanz} \\
    \midrule
    \endfirsthead
    \multicolumn{3}{l}{\textit{Tabelle \thetable\ (Fortsetzung)}} \\
    \toprule
    \textbf{Nr.} & \textbf{Quelle} & \textbf{Relevanz} \\
    \midrule
    \endhead
    \midrule
    \multicolumn{3}{r}{\textit{Fortsetzung auf nächster Seite}} \\
    \endfoot
    \bottomrule
    \multicolumn{3}{p{\linewidth}}{\textit{Anmerkung.} Basierend auf \autoref{tab:quellenbewertung} und Titel und Abstracts von \textcite{szymanski_QuantumSafeSoftwareDefinedDeterministicInternetThingsIoTHardwareEnforcedCyberSecurityCriticalInfrastructures_2024,nouma_TrustworthyEfficientDigitalTwinsPostQuantumEraHybridHardwareAssistedSignatures_2024,sharif_EIDASRegulationSurveyTechnologicalTrendsEuropeanElectronicIdentitySchemes_2022,alam_PrivatelyGeneratedKeyPairsPostQuantumCryptographyDistributedNetwork_2024,radanliev_ReviewComparisonUSEUUKRegulationsCyberRiskSecurityCurrentBlockchainTechnologies_2023}.} \\
    \endlastfoot
    1 & Szymanski, T. H. (2024). A Quantum-Safe Software-Defined Deterministic Internet of Things (IoT) with Hardware-Enforced Cyber-Security for Critical Infrastructures. Information (2078-2489), 15(4), 173. \url{https://doi.org/10.3390/info15040173} & Hoch \\
    \midrule
    2 & Nouma, S. E., \& Yavuz, A. A. (2024). Trustworthy and Efficient Digital Twins in Post-Quantum Era with Hybrid Hardware-Assisted Signatures. ACM Transactions on Multimedia Computing, Communications \& Applications, 20(6), 1–30. \url{https://doi.org/10.1145/3638250} & Hoch \\
    \midrule
    3 & Sharif, A., Ranzi, M., Carbone, R., Sciarretta, G., Marino, F. A., \& Ranise, S. (2022). The eIDAS Regulation: A Survey of Technological Trends for European Electronic Identity Schemes. Applied Sciences (2076-3417), 12(24), 12679. \url{https://doi.org/10.3390/app122412679} & Hoch \\
    \midrule
    4 & Alam, M., Hoffstein, J., \& Cambou, B. (2024). Privately Generated Key Pairs for Post Quantum Cryptography in a Distributed Network. Applied Sciences (2076-3417), 14(19), 8863. \url{https://doi.org/10.3390/app14198863} & Hoch \\
    \midrule
    5 & Radanliev, P. (2023). Review and Comparison of US, EU, and UK Regulations on Cyber Risk/Security of the Current Blockchain Technologies: Viewpoint from 2023. Review of Socionetwork Strategies, 17(2), 105–129. \url{https://doi.org/10.1007/s12626-023-00139-x} & Hoch \\
    \midrule
    6--38 & Diverse & Mittel  \\
    \midrule
    39--61 & Diverse & Niedrig \\
\end{longtable}

\subsubsection{Ergebnis Iteration 1}
\label{sec:Bewertung der identifizierten Quellen hinsichtlich ihrer Relevanz}

Iteration 1 - Bewertung der identifizierten Quellen hinsichtlich ihrer Relevanz

\begin{longtable}{L{0.5cm}L{4cm}L{1.5cm}L{7cm}}
    \caption[]{Bewertung der identifizierten Quellen hinsichtlich ihrer Relevanz}
    \label{tab:quellenbewertung} \\
    \toprule
    \textbf{Nr.} & \textbf{Quelle} & \textbf{Relevanz} & \textbf{Kommentar} \\
    \midrule
    \endfirsthead
    \multicolumn{4}{l}{\textit{Tabelle \thetable\ (Fortsetzung)}} \\
    \toprule
    \textbf{Nr.} & \textbf{Quelle} & \textbf{Relevanz} & \textbf{Kommentar} \\
    \midrule
    \endhead
    \midrule
    \multicolumn{4}{r}{\textit{Fortsetzung auf nächster Seite}} \\
    \endfoot
    \bottomrule
    \multicolumn{4}{p{\linewidth}}{\textit{Anmerkung.} Basierend auf den Abstracts aller in Spalte zwei unter \enquote{Quelle} aufgeführten Quellenangaben.} \\
    \endlastfoot
1 & Szymanski, T. H. (2024). A Quantum-Safe Software-Defined Deterministic Internet of Things (IoT) with Hardware-Enforced Cyber-Security for Critical Infrastructures. Information (2078-2489), 15(4), 173. \url{https://doi.org/10.3390/info15040173} & Hoch & Fokussiert auf die Entwicklung quantensicherer Kommunikations- und Sicherheitssysteme im Kontext von \ac{KRITIS} und Industrial IoT; adressiert explizit \ac{PQC} durch Einsatz quantensicherer Verschlüsselungsmechanismen und QKD-Netzwerke; behandelt hardwarebasierte Zugriffskontrollen, Zero Trust Architekturen und AI-gestützte Sicherheit, die für Resilienz und Sicherheitsanforderungen in \ac{KRITIS} maßgeblich sind; zwar steht \ac{SSI} und Blockchain-Technologie nicht im Mittelpunkt, jedoch zeigen die vorgestellten innovativen Konzepte und die experimentelle Validierung einen sehr hohen Anwendungs- und Erkenntniswert für den methodischen und technologischen Fortschritt in mindestens zwei zentralen Domänen (\ac{PQC}, \ac{KRITIS}); damit bietet der Beitrag substanzielle Impulse für den Schutz hochsensibler digitaler Infrastrukturen. \\
\midrule
2 & Nouma, S. E., \& Yavuz, A. A. (2024). Trustworthy and Efficient Digital Twins in Post-Quantum Era with Hybrid Hardware-Assisted Signatures. ACM Transactions on Multimedia Computing, Communications \& Applications, 20(6), 1–30. \url{https://doi.org/10.1145/3638250} & Hoch & Betont die Notwendigkeit verlässlicher und effizienter digitaler Signaturen im Kontext von Digital Twins, die primär auf IoT-Infrastrukturen für hochsensible Daten abzielen und damit eine wichtige Schnittmenge zu \ac{KRITIS} darstellen; adressiert explizit \ac{PQC} durch die Entwicklung und Umsetzung quantensicherer und hybrider Signaturlösungen—einschließlich Forward Security und Aggregation, die auch für Blockchain-basierte und dezentrale Identitätsanwendungen (insb. mit Skalierungsbedarf) hoch relevant sind; der methodische Fortschritt im Bereich hardwareunterstützter, ressourcenschonender Kryptografie bietet substanzielle Innovationsimpulse für sicherheitskritische Systeme mit beschränkten Ressourcen, wie sie für \ac{SSI}-Lösungen und die sichere Verwaltung digitaler Identitäten in \ac{KRITIS} essenziell sind; Blockchain-Technologie und \ac{SSI} werden nicht explizit vertieft, jedoch ist die Übertragbarkeit der vorgestellten Konzepte—insbesondere hybride und aggregierbare Signaturen—auf beide Domänen methodisch und praxisnah gegeben. \\
\midrule
3 & Sharif, A., Ranzi, M., Carbone, R., Sciarretta, G., Marino, F. A., \& Ranise, S. (2022). The eIDAS Regulation: A Survey of Technological Trends for European Electronic Identity Schemes. Applied Sciences (2076-3417), 12(24), 12679. \url{https://doi.org/10.3390/app122412679} & Hoch & Adressiert zentrale Entwicklungen und Herausforderungen europäischer elektronischer Identitätssysteme im Kontext der eIDAS-Regulierung; analysiert technologische Trends und ihre Auswirkungen auf Sicherheit, Datenschutz und Interoperabilität nationaler eID-Lösungen, was unmittelbar an die Domäne \ac{SSI} und deren regulatorisches Umfeld anschließt; behandelt aktuelle Technologiestandards wie OAuth 2.0, SAML und OpenID Connect, ohne explizit Blockchain- oder \ac{PQC}-Lösungen zu integrieren, beleuchtet jedoch die (in eIDAS 2.0 antizipierte) Entwicklung hin zu dezentralisierten Identitätsarchitekturen, die als Grundlage künftiger \ac{SSI}-Lösungen dienen; liefert wesentliche Erkenntnisse für die Ausgestaltung sicherer und interoperabler digitaler Identitäten in \ac{KRITIS} und gibt Impulse für die technologische und methodische Weiterentwicklung nationalübergreifender Identitätsverwaltung. \\
\midrule
4 & Alam, M., Hoffstein, J., \& Cambou, B. (2024). Privately Generated Key Pairs for Post Quantum Cryptography in a Distributed Network. Applied Sciences (2076-3417), 14(19), 8863. \url{https://doi.org/10.3390/app14198863} & Hoch & Fokussiert auf die praktische Erzeugung, Verteilung und Verifikation privat generierter post-quanten-sicherer Schlüsselpaaren in verteilten Netzwerken, mit expliziter Anwendung von Crystals-Dilithium als \ac{PQC}-Algorithmus; adressiert wesentlich die Domäne \ac{PQC} durch Integration und Umsetzung eines aktuellen Standards, was sowohl für die Sicherheit verteilter Infrastrukturen als auch für zukünftige Identitätslösungen (z.B. im Kontext von \ac{SSI}) zentral ist; Berücksichtigung von Multi-Faktor-Authentifizierung, Challenge-Response-Mechanismen und biometrielosen Verfahren bietet substanzielle methodische Impulse für die Entwicklung von sicheren, dezentralen Schlüsselmanagement- und Authentifizierungslösungen; direkte Einbindung in Blockchain- oder spezifische \ac{KRITIS} wird nicht explizit diskutiert, ist aufgrund des Protokoll- und \ac{PKI}-Fokus jedoch technisch anschlussfähig und für die Domänen \ac{SSI} und \ac{KRITIS} innovativ und relevant. \\
\midrule
5 & Radanliev, P. (2023). Review and Comparison of US, EU, and UK Regulations on Cyber Risk/Security of the Current Blockchain Technologies: Viewpoint from 2023. Review of Socionetwork Strategies, 17(2), 105–129. \url{https://doi.org/10.1007/s12626-023-00139-x} & Hoch & Vergleichende Analyse der US-, EU- und UK-Regulierung im Bereich Cyber-Risiken und Sicherheit aktueller Blockchain-Technologien, basierend auf dem Stand 2023; systematische Prüfung und Gegenüberstellung führender Standardwerke wie \ac{NIST} (US), ISO27001 (international), und neueren Regularien wie MiCA (EU) und CPMI-IOSCO, unter Einbeziehung technologieübergreifender Aspekte (u.a. \ac{PQC}, Cloud Security, IoT); liefert bedeutende Einblicke und unmittelbaren Anwendungsbezug für die Bewertung, Weiterentwicklung und Integration internationaler Cybersecurity-Standards in neue Blockchain-Projekte—insbesondere mit Blick auf die vier Domänen moderner Identitäts- und Sicherheitssysteme \\
\midrule
6 & Enaya, A., Fernando, X., \& Kashef, R. (2025). Survey of Blockchain-Based Applications for IoT. Applied Sciences (2076-3417), 15(8), 4562. \url{https://doi.org/10.3390/app15084562} & Mittel & Betrachtet zentrale Aspekte von Blockchain-Technologien und Sicherheit; Schwerpunkt auf IoT-Anwendungen und branchenspezifische Implementierungen; explizite Bezüge zu \ac{SSI}, \ac{PQC} und \ac{KRITIS} fehlen; breiter Überblick, jedoch geringere Spezifizität hinsichtlich der vier Domänen der Masterarbeit ( \ac{SSI}, Blockchain, \ac{PQC}, \ac{KRITIS}). \\
\midrule
7 & Siam, M. K., Saha, B., Hasan, M. M., Hossain Faruk, M. J., Anjum, N., Tahora, S., Siddika, A., \& Shahriar, H. (2025). Securing Decentralized Ecosystems: A Comprehensive Systematic Review of Blockchain Vulnerabilities, Attacks, and Countermeasures and Mitigation Strategies. Future Internet, 17(4), 183. \url{https://doi.org/10.3390/fi17040183} & Mittel & Systematische Analyse von Schwachstellen, Angriffsszenarien und Gegenmaßnahmen in Blockchain-Ökosystemen; konzentriert sich auf Sicherheitsaspekte von Blockchain-Technologien ohne spezifische Betrachtung von \ac{SSI}, \ac{PQC} oder \ac{KRITIS}; hoher inhaltlicher Wert für das allgemeine Verständnis von Blockchain-Sicherheit, jedoch eingeschränkte Anwendbarkeit auf alle vier Domänen der Masterarbeit. \\
\midrule
8 & Ramirez Lopez, L. J., \& Morillo Ledezma, G. G. (2025). Employing Blockchain, NFTs, and Digital Certificates for Unparalleled Authenticity and Data Protection in Source Code: A Systematic Review. Computers (2073-431X), 14(4), 131. \url{https://doi.org/10.3390/computers14040131} & Mittel & Fokussiert auf Blockchain-basierte Technologien zur Sicherung von Authentizität und Zugriffskontrolle, jedoch im Anwendungskontext akademischer Quellcode-Sicherheit; behandelt primär NFTs und digitale Zertifikate, mit begrenztem Bezug zu \ac{SSI} und ohne Einbeziehung von \ac{PQC}; adressiert die Domäne „Blockchain“ und Aspekte der Datensicherheit, weist jedoch eine geringe Relevanz für die Domänen \ac{SSI}, \ac{PQC} und \ac{KRITIS} im Kontext der Masterarbeit auf. \\
\midrule
9 & Sebestyen, H., Popescu, D. E., \& Zmaranda, R. D. (2025). A Literature Review on Security in the Internet of Things: Identifying and Analysing Critical Categories. Computers (2073-431X), 14(2), 61. \url{https://doi.org/10.3390/computers14020061} & Mittel & Umfassende Betrachtung aktueller Sicherheitsthemen und Identitätsmanagement im IoT-Kontext, unter Rückgriff auf neue Technologien wie Blockchain; Integrationspotenzial hinsichtlich Blockchain erkennbar, jedoch keine explizite Behandlung von \ac{SSI}, \ac{PQC} oder \ac{KRITIS}; bietet einen breiten Überblick zu technologischen Lösungen und Herausforderungen, adressiert jedoch nur partiell die Anforderungen und Innovationspotenziale der vier Domänen der Masterarbeit. \\
\midrule
10 & Nambundo, J. M., de Souza Martins Gomes, O., de Souza, A. D., \& Machado, R. C. S. (2025). Cybersecurity and Major Cyber Threats of Smart Meters: A Systematic Mapping Review. Energies (19961073), 18(6), 1445. \url{https://doi.org/10.3390/en18061445} & Mittel & Konzentriert sich auf Cybersecurity-Bedrohungen und Schwachstellen im Kontext von Smart Metern als Teil kritischer Infrastruktur; adressiert relevante Sicherheitsfragen in Bezug auf Energieversorgung, thematisch verwandt mit der Schutzbedarfsanalyse für \ac{KRITIS}; der Einsatz von \ac{SSI}, Blockchain-Technologien oder \ac{PQC} wird nicht explizit thematisiert; bietet wichtige Einblicke in Bedrohungsszenarien und Mitigationsstrategien für smarte Energiesysteme, bleibt jedoch im Hinblick auf innovative Identitäts- oder Kryptografielösungen und deren methodischer Integration in Smart Metering-Systeme unspezifisch. \\
\midrule
11 & Yuan, F., Huang, X., Zheng, L., Wang, L., Wang, Y., Yan, X., Gu, S., \& Peng, Y.(2025). The Evolution and Optimization Strategies of a PBFT Consensus Algorithm for Consortium Blockchains. Information (2078-2489), 16(4), 268. \url{https://doi.org/10.3390/info16040268} & Mittel & Fokussiert auf die Optimierung des PBFT-Konsensalgorithmus, was grundlegende Bedeutung für Leistung, Sicherheit und Zuverlässigkeit von Consortium Blockchains hat; leistet methodischen Beitrag zur technologischen Weiterentwicklung im Blockchain-Bereich, jedoch ohne explizite Einbindung von \ac{SSI}, \ac{PQC} oder direktem Bezug zu \ac{KRITIS}; relevante Erkenntnisse zur Verbesserung von Konsensmechanismen, die potenziell für skalierbare und sichere Blockchain-basierte Infrastrukturen adaptierbar sind, aber inhaltlich primär auf Konsensalgorithmen begrenzt. \\
\midrule
12 & Radanliev, P. (2024). Digital security by design. Security Journal, 37(4), 1640–1679. \url{https://doi.org/10.1057/s41284-024-00435-3} & Mittel & Bietet eine umfassende, technologieübergreifende Analyse aktueller Herausforderungen im Bereich digitale Sicherheit; adressiert relevante Zukunftsthemen wie AI, Blockchain und Quantencomputing im Kontext sich wandelnder Sicherheitsparadigmen; keine explizite Schwerpunktsetzung auf \ac{SSI}, \ac{PQC} oder spezifisch \ac{KRITIS}; sektorübergreifende Betrachtung liefert wertvolle Einblicke zu regulatorischen und praxisbezogenen Aspekten, bleibt jedoch in Bezug auf die integrative Anwendung innovativer Sicherheitslösungen innerhalb der vier Domänen der Masterarbeit unspezifisch. \\
\midrule
13 & Miller, T., Durlik, I., Kostecka, E., Sokołowska, S., Kozlovska, P., \& Zwolak, R. (2025). Artificial Intelligence in Maritime Cybersecurity: A Systematic Review of AI-Driven Threat Detection and Risk Mitigation Strategies. Electronics (2079-9292), 14(9), 1844. \url{https://doi.org/10.3390/electronics14091844} & Mittel & Fokus liegt auf der Anwendung von KI-gestützten Verfahren zur Erkennung und Minderung von Cyber-Bedrohungen im maritimen Sektor, adressiert damit relevante Aspekte kritischer Infrastrukturen; Schnittstellen zu Blockchain und quantenkryptographischen Ansätzen werden als Forschungsperspektiven genannt, ohne im Review zentrale methodische oder anwendungsbezogene Ausarbeitung zu bieten; \ac{SSI} wird nicht behandelt, der primäre Schwerpunkt liegt auf KI und Cybersecurity, wodurch methodische und technische Details zu \ac{SSI} und \ac{PQC} im Kontext der vier Domänen der Masterarbeit fehlen. \\
\midrule
14 & Atlam, H. F., Ekuri, N., Azad, M. A., \& Lallie, H. S. (2024). Blockchain Forensics: A Systematic Literature Review of Techniques, Applications, Challenges, and Future Directions. Electronics (2079-9292), 13(17), 3568. \url{https://doi.org/10.3390/electronics13173568} & Mittel & Umfassende Analyse von Blockchain-Technologien im digitalen Forensik-Kontext mit Schwerpunkt auf Untersuchungsmethodik und Anwendungsfeldern; adressiert insbesondere die Herausforderungen beim Nachweis und der Verfolgung von Aktivitäten auf Blockchain-Systemen und beleuchtet regulatorische Aspekte, jedoch ohne explizite Behandlung von \ac{SSI}, \ac{PQC} oder den speziellen Anforderungen kritischer Infrastrukturen; liefert wertvolle Einblicke in forensische Anwendungen der Blockchain, bleibt jedoch mit Blick auf innovative Identitäts- und Kryptografielösungen sowie den Schutz kritischer Infrastrukturen im Rahmen der Masterarbeit begrenzt anschlussfähig. \\
\midrule
15 & Yakubu, M. M., Fadzil B Hassan, M., Danyaro, K. U., Junejo, A. Z., Siraj, M., Yahaya, S., Adamu, S., \& Abdulsalam, K. (2024). A Systematic Literature Review on Blockchain Consensus Mechanisms’ Security: Applications and Open Challenges. Computer Systems Science \& Engineering, 48(6), 1437–1481. \url{https://doi.org/10.32604/csse.2024.054556} & Mittel & Umfassende systematische Analyse der Sicherheitsaspekte und Herausforderungen von Blockchain-Konsensmechanismen mit Fokus auf deren Integrität, Zuverlässigkeit und praktische Anwendungen; adressiert die Domäne „Blockchain“ in methodischer Tiefe und liefert wertvolle Erkenntnisse hinsichtlich Sicherheit, Skalierbarkeit und Energieeffizienz von Konsensprotokollen; explizite Bezüge zu \ac{SSI}, \ac{PQC} und spezifisch \ac{KRITIS} fehlen, sodass die Übertragbarkeit auf die weiteren drei Domänen des Masterarbeit-Themas begrenzt ist. \\
\midrule
16 & Oude Roelink, B., El, H. M., \& Sarmah, D. (2024). Systematic review: Comparing zk‐SNARK, zk‐STARK, and bulletproof protocols for privacy‐preserving authentication. Security \& Privacy, 7(5), 1–59. \url{https://doi.org/10.1002/spy2.401} & Mittel & Fokussiert auf den Vergleich und die Analyse moderner Zero-Knowledge-Protokolle (zk-SNARKs, zk-STARKs, Bulletproofs) mit hoher Relevanz für die Domänen Privacy und Blockchain, insbesondere im Kontext von Authentifizierung und Datenschutz; liefert methodische und performancebezogene Einblicke, jedoch ohne explizite Berücksichtigung von \ac{SSI} oder \ac{PQC}; Anbindung an \ac{KRITIS} nicht direkt gegeben, bietet jedoch Potenzial für Integration innovativer Datenschutztechnologien in Blockchain-basierte Identitäts- und Authentifizierungssysteme. \\
\midrule
17 & Cherbal, S., Zier, A., Hebal, S., Louail, L., \& Annane, B. (2024). Security in internet of things: a review on approaches based on blockchain, machine learning, cryptography, and quantum computing. Journal of Supercomputing, 80(3), 3738–3816. \url{https://doi.org/10.1007/s11227-023-05616-2} & Mittel & Umfassende Analyse sicherheitsrelevanter Technologien im IoT-Kontext, darunter Blockchain, Kryptografie und Quantencomputing, mit breitem Überblick zu Ansätzen und Herausforderungen; adressiert insbesondere die Domänen Blockchain und \ac{PQC} durch die Einbeziehung quantenkryptographischer und klassischer kryptographischer Lösungen, ohne explizite Behandlung von \ac{SSI} oder dem spezifischen Anwendungsfeld kritischer Infrastrukturen; liefert wertvolle Vergleiche und Taxonomien zu modernen Sicherheitsmechanismen im IoT, bleibt jedoch hinsichtlich der methodischen und domänenspezifischen Vertiefung für \ac{SSI} und \ac{KRITIS} begrenzt. \\
\midrule
18 & Jagarlamudi, G. K., Yazdinejad, A., Parizi, R. M., \& Pouriyeh, S. (2024). Exploring privacy measurement in federated learning. Journal of Supercomputing, 80(8), 10511–10551. \url{https://doi.org/10.1007/s11227-023-05846-4} & Mittel & Fokussiert auf die Analyse von Privacy-Maßnahmen und deren Messbarkeit im Kontext von föderiertem Lernen, mit hohem methodischen Wert zur Bewertung von Datenschutz und Sicherheitsmetriken; Bezüge zu Blockchain oder \ac{SSI} werden nicht explizit hergestellt, und \ac{PQC} ist nur als Zukunftsperspektive am Rand angedeutet; die behandelten Konzepte und resultierenden Erkenntnisse zu Privacy-Measurement-Methoden können für sichere, dezentrale Systeme (einschließlich kritischer Infrastrukturen) grundsätzlich relevant sein, liefern jedoch keine spezifische oder anwendungsbezogene Vertiefung in den vier Kernbereichen der Masterarbeit. \\
\midrule
19 & Alzoubi, Y. I., Gill, A., \& Mishra, A. (2022). A systematic review of the purposes of Blockchain and fog computing integration: classification and open issues. Journal of Cloud Computing (2192-113X), 11(1), 1–36. \url{https://doi.org/10.1186/s13677-022-00353-y} & Mittel & Systematische Analyse der Integration von Blockchain-Technologien mit Fog Computing zur Lösung sicherheitsrelevanter Herausforderungen in IoT-Anwendungen; adressiert zentrale Aspekte wie Sicherheit, Datenschutz, Zugriffs- und Vertrauensmanagement, jedoch ohne explizite Behandlung von \ac{SSI} oder konkreten Umsetzungen von \ac{PQC}; verweist auf bestehende regulatorische und technologische Herausforderungen durch aufkommende Technologien wie Quantencomputing, aber ohne methodische oder anwendungsbezogene Vertiefung in Bezug auf \ac{KRITIS} oder innovative Identitätskonzepte; bietet wertvolle Klassifikation und Überblick zu Blockchain-Anwendungen im Kontext verteilter Edge-Infrastrukturen, bleibt jedoch bezüglich der vier Kernbereiche der Masterarbeit auf die Domäne „Blockchain“ und allgemeine Sicherheitsaspekte beschränkt. \\
\midrule
20 & Kazmi, S. H. A., Hassan, R., Qamar, F., Nisar, K., \& Ibrahim, A. A. A. (2023). Security Concepts in Emerging 6G Communication: Threats, Countermeasures, Authentication Techniques and Research Directions. Symmetry (20738994), 15(6), 1147. \url{https://doi.org/10.3390/sym15061147} & Mittel & Umfassende Analyse sicherheitsrelevanter Konzepte, Bedrohungen und Authentifizierungsmethoden im Kontext der aufkommenden 6G-Kommunikation; adressiert innovative Technologien wie Künstliche Intelligenz, Quantencomputing und Föderiertes Lernen, wodurch potenzielle Schnittstellen zu \ac{PQC} und Sicherheitsanforderungen für \ac{KRITIS} bestehen; explizite Bezüge zu \ac{SSI} und Blockchain-Technologien fehlen, ebenso eine methodische Vertiefung für spezielle \ac{SSI}- oder Blockchain-basierte Authentifizierungslösungen; bietet wertvolle Einblicke in zukünftige Forschungsrichtungen und technologische Herausforderungen, jedoch begrenzte direkte Anwendbarkeit auf alle vier Kernbereiche der Masterarbeit. \\
\midrule
21 & Attkan, A., \& Ranga, V. (2022). Cyber-physical security for IoT networks: a comprehensive review on traditional, blockchain and artificial intelligence based key-security. Complex \& Intelligent Systems, 8(4), 3559–3591. \url{https://doi.org/10.1007/s40747-022-00667-z} & Mittel & Fokussiert auf Authentifizierung und Schlüsselmanagement in IoT-Netzen unter Einbeziehung klassischer, Blockchain-basierter und KI-gestützter Sicherheitsmechanismen; adressiert im Wesentlichen die Domäne \enquote{Blockchain} und bietet innovative Perspektiven zur dezentralen Verwaltung von Session-Keys sowie zu KI-basierten Angriffserkennungsmethoden im IoT-Kontext; \ac{SSI} und \ac{PQC} werden nicht explizit behandelt, ebenso fehlt die gezielte Anwendung auf \ac{KRITIS}; der umfassende Überblick zu Authentifizierung und Schlüsselmanagement bietet methodische Anschlussmöglichkeiten für \ac{SSI}-basierte Systeme oder kritische Infrastruktur, bleibt im Kern jedoch breit und technologieorientiert ohne vertiefte Ausarbeitung der vier Domänen der Masterarbeit. \\
\midrule
22 & Ray, P. P. (2023). Web3: A comprehensive review on background, technologies, applications, zero-trust architectures, challenges and future directions. Internet of Things \& Cyber Physical Systems, 3, 213–248. \url{https://doi.org/10.1016/j.iotcps.2023.05.003} & Mittel & Bietet einen breiten Überblick über die technologischen und gesellschaftlichen Grundlagen sowie Anwendungsfelder von Web3 und dezentralen Plattformen, wobei die Domäne Blockchain als zentrales Element systematisch behandelt wird; explizite Bezüge zu \ac{SSI} und Zero-Trust-Architekturen existieren, jedoch fehlt eine detaillierte methodische Analyse spezifischer \ac{SSI}-Lösungen oder Implementierungen, und \ac{PQC} wird nicht thematisiert; der Beitrag verweist auf innovative Identitätskonzepte, Anwendungsintegration und technische Herausforderungen, bleibt jedoch hinsichtlich der anwendungsbezogenen Vertiefung zu \ac{PQC} sowie spezifischen Schutzmaßnahmen für \ac{KRITIS} auf einer konzeptionellen Ebene. \\
\midrule
23 & Stach, C., Gritti, C., Bräcker, J., Behringer, M., \& Mitschang, B. (2022). Protecting Sensitive Data in the Information Age: State of the Art and Future Prospects. Future Internet, 14(11), 302. \url{https://doi.org/10.3390/fi14110302} & Mittel & Umfassende Analyse aktueller Privacy-Mechanismen im Kontext datengetriebener Smart Services, mit Fokus auf die praktische Umsetzung nutzerfreundlicher Datenschutzlösungen; adressiert zentrale Herausforderungen und praktische Einsatzfelder moderner Datenschutzverfahren, ohne jedoch explizit auf \ac{SSI}, Blockchain-Technologien, \ac{PQC} oder spezielle Schutzanforderungen kritischer Infrastrukturen einzugehen; liefert wertvolle Einblicke zu datenorientierten Schutzmechanismen und deren Limitationen, bleibt jedoch hinsichtlich der methodischen und domänenspezifischen Vertiefung für \ac{SSI}, \ac{PQC} und \ac{KRITIS} konzeptionell und generisch. \\
\midrule
24 & Farooq, M. S., Riaz, S., \& Alvi, A. (2023). Security and Privacy Issues in Software-Defined Networking (SDN): A Systematic Literature Review. Electronics (2079-9292), 12(14), 3077. \url{https://doi.org/10.3390/electronics12143077} & Mittel & Systematische Analyse von Sicherheits- und Datenschutzproblemen in Software-Defined Networks mit Schwerpunkt auf Schwachstellen, Angriffen und Sicherungsmechanismen entlang der verschiedenen Netzwerkebenen; adressiert grundlegende Herausforderungen für den Schutz moderner Netzwerkarchitekturen, insbesondere durch die Trennung von Steuer- und Datenebene—ein Aspekt, der für den Betrieb kritischer Infrastrukturen relevante Einblicke und Methoden liefert; explizite Bezüge zu Blockchain-Technologien, \ac{SSI} und \ac{PQC} fehlen, jedoch kann die vorgestellte Taxonomie und die Diskussion zukünftiger Forschungsrichtungen methodische Impulse für sichere Integrationskonzepte in verteilten, kritischen oder identitätsgetriebenen Architekturen bieten—die inhaltliche Tiefe und Anwendbarkeit bleibt jedoch primär auf SDN-spezifische Herausforderungen fokussiert. \\
\midrule
25 & Chanal, P. M., \& Kakkasageri, M. S. (2020). Security and Privacy in IoT: A Survey. Wireless Personal Communications, 115(2), 1667–1693. \url{https://doi.org/10.1007/s11277-020-07649-9} & Mittel & Bietet einen umfassenden Überblick über grundlegende Sicherheits- und Datenschutzherausforderungen im Kontext des Internet of Things mit Fokus auf ressourcenbeschränkte Geräte; adressiert Kernaspekte wie Vertraulichkeit, Integrität, Authentifizierung und Verfügbarkeit, ohne jedoch explizit auf \ac{SSI}, Blockchain-Technologien, \ac{PQC} oder spezifische Anforderungen kritischer Infrastrukturen einzugehen; liefert wertvolle konzeptionelle Grundlagen zu Sicherheitsanforderungen und -architekturen im IoT, bleibt jedoch bezüglich anwendungs- oder methodenspezifischer Vertiefung zu den vier Kernbereichen der Masterarbeit generisch. \\
\midrule
26 & Choudhary, A. (2024). Internet of Things: a comprehensive overview, architectures, applications, simulation tools, challenges and future directions. Discover Internet of Things, 4(1), 1–41. \url{https://doi.org/10.1007/s43926-024-00084-3} & Mittel & Umfassende Übersicht und Analyse der IoT-Architektur, Anwendungen und Herausforderungen; adressiert technologische, soziale und funktionale Aspekte des Internet of Things, mit generischer Betrachtung von Architekturen und Simulationsumgebungen; explizite Bezüge zu \ac{SSI}, Blockchain-Technologien, \ac{PQC} oder spezifischen Schutzanforderungen kritischer Infrastrukturen fehlen; liefert wertvolle Grundlagen für das technologische Umfeld, bleibt jedoch hinsichtlich der vier Domänen der Masterarbeit konzeptionell und unspezifisch. \\
\midrule
27 & Sikiru, I. A., Kora, A. D., Ezin, E. C., Imoize, A. L., \& Li, C.-T. (2024). Hybridization of Learning Techniques and Quantum Mechanism for IIoT Security: Applications, Challenges, and Prospects. Electronics (2079-9292), 13(21), 4153. \url{https://doi.org/10.3390/electronics13214153} & Mittel & Systematische Analyse hybrider Sicherheitsansätze in der Industrial IoT (IIoT), insbesondere durch die Kombination klassischer Lernverfahren und quantenmechanistischer Ansätze; adressiert relevante Herausforderungen und Perspektiven der IIoT-Sicherheit mit Berücksichtigung von Blockchain-Technologien und Quantum Mechanisms, wobei \ac{PQC} eher implizit thematisiert wird; explizite Vertiefung von \ac{SSI} und spezifische Anwendungen in \ac{KRITIS} fehlen, liefert jedoch Impulse für die Integration moderner Kryptografie- und Sicherheitsverfahren im industriellen Umfeld. \\
\midrule
28 & RadRadanliev, P. (2024). Artificial intelligence and quantum cryptography. Journal of Analytical Science \& Technology, 14, 1–17. \url{https://doi.org/10.1186/s40543-024-00416-6} & Mittel & Thematisiert den aktuellen Stand und die Zukunftsperspektiven an der Schnittstelle von künstlicher Intelligenz und quantenkryptografischen Verfahren, wobei insbesondere der Einfluss von AI-Methoden auf Effizienz und Robustheit kryptografischer Systeme sowie die Herausforderungen durch das \enquote{Quantum Threat}-Szenario im Zentrum stehen; explizite Bezüge zu \ac{SSI}, Blockchain-Technologien oder spezifischen Anwendungsfällen in \ac{KRITIS} fehlen; liefert wertvolle Impulse zu methodischen Innovationen in der \ac{PQC}, bleibt jedoch hinsichtlich der vier Domänen der Masterarbeit primär konzeptionell und technologisch fokussiert auf AI und Quantenkryptografie. \\
\midrule
29 & O’Donoghue, O., Vazirani, A. A., Brindley, D., \& Meinert, E. (2019). Design Choices and Trade-Offs in Health Care Blockchain Implementations: Systematic Review. Journal of Medical Internet Research, 21(5), e12426. \url{https://doi.org/10.2196/12426} & Mittel & Systematische Analyse von Architektur- und Design-Entscheidungen bei der Implementierung von Blockchain-Technologie im Kontext elektronischer Gesundheitsakten (EMR), mit Schwerpunkt auf sicherheitsrelevanten und skalierbaren Systemanforderungen sowie Trade-offs zwischen verschiedenen technischen, organisatorischen und anwendungsbezogenen Merkmalen; behandelt die Domäne Blockchain umfassend und liefert wertvolle Erkenntnisse über sicherheitsrelevante Kompromisse im Gesundheitswesen, adressiert jedoch weder \ac{SSI} noch \ac{PQC} oder \ac{KRITIS} explizit; die Untersuchung des Spannungsfelds zwischen Sicherheit, Skalierbarkeit und Datenmanagement bildet eine methodisch relevante Grundlage, bleibt aber in Bezug auf die vier zentralen Domänen der Masterarbeit auf anwendungsbezogene Blockchain-Implementierungen beschränkt. \\
\midrule
30 & Mulholland, J., Mosca, M., \& Braun, J. (2017). The Day the Cryptography Dies. IEEE Security \& Privacy, 15(4), 14–21. \url{https://doi.org/10.1109/MSP.2017.3151325} & Mittel & Überblickartige Darstellung der Auswirkungen von Quantencomputern auf bestehende kryptografische Verfahren; adressiert explizit die Bedrohung aktueller Sicherheitstechnologien (\ac{PQC}), jedoch ohne methodische oder technologische Vertiefung zu Blockchain, \ac{SSI} oder \ac{KRITIS}; bietet konzeptionelle Einblicke in Risikoszenarien und Bedrohungsmodelle, bleibt jedoch hinsichtlich innovativer Lösungsansätze oder spezifischer Anwendungsgebiete der vier Domänen der Masterarbeit unspezifisch. \\
\midrule
31 & G, C. A., \& Basarkod, P. I. (2024). A survey on blockchain security for electronic health record. Multimedia Tools and Applications: An Journal, 1–35. \url{https://doi.org/10.1007/s11042-024-19883-5} & Mittel & Fokus auf Blockchain-basierte Sicherheitslösungen für elektronische Gesundheitsakten (EHR) mit Einbindung von Deep-Learning-Methoden; adressiert primär die Domäne Blockchain durch Analyse von Datenschutz, Datensicherheit und Zugriffskontrolle im Gesundheitswesen, bietet wertvolle methodische Einblicke zur Anwendung verteilter Technologien im Bereich sensibler Daten; explizite Bezüge zu \ac{SSI}, \ac{PQC} und spezifisch \ac{KRITIS} außerhalb des Gesundheitssektors fehlen, wodurch die Anwendbarkeit auf alle vier Domänen der Masterarbeit beschränkt bleibt. \\
\midrule
32 & Batta, P., Ahuja, S., \& Kumar, A. (2024). Future Directions for Secure IoT Frameworks: Insights from Blockchain-Based Solutions: A Comprehensive Review and Future Analysis. Wireless Personal Communications: An International Journal, 139(3), 1749–1781. \url{https://doi.org/10.1007/s11277-024-11694-z} & Mittel & Systematische Untersuchung sicherer IoT-Frameworks unter Verwendung von Blockchain-Technologien; detaillierte Analyse verschiedener Algorithmen (u.a. Konsensmechanismen, \ac{RSA}, Hashing) und Plattformen (Ethereum, CoSMOS, Hyperledger Fabric), mit Fokus auf die Verbesserung von Sicherheit und Performance in IoT-Systemen; explizite Bezüge zu \ac{SSI}, \ac{PQC} und den besonderen Anforderungen kritischer Infrastrukturen fehlen; adressiert vor allem die Domäne „Blockchain“ und liefert grundlegende Einblicke zur Anwendung verteilter Sicherheitsmechanismen im IoT, bleibt jedoch hinsichtlich der vier Kerndomänen der Masterarbeit ( \ac{SSI}, Blockchain, \ac{PQC}, \ac{KRITIS}) methodisch und domänenspezifisch eingeschränkt. \\
\midrule
33 & Radanliev, P. (2024). Integrated cybersecurity for metaverse systems operating with artificial intelligence, blockchains, and cloud computing. Frontiers in Blockchain, 1–14. \url{https://doi.org/10.3389/fbloc.2024.1359130} & Mittel & Umfassende Analyse der Cybersicherheitslandschaft im Kontext integrierter Metaverse-Systeme unter Einbezug von Artificial Intelligence, Blockchain und Cloud Computing; adressiert zentrale Risikofelder, regulatorische Herausforderungen und die Rolle moderner Sicherheitstechnologien für die digitale Ökonomie, wobei insbesondere Blockchain in seiner Bedeutung für selbstverwaltete Systeme und Netzwerkgovernance diskutiert wird; explizite Vertiefungen zu \ac{SSI}, \ac{PQC} oder deren spezieller Anwendung im Umfeld kritischer Infrastrukturen fehlen, ebenso bleibt die methodische und technologische Anbindung an innovative Identitäts- und Kryptografieansätze im Rahmen der vier Domänen der Masterarbeit auf konzeptionelle Ausblicke beschränkt. \\
\midrule
34 & Hajian Berenjestanaki, M., Barzegar, H. R., El Ioini, N., \& Pahl, C. (2024). Blockchain-Based E-Voting Systems: A Technology Review. Electronics (2079-9292), 13(1), 17. \url{https://doi.org/10.3390/electronics13010017} & Mittel & Systematische Analyse von Blockchain-basierten E-Voting-Systemen mit Fokus auf Sicherheits-, Transparenz- und Integritätsaspekte; zentrale Bewertung technologischer Herausforderungen und zukünftiger Forschungsfragen, insbesondere zu Skalierbarkeit und Datenschutz – thematisch eng an die Domäne „Blockchain“ angelehnt; explizite Vertiefung von \ac{SSI}, \ac{PQC} oder die Anwendung in besonders schützenswerten \ac{KRITIS} fehlt, bietet jedoch methodische Ansätze und technische Perspektiven, die für die Entwicklung sicherer und vertrauenswürdiger Abstimmungssysteme in digitalen Infrastrukturen relevant sein können. \\
\midrule
35 & Pirbhulal, S., Chockalingam, S., Shukla, A., \& Abie, H. (2024). IoT cybersecurity in 5G and beyond: a systematic literature review. International Journal of Information Security, 23(4), 2827–2879. \url{https://doi.org/10.1007/s10207-024-00865-5} & Mittel & Systematische Literaturübersicht zu Cybersicherheitsaspekten in 5G- und Next-Generation-IoT-Umgebungen, insbesondere hinsichtlich Threats, Authentifizierung, Zugriffskontrolle, Netzwerk- und Anwendungsschicht sowie Herausforderungen durch Softwarisierung und Virtualisierung der Netze; adressiert methodisch den aktuellen Forschungsstand, evaluiert genutzte Validierungsansätze (Praxis, Simulation, Theorie) und liefert ein Kategorienschema für existierende Sicherheitsmechanismen und offene Forschungsfragen; explizite Bezüge zu \ac{SSI}, Blockchain-Technologien oder \ac{PQC} fehlen, ebenso eine gezielte Betrachtung von \ac{KRITIS}—die Branchenbeispiele (z. B. Healthcare, Energie) lassen eine indirekte Bedeutung für KRITIS erkennen, ohne diese jedoch methodisch zu vertiefen; methodische und technologische Tiefe für die vier Masterarbeitsdomänen beschränkt sich auf generische Cybersicherheitsbedrohungen und Lösungsansätze in modernen IoT/5G-Systemen. \\
\midrule
36 & Asif, M., Abrar, M., Salam, A., Amin, F., Ullah, F., Shah, S., \& AlSalman, H. (2025). Intelligent two-phase dual authentication framework for Internet of Medical Things. Scientific Reports, 15(1), 1–19. \url{https://doi.org/10.1038/s41598-024-84713-5} & Mittel & Fokus liegt auf der Entwicklung und Evaluierung eines intelligenten Zwei-Phasen-Authentifizierungsframeworks für die Internet of Medical Things (IoMT) mit Ziel der effizienten und sicheren Kommunikation sensibler Gesundheitsdaten; zentrale technische Ansätze umfassen ECDH für Schlüsselaustausch und AES-GCM für Datenverschlüsselung, wobei signifikante Verbesserungen in Effizienz und Sicherheit gegenüber klassischen Authentifizierungsmethoden nachgewiesen werden; explizite Bezüge zu \ac{SSI} und Blockchain-Technologien sowie \ac{PQC} fehlen vollständig, sodass methodische und technologische Innovationen in diesen Bereichen für den Rahmen der Masterarbeit unberücksichtigt bleiben; adressiert Schutzanforderungen im Bereich kritischer Infrastrukturen exemplarisch am Gesundheitswesen, bleibt jedoch in der Tiefe auf klassische kryptografische Verfahren und Authentifizierungsprozesse limitiert. \\
\midrule
37 & Marengo, A., \& Santamato, V. (2025). Quantum algorithms and complexity in healthcare applications: a systematic review with machine learning-optimized analysis. Frontiers in Computer Science, 1–30. \url{https://doi.org/10.3389/fcomp.2025.1584114} & Mittel & Systematische Übersicht zur Anwendung von Quantenalgorithmen und quanten-inspirierten Komplexitätsanalysen im Gesundheitswesen, mit zwei Schwerpunkten: (1) Quantum Computing für KI-basierte Analysen biomedizinischer Daten und (2) quantenkryptografische Protokolle zur Absicherung medizinischer Daten. Expliziter Bezug zur Domäne \ac{PQC} durch Analyse quantensicherer und blockchain-basierter Sicherheitsmechanismen im Healthcare-Kontext. Keine explizite Behandlung von \ac{SSI} oder dezidierten Blockchain-Architekturen außerhalb sicherheitsrelevanter Frameworks; Anwendung auf \ac{KRITIS} implizit durch den Fokus auf sichere medizinische Systeme, jedoch nicht technologieübergreifend vertieft. Insgesamt methodisch relevant für die Domäne \ac{PQC} und für Sicherheitsthemen im medizinisch-kritischen Sektor, für die vier Themenbereiche der Masterarbeit aber primär im Bereich quantensicherer Daten- und KI-Anwendungen anschlussfähig. \\
\midrule
38 & Ahakonye, L. A. C., Nwakanma, C. I., \& Kim, D.-S. (2024). Tides of Blockchain in IoT Cybersecurity. Sensors (14248220), 24(10), 3111. \url{https://doi.org/10.3390/s24103111} & Mittel & Umfassende Übersicht zu Anwendungsmöglichkeiten und Herausforderungen von Blockchain-Technologie im Bereich der IoT-Cybersicherheit, insbesondere in Verbindung mit KI-unterstützten Intrusion Detection Systemen; adressiert die Domäne „Blockchain“ grundlegend sowie deren Potenzial für Transparenz, Dezentralität und Unveränderlichkeit im IoT-Kontext; Integration von AI und Blockchain als Innovationstreiber für sichere und skalierbare IDS-Lösungen im IoT/IIoT; \ac{SSI} und \ac{PQC} werden nicht explizit behandelt; spezifische Anforderungen und Anwendungsfälle für \ac{KRITIS} werden nur indirekt adressiert; liefert wertvolle Einblicke und methodische Ansätze für die Weiterentwicklung sicherer IoT-Systeme, bleibt jedoch hinsichtlich der vier Domänen der Masterarbeit vorwiegend auf Blockchain und allgemeine Sicherheitsthemen im IoT fokussiert. \\
\midrule
39 & Nain, A., Sheikh, S., Shahid, M., \& Malik, R. (2024). Resource optimization in edge and SDN-based edge computing: a comprehensive study. Cluster Computing, 27(5), 5517–5545. \url{https://doi.org/10.1007/s10586-023-04256-8} & Niedrig & Umfassende systematische Analyse aktueller Optimierungsansätze für Ressourcenmanagement in Edge-Computing-Umgebungen, insbesondere unter Integration von Software-Defined Networking (SDN); adressiert zentrale Herausforderungen der effizienten Ressourcennutzung, Kontrollarchitekturen und Netzwerkprogrammierbarkeit, was insbesondere für leistungsfähige, latenzarme Anwendungen und Systemarchitekturen an der Netzwerkkante relevant ist; explizite Bezüge zu \ac{SSI}, Blockchain-Technologien, \ac{PQC} oder den besonderen Anforderungen kritischer Infrastrukturen fehlen; liefert dennoch wertvolle methodische Impulse für das Design verteilter, dynamischer Infrastrukturen, bleibt aber in Bezug auf die vier Domänen der Masterarbeit überwiegend allgemein und technologieorientiert. \\
\midrule
40 & Netinant, P., Saengsuwan, N., Rukhiran, M., \& Pukdesree, S. (2023). Enhancing Data Management Strategies with a Hybrid Layering Framework in Assessing Data Validation and High Availability Sustainability. Sustainability (2071-1050), 15(20), 15034. \url{https://doi.org/10.3390/su152015034} & Niedrig & Betrachtet Methoden zur nachhaltigen und hochverfügbaren Datenmigration, insbesondere durch ein hybrides Layering-Framework im Kontext von Data Management und Data Validation; adressiert primär Herausforderungen und Optimierungsstrategien im Bereich Datenkonsistenz, Datenintegrität und Verfügbarkeitsmanagement bei Migration und Transformation—relevant für unternehmensweite IT-Systeme und Logistikdaten; explizite Bezüge zu \ac{SSI}, Blockchain-Technologien, \ac{PQC} oder besonderen Anforderungen kritischer Infrastrukturen fehlen vollständig; liefert wertvolle Erkenntnisse zur Bewertung und Ausgestaltung von Datenmigrationsprozessen, bleibt jedoch hinsichtlich der vier Domänen der Masterarbeit methodisch und inhaltlich unberührt. \\
\midrule
41 & Trautman, L. J., Shackelford, S., Elzweig, B., \& Ormerod, P. (2024). Understanding Cyber Risk: Unpacking and Responding to Cyber Threats Facing the Public and Private Sectors. University of Miami Law Review, 78(3), 840–916. \url{https://repository.law.miami.edu/umlr/vol78/iss3/5/} & Niedrig & Umfassende Analyse aktueller Cyberbedrohungen und deren Auswirkungen auf öffentliche und private Sektoren, mit Fokus auf Angriffsszenarien (u.a. Ransomware, Cyberwarfare, Datenlecks), regulatorische und unternehmensbezogene Steuerungsmechanismen sowie das Zusammenspiel von Recht, Unternehmensführung und geopolitischen Risiken; adressiert zentrale Aspekte der Cyber-Risikobewertung und Reaktion auf digitale Angriffe, insbesondere aus administrativer und juristischer Perspektive; explizite Bezüge zu \ac{SSI}, Blockchain-Technologien, \ac{PQC} sowie spezifische Schutzmaßnahmen für \ac{KRITIS} fehlen; bietet wertvolle konzeptionelle Grundlagen im Bereich Cybersicherheit, Governance und Compliance, bleibt jedoch hinsichtlich der vier Schwerpunktdomänen der Masterarbeit in methodischer und technologischer Tiefe eingeschränkt. \\
\midrule
42 & Hendaoui, F., Ferchichi, A., Trabelsi, L., Meddeb, R., Ahmed, R., \& Khelifi, M. K. (2024). Advances in deep learning intrusion detection over encrypted data with privacy preservation: a systematic review. Cluster Computing, 27(7), 8683–8724. \url{https://doi.org/10.1007/s10586-024-04424-4} & Niedrig & Systematische Analyse der Fortschritte im Bereich Deep-Learning-basierter Intrusion Detection über verschlüsselte Daten mit Fokus auf Privacy-Preservation; behandelt innovative Ansätze zur Anomalieerkennung in verschlüsselten Datenströmen durch tiefe neuronale Netze, ohne auf Datenentschlüsselung angewiesen zu sein; explizite Bezüge zu \ac{SSI}, Blockchain-Technologien, \ac{PQC} oder dem Schutz kritischer Infrastrukturen fehlen; bietet wertvolle methodische und technologische Impulse zur sicheren Datenverarbeitung und Angriffserkennung in datenschutzorientierten Systemen, bleibt aber hinsichtlich der vier Kerndomänen der Masterarbeit überwiegend auf den Bereich Deep Learning und Privacy-Preserving IDS fokussiert \\
\midrule
43 & Akartuna, E. A., Johnson, S. D., \& Thornton, A. E. (2023). The money laundering and terrorist financing risks of new and disruptive technologies: a futures-oriented scoping review: The money laundering and terrorist financing risks of new and disruptive technologies: a futures-oriented scoping review. Security Journal, 36(4), 615–650. \url{https://doi.org/10.1057/s41284-022-00356-z} & Niedrig & Systematische Analyse von Geldwäsche- und Terrorismusfinanzierungsrisiken im Zusammenhang mit neuen und disruptiven Technologien, insbesondere Distributed-Ledger-Technologien (inkl. Kryptowährungen), neue Zahlungswege und FinTech; behandelt umfassend die Risiken, Methoden sowie betroffene Akteure und skizziert daraus resultierende Trends und politische Implikationen – mit klarem Bezug zur Domäne \enquote{Blockchain} und angrenzender regulatorischer Herausforderungen; explizite Vertiefungen zu \ac{SSI} oder \ac{PQC} fehlen, ebenso eine gezielte Betrachtung kritischer Infrastrukturen im engeren technischen Sinne; liefert wichtige Einblicke für die Risiko- und Bedrohungsanalyse im Kontext innovativer Finanztechnologien, bleibt aber hinsichtlich der methodischen Tiefe und direkten Anwendbarkeit für die vier Domänen der Masterarbeit beschränkt. \\
\midrule
44 & Zboril, M., \& Svatá, V. (2025). Performance comparison of cloud virtual machines. Journal of Systems \& Information Technology, 27(2), 197–213. \url{https://doi.org/10.1108/JSIT-02-2022-0040} & Niedrig & Thematischer Fokus liegt auf der vergleichenden Performancemessung von Cloud-basierten virtuellen Maschinen (VMs) bei AWS, Microsoft Azure und Google Cloud Platform, basierend auf Benchmark-Tests unter Linux; adressiert ausschließlich Infrastruktur- und Leistungsaspekte von Cloud-Diensten sowie Auswahlkriterien für IT-Betriebsmodelle; keine Verbindung zu den vier Domänen der Masterarbeit, da keine sicherheitsrelevanten, kryptografischen oder identitätsbezogenen Aspekte behandelt werden; relevante Erkenntnisse für Cloud-Infrastrukturmanagement und Benchmarking, jedoch methodisch und inhaltlich außerhalb des Kernbereichs der Masterarbeit. \\
\midrule
45 & Radanliev, P. (2024). The rise and fall of cryptocurrencies: defining the economic and social values of blockchain technologies, assessing the opportunities, and defining the financial and cybersecurity risks of the Metaverse. Financial Innovation, 10, 1–34. \url{https://doi.org/10.1186/s40854-023-00537-8} & Niedrig & Umfassende Analyse der wirtschaftlichen, sozialen und technologischen Aspekte von Blockchain-Technologien, insbesondere im Kontext von Kryptowährungen und deren Rolle im Metaverse; untersucht wirtschaftliche Chancen, Investitionsstrategien und Cybersecurity-Risiken mit interdisziplinärem Ansatz, inklusive Risikobewertung und maschinellem Lernen im Finanzsektor; explizite Bezüge zu \ac{SSI}, \ac{PQC} und dem Schutz kritischer Infrastrukturen fehlen, ebenso eine methodische oder technologische Vertiefung zu innovativen Identitäts- oder Kryptografielösungen—fokussiert primär auf ökonomische und anwendungsbezogene Fragestellungen der Blockchain im Finanz- und Metaverse-Umfeld. \\
\midrule
46 & Bunescu, L., \& Vârtei, A. M. (2024). Modern finance through quantum computing—A systematic literature review. PLoS ONE, 19(7), 1–22. \url{https://doi.org/10.1371/journal.pone.0304317} & Niedrig & Systematische Analyse des Einsatzes von Quantencomputing im Finanzsektor, mit Fokus auf Simulation, Optimierung und maschinelles Lernen; adressiert Kernaspekte der \ac{PQC} im Hinblick auf die transformative Wirkung quantenbasierter Technologien, ohne dabei explizit auf Blockchain-Technologien oder \ac{SSI} einzugehen; liefert wertvolle Einblicke in die methodische und anwendungsbezogene Entwicklung von Quantum Finance, bleibt jedoch hinsichtlich der vier thematischen Domänen der Masterarbeit ( \ac{SSI}, Blockchain, \ac{PQC}, \ac{KRITIS}) primär auf Finanzanwendungen und damit nur partiell anschlussfähig. \\
\midrule
47 & Alzoubi, Y. I., Mishra, A., \& Topcu, A. E. (2024). Research trends in deep learning and machine learning for cloud computing security. Artificial Intelligence Review: An International Science and Engineering Journal, 57(5). \url{https://doi.org/10.1007/s10462-024-10776-5} & Niedrig & Fokussiert auf den Einsatz von Deep-Learning- und Machine-Learning-Technologien zur Identifikation und Bewältigung von Cloud-Sicherheitsbedrohungen; adressiert zentrale Herausforderungen wie Anomalieerkennung, Security Automation und die Integration neuer Technologien, ohne jedoch explizit \ac{SSI}, Blockchain-Ansätze oder \ac{PQC} systematisch einzubinden; hebt methodische, datenschutzbezogene und regulatorische Fragestellungen hervor, die für den Schutz kritischer Infrastrukturen relevant sind, bleibt jedoch bezüglich der vier Kernbereiche der Masterarbeit hauptsächlich auf Cloud Security und AI-gestützte Verfahren konzentriert und bietet nur indirekte Anschlussmöglichkeiten für innovative Kryptografie- oder Identitätslösungen \\
\midrule
48 & Williamson, S. M., \& Prybutok, V. (2024). Balancing Privacy and Progress: A Review of Privacy Challenges, Systemic Oversight, and Patient Perceptions in AI-Driven Healthcare. Applied Sciences (2076-3417), 14(2), 675. \url{https://doi.org/10.3390/app14020675} & Niedrig & Kritische Analyse von Datenschutz-, Ethik- und Compliance-Herausforderungen in AI-gestützten Gesundheitssystemen, mit Fokus auf Differential Privacy und patientenzentrierte Datenverarbeitung; adressiert relevante technologische Ansätze wie Verschlüsselung und Differential Privacy sowie organisatorische und regulatorische Rahmenbedingungen—beinhaltet zudem die Herausforderungen bei der Integration von Blockchain-Technologien im healthcare-spezifischen Kontext und deren Vereinbarkeit mit der DSGVO, wodurch ein übergreifender Bezug zur Domäne Blockchain gegeben ist; \ac{SSI} und \ac{PQC} werden nicht explizit behandelt, ebenso steht die Anbindung an \ac{KRITIS} außerhalb des engeren Fokus; bietet wertvolle Erkenntnisse zu datenschutzgerechter Systemgestaltung und Interdisziplinarität im Gesundheitswesen, bleibt aber hinsichtlich methodischer Tiefe und anwendungsbezogener Integration in allen vier Kernbereichen der Masterarbeit überblicksartig und konzeptionell. \\
\midrule
49 & Tukur, M., Schneider, J., Househ, M., Dokoro, A. H., Ismail, U. I., Dawaki, M., \& Agus, M. (2023). The metaverse digital environments: a scoping review of the challenges, privacy and security issues. Frontiers in Big Data, 1–25. \url{https://doi.org/10.3389/fdata.2023.1301812} & Niedrig & Umfassende Übersicht zu Herausforderungen, Datenschutz- und Sicherheitsfragen bei der Entwicklung und Implementierung von Metaverse-Umgebungen, insbesondere infolge der pandemiebedingten Digitalisierungsschübe; adressiert wirtschaftliche, technische, ethische und soziale Herausforderungen, darunter Hard- und Softwarekosten, digitale Ungleichheit sowie Regelwerks- und Datenmanagement-Fragen; konkrete Analyse und Klassifikation von Policy-, Privacy- und Security-Problemen, mit Fokus auf privatsphärenbezogene Risiken und Governance-Anforderungen im Metaverse. Explizite Bezüge zu \ac{SSI}, Blockchain-Technologien und \ac{PQC} fehlen; \ac{KRITIS} werden nur implizit durch den Verweis auf digitale Spaltungen und gesellschaftliche Implikationen berührt. \\
\midrule
50 & Hanafi, B., Ali, M., \& Singh, D. (2025). Quantum algorithms for enhanced educational technologies. Discover Education, 4(1), 1–33. \url{https://doi.org/10.1007/s44217-025-00400-1} & Niedrig & Fokus auf die Potenziale und Herausforderungen von Quantum Computing und Quantenkryptografie in der Bildungsbranche, z. B. für personalisiertes Lernen und sichere Datenübertragung; Bezug zu \ac{PQC} nur anwendungsbezogen im Bildungskontext, ohne technische Tiefe oder Bezug zu \ac{SSI}, Blockchain oder \ac{KRITIS}; aus Sicht der vier Masterarbeitsdomänen methodisch und thematisch nur sehr eingeschränkt anschlussfähig. \\
\midrule
51 & Pillai, S. E. V. S., Nadella, G. S., Meduri, K., Priyadharsini, N. A., Bhuvanesh, A., \& Kumar, D. (2025). A walrus optimization-enhanced long short-term memory model for credit fraud detection in banking. International Journal of Information Technology: An Official Journal of Bharati Vidyapeeth’s Institute of Computer Applications and Management, 1–17. \url{https://doi.org/10.1007/s41870-025-02574-1} & Niedrig & Beschreibung einer innovativen Framework-Kombination aus Autoencoder, Long Short-Term Memory Netzwerken und Walrus Optimization Algorithm zur Verbesserung der Betrugserkennung im Bankensektor; konzentriert sich ausschließlich auf Machine-Learning-gestützte Analyse, Datenvorverarbeitung und Hyperparameteroptimierung zur Echtzeit-Erkennung betrügerischer Transaktionen in großen Datenmengen; keinerlei Behandlung oder Integration der vier Domänen der Masterarbeit; relevante methodische Beiträge beschränken sich auf KI-basierte Fraud Detection, ohne Anschlusspunkte zu den Kernthemen der Masterarbeit. \\
\midrule
52 & Priya, S. S., Vijayabhasker, R., \& Rajaram, A. (2025). Advanced Security and Efficiency Framework for Mobile Ad-Hoc Networks Using Adaptive Clustering and Optimization Techniques. Journal of Electrical Engineering \& Technology (19750102), 20(3), 1815–1826. \url{https://doi.org/10.1007/s42835-024-02119-9} & Niedrig & Fokus auf ein innovatives Sicherheits- und Effizienz-Framework für Mobile Ad-Hoc Networks (MANETs) durch adaptive Clusterbildung, AI-unterstützte Vertrauensbewertung und quantenresistente PUF-Authentifizierung; explizite Relevanz für \ac{PQC} durch QR-PUF-Komponente; \ac{SSI} und Blockchain werden nicht behandelt, ebenso fehlt eine gezielte Betrachtung kritischer Infrastrukturen; für die vier Kerndomänen der Masterarbeit somit vor allem im Kontext quantensicherer Validierung/mobiler Netzwerksicherheit anschlussfähig, ansonsten methodisch und domänenspezifisch eingeschränkt. \\
\midrule
53 & Berkani, A.-S., Moumen, H., Benharzallah, S., Yahiaoui, S., \& Bounceur, A. (2024). Blockchain Use Cases in the Sports Industry: A Systematic Review. International Journal of Networked \& Distributed Computing, 12(1), 17–40. \url{https://doi.org/10.1007/s44227-024-00022-3} & Niedrig & Fokus auf branchenspezifische Anwendungen der Blockchain-Technologie im Sportsektor (Athleten-Datenmanagement, Fandaten, NFT-Sammlerstücke); methodische und technologische Vertiefung im Hinblick auf \ac{SSI}, \ac{PQC} oder \ac{KRITIS} fehlt; relevante Erkenntnisse nur für den Bereich Blockchain-Anwendungsfälle in Sport und Entertainment, für die vier Domänen der Masterarbeit jedoch insgesamt wenig anschlussfähig. \\
\midrule
54 & 2023 PNS Annual Meeting - Copenhagen, 17-20 June 2023. (2023). Journal of the Peripheral Nervous System: JPNS, 28 Suppl 4, S3–S254. \url{https://doi.org/10.1111/jns.12585} & Niedrig & Konferenzband ohne Bezug zu \ac{SSI}, Blockchain oder \ac{PQC}. \\
\midrule
55 & Posters. (2017). FEBS Journal, 284, 102–403. \url{https://doi.org/10.1111/febs.14174} & Niedrig & Posterband, kein Bezug zu \ac{SSI}, Blockchain oder \ac{PQC}. \\
\midrule
56 & Annotated Listing of New Books. (2024). Journal of Economic Literature, 62(4), 1696–1750. \url{https://doi.org/10.1257/jel.62.4.1696} & Niedrig & Buchliste, kein Bezug zum Thema. \\
\midrule
57 & PNS Abstracts 2023. (2023). Journal of the Peripheral Nervous System, 28, S3–S254. \url{https://doi.org/10.1111/jns.12585} & Niedrig & Abstractband, kein Bezug zu SSI, Blockchain oder \ac{PQC}. \\
\midrule
58 & Elendu, C., Omeludike, E. K., Oloyede, P. O., Obidigbo, B. T., \& Omeludike, J. C. (2024). Legal implications for clinicians in cybersecurity incidents: A review. Medicine, 103(39), 1–26. \url{https://doi.org/10.1097/MD.0000000000039887} & Niedrig & Fokus auf die rechtlichen Implikationen von Cybersecurity-Vorfällen im Gesundheitswesen, insbesondere für klinisch tätige Personen; Betrachtung technologischer Entwicklungen (u. a. künstliche Intelligenz und Quantencomputing) sowie internationaler regulatorischer Unterschiede; praxisnahe Empfehlungen und Fallstudien zu Cybersecurity-Management, ethischen und juristischen Aspekten im Gesundheitssektor; keine explizite Behandlung von \ac{SSI}, Blockchain oder \ac{PQC}, \ac{KRITIS} werden durch den Gesundheitsbereich berührt, Schwerpunkt liegt jedoch auf juristischen und ethischen Fragestellungen. \\
\midrule
59 & Albshaier, L., Almarri, S., \& Hafizur Rahman, M. M. (2024). A Review of Blockchain’s Role in E-Commerce Transactions: Open Challenges, and Future Research Directions. Computers (2073-431X), 13(1), 27. \url{https://doi.org/10.3390/computers13010027} & Niedrig & Fokus auf die Anwendung von Blockchain-Technologien zur Verbesserung von Sicherheit, Transparenz und Betrugserkennung in E-Commerce-Transaktionen; betont die Rolle verteilter, unveränderlicher digitaler Ledger für den Schutz sensibler Kundendaten und die Stärkung des Vertrauens in Online-Plattformen; adressiert die Domäne „Blockchain“ vorrangig, ohne explizite Bezüge zu \ac{SSI} oder \ac{PQC}; \ac{KRITIS} werden nicht thematisiert, da der Schwerpunkt auf E-Commerce liegt; methodische und technologische Tiefe für die vier Domänen der Masterarbeit ist auf Blockchain-Anwendungen im Bereich E-Commerce beschränkt. \\
\midrule
60 & Reddy, R. C., Bhattacharjee, B., Mishra, D., \& Mandal, A. (2022). A systematic literature review towards a conceptual framework for enablers and barriers of an enterprise data science strategy. Information Systems \& e-Business Management, 20(1), 223–255. \url{https://doi.org/10.1007/s10257-022-00550-x} & Niedrig & Fokus liegt auf der systematischen Analyse von Erfolgsfaktoren und Hindernissen bei der unternehmensweiten Einführung von Data-Science-Strategien; methodische Entwicklung eines Enabler-Barrier-Frameworks für die erfolgreiche Umsetzung datengetriebener Projekte in Unternehmen; adressiert dabei organisatorische, technologische und strategische Aspekte der digitalen Transformation im breiten Kontext, jedoch ohne explizite Behandlung oder Integration von \ac{SSI}, Blockchain-Technologien, \ac{PQC} oder spezifischen Schutzanforderungen kritischer Infrastrukturen; liefert wertvolle Erkenntnisse zur Implementierung von Data Science im Unternehmensumfeld, ist für die vier Domänen der Masterarbeit methodisch und thematisch jedoch nicht anschlussfähig. \\
\midrule
61 & Kumar, Y., Marchena, J., Awlla, A. H., Li, J. J., \& Abdalla, H. B. (2024). The AI-Powered Evolution of Big Data. Applied Sciences (2076-3417), 14(22), 10176. \url{https://doi.org/10.3390/app142210176} & Niedrig & Fokus auf die Weiterentwicklung von Big-Data-Analyse und Management durch künstliche Intelligenz, mit Betonung neuer Rahmenwerke für die Charakterisierung und Handhabung großer, komplexer Datensätze. Der Beitrag stellt innovative AI-gestützte Tools (wie RAG-basierte Analyse-Bots/ChatGPT) zur Verbesserung der Datenanalyse vor und hebt methodologische Fortschritte im Bereich datengetriebene Entscheidungsunterstützung hervor. Keine explizite Behandlung oder Integration von \ac{SSI}, Blockchain-Technologien, \ac{PQC} oder besonderen Anforderungen kritischer Infrastrukturen; methodische und technologische Beiträge beschränken sich auf Big-Data-Management und AI-basierte Analytics, ohne Verknüpfung zu den vier zentralen Domänen der Masterarbeit. \\
\end{longtable}

\subsection{Zweite Iteration}

\subsubsection{Ergebnis Iteration 2}
\label{sec:Bewertung der identifizierten Quellen hinsichtlich ihrer Relevanz 2}

Iteration 2 - Bewertung der identifizierten Quellen hinsichtlich ihrer Relevanz

\begin{longtable}{L{0.5cm}L{4cm}L{1.5cm}L{7cm}}
    \caption[]{Bewertung der identifizierten Quellen hinsichtlich ihrer Relevanz}
    \label{tab:quellenbewertung_iteration2} \\
    \toprule
    \textbf{Nr.} & \textbf{Quelle} & \textbf{Relevanz} & \textbf{Kommentar} \\
    \midrule
    \endfirsthead
    \multicolumn{4}{l}{\textit{Tabelle \thetable\ (Fortsetzung)}} \\
    \toprule
    \textbf{Nr.} & \textbf{Quelle} & \textbf{Relevanz} & \textbf{Kommentar} \\
    \midrule
    \endhead
    \midrule
    \multicolumn{4}{r}{\textit{Fortsetzung auf nächster Seite}} \\
    \endfoot
    \bottomrule
    \multicolumn{4}{p{\linewidth}}{\textit{Anmerkung.} Basierend auf den Abstracts aller in Spalte zwei unter \enquote{Quelle} aufgeführten Quellenangaben.} \\
    \endlastfoot
1 & Barrett-danes, F., \& Ahmad, F. (2025). Quantum computing and cybersecurity: a rigorous systematic review of emerging threats, post-quantum solutions, and research directions (2019-2024). Discover Applied Sciences, 7(10). \url{https://doi.org/10.1007/s42452-025-07322-5} & Hoch & Setzt sich systematisch, methodisch und interdisziplinär mit den Bedrohungen durch Quantum Computing für klassische Kryptosysteme auseinander und behandelt ausdrücklich Post-Quantum-Kryptografie (PQC), hybride Frameworks (u.a. QKD), Umsetzungsherausforderungen und deren Auswirkungen insbesondere für IoT-Umgebungen. Die Arbeit analysiert sowohl den Forschungsstand als auch Implementierungs- und Migrationspfade anhand konkreter Pilotstudien und adressiert Skalierbarkeit und Wirtschaftlichkeit; Die Nutzung eines PRISMA-basierten Review-Frameworks und der Fokus auf praktische Handlungs- und Politikempfehlungen verleihen der Arbeit hohe wissenschaftliche und praxisnahe Relevanz für die Entwicklung, Migration und Absicherung quantensicherer sowie zukunftsfähiger Systeme mit starker Anschlussfähigkeit an KRITIS, PQC und angrenzende sicherheitskritische Domänen. \\
\midrule
2 & Feng, Z., Li, Z., Cui, H., \& Whitty, M. T. (2025). Identity management systems: A comprehensive review. Information (Basel), 16(9), 778. \url{https://doi.org/10.3390/info16090778} & Hoch & Der Abstract beschreibt eine umfassende und systematische Übersicht zu Blockchain-basierten Identity Management Systems (IDMSs) mit explizitem Fokus auf Self-Sovereign Identity (SSI), dezentralen Identifikatoren (DIDs), Interoperabilität und Sicherheitsanalyse über den gesamten Identitätslebenszyklus hinweg. Die Arbeit adressiert technologische und organisatorische Herausforderungen, wie Revokation, Übertragbarkeit, Interoperabilität und Quantum-Resilienz für nutzerkontrollierte Identitätsmodelle. Die PRISMA-basierte Methodik sowie die sektorübergreifende Taxonomisierung und die systematische Ableitung aktuell ungelöster Probleme und künftiger Forschungsrichtungen machen die Quelle höchst wertvoll im Umfeld SSI, Blockchain, PQC und KRITIS. \\
\midrule
3 & Akkal, M., Cherbal, S., Annane, B., Lakhlef, H., \& Kharoubi, K. (2025). Quantum, post-quantum, and blockchain approaches for securing the internet of medical things: a systematic review. Cluster Computing, 28(10). \url{https://doi.org/10.1007/s10586-025-05481-z} & Mittel & Behandelt systematisch die Bedrohungen durch Quantum Computing für IoMT und fokussiert explizit auf den Schutz medizinischer Infrastrukturen mit quantensicherer Kryptografie und Blockchain-Technologie; analysiert sowohl PQC- als auch Blockchain-basierte Sicherheitslösungen und deren Anwendung in hochsensiblen medizinischen IoT-Umgebungen, adressiert Architektur, Herausforderungen und Lösungsansätze für KRITIS-nahe Gesundheitssektoren; Fokus zu stark auf IoMT ohne SSI- und ohne Implementierungsbezug. \\
\midrule
4 & Elkhodr, M. (2025). An AI-driven framework for integrated security and privacy in Internet of Things using quantum-resistant blockchain. Future Internet, 17(6), 246. \url{https://doi.org/10.3390/fi17060246} & Mittel & Der Abstract beschreibt ein ganzheitliches Framework, das KI-basierte Security-Orchestrierung, Blockchain-gestützte Identitätsverwaltung und quantenresistente Kryptografie explizit miteinander kombiniert und anwendungsnah für IoT-Umgebungen evaluiert. Die Arbeit bietet indirekten Bezug zur technischen Umsetzung (Evaluierung) von PQC, SSI und KRITIS. \\
\midrule
5 & Kameni Tcheumaga, F. N., Umba, K., Velupillaimeikandan, P., Haque, M. M., \& Ahamed, S. I. (2025). Security of quantum federated machine learning with blockchain for electronics health records. Cureus Journal of Computer Science. \url{https://doi.org/10.7759/s44389-025-03870-4} & Mittel & Der Abstract adressiert systematisch die Integration von Quantum Federated Machine Learning (QFML), Blockchain und den Schutz elektronischer Gesundheitsakten, mit explizitem Fokus auf Security und Privacy in hochsensiblen, dezentralen Architekturen. Die Kombination aus quantenresistenter Kryptografie, Blockchain-basierter Validierung, federiertem Lernen und gezieltem Schutz kritischer Infrastrukturen im Gesundheitswesen wird als innovativer Lösungsansatz diskutiert. Die Arbeit analysiert bestehende Lücken hinsichtlich Robustheit und Sicherheit und liefert damit auch Impulse für zukünftige Forschung und Entwicklung in den Feldern PQC, KRITIS und dezentraler Identitätsarchitekturen. \\
\midrule
6 & Kurt, K. K., Timurtaş, M., Pınar, S., Ozaydin, F., \& Türkeli, S. (2025). Smart contracts, blockchain, and health policies: Past, present, and future. Information (Basel), 16(10), 853. \url{https://doi.org/10.3390/info16100853} & Mittel & Der Abstract liefert eine methodisch fundierte, systematische Übersicht zu Blockchain- und Smart-Contract-basierten Lösungen für die Verwaltung von Gesundheitsdaten und das Policy Management im Gesundheitswesen. Explizit adressiert werden Sicherheit, Datenschutz, Interoperabilität und die Rolle von Smart Contracts in der Durchsetzung und Automatisierung digitaler Gesundheitsrichtlinien. Es fehlt ein expliziter Fokus auf Post-Quantum Kryptografie in Verbindung mit Self-Sovereign Identity. \\
\midrule
7 & Qatawneh, M. (2025). A framework for security risk assessment of blockchain-based applications. Indonesian Journal of Electrical Engineering and Computer Science, 39(2), 952. \url{https://doi.org/10.11591/ijeecs.v39.i2.pp952-962} & Mittel & Der Abstract beschreibt ein systematisch entwickeltes und praxisvalidiertes Framework zur Bewertung und Mitigation von Sicherheitsrisiken in Blockchain-Anwendungen über alle Schichten hinweg. Es adressiert explizit kritische Schwachstellen wie Smart-Contract-Exploits, Sybil-Attacken und Private-Key-Compromises, integriert quantitative und qualitative Risikoanalyse und demonstriert konkrete Wirksamkeit in Form signifikanter Risikoreduktionen in einem Ethereum-Case-Study-Szenario. Das resultierende BCRAM bietet einen standardisierten, adaptierbaren Bewertungsansatz, der für sichere Systemarchitektur in KRITIS-, SSI- und PQC-orientierten Blockchain-Projekten nutzbar ist. \\
\midrule
8 & Reddy, N. R., Suryadevara, S., Reddy, K. G. R., Umamaheswari, R., Guttula, R., \& Kotoju, R. (2025). Quantum secured blockchain framework for enhancing post quantum data security. Scientific Reports, 15(1), 31048. \url{https://doi.org/10.1038/s41598-025-16315-8} & Mittel & Der Abstract stellt ein hochinnovatives, ganzheitliches Framework („QuantumShield-BC“) für die Absicherung von Blockchain-Architekturen gegen die Bedrohungen durch Quantencomputer vor: Es integriert explizit Post-Quantum-Kryptografie (z.B. Dilithium, Falcon), Quantum Key Distribution (QKD) und einen Quantum Byzantine Fault Tolerance (Q-BFT) Konsens-Mechanismus auf Basis von Quantum Random Number Generation (QRNG) für Leader- und Validator-Auswahl. Das System zeigt in experimenteller Validierung sehr geringe Latenz, hohe Durchsatzraten und vollständige Immunität gegenüber typischen Quantum-Angriffen (Shor, Grover), einschließlich effektiver Sybil-, Replay- und MITM-Abwehr. Die Architektur demonstriert praktisch den Weg für interoperable, skalierbare, hochsichere Blockchains ohne direkten Bezug zu KRITIS und SSI. \\
\midrule
9 & Tawfik, A. M., Al-Ahwal, A., Eldien, A. S. T., \& Zayed, H. H. (2025). Blockchain-based access control and privacy preservation in healthcare: a comprehensive survey. Cluster Computing, 28(8). \url{https://doi.org/10.1007/s10586-025-05308-x} & Mittel & Die Arbeit bietet eine systematische Übersicht über Blockchain-basierte Ansätze zur Zugriffssteuerung und Wahrung der Privatsphäre im Gesundheitswesen, mit besonderem Fokus auf permissioned und permissionless Frameworks, Smart Contracts, kryptographische Verfahren und Plattformen wie Hyperledger Fabric und Ethereum. Die Analyse der untersuchten Lösungen zeigt, wie feingranularer Zugriff, automatisierte Autorisierung und revisionssichere Auditierung durch Blockchain und Privacy-Preserving-Techniken realisiert werden können. Es fehlen Bezüge zu SSI. \\
\midrule
10 & Aboshosha, B. W., Zayed, M. M., Khalifa, H. S., \& Ramadan, R. A. (2025). Enhancing Internet of Things security in healthcare using a blockchain-driven lightweight hashing system. Beni-Suef University Journal of Basic and Applied Sciences, 14(1). \url{https://doi.org/10.1186/s43088-025-00644-8} & Niedrig & Fokussiert auf die Verbesserung der Datensicherheit und Integrität in IoT-basierten Gesundheitsanwendungen durch Blockchain und leichtgewichtige Hashverfahren; adressiert wesentliche Aspekte dezentraler Datenverwaltungsmodelle und kryptographischer Effizienz für ressourcenbeschränkte Geräte; PQC und SSI werden jedoch nicht behandelt, und der methodische Fokus liegt auf klassischen leichten Hashfunktionen statt quantensicheren Primitive. \\
\midrule
11 & Addula, S. R. (2025). Mobile banking adoption: A multi-factorial study on social influence, compatibility, digital self-efficacy, and perceived cost among generation Z consumers in the United States. Journal of Theoretical and Applied Electronic Commerce Research, 20(3), 192. \url{https://doi.org/10.3390/jtaer20030192} & Niedrig & Konzentriert sich auf nutzerpsychologische und soziotechnische Faktoren zur Akzeptanz mobiler Banking-Anwendungen im FinTech-Bereich mit Fokus auf Konsumentenverhalten; weder Kryptografie, Blockchain, noch dezentrale Identitätssysteme werden thematisiert; auch fehlen technische oder sicherheitsbezogene Bezüge zu PQC, SSI oder KRITIS. \\
\midrule
12 & Al Jasem, M. S., De Clark, T., \& Shrestha, A. K. (2025). Toward decentralized intelligence: A systematic literature review of blockchain-enabled AI systems. Information (Basel), 16(9), 765. \url{https://doi.org/10.3390/info16090765} & Niedrig & Die Arbeit liefert einen systematischen Überblick zu Blockchain-gestützten dezentralen KI-Systemen und adressiert damit relevante Aspekte dezentraler Architekturen, insbesondere durch die Untersuchung von Governance, Integrität, skalierbaren Konsensmechanismen sowie Sicherheits- und Datenschutzherausforderungen. Explizite technologische Bezüge zu Self-Sovereign Identity, Post-Quantum-Kryptografie oder kritischen Infrastrukturen fehlen allerdings; stattdessen steht das Zusammenwirken von Blockchain, Smart Contracts und KI im Fokus. \\
\midrule
13 & Almazroi, A. A., Alqarni, M. A., Al-Shareeda, M. A., Alkinani, M. H., Almazroey, A. A., \& Gaber, T. (2025). A bilinear pairing-based anonymous authentication scheme for 5G-assisted vehicular fog computing. Arabian Journal for Science and Engineering, 50(15), 11757–11778. \url{https://doi.org/10.1007/s13369-024-09617-y} & Niedrig & Die Arbeit konzentriert sich auf Authentifizierungsverfahren für 5G-unterstützte Vehicular Fog Computing-Systeme unter Nutzung bilinearer Parings; Im Mittelpunkt stehen effiziente Überprüfungsmechanismen, die Anonymität, Authentizität und praktische Sicherheit im Fahrzeugnetz verbessern; dezentrale Identitätsarchitekturen, Blockchain, PQC oder spezielle Konzepte für Self-Sovereign Identity oder den Schutz kritischer Infrastrukturen werden jedoch nicht behandelt. \\
\midrule
14 & Alsadie, D. (2025). Cybersecurity and artificial intelligence in unmanned aerial vehicles: Emerging challenges and advanced countermeasures. IET Information Security, 2025(1). \url{https://doi.org/10.1049/ise2/2046868} & Niedrig & Stellt eine fundierte und breit gefächerte Übersicht zu aktuellen Bedrohungen und fortschrittlichen Gegenmaßnahmen im Bereich AI-gestützter UAV-Systeme bereit und greift dabei gleich mehrere Schlüsseltechnologien des Bewertungsschemas direkt und explizit auf: Es werden konkrete Methoden der Post-Quantum-Kryptografie (PQC) und blockchain-basierte Sicherheitsmechanismen analysiert und hinsichtlich ihrer Wirksamkeit im Kontext hochsensibler und autonom agierender Drohnennetze bewertet; Kein Bezug zu KRITIS \& SSI; \\
\midrule
15 & Cavus, M., Ayan, H., Bell, M., \& Dissanayake, D. (2025). Advances in energy storage, AI optimisation, and cybersecurity for electric vehicle grid integration. Energies, 18(17), 4599. \url{https://doi.org/10.3390/en18174599} & Niedrig & Die Arbeit liefert einen integrativen und interdisziplinären Überblick zu drei zentralen Säulen hochrelevanter Technologien: Fortschritte bei sicheren dezentralen Energiespeichern, KI-basierter Optimierung für Echtzeit-Energiemanagement und den Einsatz post-quantum Kryptografie und Blockchain-Systemen für die Absicherung von Vehicle-to-Grid (V2G)-Transaktionen im Smart-Grid-Kontext. Konkret fehlt es an praktischen Beiträgen bzgl. Implementierung von SSI, PQC und Blockchain für KRITIS. \\
\midrule
16 & A, C., \& Basarkod, P. I. (2024). A survey on blockchain security for electronic health record. Multimedia Tools and Applications. \url{https://doi.org/10.1007/s11042-024-19883-5} & Niedrig & Der Abstract beschreibt eine systematische Übersicht zu Blockchain-basierten Sicherheitsansätzen für elektronische Gesundheitsakten (EHR), mit Schwerpunkt auf Datenschutz, Zugriffskontrolle, Integritätsprüfung und Effizienz, insbesondere auch mit Blick auf Deep-Learning-Ansätze im Healthcare-Umfeld. Zwar adressiert die Arbeit relevante Aspekte dezentraler vernetzter Datenstrukturen und gibt einen fundierten Vergleich vorhandener Blockchain-basierten Technologien für EHR-Sicherheit, doch stehen SSI, PQC oder KRITIS-spezifische Anwendungen nicht explizit im Fokus. \\
\midrule
17 & Guayasamín, A., Fuertes, W., Carrera, N., Tello-Oquendo, L., \& Suango, V. (2025). Blockchain-empowered e-ticket distribution system for secure and efficient transactions, validation, and audits. Annals of Telecommunications - Annales Des Télécommunications. \url{https://doi.org/10.1007/s12243-025-01125-w} & Niedrig & Der Abstract beschreibt ein Blockchain-basiertes System zur sicheren und effizienten Verwaltung, Prüfung und Auditierung von E-Tickets im Unterhaltungsbereich, insbesondere für Raffles und Veranstaltungsmanagement. Der Fokus liegt auf betrugsresistenter Ticketvergabe, Transparenz, digitaler Signierung, Hash-Funktionen und Proof-of-Work-basiertem Mining, was für die Optimierung von Fälschungsschutz und Automatisierung in öffentlichen Systemen relevant ist. Es besteht kein direkter fachlicher Bezug zu den Kerndomänen PQC, SSI, oder KRITIS. \\
\midrule
18 & He, X., Xu, G., Han, X., Wang, Q., Zhao, L., Shen, C., … Feng, D. (2025). Artificial intelligence security and privacy: a survey. Science China Information Sciences, 68(8). \url{https://doi.org/10.1007/s11432-025-4388-5} & Niedrig & Der Abstract bietet eine breite, systematische Übersicht über Sicherheits- und Datenschutzherausforderungen von KI-Systemen, inklusive Bedrohungen für Datenintegrität, Trainings-, Inferenzphasen und verteilter Umgebungen. Zentral sind klassische KI-Attacken wie Datenvergiftung, Backdoor und Adversarial Attacks, jedoch fehlt ein expliziter Bezug zu Self-Sovereign Identity, Blockchain-Integration, Post-Quantum Kryptografie oder KRITIS-spezifischen Sicherheitsarchitekturen. \\
\midrule
19 & Khagga, V., Priya., S., \& Prasad, A. M. (2025). Enhanced QoS-aware secure routing protocol for WAHNs using advanced fast double decker new binary archimedes kepler pure convolutional transformer network and cryptographic techniques. Peer-to-Peer Networking and Applications, 18(4). \url{https://doi.org/10.1007/s12083-025-02035-3} & Niedrig & Der Abstract stellt eine klassische netzwerktechnische Studie dar, die ein innovatives Routing-Protokoll für Wireless Ad-Hoc Networks (WAHNs) vorstellt und dabei verschiedene Optimierungsverfahren, Deep-Learning-Modelle und komplexe kryptografische Techniken integriert. Die Arbeit adressiert primär technische Fragen der Effizienzsteigerung, Clusterbildung, Verzögerungsminimierung und Zugangssicherheit auf Netzwerkebene, beschränkt sich aber methodisch auf klassische und KI-gestützte kryptografische Mechanismen ohne expliziten Bezug zu PQC, Blockchain, SSI oder KRITIS-spezifischen Identitäts- beziehungsweise Infrastrukturarchitekturen. \\
\midrule
20 & Kumar, M., Kaur, G., \& Rana, P. S. (2025). Performance, portability, productivity, and security in HPC cloud: a systematic literature review. The Journal of Supercomputing, 81(11). \url{https://doi.org/10.1007/s11227-025-07685-x} & Niedrig & Der Abstract liefert eine umfassende systematische Übersicht aktueller Entwicklungstrends und Herausforderungen im Bereich Cloud-basierter Hochleistungsrechner (HPC), strukturiert entlang der Aspekte Performance, Portabilität, Produktivität und Security. Während das Security-Kapitel innovative Technologien wie Trusted Execution Environments, Verschlüsselung und feingranulare Zugangskontrolle behandelt und so auf zentrale Anforderungen in sensiblen Anwendungsdomänen eingeht, fehlt ein expliziter Fokus auf Post-Quantum Kryptografie, Self-Sovereign Identity oder dezentrale Identitätsarchitekturen im KRITIS-Kontext. \\
\midrule
21 & Lubis, M., Safitra, M. F., Fakhrurroja, H., \& Muttaqin, A. N. (2025). Guarding our vital systems: A metric for critical infrastructure cyber resilience. Sensors (Basel, Switzerland), 25(15), 4545. \url{https://doi.org/10.3390/s25154545} & Niedrig & Dieses Paper adressiert direkt die Entwicklung, Messung und Erhöhung der Cyber-Resilienz für kritische Infrastrukturen anhand des InfraGuard Cybersecurity Frameworks, welches etablierte Reifegradmodelle (ISO/IEC 15504, NIST CSF, COBIT) einbindet und die gesamte Breite an Abwehr-, Verteidigungs- und Wiederherstellungsmaßnahmen strukturiert abbildet. Im Mittelpunkt stehen situative Awareness, aktive Verteidigung, Risikomanagement und Incident Recovery. Es fehlt ein expliziter Fokus auf Post-Quantum Kryptografie in Verbindung mit Self-Sovereign Identity. \\
\midrule
22 & Marengo, A., \& Santamato, V. (2025). Quantum algorithms and complexity in healthcare applications: a systematic review with machine learning-optimized analysis. Frontiers in Computer Science, 7(1584114). \url{https://doi.org/10.3389/fcomp.2025.1584114} & Niedrig & Die Studie bietet eine systematische Übersicht zu quantencomputergestützten Algorithmen, deren algorithmischer Komplexität und Anwendungsbereichen im Gesundheitswesen, insbesondere für KI-gestützte Diagnostik und die Sicherheit medizinischer Daten. Sie kombiniert fortgeschrittene Literaturanalyse (PSO, LDA, LIME) mit einer empirisch validierten Kategorisierung der Forschung in „Quantum für KI in Healthcare“ und „Quantum für Datensicherheit im Gesundheitswesen“. Besonderes Gewicht liegt auf quantenkryptografischen Protokollen, Blockchain-basierten Sicherheitsarchitekturen und hybriden Quantum-KI-Ansätzen, die innovative Lösungsansätze für den Schutz sensibler Gesundheitsdaten, die Beschleunigung diagnostischer Prozesse und die Entwicklung resilienter Infrastrukturmodelle liefern. Es fehlt ein konkreter Bezug zu SSI und KRITIS. \\
\midrule
23 & Meka, C., Palakollu, K. R., Azees, M., Rajasekaran, A. S., Das, A. K., \& Hölbl, M. (2025). A comprehensive survey on integration of machine learning with secure blockchain-based applications. Cluster Computing, 28(10). \url{https://doi.org/10.1007/s10586-025-05330-z} & Niedrig & Der Abstract beschreibt eine systematische Übersicht zu Anwendungen, Herausforderungen und offenen Problemen an der Schnittstelle von Machine Learning und Blockchain, insbesondere hinsichtlich Sicherheit, Automatisierung, Auditing und Skalierung in IoT- und Industrieszenarien. Der Beitrag legt den Fokus auf die beidseitigen Synergien beider Technologien—z.B. Fraud Detection, Schutz der Modellintegrität, sichere Prozessautomatisierung per Smart Contracts und Anomalieerkennung—und skizziert aktuelle Forschung, technische Integrationsaspekte und sektorübergreifende Use Cases. Explizite Bezüge zu PQC, Self-Sovereign Identity oder KRITIS-spezifischen Identitäts- beziehungsweise Infrastrukturarchitekturen werden jedoch nicht vertieft adressiert. \\
\midrule
24 & Mustafa, R., Sarkar, N. I., Mohaghegh, M., Pervez, S., \& Vohra, O. (2025). Cross-layer analysis of machine learning models for secure and energy-efficient IoT networks. Sensors (Basel, Switzerland), 25(12), 3720. \url{https://doi.org/10.3390/s25123720} & Niedrig & Die Arbeit verfolgt einen innovativen Ansatz zur Harmonisierung von IoT-Sicherheit und Energieeffizienz durch eine umfassende, schichtübergreifende Architektur, die spezialisierte Machine-Learning-Modelle (z. B. LSTM-Netzwerke zur Anomalieerkennung, Entscheidungsbäume zur Validierung) mit leichtgewichtiger, adaptiver Kryptografie (Speck) koppelt. Durch rollenspezifische Zugriffskontrolle (RBAC) und energieorientierte Sicherheitspolitiken adressiert die Studie zentrale Herausforderungen für ressourcenbeschränkte IoT-Geräte und demonstriert signifikante Verbesserungen bei Fehlalarmraten, Zugangssicherheit und Energieverbrauch. Explizite Bezüge zu PQC, Self-Sovereign Identity oder KRITIS-spezifischen Identitäts- beziehungsweise Infrastrukturarchitekturen werden jedoch nicht vertieft adressiert. \\
\midrule
25 & Mutahhar, A., Khanzada, T. J. S., \& Shahid, M. F. (2025). Enhanced scalability and security in blockchain-based transportation systems for mass gatherings. Information (Basel), 16(8), 641. \url{https://doi.org/10.3390/info16080641} & Niedrig & Der Abstract beschreibt ein Blockchain-basiertes System zur Erhöhung von Effizienz und Sicherheit in intelligenten Transportsystemen, insbesondere für Massengroßveranstaltungen und städtische Mobilitätsnetzwerke. Die Lösung integriert State Channels und Rollups zur Skalierungsoptimierung, erreicht hohe Transaktionsgeschwindigkeiten und adressiert die Herausforderungen der Datenmanipulation, Integrität und Verschlüsselung in urbanen Verkehrsnetzen. Die Konzeption ist praxisnah für dynamische, datenschutzsensitive Transportinfrastrukturen, adressiert aber weder explizit Self-Sovereign Identity, Post-Quantum Kryptografie noch Anwendungsfälle für KRITIS-nahe Domänen. \\
\midrule
26 & Pillai, S. E. V. S., Nadella, G. S., Meduri, K., Priyadharsini, N. A., Bhuvanesh, A., \& Kumar, D. (2025). A walrus optimization-enhanced long short-term memory model for credit fraud detection in banking. International Journal of Information Technology. \url{https://doi.org/10.1007/s41870-025-02574-1} & Niedrig & Der Abstract beschreibt ein fortgeschrittenes Framework zur Nutzung von Autoencoder, LSTM-Netzwerken und Walrus Optimization Algorithm (WOA) für die Optimierung von Kreditkartenbetrugserkennung in Bankdaten. Die Kombination von modernen Machine-Learning-Techniken und bio-inspirierten Metaheuristiken erhöht die Performance und Skalierbarkeit im Datenmanagement klassischer Finanzsysteme, bleibt aber ohne expliziten Bezug zu Blockchain, Self-Sovereign Identity, Post-Quantum Kryptografie, kritischer Infrastruktur oder dezentralen Identitätsarchitekturen. \\
\midrule
27 & Rao, C. K., Sahoo, S. K., \& Yanine, F. F. (2025). A review of IoT-based smart energy solutions for photovoltaic systems. Electrical Engineering (Berlin. Print), 107(12), 15049–15068. \url{https://doi.org/10.1007/s00202-025-03312-3} & Niedrig & Der Abstract gibt einen breiten Überblick über den Stand und die Rolle von IoT-basierten Monitoring- und Managementsystemen für Photovoltaik-Anlagen, insbesondere zur Optimierung von Energieeffizienz, Datenanalyse, Cloud-Integration und Betriebsplanung. Der Schwerpunkt liegt auf technologischen Innovationen rund um IoT, Smart Grids, Energieverwaltung und Echtzeitdatenerfassung für industrielle und wissenschaftliche Anwendungen. Methodisch und thematisch fehlt jedoch ein Bezug zu Self-Sovereign Identity, Blockchain, Post-Quantum Kryptografie oder zur Absicherung kritischer digitaler Versorgungsinfrastrukturen. \\
\midrule
28 & Sefati, S. S., Arasteh, B., Halunga, S., \& Fratu, O. (2025). A comprehensive survey of cybersecurity techniques based on quality of service (QoS) on the Internet of Things (IoT). Cluster Computing, 28(12). \url{https://doi.org/10.1007/s10586-025-05449-z} & Niedrig & Der Abstract liefert einen systematischen Überblick zu aktuellen Cybersecurity-Techniken zur QoS-bewussten Absicherung von IoT-Systemen, mit besonderem Fokus auf die Trade-offs zwischen Sicherheitsmechanismen (etwa Anomalieerkennung, Hybrid- und KI-basierte Methoden, Blockchain, Federated Learning) und kritischen Leistungsparametern (Latenz, Durchsatz, Energieverbrauch) in ressourcenbeschränkten Umgebungen. Der Beitrag beleuchtet vielseitige Angriffsszenarien (DDoS, MITM, Datenextraktion), validiert aktuelle Detektionsparadigmen und hebt offene Forschungsfelder hervor, etwa adaptive und echtzeitfähige Security-Modelle sowie Recovery-Maßnahmen nach Angriffen. Explizite Bezüge zu PQC, SSI oder KRITIS-spezifischen Architekturparadigmen werden nicht gesetzt. \\
\midrule
29 & Shahzad, M., Rizvi, S., Khan, T. A., Ahmad, S., \& Ateya, A. A. (2025). An exhaustive parametric analysis for securing SDN through traditional, AI/ML, and blockchain approaches: A systematic review. International Journal of Networked and Distributed Computing, 13(1). \url{https://doi.org/10.1007/s44227-024-00055-8} & Niedrig & Das Paper bietet eine umfassende, systematische Analyse und Vergleich der wichtigsten Sicherheitsansätze für Software-defined Networking (SDN), inklusive klassischer, Machine-Learning- und Blockchain-basierter Methoden. Die Literaturauswertung demonstriert, dass Blockchain-basierte Mechanismen für Flussregelung, Datenvalidierung, Parser und Controller-Authentifizierung die Robustheit und Angriffsresistenz von SDN signifikant steigern und Aspekte wie Integrität, Trust und Dezentralität adressieren. Machine-Learning-Technologien (CNN, SVM, KNN) liefern praxisnahe Mehrwerte bei der Angriffserkennung. Jedoch bleibt ein expliziter Bezug zu Post-Quantum Kryptografie, Self-Sovereign Identity oder KRITIS-spezifischen Identitätsarchitekturen aus. \\
\midrule
30 & Singh, B., Indu, S., \& Majumdar, S. (2025). Comparative analysis of intrusion detection models using quantum machine learning techniques. Circuits, Systems, and Signal Processing. \url{https://doi.org/10.1007/s00034-025-03256-w} & Niedrig & Der Abstract und die zitierten Studien zeigen, dass Quantum Machine Learning-Techniken bei der Intrusion Detection gegenüber klassischen Machine Learning-Ansätzen, insbesondere in großen Netzwerken und komplexen Datenlagen, signifikante Vorteile in Genauigkeit und Performanz bieten können. Die experimentellen Vergleiche (z.B. QSVM, QCNN, VQC) über mehrere bekannte Datensätze belegen, dass quantenunterstützte Modelle wie das QML-IDS eine robustere Erkennung von Angriffsmustern (z.B. DDoS, BruteForce, Reconnaissance) und eine bessere Generalisierbarkeit im Kontext von Post-Quantum Cryptography und modernen Cyberbedrohungen erreichen. Auch werden praktische Herausforderungen und die Skalierbarkeit hybrider Modelle thematisiert Ein expliziter Bezug zu Post-Quantum Kryptografie, Self-Sovereign Identity oder KRITIS-spezifischen Identitätsarchitekturen bleibt aus.\\
\midrule
31 & Tom, A. K., Khraisat, A., Jan, T., Whaiduzzaman, M., Nguyen, T. D., \& Alazab, A. (2025). Survey of federated learning for cyber threat intelligence in industrial IoT: Techniques, applications and deployment models. Future Internet, 17(9), 409. \url{https://doi.org/10.3390/fi17090409} & Niedrig & Die Arbeit liefert einen detaillierten und breit gefächerten Überblick zu aktuellen Methoden für cyber threat intelligence (CTI) im Kontext industrieller IoT-Umgebungen (IIoT) unter expliziter Berücksichtigung von föderiertem Lernen (FL) als Schlüsseltechnologie für datenschutzwahrende, skalierbare und dezentrale Bedrohungsanalyse. Die systematische Betrachtung von FL-Architekturen, Aggregationsstrategien (z.B. FedAvg, FedProx, Krum) und deren Anwendungen auf Intrusion Detection, Malware-Analyse, Botnet-Mitigation und Anomalieerkennung demonstriert praxisrelevante Fortschritte für KRITIS-nahe und industriell ausgerichtete Sicherheitsarchitekturen. Es fehlen Bezüge zu SSI, Blockchain und PQC. \\
\midrule
32 & Zhang, Y., Zhao, K., Yang, Y., \& Zhou, Z. (2025). Real-time service migration in edge networks: A survey. Journal of Sensor and Actuator Networks, 14(4), 79. \url{https://doi.org/10.3390/jsan14040079} & Niedrig & Der Abstract stellt eine umfassende systematische Übersicht zu Echtzeit-Service-Migration in Edge-Netzwerken bereit, einschließlich Architekturen, Modellen, Motivationen, Techniken und Anwendungsszenarien (z.B. Smart Cities, Smart Homes, Smart Manufacturing). Der Schwerpunkt liegt auf Algorithmen, Modellen und Methoden zur Reduktion von Latenz, Lastverteilung und dynamischer Ressourcenzuteilung für zeitkritische, verteilte Dienste. Sicherheit, Privacy, PQC, Blockchain oder dezentrale Identitätsarchitekturen werden nicht explizit adressiert. Der Beitrag ist für Netzwerkarchitektur und Echtzeitfähigkeit relevant, liefert aber keinen direkten methodischen oder konzeptionellen Beitrag zu SSI/PQC/KRITIS. \\
\midrule
33 & Zinabu, N. G., Marye, Y. W., Tune, K. K., \& Demilew, S. A. (2025). Comprehensive analysis of lightweight cryptographic algorithms for battery‐limited Internet of Things devices. International Journal of Distributed Sensor Networks, 2025(1). \url{https://doi.org/10.1155/dsn/9639728} & Niedrig & Die Arbeit liefert eine systematische und breit angelegte Analyse zu aktuellen Lightweight-Kryptografie-Algorithmen speziell für energie- und ressourcenbeschränkte IoT-Geräte. Verglichen werden dabei unter anderem Algorithmen wie ASCON, SPECK, PRINCE, TWINE und modifizierte Varianten von AES-128 hinsichtlich Effizienz, Sicherheit, Energieverbrauch und Implementierungskomplexität in konkreten Hardware- und Softwareszenarien. Besonders hervorgehoben werden die trade-offs zwischen Durchsatz, Speicher- und Energiebedarf sowie Angriffstoleranz. Der systematische Review zeigt, dass etwa ASCON als sichere, performante Allround-Lösung für Authenticated Encryption anerkannt ist, während SPECK in Szenarien mit höchsten Durchsatzanforderungen durch Einfachheit punktet, jedoch unter größerer kritischer Scrutiny in puncto langfristiger Sicherheit steht. PQC- oder SSI-Verfahren werden in diesem Kontext nicht umfassend adressiert. \\
\midrule
34 & Zreikat, A. I., AlArnaout, Z., Abadleh, A., Elbasi, E., \& Mostafa, N. (2025). The integration of the Internet of Things (IoT) applications into 5G networks: A review and analysis. Computers, 14(7), 250. \url{https://doi.org/10.3390/computers14070250} & Niedrig & Die Arbeit bietet einen fundierten Review und eine Analyse zur Integration von IoT-Anwendungen in 5G-Netzwerke mit Schwerpunkt auf Konnektivität, Datenrate, Latenz, Interoperabilität und Anwendungsvielfalt (Smart Cities, Industrie 4.0, Healthcare etc.). Methodisch werden technische Potenziale (wie Network Slicing, Edge Computing, massive Machine-Type-Kommunikation) und Herausforderungen (Sicherheit, Energie, Netzmanagement) systematisch aufgearbeitet. Zwar werden Security und Privacy angesprochen, jedoch gibt es keinerlei dezidierte Analyse zu PQC, Self-Sovereign Identity, Blockchain, dezentrale Identitätsarchitekturen oder KRITIS-spezifische Schutzmaßnahmen. \\
\end{longtable}


\newpage
\section{Artefaktentwicklung Iteration 1} \label{sec:Anhang_Artefaktentwicklung Iteration 1}

\subsection{Zertifikatserstellungsworkflow}

\refstepcounter{manualListingCounter}
\label{lst:Zertifikatserstellungsworkflow}
\begin{lstlisting}[language=bash, caption={Listing \arabic{lstlisting}: Zertifikatserstellungsworkflow}, numbers=left, frame=single]
# Openssl 3.5.4 LTS (version mit PQC support)
OpenSSL 3.5.4 30 Sep 2025 (Library: OpenSSL 3.5.4 30 Sep 2025)

# ML-DSA-87 Private Key (höchste Sicherheit für Root)
openssl genpkey -algorithm mldsa87 -out rootCA.key

# Self-signed Root Certificate (10 Jahre)
openssl req -x509 -new -key rootCA.key -out rootCA.crt \
  -days 3650 -subj "/CN=My PQC Root CA/O=MyOrg/C=DE" \
  -addext "basicConstraints=critical,CA:TRUE" \
  -addext "keyUsage=critical,keyCertSign,cRLSign"

# ML-DSA-65 Private Keys für Sidecar Proxy
openssl genpkey -algorithm mldsa65 -out server.key

# Certificate Signing Requests (CSRs) für Sidecar Proxy mit SAN
openssl req -new -key server.key -out server.csr \
    -subj "/CN=server proxy/O=FM/C=DE" \
    -addext "subjectAltName=DNS:issuer,DNS:pqc-sidecarproxy-issuer,DNS:host.docker.internal,DNS:localhost,IP:127.0.0.1"

# Signierung mit ML-DSA-65 (Balance zwischen Sicherheit und Performance)
openssl x509 -req -in von-webserver-pqc-proxy.csr \
  -CA rootCA.crt -CAkey rootCA.key -CAcreateserial \
  -out von-webserver-pqc-proxy.crt -days 365 -sha3-256 \
  -copy_extensions copy
\end{lstlisting}

\subsection{Dockerfile: Sidecar Proxy nginx}

\refstepcounter{manualListingCounter}
\label{lst:Dockerfile-Sidecar-Proxy-nginx}
\begin{lstlisting}[language=bash, caption={Listing \arabic{lstlisting}: Dockerfile - Sidecar Proxy nginx}, numbers=left, frame=single]
# Nginx with OpenSSL 3.5.4 (LTS) + OQS Provider for Post-Quantum Cryptography
# Based on: https://github.com/open-quantum-safe/oqs-demos/blob/main/nginx/Dockerfile
#
# Customizations:
# - OpenSSL 3.5.4 (LTS with Security Fixes)
# - Uses custom certificates (mounted via volume)
# - ML-KEM-768 Key Exchange enabled

# Define build arguments for version tags, installation paths, and configurations
ARG ALPINE_VERSION=3.21
ARG OPENSSL_TAG=openssl-3.5.4
ARG LIBOQS_TAG=0.13.0
ARG OQSPROVIDER_TAG=0.9.0
ARG NGINX_VERSION=1.28.0
ARG BASEDIR="/opt"
ARG INSTALLDIR=${BASEDIR}/nginx

# Specify supported signature and key encapsulation mechanisms (KEM) algorithms
ARG SIG_ALG="mldsa65"
ARG DEFAULT_GROUPS=X25519MLKEM768:mlkem768:x25519:mlkem1024

# Stage 1: Build - Compile and assemble all necessary components and dependencies
FROM alpine:${ALPINE_VERSION} AS intermediate
ARG OPENSSL_TAG
ARG LIBOQS_TAG
ARG OQSPROVIDER_TAG
ARG NGINX_VERSION
ARG BASEDIR
ARG INSTALLDIR
ARG SIG_ALG
ARG DEFAULT_GROUPS
ARG OSSLDIR=${BASEDIR}/openssl/.openssl

# Install required build tools and system dependencies
RUN apk update && apk --no-cache add \
    build-base linux-headers libtool \
    automake autoconf make cmake ninja \
    openssl openssl-dev git wget pcre-dev

# Download and prepare source files needed for the build process
WORKDIR /opt
RUN git clone --depth 1 --branch ${LIBOQS_TAG} https://github.com/open-quantum-safe/liboqs \
    && git clone --depth 1 --branch ${OQSPROVIDER_TAG} https://github.com/open-quantum-safe/oqs-provider.git \
    && git clone --depth 1 --branch ${OPENSSL_TAG} https://github.com/openssl/openssl.git \
    && wget -q nginx.org/download/nginx-${NGINX_VERSION}.tar.gz \
    && tar -zxf nginx-${NGINX_VERSION}.tar.gz \
    && rm nginx-${NGINX_VERSION}.tar.gz

# Build and install OpenSSL with shared libraries (for curl and nginx)
WORKDIR /opt/openssl
RUN ./Configure --prefix=${OSSLDIR} \
                --openssldir=${OSSLDIR}/ssl \
                shared \
                enable-fips \
    && make -j"$(nproc)" \
    && make install_sw install_ssldirs

# Configure OpenSSL to support the oqs-provider
RUN cp /opt/openssl/apps/openssl.cnf ${OSSLDIR}/ssl/ && \
    sed -i "s/default = default_sect/default = default_sect\noqsprovider = oqsprovider_sect/g" ${OSSLDIR}/ssl/openssl.cnf && \
    sed -i "s/\[default_sect\]/\[default_sect\]\nactivate = 1\n\[oqsprovider_sect\]\nactivate = 1\n/g" ${OSSLDIR}/ssl/openssl.cnf && \
    sed -i "s/providers = provider_sect/providers = provider_sect\nssl_conf = ssl_sect\n\n\[ssl_sect\]\nsystem_default = system_default_sect\n\n\[system_default_sect\]\nGroups = \$ENV\:\:DEFAULT_GROUPS\n/g" ${OSSLDIR}/ssl/openssl.cnf && \
    sed -i "s/HOME\t\t\t= ./HOME\t\t= .\nDEFAULT_GROUPS\t= ${DEFAULT_GROUPS}/g" ${OSSLDIR}/ssl/openssl.cnf

# Build and install liboqs
WORKDIR /opt/liboqs/build
RUN cmake -G"Ninja"  \
    -DOQS_DIST_BUILD=ON  \
    -DBUILD_SHARED_LIBS=OFF  \
    -DCMAKE_INSTALL_PREFIX="${INSTALLDIR}" ..  \
    && ninja -j"$(nproc)" && ninja install

# Build and install Nginx with shared OpenSSL
WORKDIR /opt/nginx-${NGINX_VERSION}
RUN ./configure --prefix=${INSTALLDIR} \
    --with-debug --with-http_ssl_module  \
    --with-cc-opt="-I${OSSLDIR}/include" \
    --with-ld-opt="-L${OSSLDIR}/lib64 -Wl,-rpath,${OSSLDIR}/lib64" \
    --without-http_gzip_module && \
    make -j"$(nproc)" && make install

# Build and install OQS provider
WORKDIR /opt/oqs-provider
RUN ln -s "/opt/nginx/include/oqs" "${OSSLDIR}/include" && \
    rm -rf build && \
    cmake -DCMAKE_BUILD_TYPE=Debug \
          -DOPENSSL_ROOT_DIR="${OSSLDIR}" \
          -DCMAKE_PREFIX_PATH="${INSTALLDIR}" \
          -S . -B build && \
    cmake --build build && \
    MODULESDIR=$(find "${OSSLDIR}" -name ossl-modules -type d | head -1) && \
    export MODULESDIR && \
    cp build/lib/oqsprovider.so "${MODULESDIR}" && \
    rm -rf "${INSTALLDIR:?}/lib64"

# Build curl with shared OpenSSL 3.5.4
ARG CURL_VERSION=8.11.1
WORKDIR /opt
RUN wget -q https://curl.se/download/curl-${CURL_VERSION}.tar.gz && \
    tar -zxf curl-${CURL_VERSION}.tar.gz && \
    rm curl-${CURL_VERSION}.tar.gz

WORKDIR /opt/curl-${CURL_VERSION}
RUN LDFLAGS="-Wl,-rpath,${OSSLDIR}/lib64 -L${OSSLDIR}/lib64" \
    PKG_CONFIG_PATH="${OSSLDIR}/lib64/pkgconfig" \
    ./configure \
    --prefix=${INSTALLDIR} \
    --with-openssl=${OSSLDIR} \
    --with-ca-bundle=/etc/ssl/certs/ca-certificates.crt \
    --disable-manual \
    --disable-ldap \
    --disable-ldaps \
    --without-libpsl \
    --without-zlib \
    --without-brotli \
    --without-zstd && \
    make -j"$(nproc)" && \
    make install

# Minimize image size by stripping binaries
WORKDIR ${INSTALLDIR}
ENV PATH="${INSTALLDIR}/sbin:${INSTALLDIR}/bin:${OSSLDIR}/bin:${PATH}"

RUN set -ex && \
    strip "${OSSLDIR}/lib64/"*.a \
          "${OSSLDIR}/lib64/ossl-modules/oqsprovider.so" \
          "${INSTALLDIR}/sbin/"* \
          "${INSTALLDIR}/bin/curl" \
          "${OSSLDIR}/bin/openssl" && \
    mkdir -p certs

# Stage 2: Runtime - Create a lightweight image with essential binaries and configurations
FROM alpine:${ALPINE_VERSION}
ARG INSTALLDIR
ARG BASEDIR
ARG OSSLDIR=${BASEDIR}/openssl/.openssl

# Install runtime dependencies
RUN apk update && apk --no-cache add pcre-dev ca-certificates

# Copy compiled artifacts and configuration from the intermediate stage
COPY --from=intermediate ${INSTALLDIR} ${INSTALLDIR}
COPY --from=intermediate ${OSSLDIR} ${OSSLDIR}

# Link logs to Docker collector
RUN set -ex && \
    mkdir -p "${INSTALLDIR}/logs" && \
    ln -sf /dev/stdout "${INSTALLDIR}/logs/access.log" && \
    ln -sf /dev/stderr "${INSTALLDIR}/logs/error.log"

# Expose HTTPS port
# EXPOSE 443

# Set OpenSSL configuration environment
# From Nginx 1.25.2: "nginx does not try to load OpenSSL configuration if the
# --with-openssl option was used to build OpenSSL and the OPENSSL_CONF
# environment variable is not set." Hence we must explicitly set OPENSSL_CONF.
ENV PATH="${INSTALLDIR}/sbin:${INSTALLDIR}/bin:${OSSLDIR}/bin:${PATH}" \
    OPENSSL_CONF="${OSSLDIR}/ssl/openssl.cnf" \
    DEFAULT_GROUPS="X25519MLKEM768:mlkem768:x25519:mlkem1024"

# Create non-root user and update permissions
RUN addgroup -g 1000 -S oqs \
 && adduser --uid 1000 -S oqs -G oqs \
 && chown -R oqs:oqs "${INSTALLDIR}"

# Run as non-root user
USER oqs
WORKDIR ${INSTALLDIR}

STOPSIGNAL SIGTERM
CMD ["nginx", "-c", "nginx-conf/nginx.conf", "-g", "daemon off;"]
\end{lstlisting}



\refstepcounter{manualListingCounter}
\label{lst:nginx_holder.conf}
\begin{lstlisting}[language=bash, caption={Listing \arabic{lstlisting}: nginx\_holder.conf}, numbers=left, frame=single]

# OQS Nginx Configuration for VON Network Webserver Reverse Proxy
# Post-Quantum Cryptography enabled with ML-KEM

worker_processes auto;
error_log /opt/nginx/logs/error.log info;
pid /opt/nginx/logs/nginx.pid;

events {
    worker_connections 1024;
}

http {
    include /opt/nginx/conf/mime.types;
    default_type application/octet-stream;

    log_format main '$remote_addr - $remote_user [$time_local] "$request" '
                    '$status $body_bytes_sent "$http_referer" '
                    '"$http_user_agent" "$http_x_forwarded_for"';

    access_log /opt/nginx/logs/access.log main;

    sendfile on;
    keepalive_timeout 65;

    # Upstream: Holder Agent Inbound Transport (Port 8030)
    upstream holder_inbound {
        server holder:8030;
    }

    # Upstream: Holder Agent Admin API (Port 8031)
    upstream holder_admin {
        server holder:8031;
    }

    # HTTPS Server for Holder Inbound Transport (Port 8030)
    server {
        listen 8030 ssl;
        server_name pqc-sidecarproxy-holder;

        # SSL Certificates (custom ML-DSA-65 certificates)
        ssl_certificate /opt/nginx/certs/holder.crt;
        ssl_certificate_key /opt/nginx/certs/holder.key;

        # TLS 1.3 with Post-Quantum Cryptography
        ssl_protocols TLSv1.3;
        ssl_ecdh_curve X25519MLKEM768;
        # Quantum-Safe Key Exchange Groups (ML-KEM from NIST FIPS-203)
        # Groups are set via DEFAULT_GROUPS environment variable
        # Default: mlkem768:x25519:mlkem1024
        # TLS 1.3 cipher suites are automatically selected

        ssl_prefer_server_ciphers off;

        # Reverse Proxy to Holder Inbound Transport
        location / {
            proxy_pass http://holder_inbound;
            proxy_set_header Host $host;
            proxy_set_header X-Real-IP $remote_addr;
            proxy_set_header X-Forwarded-For $proxy_add_x_forwarded_for;
            proxy_set_header X-Forwarded-Proto https;

            # Timeouts
            proxy_connect_timeout 60s;
            proxy_send_timeout 60s;
            proxy_read_timeout 60s;
        }
    }

    # HTTPS Server for Holder Admin API (Port 8031)
    server {
        listen 8031 ssl;
        server_name pqc-sidecarproxy-holder-admin;

        # SSL Certificates (ML-DSA-65)
        ssl_certificate /opt/nginx/certs/holder.crt;
        ssl_certificate_key /opt/nginx/certs/holder.key;

        # TLS 1.3 with Post-Quantum Cryptography
        ssl_protocols TLSv1.3;
        ssl_ecdh_curve X25519MLKEM768;
        # Quantum-Safe Key Exchange Groups (ML-KEM from NIST FIPS-203)
        # Groups are set via DEFAULT_GROUPS environment variable
        # Default: X25519MLKEM768:mlkem768:x25519:mlkem1024

        ssl_prefer_server_ciphers off;

        # Reverse Proxy to Holder Admin API
        location / {
            proxy_pass http://holder_admin;
            proxy_set_header Host $host;
            proxy_set_header X-Real-IP $remote_addr;
            proxy_set_header X-Forwarded-For $proxy_add_x_forwarded_for;
            proxy_set_header X-Forwarded-Proto https;

            # Timeouts
            proxy_connect_timeout 60s;
            proxy_send_timeout 60s;
            proxy_read_timeout 60s;
        }

        # Health Check
        location /health {
            access_log off;
            return 200 "Holder Admin PQC Proxy OK\n";
            add_header Content-Type text/plain;
        }
    }
}
\end{lstlisting}

\subsection{docker-compose.yml: DLT-Infrastruktur}

\refstepcounter{manualListingCounter}
\label{lst:docker-compose.yml-DLT-Infrastruktur}
\begin{lstlisting}[language=bash, caption={Listing \arabic{lstlisting}: docker-compose.yml: DLT-Infrastruktur}, numbers=left, frame=single]
version: '3'
services:
  #
  # Client
  #
  client:
    image: von-network-base
    command: ./scripts/start_client.sh
    environment:
      - IP=${IP}
      - IPS=${IPS}
      - DOCKERHOST=${DOCKERHOST}
      - RUST_LOG=${RUST_LOG}
    networks:
      - von
    volumes:
      - client-data:/home/indy/.indy_client
      - ./tmp:/tmp

  #
  # Webserver
  #
  webserver:
    image: von-network-base
    command: bash -c 'sleep 10 && ./scripts/start_webserver.sh'
    container_name: von-webserver
    environment:
      - IP=${IP}
      - IPS=${IPS}
      - DOCKERHOST=${DOCKERHOST}
      - LOG_LEVEL=${LOG_LEVEL}
      - RUST_LOG=${RUST_LOG}
      - GENESIS_URL=${GENESIS_URL}
      - LEDGER_SEED=${LEDGER_SEED}
      - LEDGER_CACHE_PATH=${LEDGER_CACHE_PATH}
      - MAX_FETCH=${MAX_FETCH:-50000}
      - RESYNC_TIME=${RESYNC_TIME:-120}
      - POOL_CONNECTION_ATTEMPTS=${POOL_CONNECTION_ATTEMPTS:-5}
      - POOL_CONNECTION_DELAY=${POOL_CONNECTION_DELAY:-10}
      - REGISTER_NEW_DIDS=${REGISTER_NEW_DIDS:-True}
      - ENABLE_LEDGER_CACHE=${ENABLE_LEDGER_CACHE:-True}
      - ENABLE_BROWSER_ROUTES=${ENABLE_BROWSER_ROUTES:-True}
      - DISPLAY_LEDGER_STATE=${DISPLAY_LEDGER_STATE:-True}
      - LEDGER_INSTANCE_NAME=${LEDGER_INSTANCE_NAME:-localhost}
      - LEDGER_DESCRIPTION=${LEDGER_DESCRIPTION}
      - WEB_ANALYTICS_SCRIPT=${WEB_ANALYTICS_SCRIPT}
      - INFO_SITE_TEXT=${INFO_SITE_TEXT}
      - INFO_SITE_URL=${INFO_SITE_URL}
      - INDY_SCAN_URL=${INDY_SCAN_URL}
      - INDY_SCAN_TEXT=${INDY_SCAN_TEXT}
    networks:
      - von
    # ports:
    #   - ${WEB_SERVER_HOST_PORT:-9000}:8000
    volumes:
      - ./config:/home/indy/config
      - ./server:/home/indy/server
      - webserver-cli:/home/indy/.indy-cli
      - webserver-ledger:/home/indy/ledger

  #
  # Synchronization test
  #
  synctest:
    image: von-network-base
    command: ./scripts/start_synctest.sh
    environment:
      - IP=${IP}
      - IPS=${IPS}
      - DOCKERHOST=${DOCKERHOST}
      - LOG_LEVEL=${LOG_LEVEL}
      - RUST_LOG=${RUST_LOG}
    networks:
      - von
    ports:
      - ${WEB_SERVER_HOST_PORT:-9000}:8000
    volumes:
      - ./config:/home/indy/config
      - ./server:/home/indy/server
      - webserver-cli:/home/indy/.indy-cli
      - webserver-ledger:/home/indy/ledger

  #
  # Nodes
  #
  nodes:
    image: von-network-base
    command: ./scripts/start_nodes.sh
    networks:
      - von
    ports:
      - 9701:9701
      - 9702:9702
      - 9703:9703
      - 9704:9704
      - 9705:9705
      - 9706:9706
      - 9707:9707
      - 9708:9708
    environment:
      - IP=${IP}
      - IPS=${IPS}
      - DOCKERHOST=${DOCKERHOST}
      - LOG_LEVEL=${LOG_LEVEL}
      - RUST_LOG=${RUST_LOG}
    volumes:
      - nodes-data:/home/indy/ledger

  node1:
    image: von-network-base
    command: ./scripts/start_node.sh 1
    networks:
      - von
    ports:
      - 9701:9701
      - 9702:9702
    container_name: von-node1
    environment:
      - IP=${IP}
      - IPS=${IPS}
      - DOCKERHOST=${DOCKERHOST}
      - LOG_LEVEL=${LOG_LEVEL}
      - RUST_LOG=${RUST_LOG}
    volumes:
      - node1-data:/home/indy/ledger

  node2:
    image: von-network-base
    command: ./scripts/start_node.sh 2
    networks:
      - von
    ports:
      - 9703:9703
      - 9704:9704
    container_name: von-node2
    environment:
      - IP=${IP}
      - IPS=${IPS}
      - DOCKERHOST=${DOCKERHOST}
      - LOG_LEVEL=${LOG_LEVEL}
      - RUST_LOG=${RUST_LOG}
    volumes:
      - node2-data:/home/indy/ledger

  node3:
    image: von-network-base
    command: ./scripts/start_node.sh 3
    networks:
      - von
    ports:
      - 9705:9705
      - 9706:9706
    container_name: von-node3
    environment:
      - IP=${IP}
      - IPS=${IPS}
      - DOCKERHOST=${DOCKERHOST}
      - LOG_LEVEL=${LOG_LEVEL}
      - RUST_LOG=${RUST_LOG}
    volumes:
      - node3-data:/home/indy/ledger

  node4:
    image: von-network-base
    command: ./scripts/start_node.sh 4
    networks:
      - von
    ports:
      - 9707:9707
      - 9708:9708
    container_name: von-node4
    environment:
      - IP=${IP}
      - IPS=${IPS}
      - DOCKERHOST=${DOCKERHOST}
      - LOG_LEVEL=${LOG_LEVEL}
      - RUST_LOG=${RUST_LOG}
    volumes:
      - node4-data:/home/indy/ledger

  # Post-Quantum Nginx Reverse Proxy für VON Network Webserver
  pqc-sidecarproxy-webserver:
    build:
      context: ./pqc_sidecarproxy_nginx
      dockerfile: Dockerfile
      args:
        OPENSSL_TAG: openssl-3.5.4
        LIBOQS_TAG: 0.13.0
        OQSPROVIDER_TAG: 0.9.0
        NGINX_VERSION: 1.28.0
        SIG_ALG: mldsa65
        DEFAULT_GROUPS: X25519MLKEM768:mlkem768:x25519:mlkem1024
    container_name: von-pqc-sidecarproxy-webserver
    environment:
      # OpenSSL Configuration
      - OPENSSL_CONF=/opt/openssl/.openssl/ssl/openssl.cnf
      # Post-Quantum Key Exchange Groups
      - DEFAULT_GROUPS=X25519MLKEM768:mlkem768:x25519:mlkem1024
    networks:
      - von
      - sidecarproxy
    ports:
      - 8000:8000  # HTTPS with Post-Quantum Cryptography (ML-KEM-768)
    volumes:
      # Custom nginx configuration for reverse proxy
      - ./pqc_sidecarproxy_nginx/nginx-conf/nginx_webserver.conf:/opt/nginx/nginx-conf/nginx.conf:ro
      # Custom ML-DSA-65 certificates
      - ./pqc_sidecarproxy_nginx/certs:/opt/nginx/certs:ro
      # Logs
      - nginx-logs:/opt/nginx/logs
    depends_on:
      - webserver
    restart: unless-stopped
    healthcheck:
      test: ["CMD", "curl", "-k", "-f", "https://localhost:8000/health"]
      interval: 30s
      timeout: 10s
      retries: 5
      start_period: 10s

networks:
  von:
  sidecarproxy:

volumes:
  client-data:
  webserver-cli:
  webserver-ledger:
  node1-data:
  node2-data:
  node3-data:
  node4-data:
  nodes-data:
  nginx-logs:
\end{lstlisting}

\subsection{Dockerfile: acapy-base}

\refstepcounter{manualListingCounter}
\label{lst:Dockerfile-acapy-base}
\begin{lstlisting}[language=bash, caption={Listing \arabic{lstlisting}: Dockerfile - acapy-base: Revocation Registry}, numbers=left, frame=single]
ARG python_version=3.12
FROM python:${python_version}-slim-bookworm AS build

RUN pip install --no-cache-dir poetry==2.1.1

WORKDIR /src

COPY ./pyproject.toml ./poetry.lock ./
RUN poetry install --no-root

COPY ./acapy_agent ./acapy_agent
COPY ./README.md /src
RUN poetry build

FROM python:${python_version}-slim-bookworm AS main

ARG uid=1001
ARG user=aries
ARG acapy_name="acapy-agent"
ARG acapy_version
ARG acapy_reqs=[didcommv2]

ENV HOME="/home/$user" \
    APP_ROOT="/home/$user" \
    LC_ALL=C.UTF-8 \
    LANG=C.UTF-8 \
    PIP_NO_CACHE_DIR=off \
    PYTHONUNBUFFERED=1 \
    PYTHONIOENCODING=UTF-8 \
    RUST_LOG=warn \
    SHELL=/bin/bash \
    SUMMARY="$acapy_name image" \
    DESCRIPTION="$acapy_name provides a base image for running acapy agents in Docker. \
    This image layers the python implementation of $acapy_name $acapy_version. Based on Debian Buster."

LABEL summary="$SUMMARY" \
    description="$DESCRIPTION" \
    io.k8s.description="$DESCRIPTION" \
    io.k8s.display-name="$acapy_name $acapy_version" \
    name=$acapy_name \
    acapy.version="$acapy_version" \
    maintainer=""

# Add aries user
RUN useradd -U -ms /bin/bash -u $uid $user

# Install environment
RUN apt-get update && \
    apt-get install -y --no-install-recommends \
    apt-transport-https \
    ca-certificates \
    curl \
    git \
    libffi-dev \
    libgmp10 \
    libncurses5 \
    libncursesw5 \
    openssl \
    sqlite3 \
    zlib1g && \
    apt-get autopurge -y && \
    apt-get clean -y && \
    rm -rf /var/lib/apt/lists/* /usr/share/doc/*

WORKDIR $HOME

# Add local binaries and aliases to path
ENV PATH="$HOME/.local/bin:$PATH"

# - In order to drop the root user, we have to make some directories writable
#   to the root group as OpenShift default security model is to run the container
#   under random UID.
RUN usermod -a -G 0 $user

# Create standard directories to allow volume mounting and set permissions
# Note: PIP_NO_CACHE_DIR environment variable should be cleared to allow caching
RUN mkdir -p \
    $HOME/.acapy_agent \
    $HOME/.cache/pip/http \
    $HOME/.indy_client \
    $HOME/ledger/sandbox/data \
    $HOME/log

# The root group needs access the directories under $HOME/.indy_client and $HOME/.acapy_agent for the container to function in OpenShift.
RUN chown -R $user:root $HOME/.indy_client $HOME/.acapy_agent && \
    chmod -R ug+rw $HOME/log $HOME/ledger $HOME/.acapy_agent $HOME/.cache $HOME/.indy_client

# Create /home/indy and symlink .indy_client folder for backwards compatibility with artifacts created on older indy-based images.
RUN mkdir -p /home/indy
RUN ln -s /home/aries/.indy_client /home/indy/.indy_client

# Install ACA-py from the wheel as $user,
# and ensure the permissions on the python 'site-packages' and $HOME/.local folders are set correctly.
USER $user
COPY --from=build /src/dist/acapy_agent*.whl .
RUN acapy_agent_package=$(find ./ -name "acapy_agent*.whl" | head -n 1) && \
    echo "Installing ${acapy_agent_package} ..." && \
    pip install --no-cache-dir --find-links=. ${acapy_agent_package}${acapy_reqs} && \
    rm acapy_agent*.whl && \
    chmod +rx $(python -m site --user-site) $HOME/.local

ENTRYPOINT ["aca-py"]
\end{lstlisting}

\subsection{docker-compose.yml: Revocation Registry}

\refstepcounter{manualListingCounter}
\label{lst:docker-compose.yml-Revocation-Registry}
\begin{lstlisting}[language=bash, caption={Listing \arabic{lstlisting}: docker-compose.yml: Revocation Registry}, numbers=left, frame=single]
services:
  ngrok-tails-server:
    image: ngrok/ngrok
    networks:
      - tails-server
    ports:
      - 4044:4040
    command: start --all
    environment:
      - NGROK_CONFIG=/etc/ngrok.yml
      - NGROK_AUTHTOKEN=${NGROK_AUTHTOKEN}
    volumes:
      - ./ngrok.yml:/etc/ngrok.yml
  tails-server:
    build:
      context: ..
      dockerfile: docker/Dockerfile.tails-server
    networks:
      - tails-server
    command: >
      tails-server
        --host 0.0.0.0
        --port 6543
        --storage-path $STORAGE_PATH
        --log-level $LOG_LEVEL
        --log-config $LOGGING_CONFIG
  tester:
    build:
      context: ..
      dockerfile: docker/Dockerfile.test

  # Post-Quantum Nginx Reverse Proxy für Tails Server
  pqc-sidecarproxy-tails-server:
    build:
      context: ./pqc_sidecarproxy_nginx
      dockerfile: Dockerfile
      args:
        OPENSSL_TAG: openssl-3.5.4
        LIBOQS_TAG: 0.13.0
        OQSPROVIDER_TAG: 0.9.0
        NGINX_VERSION: 1.28.0
        SIG_ALG: mldsa65
        DEFAULT_GROUPS: X25519MLKEM768:mlkem768:x25519:mlkem1024
    container_name: pqc-sidecarproxy-tails-server
    environment:
      # OpenSSL Configuration
      - OPENSSL_CONF=/opt/openssl/.openssl/ssl/openssl.cnf
      # Post-Quantum Key Exchange Groups
      - DEFAULT_GROUPS=X25519MLKEM768:mlkem768:x25519:mlkem1024
    networks:
      - tails-server
      - von_sidecarproxy
    ports:
      - 6543:6543  # HTTPS with Post-Quantum Cryptography (ML-KEM-768)
    volumes:
      # Custom nginx configuration for reverse proxy
      - ./pqc_sidecarproxy_nginx/nginx-conf/nginx_tails-server.conf:/opt/nginx/nginx-conf/nginx.conf:ro
      # Custom ML-DSA-65 certificates
      - ./pqc_sidecarproxy_nginx/certs:/opt/nginx/certs:ro
      # Logs
      - nginx-logs:/opt/nginx/logs
    depends_on:
      - tails-server
    restart: unless-stopped
    healthcheck:
      test: ["CMD", "curl", "-k", "-f", "https://localhost:6543/health"]
      interval: 30s
      timeout: 10s
      retries: 5
      start_period: 10s

networks:
  tails-server:
  von_sidecarproxy:
    external: true

volumes:
  nginx-logs:
\end{lstlisting}

\subsection{docker-compose.yml: SSI-Agenten}

\refstepcounter{manualListingCounter}
\label{lst:docker-compose.yml-SSI-Agenten}
\begin{lstlisting}[language=bash, caption={Listing \arabic{lstlisting}: docker-compose.yml: SSI-Agenten}, numbers=left, frame=single]

version: '3.8'

services:
  issuer:
    build:
      context: ..
      dockerfile: hopE/Dockerfile.acapy-base
    image: acapy-base
    container_name: issuer-agent
    environment:
      - DOCKERHOST=${DOCKERHOST:-host.docker.internal}
      - GENESIS_URL=${GENESIS_URL:-https://host.docker.internal:8000/genesis}
      - LEDGER_URL=${LEDGER_URL:-https://host.docker.internal:8000}
      - PUBLIC_TAILS_URL=https://host.docker.internal:6543
      - TAILS_FILE_COUNT=100
    networks:
      - hope-issuer
    volumes:
      - issuer-data:/home/aries/.acapy_agent
    extra_hosts:
      - "host.docker.internal:host-gateway"
    command: >
      start
      --label "Issuer Agent"
      --inbound-transport http 0.0.0.0 8020
      --outbound-transport http
      --endpoint https://host.docker.internal:8020
      --admin 0.0.0.0 8021
      --admin-insecure-mode
      --auto-provision
      --wallet-type askar
      --wallet-name issuer_wallet
      --wallet-key issuer_wallet_key_000000000000
      --genesis-url https://host.docker.internal:8000/genesis
      --log-level info
      --auto-accept-invites
      --auto-accept-requests
      --auto-ping-connection
      --auto-respond-credential-proposal
      --auto-respond-credential-offer
      --auto-respond-credential-request
      --auto-verify-presentation
      --public-invites
      --preserve-exchange-records
      --tails-server-base-url https://host.docker.internal:6543
      --notify-revocation
    healthcheck:
      test: ["CMD", "curl", "-f", "http://localhost:8021/status/ready"]
      interval: 30s
      timeout: 10s
      retries: 5
      start_period: 30s

  holder:
    build:
      context: ..
      dockerfile: hopE/Dockerfile.acapy-base
    image: acapy-base
    container_name: holder-agent
    environment:
      - DOCKERHOST=${DOCKERHOST:-host.docker.internal}
      - GENESIS_URL=${GENESIS_URL:-https://host.docker.internal:8000/genesis}
      - LEDGER_URL=${LEDGER_URL:-https://host.docker.internal:8000}
      - PUBLIC_TAILS_URL=https://host.docker.internal:6543
    networks:
      - hope-holder
    volumes:
      - holder-data:/home/aries/.acapy_agent
    extra_hosts:
      - "host.docker.internal:host-gateway"
    command: >
      start
      --label "Holder Agent"
      --inbound-transport http 0.0.0.0 8030
      --outbound-transport http
      --endpoint https://host.docker.internal:8030
      --admin 0.0.0.0 8031
      --admin-insecure-mode
      --auto-provision
      --wallet-type askar
      --wallet-name holder_wallet
      --wallet-key holder_wallet_key_000000000000
      --genesis-url https://host.docker.internal:8000/genesis
      --log-level info
      --auto-accept-invites
      --auto-accept-requests
      --auto-ping-connection
      --auto-respond-credential-offer
      --auto-store-credential
      --public-invites
      --tails-server-base-url https://host.docker.internal:6543
      --preserve-exchange-records
    healthcheck:
      test: ["CMD", "curl", "-f", "http://localhost:8031/status/ready"]
      interval: 30s
      timeout: 10s
      retries: 5
      start_period: 30s

  verifier:
    build:
      context: ..
      dockerfile: hopE/Dockerfile.acapy-base
    image: acapy-base
    container_name: verifier-agent
    environment:
      - DOCKERHOST=${DOCKERHOST:-host.docker.internal}
      - GENESIS_URL=${GENESIS_URL:-https://host.docker.internal:8000/genesis}
      - LEDGER_URL=${LEDGER_URL:-https://host.docker.internal:8000}
      - PUBLIC_TAILS_URL=https://host.docker.internal:6543
    networks:
      - hope-verifier
    volumes:
      - verifier-data:/home/aries/.acapy_agent
    extra_hosts:
      - "host.docker.internal:host-gateway"
    command: >
      start
      --label "Verifier Agent"
      --inbound-transport http 0.0.0.0 8040
      --outbound-transport http
      --endpoint https://host.docker.internal:8040
      --admin 0.0.0.0 8041
      --admin-insecure-mode
      --auto-provision
      --wallet-type askar
      --wallet-name verifier_wallet
      --wallet-key verifier_wallet_key_00000000000
      --genesis-url https://host.docker.internal:8000/genesis
      --log-level info
      --auto-accept-invites
      --auto-accept-requests
      --auto-ping-connection
      --auto-verify-presentation
      --public-invites
      --tails-server-base-url https://host.docker.internal:6543
      --preserve-exchange-records
    healthcheck:
      test: ["CMD", "curl", "-f", "http://localhost:8041/status/ready"]
      interval: 30s
      timeout: 10s
      retries: 5
      start_period: 30s

  pqc-sidecarproxy-issuer:
    build:
      context: ./pqc_sidecarproxy_nginx
      dockerfile: Dockerfile
      args:
        OPENSSL_TAG: openssl-3.5.4
        LIBOQS_TAG: 0.13.0
        OQSPROVIDER_TAG: 0.9.0
        NGINX_VERSION: 1.28.0
        SIG_ALG: mldsa65
        DEFAULT_GROUPS: X25519MLKEM768:mlkem768:x25519:mlkem1024
    container_name: pqc-sidecarproxy-issuer
    environment:
      # OpenSSL Configuration
      - OPENSSL_CONF=/opt/openssl/.openssl/ssl/openssl.cnf
      # Post-Quantum Key Exchange Groups
      - DEFAULT_GROUPS=X25519MLKEM768:mlkem768:x25519:mlkem1024
    networks:
      - von_sidecarproxy
      - hope-issuer
    ports:
      - "8020:8020"  # Issuer Inbound Transport HTTPS (ML-KEM-768)
      - "8021:8021"  # Issuer Admin API HTTPS (ML-KEM-768)
    volumes:
      # Custom nginx configuration for reverse proxy
      - ./pqc_sidecarproxy_nginx/nginx-conf/nginx_issuer.conf:/opt/nginx/nginx-conf/nginx.conf:ro
      # Custom ML-DSA-65 certificates
      - ./pqc_sidecarproxy_nginx/certs:/opt/nginx/certs:ro
      # Logs
      - nginx-logs:/opt/nginx/logs
    depends_on:
      - issuer
      - holder
      - verifier
    restart: unless-stopped
    healthcheck:
      test: ["CMD", "curl", "-k", "-f", "https://localhost:8021/health"]
      interval: 30s
      timeout: 10s
      retries: 5
      start_period: 10s

  pqc-sidecarproxy-holder:
    build:
      context: ./pqc_sidecarproxy_nginx
      dockerfile: Dockerfile
      args:
        OPENSSL_TAG: openssl-3.5.4
        LIBOQS_TAG: 0.13.0
        OQSPROVIDER_TAG: 0.9.0
        NGINX_VERSION: 1.28.0
        SIG_ALG: mldsa65
        DEFAULT_GROUPS: X25519MLKEM768:mlkem768:x25519:mlkem1024
    container_name: pqc-sidecarproxy-holder
    environment:
      # OpenSSL Configuration
      - OPENSSL_CONF=/opt/openssl/.openssl/ssl/openssl.cnf
      # Post-Quantum Key Exchange Groups
      - DEFAULT_GROUPS=X25519MLKEM768:mlkem768:x25519:mlkem1024
    networks:
      - von_sidecarproxy
      - hope-holder
    ports:
      - "8030:8030"  # Holder Inbound Transport HTTPS (ML-KEM-768)
      - "8031:8031"  # Holder Admin API HTTPS (ML-KEM-768)
    volumes:
      # Custom nginx configuration for reverse proxy
      - ./pqc_sidecarproxy_nginx/nginx-conf/nginx_holder.conf:/opt/nginx/nginx-conf/nginx.conf:ro
      # Custom ML-DSA-65 certificates
      - ./pqc_sidecarproxy_nginx/certs:/opt/nginx/certs:ro
      # Logs
      - nginx-logs:/opt/nginx/logs
    depends_on:
      - issuer
      - holder
      - verifier
    restart: unless-stopped
    healthcheck:
      test: ["CMD", "curl", "-k", "-f", "https://localhost:8031/health"]
      interval: 30s
      timeout: 10s
      retries: 5
      start_period: 10s

  pqc-sidecarproxy-verifier:
    build:
      context: ./pqc_sidecarproxy_nginx
      dockerfile: Dockerfile
      args:
        OPENSSL_TAG: openssl-3.5.4
        LIBOQS_TAG: 0.13.0
        OQSPROVIDER_TAG: 0.9.0
        NGINX_VERSION: 1.28.0
        SIG_ALG: mldsa65
        DEFAULT_GROUPS: X25519MLKEM768:mlkem768:x25519:mlkem1024
    container_name: pqc-sidecarproxy-verifier
    environment:
      # OpenSSL Configuration
      - OPENSSL_CONF=/opt/openssl/.openssl/ssl/openssl.cnf
      # Post-Quantum Key Exchange Groups
      - DEFAULT_GROUPS=X25519MLKEM768:mlkem768:x25519:mlkem1024
    networks:
      - von_sidecarproxy
      - hope-verifier
    ports:
      - "8040:8040"  # Verifier Inbound Transport HTTPS (ML-KEM-768)
      - "8041:8041"  # Verifier Admin API HTTPS (ML-KEM-768)
    volumes:
      # Custom nginx configuration for reverse proxy
      - ./pqc_sidecarproxy_nginx/nginx-conf/nginx_verifier.conf:/opt/nginx/nginx-conf/nginx.conf:ro
      # Custom ML-DSA-65 certificates
      - ./pqc_sidecarproxy_nginx/certs:/opt/nginx/certs:ro
      # Logs
      - nginx-logs:/opt/nginx/logs
    depends_on:
      - issuer
      - holder
      - verifier
    restart: unless-stopped
    healthcheck:
      test: ["CMD", "curl", "-k", "-f", "https://localhost:8041/health"]
      interval: 30s
      timeout: 10s
      retries: 5
      start_period: 10s

networks:
  hope-issuer:
  hope-holder:
  hope-verifier:
  von_sidecarproxy:
    external: true

volumes:
  issuer-data:
  holder-data:
  verifier-data:
  nginx-logs:
\end{lstlisting}

\subsection{Eigenkompilation eines Chromium-Browsers mit PQC-Unterstützung}
\label{Eigenkompilation eines Chromium-Browsers mit PQC-Unterstützung}

Zur experimentellen Evaluation von Verfahren der Post-Quanten-Kryptographie (PQC) im Kontext realer Webbrowser wurde ein Chromium-basierter Browser mit erweiterten TLS-Fähigkeiten selbst kompiliert. Grundlage bildete das von Open Quantum Safe (OQS) bereitgestellte Chromium-Demoprojekt, das eine Integration der \emph{liboqs}-Bibliothek in die TLS-Implementierung von Chromium (BoringSSL) demonstriert und hybride sowie rein PQ-basierte Schlüsselaustausch- und Signaturverfahren bereitstellt \parencite{open-quantum-safe_OqsdemosChromium643ef99297fe8c6ebd3587b5dd238d5e7a457037openquantumsafeoqsdemos_,open-quantum-safe_OqsdemosChromiumREADMELinuxmd643ef99297fe8c6ebd3587b5dd238d5e7a457037openquantumsafeoqsdemos_}. Ziel war es, eine lauffähige Build-Umgebung unter Linux aufzusetzen, den OQS-angepassten Quellcode zu beziehen, die notwendigen Abhängigkeiten zu installieren und anschließend ein reproduzierbares Build-Artefakt des Browsers mit PQC-Unterstützung zu erzeugen.

Die Einrichtung der Build-Umgebung erfolgte weitgehend entsprechend der offiziellen Linux-Build-Dokumentation des Chromium-Projekts \parencite{_ChromiumDocsCheckingoutbuildingChromiumLinux_}. Hierzu wurden zunächst die von Chromium bereitgestellten \texttt{depot\_tools} geklont und in den \texttt{PATH} eingebunden, um Werkzeuge wie \texttt{fetch}, \texttt{gclient} und \texttt{gn} verwenden zu können. Anschließend wurde ein separates Arbeitsverzeichnis angelegt und über \texttt{fetch} ein Chromium-Checkout inklusive aller benötigten Abhängigkeiten durchgeführt. Unter einer aktuellen Ubuntu-Linux-Distribution wurden im nächsten Schritt die von Chromium empfohlenen Systemabhängigkeiten installiert, etwa über das Skript \texttt{build/install-build-deps.sh}, welches Compiler, Entwicklungsbibliotheken und Laufzeitbibliotheken für das spätere Linken der Browser-Binärdateien bereitstellt \parencite{_ChromiumDocsCheckingoutbuildingChromiumLinux_}. Nach Abschluss dieser Vorbereitungen wurden mittels \texttt{gclient runhooks} die Chromium-spezifischen Hooks ausgeführt, um zusätzliche Werkzeuge und vorkompilierte Komponenten nachzuladen.

Aufbauend auf dieser Standard-Umgebung wurde der von OQS bereitgestellte Chromium-Zweig eingebunden, der Anpassungen an BoringSSL sowie die Einbindung von \texttt{liboqs} enthält \parencite{open-quantum-safe_OqsdemosChromium643ef99297fe8c6ebd3587b5dd238d5e7a457037openquantumsafeoqsdemos_,open-quantum-safe_OqsdemosChromiumREADMELinuxmd643ef99297fe8c6ebd3587b5dd238d5e7a457037openquantumsafeoqsdemos_}. Dazu wurde das entsprechende Repository aus dem OQS-Demoprojekt geklont und gemäß der dort beschriebenen Struktur so in die bestehende Chromium-Arbeitsumgebung integriert, dass die PQC-Erweiterungen anstelle der unveränderten Upstream-Kryptographiebibliothek verwendet werden. Zentral war dabei die Übernahme der in der OQS-Dokumentation beschriebenen Build-Konfigurationen, insbesondere GN-Argumente, die das Linken gegen \texttt{liboqs} aktivieren und die experimentellen PQ- bzw. Hybrid-Ciphersuites in der TLS-Konfiguration von Chromium einschalten \parencite{open-quantum-safe_OqsdemosChromiumREADMELinuxmd643ef99297fe8c6ebd3587b5dd238d5e7a457037openquantumsafeoqsdemos_}. Diese Konfiguration wurde in einer eigenen Build-Directory, etwa \texttt{out/oqs-Default}, über den Aufruf \texttt{gn gen} mit den projektspezifischen Argumenten erzeugt.

Im Anschluss daran erfolgte der eigentliche Kompiliervorgang des Browsers mit dem von Chromium vorgesehenen Build-Werkzeug \texttt{autoninja}, das die GN-Konfiguration nutzt, um alle notwendigen Targets effizient zu bauen \parencite{_ChromiumDocsCheckingoutbuildingChromiumLinux_}. Durch den Aufruf von \texttt{autoninja -C out/oqs-Default chrome} wurde eine Browser-Binärdatei erzeugt, die die OQS-Erweiterungen in der TLS-Schicht enthält. Der resultierende Browser konnte direkt aus dem Build-Verzeichnis gestartet und gegen PQC-fähige Testserver genutzt werden, um TLS-Verbindungen mit hybriden oder rein PQ-basierten Schlüsselaustauschmechanismen zu etablieren. Die so aufgebaute Umgebung ermöglicht eine kontrollierte experimentelle Analyse der praktischen Auswirkungen von PQC im Browserkontext, etwa hinsichtlich Kompatibilität, Performance und Protokollhandshake, auf Basis eines realen Chromium-Builds mit explizit aktivierter Post-Quanten-Kryptographie.


\subsection{Issuer Agent Boot Logs}

\refstepcounter{manualListingCounter}
\label{lst:Issuer-Agent-Boot-Logs}
\begin{lstlisting}[language=bash, caption={Listing \arabic{lstlisting}: Issuer Agent Boot Logs}, numbers=left, frame=single]
Executing task in folder ferris: docker logs --tail 1000 -f 179e43b336fa16b399efa7326cdc0b8bfa6ab24c8f86a2b4b630fe57fc382064 

2025-11-28 23:49:38,921 acapy_agent.config.default_context INFO Registering default plugins
2025-11-28 23:49:39,083 acapy_agent.config.default_context INFO Registering askar plugins
2025-11-28 23:49:39,308 acapy_agent.config.ledger INFO Fetching genesis transactions from: https://host.docker.internal:8000/genesis
2025-11-28 23:49:46,340 acapy_agent.core.profile INFO Create profile manager: askar
2025-11-28 23:49:46,827 acapy_agent.config.wallet INFO Created new profile - Profile name: issuer_wallet, backend: askar
2025-11-28 23:49:46,829 acapy_agent.config.wallet INFO No public DID created
2025-11-28 23:49:46,885 acapy_agent.config.ledger INFO Ledger configuration complete
2025-11-28 23:49:46,885 acapy_agent.core.conductor INFO Ledger configured successfully.
2025-11-28 23:49:46,893 acapy_agent.core.conductor INFO Wallet type record not found.
2025-11-28 23:49:46,894 acapy_agent.core.conductor INFO New agent. Setting wallet type to askar.
2025-11-28 23:49:47,028 acapy_agent.config.banner INFO 
::::::::::::::::::::::::::::::::::::::::::::::
::               Issuer Agent               ::
::                                          ::
::                                          ::
:: Inbound Transports:                      ::
::                                          ::
::   - http://0.0.0.0:8020                  ::
::                                          ::
:: Outbound Transports:                     ::
::                                          ::
::   - http                                 ::
::   - https                                ::
::                                          ::
:: Administration API:                      ::
::                                          ::
::   - http://0.0.0.0:8021                  ::
::                                          ::
::                               ver: 1.3.2 ::
::::::::::::::::::::::::::::::::::::::::::::::

2025-11-28 23:49:47,028 acapy_agent.config.banner INFO 
::::::::::::::::::::::::::::::::::::::::::::::::::::::::::::::::::::::::::::::::::::::
::                             DEPRECATION NOTICE:                                  ::
:: -------------------------------------------------------------------------------- ::
:: Receiving a core DIDComm protocol with the `did:sov:BzCbsNYhMrjHiqZDTUASHg;spec` ::
:: prefix is deprecated. All parties sending this prefix should be notified that    ::
:: support for receiving such messages will be removed in a future release. Use     ::
:: https://didcomm.org/ instead.                                                    ::
:: -------------------------------------------------------------------------------- ::
:: Aries RFC 0160: Connection Protocol is deprecated and support will be removed in ::
:: a future release; use RFC 0023: DID Exchange instead.                            ::
:: -------------------------------------------------------------------------------- ::
:: Aries RFC 0036: Issue Credential 1.0 is deprecated and support will be removed   ::
:: in a future release; use RFC 0453: Issue Credential 2.0 instead.                 ::
:: -------------------------------------------------------------------------------- ::
:: Aries RFC 0037: Present Proof 1.0 is deprecated and support will be removed in a ::
:: future release; use RFC 0454: Present Proof 2.0 instead.                         ::
::::::::::::::::::::::::::::::::::::::::::::::::::::::::::::::::::::::::::::::::::::::

2025-11-28 23:49:47,029 acapy_agent.core.conductor INFO Wallet version storage record not found.
2025-11-28 23:49:47,030 acapy_agent.core.conductor INFO No upgrade from version was found from wallet or via --from-version startup argument. Defaulting to v0.7.5.
2025-11-28 23:49:47,031 acapy_agent.core.conductor INFO Upgrade configurations available. Initiating upgrade.
2025-11-28 23:49:47,033 acapy_agent.commands.upgrade INFO No ACA-Py version found in wallet storage.
2025-11-28 23:49:47,033 acapy_agent.commands.upgrade INFO Selecting v0.7.5 as --from-version from the config.
2025-11-28 23:49:47,033 acapy_agent.commands.upgrade INFO Running upgrade process for v0.8.1
2025-11-28 23:49:47,034 acapy_agent.commands.upgrade INFO No records of <class 'acapy_agent.connections.models.conn_record.ConnRecord'> found
2025-11-28 23:49:47,040 acapy_agent.commands.upgrade INFO acapy_version storage record set to v1.3.2
2025-11-28 23:49:47,042 acapy_agent.core.conductor INFO Listening...
2025-11-28 23:49:52,008 aiohttp.access INFO 127.0.0.1 [28/Nov/2025:23:49:52 +0000] "GET /status/ready HTTP/1.1" 200 138 "-" "curl/7.88.1"
2025-11-28 23:50:22,041 aiohttp.access INFO 127.0.0.1 [28/Nov/2025:23:50:22 +0000] "GET /status/ready HTTP/1.1" 200 138 "-" "curl/7.88.1"
\end{lstlisting}


\newpage
\section{Artefaktentwicklung Iteration 2}
\label{sec:Anhang_Artefaktentwicklung Iteration 2}