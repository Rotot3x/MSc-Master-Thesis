\subsection{Zielsetzung und Forschungsfragen} \label{sec:Zielsetzung und Forschungsfragen}

- Konkretisierung der vier Hauptforschungsfragen \\
- Erwartete Beiträge zur Wissenschaft und Praxis



Angesichts dessen sind die zentralen Forschungsfragen dieser Arbeit in \autoref{tab:forschungsfragen} dargestellt.

\begin{longtable}{L{0.5cm}L{3.5cm}L{10cm}}
    \caption{Forschungsfragen}
    \label{tab:forschungsfragen} \\
    \toprule
    \textbf{Nr.} & \textbf{Kategorie} & \textbf{Forschungsfrage} \\
    \midrule
    \endfirsthead
    \multicolumn{3}{l}{\textit{Tabelle \thetable\ (Fortsetzung)}} \\
    \toprule
    \textbf{Nr.} & \textbf{Kategorie} & \textbf{Forschungsfrage} \\
    \midrule
    \endhead
    \midrule
    \multicolumn{3}{r}{\textit{Fortsetzung auf nächster Seite}} \\
    \endfoot
    \bottomrule
    \multicolumn{3}{p{\linewidth}}{\textit{Anmerkung.} Eigene Darstellung.} \\
    \endlastfoot
    1 & Systemarchitektur \newline \& \newline Compliance &
    Wie kann ein blockchain-basiertes \ac{SSI}-System unter Einsatz von \ac{PQC} gestaltet werden, um die regulatorischen und technischen Anforderungen von \ac{KRITIS} nachhaltig zu erfüllen? \\
    \midrule
    2 & Algorithmenauswahl \newline \& \newline Sicherheitsbewertung &
    Welche \ac{PQC}-Algorithmen eignen sich für die Integration in \ac{SSI}-Systeme hinsichtlich Sicherheit und Interoperabilität, insbesondere im Kontext von \ac{KRITIS}? \\
    \midrule
    3 & Kryptografische Agilität &
    Welche kryptografischen Agilitätsmechanismen sind erforderlich, um zukünftige \ac{PQC}-Algorithmenupdates ohne Systemunterbrechung zu ermöglichen? \\
\end{longtable}