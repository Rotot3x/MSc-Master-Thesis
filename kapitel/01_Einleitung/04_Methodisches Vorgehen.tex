\subsection{Methodisches Vorgehen} \label{sec:Methodisches Vorgehen}

- Design Science Research Ansatz \\
- Abgrenzung zu realen KRITIS-Umgebungen


\textbf{Systematische Literaturrecherche nach PRISMA 2020 und unsystematische Recherche im DSR-Kontext:} Die Arbeit folgt einem hybriden Review-Ansatz, der zeitlich und funktional klar differenziert ist. Die initiale, systematische Literaturrecherche nach ausgewählten Methoden der PRISMA 2020-Richtlinien (Kapitel 3.1) dient ausschließlich der \textbf{initialen Problemidentifikation, Ergrundung des Forschungsstands und Ableitung der Forschungslücke}. Diese systematische Vorgehensweise gewährleistet einen transparenten reproduzierbaren Prozess und verbessert die Berichtqualität \fixme{PRISMA-Quelle}. \fixme{Vertiefung relevanter Themenbereiche basierend auf den Erkenntnissen aus der Artefakt-Entwicklung erfolgt unsystematisch...} Abgrenzung Herleitung des Themas und Relevanz systematisch vs. DSR unsystematische Literaturrecherche bedarfsspezifisch?