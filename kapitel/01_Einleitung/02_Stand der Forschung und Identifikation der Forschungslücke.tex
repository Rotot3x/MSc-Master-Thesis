\subsection{Stand der Forschung und Identifikation der Forschungslücke} \label{sec:Stand der Forschung und Identifikation der Forschungslücke}

\fixme{Mix aus Expose und Masterarbeit auf Basis der beiden Iterationen der Literaturrecherchen.}

\fixme{

Die systematische Literaturrecherche zum gegenwärtigen Stand der Forschung wurde nach PRISMA-2020-Standards durchgeführt und in zwei iterativen Phasen organisiert. Dieser Ansatz ermöglicht es, sowohl die initiale Problemidentifikation als auch die Aktualisierung der Wissensbasis unter Berücksichtigung des dynamischen Charakters des Forschungsfeldes abzubilden. Die erste Iteration am 30. Mai 2025 wurde im Rahmen des Exposés durchgeführt und etablierte die methodische Grundlage für das gesamte Forschungsprojekt, indem sie 61 relevante Quellen identifizierte und systematisch klassifizierte. Die zweite Iteration am 02. November 2025 ergänzte diese Erkenntnisbasis um aktualisierte Literatur aus dem Zeitraum Mai bis November 2025 und prägte damit ein umfassendes Verständnis des gegenwärtigen Forschungsstands mit zusätzlich 34 identifizierten Quellen.

Die Entwicklung von Blockchain-Technologien und dezentralen Identitätskonzepten bildet die Grundlage für Self-Sovereign Identity (SSI). Giannopoulou und Wang definieren SSI als ein Identitätsmanagementsystem ohne zentrale Autoritäten, das auf dezentralen Architekturen basiert und Sicherheit, Privatsphäre sowie individuelle Selbstbestimmung betont. Die systematische Recherche adressierte vier zentrale, interdisziplinäre Domänen, die das Kernforschungsinteresse dieser Arbeit konstituieren. Die erste Domäne, Self-Sovereign Identity, umfasst dezentrale Identitätssysteme, in denen Nutzer die Kontrolle über ihre Identitätsdaten behalten und deren Weitergabe eigenverantwortlich regeln können. Dies wird technisch durch Standards wie Decentralized Identifiers (DIDs), Verifiable Credentials (VCs) und Verifiable Presentations (VPs) realisiert. Die zweite Domäne, Blockchain-Technologie als Identitätsinfrastruktur, adressiert die Nutzung von Distributed-Ledger-Technology (DLT) zur Sicherstellung von Integrität, Transparenz und Manipulationssicherheit im dezentralen Identitätsmanagement, insbesondere durch permissioned und permissionless Ledger-Architekturen. Die dritte Domäne, Post-Quantum Cryptography (PQC), untersucht kryptografische Verfahren, die gegen Angriffe durch leistungsstarke Quantencomputer resistent sind, einschließlich der aktuellen NIST-Standards (ML-DSA, ML-KEM, SLH-DSA) und ihrer kryptoagilen Implementierung. Die vierte Domäne schließlich, kritische Infrastrukturen (KRITIS), konzentriert sich auf die spezifischen Sicherheits-, Compliance- und Resilienzanforderungen von Systemen mit essentieller Gemeinwohlfunktion gemäß BSI- und internationalen Standards.

Die Bewertung der identifizierten Literatur zeigt, dass lediglich fünf Publikationen aus der ersten Iteration und zwei Publikationen aus der zweiten Iteration eine hohe Relevanz besitzen und gleichzeitig innovative Beiträge zu mindestens zwei Kernbereichen liefern, während die Mehrzahl der Quellen entweder thematisch breiter aufgestellt ist oder nur einzelne Teilaspekte abdeckt. Die sachliche Synthese dieser Hochrelevanz-Arbeiten offenbart ein differenziertes Bild der gegenwärtigen Forschungslandschaft.

Aus der ersten Iteration zeigen Alam, Hoffstein und Cambou konkrete Ansätze zur praktischen Erzeugung und Verteilung quantensicherer Schlüsselpaare auf Basis von Crystals-Dilithium in verteilten Netzwerken und adressieren dabei wesentlich die Integration von Multi-Faktor-Authentifizierung und Challenge-Response-Mechanismen, was sowohl für die Sicherheit verteilter Infrastrukturen als auch für zukünftige SSI-basierte Identitätslösungen zentral ist. Szymanski demonstriert die technische Machbarkeit quantensicherer, hardwaregestützter Kommunikationssysteme für das Industrial Internet of Things im KRITIS-Kontext und behandelt dabei Zero-Trust-Architekturen sowie KI-gestützte Sicherheitsmechanismen. Der Beitrag zeigt explizit den Einsatz quantensicherer Verschlüsselungsmechanismen und QKD-Netzwerke zur Gewährleistung von Resilienz und Sicherheitsanforderungen in kritischen Infrastrukturen. Nouma und Yavuz adressieren die praktische Integration quantensicherer, hybrider Hardware-assistierter Signaturen in Digital Twins und IoT-Infrastrukturen, was für ressourcenbeschränkte SSI-Lösungen unmittelbare Relevanz besitzt. Sharif et al. bieten eine umfassende Übersicht der europäischen regulatorischen Entwicklung hin zu dezentralisierten eID-Lösungen im eIDAS-Rahmenwerk und beleuchten die in eIDAS 2.0 antizipierte Entwicklung hin zu dezentralisierten Identitätsarchitekturen, die als Grundlage künftiger SSI-Lösungen dienen. Radanliev liefert eine vergleichende Analyse internationaler Cybersecurity-Standards (NIST, ISO 27001, MiCA) und deren Anwendung auf blockchain-basierte Systeme, mit explizitem Bezug zu regulatorischen Anforderungen für kritische Infrastrukturen sowie systematischer Prüfung neuerer Regularien unter Einbeziehung technologieübergreifender Aspekte wie PQC, Cloud Security und IoT.

Die zweite Iteration identifizierte zwei weitere hochrelevante Quellen. Barrett-Danes und Ahmad setzen sich systematisch, methodisch und interdisziplinär mit den Bedrohungen durch Quantum Computing für klassische Kryptosysteme auseinander und behandeln ausdrücklich Post-Quantum-Kryptografie, hybride Frameworks einschließlich QKD, Umsetzungsherausforderungen und deren Auswirkungen insbesondere für IoT-Umgebungen. Die Arbeit analysiert sowohl den Forschungsstand als auch Implementierungs- und Migrationspfade anhand konkreter Pilotstudien und adressiert Skalierbarkeit und Wirtschaftlichkeit. Die Nutzung eines PRISMA-basierten Review-Frameworks und der Fokus auf praktische Handlungs- und Politikempfehlungen verleihen der Arbeit hohe wissenschaftliche und praxisnahe Relevanz für die Entwicklung, Migration und Absicherung quantensicherer sowie zukunftsfähiger Systeme mit starker Anschlussfähigkeit an KRITIS, PQC und angrenzende sicherheitskritische Domänen. Feng et al. beschreiben eine umfassende und systematische Übersicht zu Blockchain-basierten Identity Management Systems mit explizitem Fokus auf Self-Sovereign Identity, dezentralen Identifikatoren, Interoperabilität und Sicherheitsanalyse über den gesamten Identitätslebenszyklus hinweg. Die Arbeit adressiert technologische und organisatorische Herausforderungen wie Revokation, Übertragbarkeit, Interoperabilität und Quantum-Resilienz für nutzerkontrollierte Identitätsmodelle. Die PRISMA-basierte Methodik sowie die sektorübergreifende Taxonomisierung und die systematische Ableitung aktuell ungelöster Probleme und künftiger Forschungsrichtungen machen die Quelle höchst wertvoll im Umfeld SSI, Blockchain, PQC und KRITIS.

Arbeiten mittlerer Relevanz vertiefen beispielsweise Sicherheit in IoT-Kontexten oder analysieren Blockchain in der digitalen Forensik. Atlam et al. bieten eine umfassende Analyse von Blockchain-Technologien im digitalen Forensik-Kontext mit Schwerpunkt auf Untersuchungsmethodik und Anwendungsfeldern, adressieren insbesondere die Herausforderungen beim Nachweis und der Verfolgung von Aktivitäten auf Blockchain-Systemen und beleuchten regulatorische Aspekte, jedoch ohne explizite Behandlung von SSI, PQC oder den speziellen Anforderungen kritischer Infrastrukturen. Enaya, Fernando und Kashef betrachten zentrale Aspekte von Blockchain-Technologien und Sicherheit mit Schwerpunkt auf IoT-Anwendungen und branchenspezifische Implementierungen, jedoch fehlen explizite Bezüge zu SSI, PQC und KRITIS, was die Anwendbarkeit auf alle vier Domänen einschränkt. Weniger relevante Quellen bieten überwiegend nur ergänzende Einblicke, etwa zu Cloud-Performance oder KI-gestützter Datenanalyse. Kumar et al. fokussieren auf die Weiterentwicklung von Big-Data-Analyse und Management durch künstliche Intelligenz, mit Betonung neuer Rahmenwerke für die Charakterisierung und Handhabung großer, komplexer Datensätze, jedoch ohne explizite Behandlung oder Integration von SSI, Blockchain-Technologien, PQC oder besonderen Anforderungen kritischer Infrastrukturen.

Trotz dieser differenzierten Forschungslandschaft offenbart die systematische Analyse eine signifikante Forschungslücke. Blockchain-basierte SSI-Ansätze werden als Alternative zu zentralisierten Systemen erforscht, doch die aktuelle Forschung weist weiterhin erhebliche Lücken auf. Vor allem mangelt es an der Integration quantenresistenter Algorithmen in SSI-Architekturen. Lösungen sind meist nicht auf PQC ausgelegt, wie Dutto et al. beschreiben. Die Entwicklung und Bewertung PQC-basierter Ansätze ist daher ein offenes Feld. Ebenso bestehen Defizite in der spezifischen Anpassung dieser Lösungen an die regulatorischen und technischen Anforderungen von KRITIS-Sektoren, wie vom Bundesamt für Sicherheit in der Informationstechnik dargestellt. Der Bundesbeauftragte für den Datenschutz und die Informationsfreiheit fordert, dass digitale Identitätslösungen Datenschutz und Compliance gewährleisten müssen. Bezüglich der technischen Anforderungen wird ein rein softwarebasierter Schutz als unzureichend angesehen.

Bislang existiert kein wissenschaftlicher Ansatz, der SSI, PQC, hardwarebasierte Sicherheit und regulatorische Konformität domänenspezifisch für KRITIS integriert und evaluiert. Die identifizierten Hochrelevanz-Arbeiten adressieren die Domänen SSI, Blockchain und PQC vorwiegend isoliert, ohne ihre systemische Zusammenführung methodisch und implementierungstechnisch zu klären. Darüber hinaus besteht ein Defizit an designwissenschaftlich fundierten Artefaktentwicklungen, die unter Anwendung von Design Science Research Methodologie solche integrierten Systeme systematisch konzipieren, prototypisch umsetzen und validieren. Hinsichtlich der spezifischen KRITIS-Anforderungen zeigt sich, dass einzelne Technologien adressiert werden, ohne jedoch ihre Vereinbarkeit mit BSI-Standards und Compliance-Anforderungen kritischer Infrastrukturen systematisch zu analysieren. Besonders kritisch ist die Unterrepräsentation von Mechanismen für kryptografische Agilität, also die Fähigkeit eines Systems, künftige Algorithmenupdates ohne Systemunterbrechung zu absorbieren, im Kontext von blockchain-basierten SSI-Systemen, obwohl diese für die Langzeitresilienz gegen die Quantenbedrohung fundamental sind.

Diese Lücke in der wissenschaftlichen Literatur bildet das Fundament dieser Masterarbeit. Sie wird durch die vier in Kapitel 1.3 formulierten Forschungsfragen strukturiert und motiviert einen Design-Science-Forschungsansatz, der sowohl die theoretische Erkenntnis als auch die praktische Prototypentwicklung in den Blick nimmt. Die vorliegende Arbeit adressiert diese Lücke durch die iterative Konzeption, Implementierung und Evaluation eines blockchain-basierten, quantenresistenten SSI-Prototyps für kritische Infrastrukturen, wodurch eine wissenschaftlich fundierte Grundlage für zukünftige Deployments in echten KRITIS-Umgebungen geschaffen wird.

}


