\subsection{Problemstellung und Motivation} \label{sec:Problemstellung und Motivation}

- Quantenbedrohung für aktuelle SSI-Systeme \\
- Herausforderungen in KRITIS \\
- Notwendigkeit kryptoagiler Lösungen


Die Migration zu PQC erfordert eine sorgfältige Planung mehrerer kritischer Aspekte. Organisationen müssen robuste Interoperabilitätsstrategien entwickeln, um PQC-Algorithmen nahtlos in bestehende kryptographische Systeme zu integrieren, wobei umfangreiche Kompatibilitätstests durchzuführen sind. Die Auswahl geeigneter PQC-Algorithmen sollte unter Berücksichtigung von Sicherheitsgarantien, Leistungsmerkmalen und Implementierungskomplexität erfolgen \parencite[S. 3]{mamatha_PostQuantumCryptographySecuringDigitalCommunicationQuantumEra_2024}.

Ein zentrales Herausforderung stellt die Kompatibilität mit Legacy-Systemen dar. Implementierer sollten umfassende Migrationspläne entwickeln, die gestaffelte Deployment-Strategien und Rückwärtskompatibilität vorsehen, um Störungen zu minimieren \parencite[S. 3]{mamatha_PostQuantumCryptographySecuringDigitalCommunicationQuantumEra_2024}. Gleichzeitig müssen Organisationen einschlägige kryptographische Standards und regulatorische Anforderungen erfüllen. Eine enge Zusammenarbeit mit Regulierungsbehörden und Industrie-Stakeholdern ist notwendig, um Compliance-Anforderungen zu navigieren und rechtliche Rahmenbedingungen einzuhalten \parencite[S. 3]{mamatha_PostQuantumCryptographySecuringDigitalCommunicationQuantumEra_2024}.