\subsection{Aufbau der Arbeit} \label{sec:Aufbau der Arbeit}

- Aufbau und Struktur der Masterarbeit



\subsection{ABGRENZUNG} \label{ABGRENZUNG}


\fixme{Klare Abgrenzung zu realen KRITIS-Umgebungen

Limitationen und Übertragbarkeit der Erkenntnisse}

- Keine prozessualen oder organisatorischen Compliance Anforderungen, nur technisch!
- Compliance Anforderungen eingeschränkt => ausrichtung KRTISI aber technische Umsetzung im Labor.


Limitation der ARbeit durch Umsetzungsherausforderungen

2 Implementierungszyklen

ebd.

Listings verzeichnis ==> Quellcode verzeichnis

Wegen begrenztem Umfang der Masterarbeit
    - Fokus auf SSI (von-network und indy tails server für labor only!) \\
    - 2 DSR Zyklen \\
    - funktionale Anforderungen \\ 
    - Compliance Anforderungen \\



\textbf{Abgrenzung und Limitationen der Evaluation}

Im Rahmen des FEDS-Frameworks fokussiert sich diese Arbeit bewusst auf das Evaluationsziel der \textit{Efficacy} (Wirksamkeit) und der \textit{Fidelity} (Übereinstimmung mit Anforderungen). Ziel ist der Nachweis der technischen Machbarkeit (Technical Feasibility) und der regulatorischen Konformität (Compliance Readiness) der entwickelten Artefakte.

Eine quantitative Evaluierung der \textit{Efficiency} (Performance, Latenzzeiten, Durchsatz) ist explizit \textbf{nicht} Bestandteil dieser Untersuchung. Diese Abgrenzung erfolgt aus folgenden Gründen:

\begin{itemize}
    \item \textbf{Primat der Korrektheit:} Im Kontext von KRITIS und Post-Quantum-Kryptografie ist die funktionale Korrektheit der Implementierung und die Einhaltung spezifizierter Sicherheitsparameter (z.\,B. NIST Security Levels) die notwendige Vorbedingung für jeden operativen Einsatz. Performance-Optimierungen sind erst sinnvoll, wenn die funktionale Integrität zweifelsfrei bewiesen ist.
    \item \textbf{Status der Standardisierung:} Da die verwendeten Algorithmen (ML-KEM, ML-DSA) und deren Integration in SSI-Protokolle (DIDComm v2) sich teilweise noch in der Standardisierungs- oder frühen Adoptionsphase befinden, liegt der wissenschaftliche Mehrwert primär in der Erforschung der Integrationsfähigkeit und Interoperabilität, nicht in der Laufzeitmessung von Referenzimplementierungen, die sich noch signifikant ändern können.
    \item \textbf{Artifizielle Evaluationsumgebung:} Die Evaluation findet bewusst in einer rein artifiziellen Laborumgebung statt, um isolierte funktionale Nachweise ohne Störfaktoren realer Netzwerke zu erbringen. Rückschlüsse auf die Performance in produktiven, verteilten Systemen wären auf Basis dieser Umgebung von begrenzter Validität.
\end{itemize}
