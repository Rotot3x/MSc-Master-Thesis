\newpage
\section{Theoretische Grundlagen} \label{sec:Theoretische Grundlagen}

\subsection{Self-Sovereign Identity} \label{sec:Self-Sovereign Identity}

\ac{SSI} ist ein dezentrales Identitätsparadigma, das Nutzern die vollständige Kontrolle über ihre persönlichen Daten und digitalen Identitäten ermöglicht \parencite[S. 1]{satybaldy_TaxonomyChallengesSelfSovereignIdentitySystems_2024}. Im Gegensatz zu traditionellen zentralisierten oder föderalen Identitätsmanagementsystemen, bei denen Identitätsanbieter als Intermediäre agieren, verleiht \ac{SSI} dem Einzelnen die Autorität, eigenständig über die Verfügbarkeit und Weitergabe seiner digitalen Identitätsattribute zu bestimmen \parencite[S. 1]{schardong_SelfSovereignIdentitySystematicReviewMappingTaxonomy_2022}. Diese Kontrollverschiebung nach \textcite[S. 12]{preukschat_SelfsovereignIdentityDecentralizeddigitalidentityverifiablecredentials_2021} vom Issuer und Verifier hin zum User wird in \autoref{fig:ssi_shift_in_control} dargestellt.

\begin{figure}[H]
    \centering
    \includegraphics[width=\linewidth]{ssi_shift_in_control.png}
    \caption{Kontrollverschiebung vom Issuer und Verifier zum User}
    \begin{flushleft}
    \textit{Anmerkung.} Aus \textcite[S. 12]{preukschat_SelfsovereignIdentityDecentralizeddigitalidentityverifiablecredentials_2021}.
    \end{flushleft}
    \label{fig:ssi_shift_in_control}
\end{figure}

Die Abgrenzung gegenüber anderen Identitätsmanagementsystemen wird durch die zehn Prinzipien von \ac{SSI} \textcite{allen_PathSelfSovereignIdentity_2016} verdeutlicht, die in \autoref{tab:allen_ten_ssi_principles} aufgeführt sind. Diese Prinzipien waren als Reaktion auf die Nachteile bisheriger digitaler Identitätsmanagementsysteme entstanden, beschreiben keine spezifische technische Lösung, aber fungieren als konzeptionelle Anforderungen für die Umsetzung einer \ac{SSI}-Lösung \parencite[S. 12--13]{strueker_SelfSovereignIdentityGrundlagenAnwendungenundPotenzialeportablerdigitalerIdentitaeten_2021}.

\subsubsection{Kernkomponenten der SSI-Architektur}

Ein funktionierendes \ac{SSI}-System (\autoref{fig:ssi_architektur}) besteht aus drei essentiellen Rollen, unterstützenden Technologiekomponenten und einer darunterliegenden dezentralen Infrastruktur \parencite[S. 4]{naghmouchi_SystematicReviewLayeredFrameworkPrivacybyDesignSelfSovereignIdentitySystems_2025}. Die Interaktion zwischen autonomen Akteuren erfolgt durch eigenständig generierte Identifikatoren und kryptographisches Material, unterstützt durch Wallet-Applikationen \parencite[S. 4]{naghmouchi_SystematicReviewLayeredFrameworkPrivacybyDesignSelfSovereignIdentitySystems_2025}.

\begin{figure}[H]
    \centering
    \includegraphics[width=\linewidth]{ssi_architektur.png}
    \caption{SSI-Systemarchitektur}
    \begin{flushleft}
    \textit{Anmerkung.} Aus \textcite[S. 4]{naghmouchi_SystematicReviewLayeredFrameworkPrivacybyDesignSelfSovereignIdentitySystems_2025}.
    \end{flushleft}
    \label{fig:ssi_architektur}
\end{figure}

Der Issuer ist eine vertrauenswürdige Partei, dessen Identitätsinformationen und Public Key öffentlich einsehbar sind \parencite[S. 20]{strueker_SelfSovereignIdentityGrundlagenAnwendungenundPotenzialeportablerdigitalerIdentitaeten_2021}. Der Holder steht im Zentrum des \ac{SSI}-Modells, fordert Credentials von Issuern an, speichert diese in einem digitalen Wallet und präsentiert bei Bedarf Nachweise an Verifier \parencite[S. 4]{naghmouchi_SystematicReviewLayeredFrameworkPrivacybyDesignSelfSovereignIdentitySystems_2025}. Der Verifier fordert Nachweise an, prüft deren Validität und gewährt oder verweigert auf dieser Basis Zugang zu Diensten \parencite[S. 4]{naghmouchi_SystematicReviewLayeredFrameworkPrivacybyDesignSelfSovereignIdentitySystems_2025}. Diese Rollen sind austauschbar, sodass ein Nutzer kontextabhängig zwischen ihnen wechseln kann \parencite[S. 4]{naghmouchi_SystematicReviewLayeredFrameworkPrivacybyDesignSelfSovereignIdentitySystems_2025}.

\subsubsection{Infrastruktur}

Die Infrastruktur-Schicht bildet die Basis von \ac{SSI}-Systemen und dient als \ac{DPKI} zur Verwaltung von Public Keys, als \ac{VDR} \parencite[S. 201]{preukschat_SelfsovereignIdentityDecentralizeddigitalidentityverifiablecredentials_2021} für öffentliche Identifikatoren und \ac{DID}-Dokumente, sowie zur Veröffentlichung von Credential Schemas, Definitions und Revokationsinformationen \parencite[S. 4--5, 7]{naghmouchi_SystematicReviewLayeredFrameworkPrivacybyDesignSelfSovereignIdentitySystems_2025}. Die Infrastruktur kann \ac{DLT}, Blockchain-Technologie und/oder Web-basierte Technologie nutzen \parencite[S. 8]{naghmouchi_SystematicReviewLayeredFrameworkPrivacybyDesignSelfSovereignIdentitySystems_2025}. Dezentrale Infrastrukturen werden bevorzugt eingesetzt, um die Prinzipien der Nutzerautonomie und -kontrolle zu erfüllen, da eine private Infrastruktur unter Kontrolle weniger Entitäten die Selbstsouveränität der Nutzer nicht gewährleisten kann \parencite[S. 4]{naghmouchi_SystematicReviewLayeredFrameworkPrivacybyDesignSelfSovereignIdentitySystems_2025}.

\subsubsection{Identifikatoren und kryptographisches Material}

\ac{DID} sind weltweit eindeutige, kryptographisch verifizierbare Identifikatoren, die dem Aufbau dargestellt in \autoref{fig:ssi_did_aufbau} folgen und es verschiedenen Parteien ermöglichen, mehrere Identitäten zu kontrollieren, um ihre Privatsphäre zu schützen \parencite[S. 22]{strueker_SelfSovereignIdentityGrundlagenAnwendungenundPotenzialeportablerdigitalerIdentitaeten_2021}. In Verbindung mit kryptographischem Material wie Public Keys werden sie zur Identifikation von \ac{SSI}-Nutzern verwendet, unabhängig von ihrer eingenommenen Rolle \parencite[S. 5]{naghmouchi_SystematicReviewLayeredFrameworkPrivacybyDesignSelfSovereignIdentitySystems_2025}. Die Authentifizierung über Identifikatoren stellt die zwei grundlegenden Funktionalitäten des Identitätsmanagements (Identifikation und Authentifizierung) bereit \parencite[S. 5]{naghmouchi_SystematicReviewLayeredFrameworkPrivacybyDesignSelfSovereignIdentitySystems_2025}.

\begin{figure}[H]
    \centering
    \includegraphics[width=\linewidth]{ssi_did_aufbau.png}
    \caption{Bestandteile eines DID}
    \begin{flushleft}
    \textit{Anmerkung.} Aus \textcite[S. 4]{naghmouchi_SystematicReviewLayeredFrameworkPrivacybyDesignSelfSovereignIdentitySystems_2025}.
    \end{flushleft}
    \label{fig:ssi_did_aufbau}
\end{figure}

SSI nutzt Identifikatoren, die keine Registrierungsstellen oder Identity Provider erfordern, wobei die häufigsten Identifikatoren dem \ac{DID}-Standard des W3C folgen \parencite[S. 5]{naghmouchi_SystematicReviewLayeredFrameworkPrivacybyDesignSelfSovereignIdentitySystems_2025}. \ac{DID} werden von ihren Eigentümern erstellt, sind mit einem Public Key verknüpft, der einem Private Key zugeordnet ist, sodass die Eigentümerschaft durch Challenge-Response-Verfahren verifiziert werden kann \parencite[S. 5]{naghmouchi_SystematicReviewLayeredFrameworkPrivacybyDesignSelfSovereignIdentitySystems_2025}. Jede \ac{DID} ist mit einem \ac{DID}-Dokument verbunden, in welchem ein Service-Endpoint enthalten ist \parencite[S. 23]{strueker_SelfSovereignIdentityGrundlagenAnwendungenundPotenzialeportablerdigitalerIdentitaeten_2021}.

\subsubsection{Credentials und Presentations}

\ac{VC} sind digital signierte Sammlungen von Attributen, die einem Standard unterliegen und kryptographisch verifizierbar sind \parencite[S. 16--17]{strueker_SelfSovereignIdentityGrundlagenAnwendungenundPotenzialeportablerdigitalerIdentitaeten_2021}. \ac{VC} enthalten Revokationsmechanismen und kryptographisches Material, welches die Integrität des Credentials, die Identifizierung des Issuers und Nicht-Abstreitbarkeit gewährleistet \parencite[S. 5]{schardong_SelfSovereignIdentitySystematicReviewMappingTaxonomy_2022}. Credentials enthalten einen Proof, der sie hinsichtlich Urheberschaft (Issuer-Seite) und Besitz (Holder-Seite) verifizierbar macht \parencite[S. 5]{naghmouchi_SystematicReviewLayeredFrameworkPrivacybyDesignSelfSovereignIdentitySystems_2025}. Credentials werden üblicherweise mit Schlüsseln signiert, die mit vom Nutzer erstellten \ac{DID} verknüpft sind \parencite[S. 5]{naghmouchi_SystematicReviewLayeredFrameworkPrivacybyDesignSelfSovereignIdentitySystems_2025}.

Ein Credential ist mit einem Credential Schema verbunden, das Attribute, Formate und Typen beschreibt, während eine Credential Definition die Issuer-spezifische Instanz mit kryptographischen Signatur- und Verifikationsmethoden darstellt \parencite[S. 5]{naghmouchi_SystematicReviewLayeredFrameworkPrivacybyDesignSelfSovereignIdentitySystems_2025}. Selective Disclosure definiert, dass der Holder nur spezifische Attribute aus dem Credential offenbaren muss, nicht das gesamte Credential \parencite[S. 29]{strueker_SelfSovereignIdentityGrundlagenAnwendungenundPotenzialeportablerdigitalerIdentitaeten_2021}. Die Präsentationsphase besteht darin, dass der Holder die angeforderten Attribute einem Verifier präsentiert. Dies geschieht entweder durch Vorlage vollständiger Credentials oder durch Selektion und Kombination einzelner Attribute aus verschiedenen Credentials mittels \ac{VP} \parencite[S. 5]{naghmouchi_SystematicReviewLayeredFrameworkPrivacybyDesignSelfSovereignIdentitySystems_2025}.

\subsubsection{Wallet-Applikation (Edge Agent und Cloud Agent)}

Kommunikationsprotokolle zwischen Peers, die Speicherung samt Austausch von \ac{VC} und \ac{VP}, sowie das Signieren und Verifizieren von Credentials werden über eine Wallet-Applikation abgewickelt \parencite[S. 5]{naghmouchi_SystematicReviewLayeredFrameworkPrivacybyDesignSelfSovereignIdentitySystems_2025}. Digital Wallets ermöglichen Holdern ebenfalls die Generierung \parencite[S. 5]{naghmouchi_SystematicReviewLayeredFrameworkPrivacybyDesignSelfSovereignIdentitySystems_2025}, die sichere Speicherung und Verwaltung von Private Keys und \ac{DID}-Dokumenten \parencite[S. 18]{strueker_SelfSovereignIdentityGrundlagenAnwendungenundPotenzialeportablerdigitalerIdentitaeten_2021}, sowie deren Veröffentlichung auf der Infrastruktur, wodurch eine Befähigung zur Peer-to-Peer-Authentifizierung durch \ac{DID}-Resolution stattfinden kann \parencite[S. 5]{naghmouchi_SystematicReviewLayeredFrameworkPrivacybyDesignSelfSovereignIdentitySystems_2025}. Ein Wallet besteht typischerweise aus einem Cloud Agent, der als Online-Service in der Cloud läuft, und einem Edge Agent, der auf dem Gerät des Nutzers installiert ist \parencite[S. 5]{naghmouchi_SystematicReviewLayeredFrameworkPrivacybyDesignSelfSovereignIdentitySystems_2025}.

\subsubsection{Vertrauensmechanismus und Verifikationsprozess}

Das Vertrauen in \ac{SSI}-Systemen basiert nicht auf der Abhängigkeit von zentralisierten Intermediären, sondern auf kryptographischen und dezentralen Mechanismen. Wenn ein Holder eine \ac{VP} an einen Verifier sendet, prüft dieser wer das zugrundeliegende Credential signiert hat, ob die Presentation korrekt konstruiert ist, und ob das Credential noch gültig ist und nicht revocated wurde \parencite[S. 6]{schardong_SelfSovereignIdentitySystematicReviewMappingTaxonomy_2022}. Der Verifier kann dann eigenständig entscheiden, ob er dem Issuer vertraut oder nicht \parencite[S. 6]{schardong_SelfSovereignIdentitySystematicReviewMappingTaxonomy_2022}. Revokationsinformationen sind öffentlich verfügbar und können anonym abgefragt werden, ohne dass die spezifische Identität des Abfragenden offengelegt wird \parencite[S. 6]{schardong_SelfSovereignIdentitySystematicReviewMappingTaxonomy_2022}.

Ein kritischer technischer Enabler ist der \ac{ZKP}, ein kryptographisches Verfahren, welches es einem Prover ermöglicht, einem Verifier zu beweisen, dass eine Aussage wahr ist, ohne die zugrundeliegenden Daten preiszugeben \parencite[S. 6]{schardong_SelfSovereignIdentitySystematicReviewMappingTaxonomy_2022}. Dies ermöglicht es Holdern beispielsweise, nachzuweisen, dass sie älter als 18 Jahre sind, ohne ihr genaues Geburtsdatum preiszugeben \parencite[S. 6]{schardong_SelfSovereignIdentitySystematicReviewMappingTaxonomy_2022}.

\subsection{Blockchain-Technologie}
\label{sec:Blockchain-Technologie}

Die Blockchain-Technologie wird in der Literatur als eine spezifische 
Ausprägung der \ac{DLT} klassifiziert, die die für die \ac{SSI}-Architektur erforderlichen Charakteristika Zuverlässigkeit, Unveränderlichkeit, Transparenz, Irreversibilität und Auditierbarkeit durch zeitchronologische Anordnung erfüllt \parencite[S. 27]{strueker_SelfSovereignIdentityGrundlagenAnwendungenundPotenzialeportablerdigitalerIdentitaeten_2021}.

Strukturell besteht eine Blockchain aus einer sequenziellen Kette von Datenblöcken. Jeder Block setzt sich aus den beiden Hauptkomponenten Block-Header und Block-Body zusammen \parencite[S. 647]{moosavi_BlockchainTechnologyStructureApplicationsSurvey_2024}. 

Der Block-Header enthält essenzielle Metadaten, darunter den Zeitstempel, eine Nonce (für Proof-of-Work-Systeme) und vor allem den kryptographischen Hash-Wert des vorherigen Blocks (Previous Block Hash) \parencite[S. 648]{moosavi_BlockchainTechnologyStructureApplicationsSurvey_2024}. Durch diese kryptographische Verkettung (\autoref{fig:blockchain_sequence}) entsteht eine untrennbare Historie. Die Modifikation eines einzigen Transaktionsdatensatzes würde den Hash des Blocks verändern, was wiederum die Ungültigkeit aller nachfolgenden Blöcke zur Folge hätte \parencite[S. 648]{moosavi_BlockchainTechnologyStructureApplicationsSurvey_2024}. Der Merkle-Root-Hash im Block-Header ermöglicht somit eine effiziente Verifikation der Datenintegrität, ohne alle Knoten der Merkle-Tree-Struktur überprüfen zu müssen \parencite[S. 4]{kuznetsov_MerkleTreesBlockchainStudyCollisionProbabilitySecurityImplications_2024}.

Der Block-Body enthält die eigentlichen Transaktionen, die häufig über 
einen Merkle Tree (Hash-Baum) effizient organisiert sind \parencite[S. 648]{moosavi_BlockchainTechnologyStructureApplicationsSurvey_2024}.

\begin{figure}[H]
    \centering
    \includegraphics[width=\linewidth]{blockchain_sequence}
    \caption{Blockchain Sequenz}
    \begin{flushleft}
    \textit{Anmerkung.} Eigene Darstellung nach \textcite[S. 648]{moosavi_BlockchainTechnologyStructureApplicationsSurvey_2024}.
    \end{flushleft}
    \label{fig:blockchain_sequence}
\end{figure}

\subsubsection{Klassifizierung von Ledger-Architekturen}

Blockchain-Netzwerke werden primär nach dem Zugriffsprivileg in die beiden Modelle Permissionless und Permissioned klassifiziert (\autoref{fig:blockchain_classification}) 
\parencite[S. 649]{moosavi_BlockchainTechnologyStructureApplicationsSurvey_2024}.

Permissionless Blockchains sind vollständig offen, jeder Teilnehmer kann dem Netzwerk beitreten, lesen und schreiben \parencite[S. 649]{moosavi_BlockchainTechnologyStructureApplicationsSurvey_2024}.

Permissioned Blockchains werden in zwei Hauptgruppen unterteilt, private Blockchains und Konsortium-Blockchains.
Bei der ersten Gruppe hat nur eine ausgewählte Anzahl von Knoten Zugriff auf das Netzwerk und kann Transaktionen validieren. Diese Einschränkung beschleunigt die Transaktionen, sodass selbst bei wachsender Blockchain die Transaktionszeit effizient bleibt. In dieser Art von Blockchain können vollständig sichere Transaktionen durchgeführt werden. Der Hauptnachteil einer privaten Blockchain besteht darin, dass die Validierung zentralisiert wird, da sie unter der Kontrolle einer einzelnen Organisation steht.
Eine Konsortium-Blockchain weist einige Besonderheiten einer öffentlichen oder privaten Blockchain auf, bei der mehrere Organisationen das Netzwerk kontrollieren \parencite[S. 649--650]{moosavi_BlockchainTechnologyStructureApplicationsSurvey_2024}.

Permissioned Blockchains werden für Identitätsmanagement-Lösungen oft präferiert, da sie Datenschutzanforderungen, wie Data Segregation, besser abbilden können \parencite[S. 8--10]{naghmouchi_SystematicReviewLayeredFrameworkPrivacybyDesignSelfSovereignIdentitySystems_2025}.

\begin{figure}[H]
    \centering
    \includegraphics[width=\linewidth]{blockchain_classification}
    \caption{Blockchain Klassifizierung}
    \begin{flushleft}
    \textit{Anmerkung.} Eigene Darstellung nach \textcite[S. 649]{moosavi_BlockchainTechnologyStructureApplicationsSurvey_2024}.
    \end{flushleft}
    \label{fig:blockchain_classification}
\end{figure}

\subsubsection{Die Rolle der Blockchain in SSI-Systemen}

In einer \ac{SSI}-Architektur fungiert die Blockchain nicht als Speicherort für Identitätsdaten selbst \parencite[S. 27]{strueker_SelfSovereignIdentityGrundlagenAnwendungenundPotenzialeportablerdigitalerIdentitaeten_2021}, sondern als \ac{VDR} \parencite[S. 7]{satybaldy_TaxonomyChallengesSelfSovereignIdentitySystems_2024}. Sie dient als manipuliationssichere Quelle für kryptographisches Material, das für die Verifikation von Credentials notwendig ist, ohne dabei personenbezogene Daten preiszugeben \parencite[S. 27]{strueker_SelfSovereignIdentityGrundlagenAnwendungenundPotenzialeportablerdigitalerIdentitaeten_2021}.

Der Ledger übernimmt im \ac{SSI}-Kontext spezifische, klar definierte Aufgaben:

\textbf{\ac{DPKI}:} Anstatt sich auf eine zentrale \ac{CA} zu verlassen, werden Public Keys (in Form von \ac{DID}-Dokumenten) auf der Blockchain verankert \parencite[S. 17]{strueker_SelfSovereignIdentityGrundlagenAnwendungenundPotenzialeportablerdigitalerIdentitaeten_2021}. Da der Ledger manipulationssicher ist, kann jeder Verifier den Public Key eines Issuers zweifelsfrei abrufen und dessen Signaturen validieren, ohne den Issuer direkt kontaktieren zu müssen \parencite[S. 21]{strueker_SelfSovereignIdentityGrundlagenAnwendungenundPotenzialeportablerdigitalerIdentitaeten_2021}.

\textbf{Schema- und Credential-Definitionen:} Um Interoperabilität zu gewährleisten, werden die Strukturdaten von Credentials (Schemata) auf dem Ledger publiziert. Dies stellt sicher, dass alle Parteien dieselbe Datenstruktur für ein spezifisches Credential verwenden und interpretieren können \parencite[S. 19]{strueker_SelfSovereignIdentityGrundlagenAnwendungenundPotenzialeportablerdigitalerIdentitaeten_2021}.

\textbf{Revocation Registries:} Ein kritischer Aspekt ist die Revocation von Credentials. \ac{SSI}-Systeme nutzen hierfür oft kryptographische Akkumulatoren auf der Blockchain \parencite[S. 26--28]{strueker_SelfSovereignIdentityGrundlagenAnwendungenundPotenzialeportablerdigitalerIdentitaeten_2021}. Diese erlauben es einem Holder, mittels eines Non-Revocation Proofs zu beweisen, dass sein Credential nicht widerrufen wurde, ohne dass der Issuer bei jeder Prüfung kontaktiert werden muss \parencite[S. 29]{strueker_SelfSovereignIdentityGrundlagenAnwendungenundPotenzialeportablerdigitalerIdentitaeten_2021}.

\subsection{Post-Quantum Kryptografie} \label{sec:Post-Quantum Kryptografie}

\ac{PQC} befasst sich mit kryptographischen Verfahren, die entwickelt wurden, um Angriffen durch Quantencomputer zu widerstehen \parencite[S. 2]{mamatha_PostQuantumCryptographySecuringDigitalCommunicationQuantumEra_2024}, da davon ausgegangen wird, dass populäre Public-Key-Algorithmen wie \ac{RSA}, DSA und ECDSA durch den Shor-Algorithmus auf hinreichend leistungsfähigen Quantencomputern gebrochen werden \parencite[S. 1--2]{bernstein_PostquantumCryptography_2009}. Während praktische Quantencomputer noch in der Entwicklung sind \parencite[S. 1--2]{chhetri_PostQuantumCryptographyQuantumSafeSecurityComprehensiveSurvey_2025}, arbeiten Regierungen, Normungsorganisationen und Akteure der Industrie bereits an Übergangsstrategien, um sensible Informationen und \ac{KRITIS} langfristig zu schützen \parencite[S. 6]{mamatha_PostQuantumCryptographySecuringDigitalCommunicationQuantumEra_2024}.

\pagebreak

\subsubsection{Abgrenzung zu Quantum Cryptography}

\ac{PQC} und Quantum Cryptography unterscheiden sich grundlegend in Scope, Sicherheitsgrundlagen und praktischer Anwendbarkeit. Während Quantum Cryptography ausschließlich der Schlüsselexpansion, der Erweiterung eines kurzen gemeinsamen Geheimnisses in einen längeren Bitstrom, dient,deckt \ac{PQC} ein breites Spektrum kryptographischer Aufgaben ab, einschließlich digitaler Signaturen und Public-Key-Verschlüsselung \parencite[S. 13--14]{bernstein_PostquantumCryptography_2009}.

Ein entscheidender Unterschied liegt in den Sicherheitsannahmen. Während \ac{PQC} teilweise auf Vermutungen beruht, basiert Quantum Cryptography theoretisch auf etablierten Gesetzen der Quantenmechanik. Praktisch jedoch erfordert Quantum Cryptography spezialisierte, kostspielige Hardware und direkte Glasfaserverbindungen, wohingegen \ac{PQC}-Systeme mit konventioneller Computertechnik implementierbar sind und über bestehende Netzwerke funktionieren. Diese Unterschiede machen \ac{PQC} zur praktikablen Lösung für den Schutz vor Quantencomputern im Internet-Maßstab, während Quantum Cryptography auf spezialisierten Anwendungen mit hohem Budgetaufwand beschränkt bleibt \parencite[S. 13--14]{bernstein_PostquantumCryptography_2009}.

\subsubsection{Kategorien von PQC-Algorithmen}

Die prominentesten \ac{PQC}-Algorithmen lassen sich in mehrere Kategorien einteilen, die in \autoref{tab:übersicht-pqc-algorithmus-kategorien} dargestellt sind.

\begin{longtable}{L{2cm}L{4cm}L{4cm}L{2cm}}
    \caption{Übersicht ausgewählter PQC-Algorithmus-Kategorien} \\
    \label{tab:übersicht-pqc-algorithmus-kategorien} \\
    \toprule
    \textbf{Typ} & \textbf{Vorteil} & \textbf{Nachteil} & \textbf{Algorithmus} \\
    \midrule
    \endfirsthead
    \multicolumn{3}{l}{\textit{Tabelle \thetable\ (Fortsetzung)}} \\
    \toprule
    \textbf{Typ} & \textbf{Vorteil} & \textbf{Nachteil} & \textbf{Algorithmus} \\
    \midrule
    \endhead
    \midrule
    \multicolumn{4}{r}{\textit{Fortsetzung auf nächster Seite}} \\
    \endfoot
    \bottomrule
    \multicolumn{4}{p{\linewidth}}{\textit{Anmerkung.} Eigene Darstellung nach \textcite[S. 2]{mamatha_PostQuantumCryptographySecuringDigitalCommunicationQuantumEra_2024}.} \\
    \endlastfoot
    Gitter & Hohe Betriebsgeschwindigkeit & Schwierige Parametereinstellungen & CRYSTALS-DILITHIUM, FALCON \\
    \midrule
    Code & Kleine Signaturgrößen, Hohe Betriebsgeschwindigkeit & Große Schlüsselgrößen & Classic McEliece, BIKE, HQC \\
    \midrule
    Hash & Sicherheitsnachweis möglich & Große Signaturgrößen & SPHINCS+, PICNIC \\
    \midrule
    Multivariat & Schnelle Ver- und Entschlüsslungsgeschwindigkeiten & Große Schlüsselgrößen & Rainbow \\
\end{longtable}

Die gitterbasierte Kryptografie nutzt die Schwierigkeit mathematischer Probleme auf Gittern wie dem Shortest Vector Problem oder dem Learning With Errors Problem, die als quantenresistent gelten \parencite[S. 2]{mamatha_PostQuantumCryptographySecuringDigitalCommunicationQuantumEra_2024}. Bei der codebasierten Kryptografie werden fehlerkorrigierende Codes wie Goppa-Codes oder McEliece-Kryptosysteme eingesetzt, deren Sicherheit auf der Schwierigkeit des Dekodieren bestimmter linearer Codes beruht \parencite[S. 2]{mamatha_PostQuantumCryptographySecuringDigitalCommunicationQuantumEra_2024}. Die hashbasierte Kryptografie basiert auf kryptographischen Hash-Funktionen zur Bereitstellung digitaler Signaturen und ist nicht direkt durch Quantenangriffe gefährdet \parencite[S. 2]{mamatha_PostQuantumCryptographySecuringDigitalCommunicationQuantumEra_2024}. Die multivariate Polynom-Kryptografie hingegen stützt sich auf die Schwierigkeit, Systeme multivariater Polynomgleichungen zu lösen, was als quantenresistent gilt.

\subsubsection{NIST-Standardisierung der PQC-Algorithmen}

Im August 2024 hat das \ac{NIST} drei Federal Information Processing Standards veröffentlicht, die quantenresistente kryptographische Verfahren für die Bundesbehörden der USA verpflichtend einführen \parencite[S. i]{nationalinstituteofstandardsandtechnologyus_ModulelatticebasedDigitalSignaturestandard_2024,nationalinstituteofstandardsandtechnologyus_ModulelatticebasedKeyencapsulationMechanismstandard_2024,nationalinstituteofstandardsandtechnologyus_StatelessHashbasedDigitalsignaturestandard_2024}. Mit FIPS 203 wurde ML-KEM als gitterbasierter Schlüsselkapselungsmechanismus standardisiert \parencite[S. i]{nationalinstituteofstandardsandtechnologyus_ModulelatticebasedKeyencapsulationMechanismstandard_2024}. FIPS 204 spezifiziert ML-DSA als gitterbasierten digitalen Signaturalgorithmus, der aus CRYSTALS-DILITHIUM abgeleitet wurde \parencite[S. 1]{nationalinstituteofstandardsandtechnologyus_ModulelatticebasedDigitalSignaturestandard_2024} und als sicher gegen Angreifer mit großen Quantencomputern gilt \parencite[S. i]{nationalinstituteofstandardsandtechnologyus_ModulelatticebasedDigitalSignaturestandard_2024}. Als drittes Verfahren wurde mit FIPS 205 der zustandslose hashbasierte Signaturalgorithmus SLH-DSA standardisiert, der auf SPHINCS+ basiert und dessen Sicherheit auf der Schwierigkeit des Findens von Urbildern für kryptographische Hashfunktionen beruht \parencite[S. 1]{nationalinstituteofstandardsandtechnologyus_StatelessHashbasedDigitalsignaturestandard_2024}.

\subsection{Kritische Infrastrukturen und Compliance}
\label{sec:Kritische Infrastrukturen und Compliance}

Gemäß \textcite[§ 2 Abs. 10 Satz 1]{bundesministeriumderjustiz_GesetzUeberBundesamtfuerSicherheitInformationstechnikBSIGesetzBSIG_2009} werden \ac{KRITIS} als Einrichtungen, Anlagen oder Teile davon definiert, die zwei zentrale Kriterien erfüllen. Zum einen müssen sie einem der Sektoren Energie, Informationstechnik und Telekommunikation, Transport und Verkehr, Gesundheit, Wasser, Ernährung, Finanz- und Versicherungswesen oder Siedlungsabfallentsorgung angehören. Zum anderen ist eine hohe Bedeutung für das Funktionieren des Gemeinwesens erforderlich, da ein Ausfall oder eine Beeinträchtigung dieser Systeme erhebliche Versorgungsengpässe oder Gefährdungen für die öffentliche Sicherheit zur Folge hätte.

Die konkrete Bestimmung, welche Einrichtungen und Anlagen als \ac{KRITIS} gelten, wird nach \textcite[§ 10 Abs. 1]{bundesministeriumderjustiz_GesetzUeberBundesamtfuerSicherheitInformationstechnikBSIGesetzBSIG_2009} durch eine Rechtsverordnung des Bundesministeriums des Innern, für Bau und Heimat festgelegt. Diese Verordnung erfolgt nach Anhörung von Vertretern der Wissenschaft, der betroffenen Betreiber und der betroffenen Wirtschaftsverbände im Einvernehmen mit zahlreichen weiteren Bundesministerien. Die Verordnung bestimmt dabei die in den jeweiligen Sektoren als kritisch anzusehenden Dienstleistungen sowie deren als bedeutend anzusehende Versorgungsgrade. Zur Operationalisierung dieser Kriterien werden branchenspezifische Schwellenwerte herangezogen, die für jede als kritisch anzusehende Dienstleistung im jeweiligen Sektor individuell festgelegt werden. Diese systematische Klassifizierung ermöglicht es, die Schutzmaßnahmen und Compliance-Anforderungen angemessen auf die Risiken und Bedeutsamkeit der einzelnen Infrastrukturen abzustimmen.

\subsubsection{Regulatorische Anforderungen}

Die regulatorischen Anforderungen an die Informationssicherheit von \ac{KRITIS} erfordern den Einsatz kryptographischer Verfahren, die dem Stand der Technik entsprechen. Gemäß \textcite[§ 8a Abs. 1]{bundesministeriumderjustiz_GesetzUeberBundesamtfuerSicherheitInformationstechnikBSIGesetzBSIG_2009} muss ein \ac{KRITIS}-Betreiber die zur Erbringung der kritischen Dienstleistung notwendigen informationstechnischen Systeme und Komponenten so gestalten, dass diese gegen Störungen der Verfügbarkeit, Integrität, Authentizität und Vertraulichkeit angemessen geschützt sind \parencite[S. 7]{bsi_KonkretisierungKRITISAnforderungen8aAbsatz1undAbsatz1aBSIG_2024}. Mit Blick auf die fortschreitende Entwicklung von Quantencomputern gewinnt die Transition zu \ac{PQC} zunehmend an Bedeutung für die Erfüllung regulatorischer Anforderungen.

Die \ac{DSGVO} fordert in Artikel 32 geeignete technische und organisatorische Maßnahmen zum Schutz personenbezogener Daten, einschließlich Verschlüsselung als Sicherheitsmaßnahme nach dem Stand der Technik \parencite[S. 51]{daseuropaeischeparlamentundderratdereuropaeischenunion_VerordnungEU2016679EuropaeischenParlamentsundRatesvom27April2016_2016}. Diese Anforderung wird insbesondere für langfristig schutzbedürftige Daten relevant, da diese angesichts der \enquote{Harvest Now, Decrypt Later}-Strategie \parencite[S. 344--345]{geremew_PreparingCriticalInfrastructurePostQuantumCryptographyStrategiesTransitioningAheadCryptanalyticallyRelevantQuantumComputing_2024} gefährdet sind. Zur Begegnung dieser Bedrohung wird eine rechtzeitige Migration zu quantensicheren Signaturverfahren empfohlen, insbesondere für Anwendungsfälle mit langen Migrationszeiten \parencite[S. 21--22]{bsi_BSITR021021KryptographischeVerfahrenEmpfehlungenundSchluessellaengenVersion202501_2025}.

Für die Implementierung kryptographischer Systeme in \ac{KRITIS}-Umgebungen ist dabei die Kryptoagilität (Kapitel~\ref{sec:Kryptoagilitaet}) von zentraler Bedeutung, die es ermöglicht, kryptographische Algorithmen, Schlüssellängen und Schlüsselgenerierungsverfahren ohne wesentliche Änderungen am Gesamtsystem auszutauschen \parencite[S. xii]{bsi_BSITR021021KryptographischeVerfahrenEmpfehlungenundSchluessellaengenVersion202501_2025}. Das \ac{BSI} empfiehlt gemeinsam mit europäischen Partnerbehörden, die Migration zu quantensicheren Verfahren für hochsensitive Anwendungen bis spätestens Ende 2030 abzuschließen \parencite[S. 30]{bsi_BSITR021021KryptographischeVerfahrenEmpfehlungenundSchluessellaengenVersion202501_2025}.

\subsubsection{Compliance-Vorgaben}

Das \ac{BSI} konkretisiert die regulatorischen Anforderungen durch die Veröffentlichung technischer Richtlinien, welche die Grundlage für die Erfüllung der Compliance-Vorgaben bilden \parencite[S. 18--21]{bsi_BSITR021021KryptographischeVerfahrenEmpfehlungenundSchluessellaengenVersion202501_2025}. Die Technische Richtlinie TR-02102-1 empfiehlt konkrete kryptographische Verfahren und Schlüssellängen, die das erforderliche Sicherheitsniveau von mindestens 120 Bit erreichen \parencite[S. 18--19]{bsi_BSITR021021KryptographischeVerfahrenEmpfehlungenundSchluessellaengenVersion202501_2025}.

Im Kontext der kryptographischen Sicherheit fordert \textcite[Control A.8.24]{iso/iec_ISOIEC270012022InformationsecuritycybersecurityprivacyprotectionInformationsecuritymanagement_2022}, dass Regeln für den wirksamen Einsatz von Kryptografie, einschließlich Schlüsselverwaltung, definiert und umgesetzt werden müssen \parencite[S. 17]{iso/iec_ISOIEC270012022InformationsecuritycybersecurityprivacyprotectionInformationsecuritymanagement_2022}. Dies umfasst auch die Berücksichtigung zukünftiger kryptographischer Entwicklungen und der Notwendigkeit zur Transition.

Die vorliegende Technische Richtlinie empfiehlt den hybriden Einsatz quantensicherer Schlüsseleinigungsverfahren in Kombination mit klassischen Verfahren zum langfristigen Schutz der Vertraulichkeit \parencite[S. 21]{bsi_BSITR021021KryptographischeVerfahrenEmpfehlungenundSchluessellaengenVersion202501_2025}.

Die konkrete Umsetzung dieser Anforderungen erfordert dabei die Berücksichtigung spezifischer Empfehlungen für Transportprotokolle. Standards für die Nutzung quantensicherer Verfahren in \ac{TLS} werden derzeit entwickelt und erprobt, wobei das \ac{BSI} beabsichtigt, quantensichere Verfahren in hybrider Nutzung mit empfohlenen klassischen Verfahren zu empfehlen, sobald geeignete Standards verabschiedet wurden \parencite[S. 6]{bsi_TechnischeRichtlinieTR021022KryptographischeVerfahrenEmpfehlungenundSchlussellangenTeil2VerwendungTransport_2025}.


\subsection{Kryptoagilität} \label{sec:Kryptoagilitaet}

Kryptoagilität beschreibt die Fähigkeit kryptographischer Systeme, verwendete Verfahren, Parameter und Implementierungen flexibel und mit vertretbarem Aufwand an neue Sicherheitsanforderungen anzupassen, ohne die grundlegende Systemarchitektur vollständig überarbeiten zu müssen \parencite[S. 101--102]{mehrez_CryptoAgilityProperties_2018}. Diese Fähigkeit ist zentral, um auf neu entdeckte Schwachstellen, technologische Fortschritte oder disruptive Bedrohungen wie Quantencomputer zu reagieren. Dies erfolgt durch den Austausch kryptographischer Algorithmen oder die Ergänzung zusätzlicher Schutzmechanismen \parencites[S. 99--100]{mehrez_CryptoAgilityProperties_2018}[S. 1]{marchesi_SurveyCryptoagilityAgilePracticeslightquantumresistance_2025}.

Um die in \autoref{tab:kryptoagilitaets-eigenschaften} dargestellten Eigenschaften der Kryptoagilität zu erfüllen, basieren Kryptoagile Architekturen auf modularen Komponenten \parencite[S. 102--103]{mehrez_CryptoAgilityProperties_2018}.

\begin{longtable}{L{3cm}L{11cm}}
    \caption{Eigenschaften der Kryptoagilität} \\
    \label{tab:kryptoagilitaets-eigenschaften} \\
    \toprule
    \textbf{Eigenschaft} & \textbf{Beschreibung} \\
    \midrule
    \endfirsthead
    \multicolumn{2}{l}{\textit{Tabelle \thetable\ (Fortsetzung)}} \\
    \toprule
    \textbf{Eigenschaft} & \textbf{Beschreibung} \\
    \midrule
    \endhead
    \midrule
    \multicolumn{2}{r}{\textit{Fortsetzung auf nächster Seite}} \\
    \endfoot
    \bottomrule
    \multicolumn{2}{p{\linewidth}}{\textit{Anmerkung.} Eigene Darstellung basierend auf \textcite[S. 102--103]{mehrez_CryptoAgilityProperties_2018}.} \\
    \endlastfoot
    Extensibility & Fähigkeit, neue Algorithmen oder Parameter effizient zum System hinzuzufügen \\
    \midrule
    Removability & Fähigkeit, veraltete oder anfällige kryptographische Verfahren kontrolliert zu deaktivieren \\
    \midrule
    Fungibility & Leichte Austauschbarkeit von Sicherheitskomponenten durch standardisierte Identifikatoren \\
    \midrule
    Interoperability & Kompatibilität zwischen verschiedenen Implementierungen basierend auf Spezifikationen \\
    \midrule
    Updateability & Sichere Aktualisierung oder Korrektur kryptographischer Algorithmen Implementierungen \\
    \midrule
    Flexibility & Grad der Anpassungsfähigkeit je nach Implementierungsebene (Hardware < Betriebssystem < Anwendung < Skriptsprachen) \\
    \midrule
    Compatibility & Neue Softwaremodule und Patches müssen auf bestehender Hardware lauffähig sein \\
    \midrule
    Reversibility & Fähigkeit, fehlgeschlagene Softwareupdates rückgängig zu machen und zu vorherigen, funktionierenden Versionen zurückzukehren \\
    \midrule
    Transition Mechanisms & Geschützte Aushandlung neuer Algorithmen gegen Downgrade-Attacken \\
    \midrule
    Backward Compatibility & Unterstützung älterer Algorithmen während begrenzter Übergangsphasen \\
\end{longtable}

\textcite[S. 667]{kreutzer_Kryptoagilitaet_2024a} heben hervor, dass kryptoagile Systeme eine Voraussetzung für langfristige Sicherheit und für die Aufnahme kryptographischer Innovationen darstellen. Entsprechende Sicherheitsarchitekturen müssen daher bereits im Design vorsehen, dass kryptographische Komponenten aktualisiert, kombiniert und betrieblich migriert werden können, um auf Innovationsdruck, Erosion bestehender Verfahren und inkompatible Standards reagieren zu können \parencite[S. 667--671]{kreutzer_Kryptoagilitaet_2024a}. Kryptoagilität wird in diesem Sinne als Sichtweisen verschiedener Handlungsfelder verstanden, das sich von kryptographischen Bibliotheken über Systemarchitekturen bis hin zu Entscheidungsprozessen und Governance-Strukturen erstreckt \parencite[S. 669--670]{kreutzer_Kryptoagilitaet_2024a}.