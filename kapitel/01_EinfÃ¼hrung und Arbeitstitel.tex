\newpage
\section{Einführung und Arbeitstitel} \label{sec:Einführung und Arbeitstitel}

Die Zuverlässigkeit, Leistungsfähigkeit und Sicherheit kritischer Infrastrukturen (\acsu{KRITIS}) sind elementare nationale Prioritäten, da sie für die moderne Gesellschaft essenziell sind \parencite[S. 1]{alcaraz_CriticalInfrastructureProtectionRequirementschallenges21stcentury_2015}. Cyberangriffe zählen zu den größten Bedrohungen für Organisationen unterschiedlichster Sektoren, insbesondere für \ac{KRITIS} \parencite[S. 1]{osliak_CyberThreatIntelligencecriticalinfrastructuresecurity_2023}.

Die zunehmende Online-Verarbeitung und -Speicherung von Daten verschiebt den Security Perimeter hin zur Identität, deren Verwaltung zum zentralen Aspekt der Cybersicherheit wird \parencite[S. 191]{josang_CybersecurityTechnologyGovernance_2025}. \ac{IAM} trägt wesentlich zur Gefahrenabwehr bei \parencite[S. 4]{nosouhi_ValueStrongIdentityAccessManagementICSOTSecurity_2024}, denn es umfasst sowohl die Registrierung und Authentifizierung von Identitäten (\ac{IdM}) als auch die Vergabe und Kontrolle von Zugriffsrechten (\ac{AM}) \parencite[S. 191]{josang_CybersecurityTechnologyGovernance_2025}.

Gerade \ac{IdM}-Systeme sind attraktive Ziele für Cyberangriffe, da sie Identitätsdiebstahl und Überwachungsangriffe erleichtern \parencite[S. 68]{fritsch_IdentityManagementTargetcyberwar_2020}. Das exponentielle Wachstum des \ac{IoT} macht zentralisierte, \ac{PKI}-basierte \ac{IdM}-Systeme zunehmend ungeeignet, aufgrund von Single-Point-of-Failure, Skalierungsdefiziten und fehlender Nutzerkontrolle \parencite[S. 1]{ramirez-gordillo_DecentralizedIdentityManagementInternetThingsIoTDevicesUsingIOTABlockchainTechnology_2025}.

Neue Technologien und Protokolle, insbesondere das \ac{SSI}-Konzept, bieten stattdessen dezentrale, nutzerkontrollierte digitale Identitäten und nutzen dazu digitale Wallets sowie Blockchain-Technologien \parencite[S. 6]{allendelopez_SelfSovereignIdentityFutureIdentitySelfSovereignityDigitalWalletsBlockchain_2020}. Wie klassische \ac{PKI} setzt \ac{SSI} auf asymmetrische Kryptografie und digitale Signaturen.

Quantencomputer eröffnen zwar neue Rechenpotenziale, gefährden jedoch bei Erreichung des \gls{CRQC} durch beispielsweise Shor's Algorithmus die Sicherheit bestehender Kryptosysteme \parencite[S. 338, 341]{geremew_PreparingCriticalInfrastructurePostQuantumCryptographyStrategiesTransitioningAheadCryptanalyticallyRelevantQuantumComputing_2024}. Dadurch werden insbesondere klassische \ac{PKI}-Verfahren auch in \ac{SSI}-Systemen verwundbar \parencite[S. 2-3]{solavagione_TransitionSelfSovereignIdentityPostQuantumCryptography_2025}. Organisationen müssen ihre kryptografischen Systeme daher frühzeitig auf quantenresistente Verfahren umstellen.

Dieses Exposé skizziert die Entwicklung eines blockchain-basierten \ac{SSI}-Prototyps mit \ac{PQC} für \ac{KRITIS}, geleitet durch den \ac{DSR}-Ansatz:

\begin{center}
    \textbf{\enquote{Design Science Research-geleitete Entwicklung und Evaluation eines blockchain-basierten Self-Sovereign-Identity-Prototypen mit Post-Quantum-Kryptografie für kritische Infrastrukturen}}
\end{center}