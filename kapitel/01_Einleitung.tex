\newpage
\section{Einleitung} \label{sec:Einleitung}

\subsection{Problemstellung und Motivation} \label{sec:Problemstellung und Motivation}

Kritische Infrastrukturen bilden das Rückgrat moderner Gesellschaften und sind essenziell für die nationale Sicherheit, wirtschaftliche Stabilität sowie die öffentliche Gesundheit und Sicherheit \parencite[S. 53]{alcaraz_CriticalInfrastructureProtectionRequirementschallenges21stcentury_2015}. Die fortschreitende Digitalisierung und Vernetzung dieser Systeme führt jedoch zu einer signifikanten Erweiterung der Angriffsfläche, wodurch kritische Infrastrukturen zunehmend komplexen Cyberangriffen ausgesetzt sind \parencite[S. 1]{nosouhi_ValueStrongIdentityAccessManagementICSOTSecurity_2024}. Diese Entwicklung wird durch die Konvergenz von \ac{IT} und \ac{OT} verstärkt, die traditionell isolierte industrielle Steuerungssysteme mit Unternehmensnetzwerken verbindet und neue Sicherheitsrisiken schafft \parencite[S. 1]{osliak_CyberThreatIntelligencecriticalinfrastructuresecurity_2023}.

Die zunehmende Online-Verarbeitung und -Speicherung von Daten verschiebt den Security Perimeter hin zur Identität, deren Verwaltung zum zentralen Aspekt der Cybersicherheit wird \parencite[S. 191]{josang_CybersecurityTechnologyGovernance_2025}. Identity Management nimmt in diesem Kontext eine zentrale Stellung ein, da es die Identifikation, Verifikation und Autorisierung von Personen sowie die Verwaltung von Zugriffsrechten auf kritische Ressourcen ermöglicht \parencite[S. 61]{fritsch_IdentityManagementTargetcyberwar_2020}. Identity Management stellt somit eine \ac{KRITIS} dar, die vielen anderen gesellschaftlich relevanten \ac{KRITIS} und Funktionen zugrunde liegt \parencite[S. 61]{fritsch_IdentityManagementTargetcyberwar_2020}. Die besondere Brisanz ergibt sich daraus, dass kompromittierte digitale Identitäten nicht nur den Zugang zu kritischen Systemen ermöglichen, sondern auch als Ausgangspunkt für weitreichende Angriffe dienen können \parencite[S. 3]{nosouhi_ValueStrongIdentityAccessManagementICSOTSecurity_2024}. Analysen zeigen, dass \ac{IAM}-basierte Mitigationsansätze 42\% der Angriffsvektoren beim \enquote{Initial Access} und 86\% der Angriffsvektoren beim \enquote{Lateral Movement} effektiv adressieren können \parencite[S. 3--4]{nosouhi_ValueStrongIdentityAccessManagementICSOTSecurity_2024}.

Trotz dieser Bedeutung arbeiten herkömmliche Identity-Management-Systeme überwiegend mit zentralisierten Frameworks, die inhärente Schwachstellen aufweisen. Zu diesen zählen Single Points of Failure, eingeschränkte Skalierbarkeit und begrenzte Datenschutzfunktionen \parencite[S. 1]{ramirez-gordillo_DecentralizedIdentityManagementInternetThingsIoTDevicesUsingIOTABlockchainTechnology_2025}. In solchen Modellen besitzen Nutzer keine effektive Kontrolle über ihre digitalen Identitäten, da Authentifikatoren, Daten und Credentials in Datenbanken Dritter gespeichert werden, die letztlich die Hoheit ausüben \parencite[S. 15]{allendelopez_SelfSovereignIdentityFutureIdentitySelfSovereignityDigitalWalletsBlockchain_2020}. Dies führt zu erheblichen Datenschutzrisiken und schränkt die Nutzerautonomie ein, da Rechte wie Einwilligung, Löschung (\enquote{Recht auf Vergessenwerden}), Portabilität und Pseudonymisierung kaum technisch garantiert werden können \parencite[S. 18--19]{allendelopez_SelfSovereignIdentityFutureIdentitySelfSovereignityDigitalWalletsBlockchain_2020}.

Das Konzept der \ac{SSI} adressiert diese architektonischen Limitierungen durch einen fundamentalen Paradigmenwechsel. Die Souveränität über die Identitätsverwaltung wird direkt zum Nutzer verlagert, der seine digitalen Assets und Nachweise mithilfe von lokalen Wallets autonom kontrolliert \parencite[S. 7]{allendelopez_SelfSovereignIdentityFutureIdentitySelfSovereignityDigitalWalletsBlockchain_2020}. Technologisch wird dieser Ansatz durch die Integration von \ac{DID} und \ac{DLT} realisiert, die als dezentrale Vertrauensanker fungieren und dadurch die Abhängigkeit von zentralen Zertifizierungsstellen auflösen \parencite[S. 5--6]{ramirez-gordillo_DecentralizedIdentityManagementInternetThingsIoTDevicesUsingIOTABlockchainTechnology_2025}. Ergänzend ermöglichen \ac{VC} und \ac{VP} einen kryptografisch gesicherten Austausch von Identitätsdaten, bei dem der Nutzer selektiv entscheiden kann, welche Informationen er wem präsentiert, ohne dabei die Datenhoheit abzugeben \parencite[S. 71, 74, 77]{allendelopez_SelfSovereignIdentityFutureIdentitySelfSovereignityDigitalWalletsBlockchain_2020}. Diese Architektur eliminiert durch die verteilte Validierung von Transaktionen systemisch zentrale Angriffsvektoren, was die Resilienz gegenüber Ausfällen und Manipulationen signifikant erhöht \parencite[S. 32]{ramirez-gordillo_DecentralizedIdentityManagementInternetThingsIoTDevicesUsingIOTABlockchainTechnology_2025}.

Während \ac{SSI} die architektonische Schwachstellen beseitigt, entwickeln sich die rasanten Fortschritte im Quantum Computing parallel zu einer tiefgreifenden Bedrohung für traditionelle kryptografische Systeme \parencite[S. 1]{mamatha_PostQuantumCryptographySecuringDigitalCommunicationQuantumEra_2024}. Diese Bedrohung ergibt sich aus der Fähigkeit von Quantencomputern, spezifische mathematische Probleme schneller zu lösen als klassische Computer \parencite[S. 339]{geremew_PreparingCriticalInfrastructurePostQuantumCryptographyStrategiesTransitioningAheadCryptanalyticallyRelevantQuantumComputing_2024}. Nach Einschätzung der \ac{CISA} könnte die kritische Phase dieser Entwicklung bereits um 2030 mit dem Erreichen von \gls{CRQC} eintreten \parencite[S. 344]{geremew_PreparingCriticalInfrastructurePostQuantumCryptographyStrategiesTransitioningAheadCryptanalyticallyRelevantQuantumComputing_2024}. 

Zwei bedeutende Quantenalgorithmen sind hierbei von besonderer Relevanz \parencite[S. 341]{geremew_PreparingCriticalInfrastructurePostQuantumCryptographyStrategiesTransitioningAheadCryptanalyticallyRelevantQuantumComputing_2024}. Der von \textcite[S. 1]{shor_AlgorithmsQuantumComputationdiscretelogarithmsfactoring_1994} entwickelte Shor-Algorithmus löst Probleme der ganzzahligen Faktorisierung und des diskreten Logarithmus effizient und bedroht somit die Sicherheit vieler asymmetrischer Kryptosysteme wie \ac{RSA} und \ac{ECC} \parencite[S. 341]{geremew_PreparingCriticalInfrastructurePostQuantumCryptographyStrategiesTransitioningAheadCryptanalyticallyRelevantQuantumComputing_2024}. Der von \textcite[S. 1]{grover_FastQuantumMechanicalalgorithmdatabasesearch_1996} entwickelte Grover-Algorithmus bietet eine quadratische Beschleunigung für die Suche in unstrukturierten Datenbanken und verringert die Sicherheit symmetrischer Kryptografie durch Halbierung der effektiven Schlüsselgröße \parencite[S. 1]{bernstein_GroverVsMcEliece_2010}.

Die Bedrohung durch Quantencomputer manifestiert sich besonders in der sogenannten \enquote{Harvest Now, Decrypt Later}-Strategie, bei der Angreifer verschlüsselte Daten bereits heute abfangen und speichern, um sie später mit Quantencomputern zu entschlüsseln \parencite[S. 344--345]{geremew_PreparingCriticalInfrastructurePostQuantumCryptographyStrategiesTransitioningAheadCryptanalyticallyRelevantQuantumComputing_2024}. Diese Angriffsstrategie gefährdet insbesondere Daten mit langfristiger Schutzbedürftigkeit, wie sie in kritischen Infrastrukturen regelmäßig verarbeitet werden \parencite[S. 338--339, 345]{geremew_PreparingCriticalInfrastructurePostQuantumCryptographyStrategiesTransitioningAheadCryptanalyticallyRelevantQuantumComputing_2024}. Klassische kryptografische Systeme, die derzeit \ac{IAM}-Infrastrukturen absichern, werden somit mittelfristig ihre Schutzwirkung verlieren \parencite[S. 339, 341, 351]{geremew_PreparingCriticalInfrastructurePostQuantumCryptographyStrategiesTransitioningAheadCryptanalyticallyRelevantQuantumComputing_2024}.

Als Antwort darauf entwickelt die Kryptografie-Community quantenresistente Verfahren, die als \ac{PQC} zusammengefasst werden \parencite[S. 1]{mamatha_PostQuantumCryptographySecuringDigitalCommunicationQuantumEra_2024}. Das \ac{NIST} finalisierte im August 2024 die ersten drei \ac{PQC}-Standards (ML-KEM, ML-DSA, SLH-DSA), die auf alternativen mathematischen Problemen wie Gitterstrukturen basieren \parencite{nationalinstituteofstandardsandtechnologyus_ModulelatticebasedDigitalSignaturestandard_2024,nationalinstituteofstandardsandtechnologyus_ModulelatticebasedKeyencapsulationMechanismstandard_2024,nationalinstituteofstandardsandtechnologyus_StatelessHashbasedDigitalsignaturestandard_2024} und in der Umstellung von Internet Protokollen eingesetzt werden sollen \parencite{solavagione_TransitionSelfSovereignIdentityPostQuantumCryptography_2025}. In diesem Kontext offenbart sich jedoch eine kritische Sicherheitslücke bei dezentralen Identitäten, da sich aktuelle \ac{SSI}-Implementierungen auf klassische kryptografische Primitive stützen, die durch künftige Quantencomputer kompromittiert werden können \parencite[S. 2--3]{solavagione_TransitionSelfSovereignIdentityPostQuantumCryptography_2025}. Um die strukturellen Resilienzvorteile der Dezentralisierung auch langfristig für kritische Infrastrukturen zu erhalten, ist eine Migration auf \ac{PQC} erforderlich.

\subsection{Stand der Forschung und Identifikation der Forschungslücke} \label{sec:Stand der Forschung und Identifikation der Forschungslücke}

Die systematische Literaturrecherche (Kapitel~\ref{sec:Systematische Literaturrecherche}, \ref{sec:Anhang_Dokumentation_der_systematischen_Literaturrecherche}) zum gegenwärtigen Stand der Forschung wurde nach ausgewählten Methoden des \ac{PRISMA} 2020 Standards \parencite[S. 1]{page_PRISMA2020Statementupdatedguidelinereportingsystematicreviews_2021} durchgeführt und in zwei iterativen Phasen organisiert. Dieser Ansatz ermöglicht es, sowohl die initiale Problemidentifikation als auch die Aktualisierung der Wissensbasis unter Berücksichtigung des dynamischen Charakters des Forschungsfeldes abzubilden. Folgend die Synthese der Ergebnisse der beiden Iterationen der Literaturrecherche (\autoref{tab:quellenuebersicht_Iteration1} und \autoref{tab:quellenuebersicht_iteration2}).

Im Bereich der dezentralisierten Identitätsverwaltung existiert eine wachsende Forschungstätigkeit, die technologische Innovationen mit regulatorischen Anforderungen verbindet. Die Arbeiten in diesem Feld lassen sich in vier zentrale Themencluster einordnen, die jeweils unterschiedliche Aspekte der SSI-Technologie und ihrer sicherheitstechnischen Anforderungen für kritische Infrastrukturen adressieren.

Der erste Cluster umfasst grundlegende Arbeiten zu Blockchain-basierten Identitätsmanagementsystemen. \textcite[S. 11]{feng_IdentityManagementSystemsComprehensiveReview_2025} zeigt auf, dass Blockchain-Technologie neue Paradigmen für dezentrale Identitätsverwaltung eröffnet, in denen Nutzer die Kontrolle über ihre Identitätsdaten zurückgewinnen und ihre Abhängigkeit von zentralisierten Instanzen reduzieren können. Der historische Entwicklungspfad führt von traditionellen zentralisierten und föderalen Architekturmodellen hin zu Blockchain-basierten dezentralisierten Lösungen, die verteilte Ledger-Technologien als vertrauensanker einsetzen \parencite[S. 11]{feng_IdentityManagementSystemsComprehensiveReview_2025}. Allerdings dokumentiert die umfassende Analyse unzureichend adressierte Sicherheitsherausforderungen wie Interoperabilität, Revocation-Management und Quantenresilienz \parencite[S. 2]{feng_IdentityManagementSystemsComprehensiveReview_2025}. Diese Lücken in der aktuellen Forschung sind besonders kritisch, da sie die praktische Deploybarkeit dezentraler Identitätslösungen in regulativ kontrollierten Sektoren wie kritischen Infrastrukturen einschränken.

Der zweite Cluster adressiert die Quantum-Computing-Bedrohung und die Standardisierung von Post-Quantum-Kryptografie. \textcite[S. 2--3]{barrett-danes_QuantumComputingCybersecurityrigoroussystematicreviewemergingthreatspostquantumsolutionsresearchdirections_2025} dokumentieren, dass trotz rascher Fortschritte bei \ac{PQC}-Lösungen erhebliche Lücken zwischen theoretischer Robustheit und praktischer Implementierbarkeit bestehen. \textcite[S. 2--3]{barrett-danes_QuantumComputingCybersecurityrigoroussystematicreviewemergingthreatspostquantumsolutionsresearchdirections_2025} weisen explizit auf die \ac{NIST}-Standardisierung von vier \ac{PQC}-Algorithmen im Jahr 2024 hin, vermerken jedoch, dass zwei vorangegangene Kandidaten während des Testens kompromittiert wurden und selbst die approbierten Algorithmen weitere praktische Validierung und Optimierung für ressourcenbegrenzte Umgebungen benötigen. Insbesondere im Kontext von Internet-of-Things-Geräten zeigen sich erhebliche Skalierungsprobleme und Integrationskomplexität \parencite[S. 2--3]{barrett-danes_QuantumComputingCybersecurityrigoroussystematicreviewemergingthreatspostquantumsolutionsresearchdirections_2025}. Die Autoren identifizieren ein kritisches Ungleichgewicht. Während 81\% der analysierten Literatur einen dominanten Fokus auf die Entwicklung, Bewertung und Optimierung von kryptografischen \ac{PQC}-Frameworks setzen, adressieren nur 4,8\% der Studien tatsächliche Implementierungsbarrieren \parencite[S. 14--15]{barrett-danes_QuantumComputingCybersecurityrigoroussystematicreviewemergingthreatspostquantumsolutionsresearchdirections_2025}. Dieser starke Ungleichgewicht in der Forschungslandschaft weist auf eine fundamentale Forschungslücke zwischen akademischer Lösung und praktischer Machbarkeit hin. Komplementär dazu erörtert \textcite[S. 1]{szymanski_QuantumSafeSoftwareDefinedDeterministicInternetThingsIoTHardwareEnforcedCyberSecurityCriticalInfrastructures_2024} eine neuartige Software-Defined-Deterministic-IIoT-Architektur mit quantensicheren Verschlüsselungsverfahren für kritische Infrastrukturen. Der Ansatz kombiniert deterministische Netzwerkmechanismen mit \ac{FPGA}-basierter Hardwareverstärkung, um DoS/DDoS-Angriffe vollständig zu eliminieren und Zero-Trust-Architekturen hardwaregestützt umzusetzen \parencite[S. 1--2]{szymanski_QuantumSafeSoftwareDefinedDeterministicInternetThingsIoTHardwareEnforcedCyberSecurityCriticalInfrastructures_2024}. Das Konzept der Authenticated Encrypted Deterministic Channels  zeigt, dass Architektur-Design und Kryptografie eng miteinander verflochten sein müssen, um kritische Infrastrukturen angemessen zu schützen.

Der dritte Cluster befasst sich mit hybriden Ansätzen und fortgeschrittenen Signaturschemata für ressourcenlimitierte Umgebungen. \textcite[S. 2]{nouma_TrustworthyEfficientDigitalTwinsPostQuantumEraHybridHardwareAssistedSignatures_2024} präsentieren Hardware-Assisted Efficient Signatures, die Forward-Secure \ac{PQC}-Signaturen mit hoher Signer-Efficiency kombinieren. Die Autoren zeigen, dass \ac{NIST}-\c{PQC}-Standards für embedded IoT-Komponenten faktisch ungeeignet sind \parencite[S. 2--3]{nouma_TrustworthyEfficientDigitalTwinsPostQuantumEraHybridHardwareAssistedSignatures_2024}. Dies wird durch Arbeiten zur privat generierten Schlüsselpaaren von \textcite[S. 1--2]{alam_PrivatelyGeneratedKeyPairsPostQuantumCryptographyDistributedNetwork_2024} ergänzt, die Challenge-Response-Mechanismen für dezentrale \ac{PQC}-Keygenerierung demonstrieren und damit einen Brückenschlag zwischen dezentraler Kontrolle und zentraler Verifikation ermöglichen. Der Einsatz von CRYSTALS-Dilithium als standardisierte \ac{PQC}-Algorithmus zeigt, dass gitterbasierte Verfahren auch für verteilte Schlüsselerzeugung praktikabel sind \parencite[S. 17]{alam_PrivatelyGeneratedKeyPairsPostQuantumCryptographyDistributedNetwork_2024}.

Der vierte Cluster behandelt regulatorische und standardisierungstechnische Anforderungen an dezentrale Identitätssysteme. \textcite[S. 23--24]{sharif_EIDASRegulationSurveyTechnologicalTrendsEuropeanElectronicIdentitySchemes_2022} zeigen, dass die europäische eIDAS-2.0-Regulierung einen Paradigmenwechsel von zentralisierten zu selbstbestimmten Identitätsmodellen vorsieht, ohne dabei konkrete technische Anforderungen an Quantenresilienz in dezentralen Wallets zu spezifizieren. Ähnlich dokumentiert die internationale Regulatory-Landscape, dass Cybersicherheitsstandards für kritische Infrastrukturen zwar zunehmend Post-Quantum-Kryptografie mandatieren, dabei aber Implementierungspfade und konkrete Integrationsvorgaben für dezentrale Systeme noch nicht ausreichend entwickelt haben \parencite[S. 105--106]{radanliev_ReviewComparisonUSEUUKRegulationsCyberRiskSecurityCurrentBlockchainTechnologies_2023}.

Zusammenfassend offenbart sich eine Forschungslücke an der Schnittstelle zwischen dezentraler Identitätsverwaltung, Post-Quantum-Kryptografie und kritischer Infrastruktur-Sicherheit. Während sowohl SSI-Technologien als auch PQC-Standards als Einzeltechnologien einen ausgereiften Standardisierungsgrad erreicht haben, existiert bislang praktisch keine systematische Forschungsarbeit zur kombinierten Evaluation und zum sicherheitsfokussierten Design eines quantensicheren SSI-Prototypen, der gleichzeitig KRITIS-spezifische Anforderungen und regulatorische Compliance erfüllt.

\subsection{Zielsetzung und Forschungsfragen} \label{sec:Zielsetzung und Forschungsfragen}

Die vorliegende Masterarbeit adressiert die in Kapitel~\ref{sec:Stand der Forschung und Identifikation der Forschungslücke} identifizierte Forschungslücke an der Schnittstelle zwischen dezentraler Identitätsverwaltung, Post-Quantum-Kryptografie und kritischer Infrastruktur-Sicherheit. Die übergeordnete Zielsetzung ist die Entwicklung und empirische Evaluation eines funktionsfähigen blockchain-basierten SSI-Prototypen mit integrierter Post-Quantum-Kryptografie im Kontext von \ac{KRITIS}. Im Kern wird untersucht, ob und unter welchen architektonischen Bedingungen NIST-standardisierte PQC-Algorithmen nicht-invasiv in bestehende SSI-Frameworks integriert werden können, ohne sowohl die Dezentralisierungsprinzipien der SSI-Technologie als auch die regulatorischen Anforderungen von \ac{KRITIS} zu gefährden. Die zentrale Herausforderung besteht darin, ob Quantensicherheit konsistent über mehrere architektonische Schichten hinweg operationalisierbar ist und damit eine praktikable Synthese der drei Anforderungsdimensionen Dezentralisierung, Quantensicherheit und KRITIS-Compliance gelingen kann.

Aus dieser übergeordneten Zielsetzung ergeben sich drei konkrete und operationalisierbare Forschungsfragen, die die \enquote{Technical Feasibility} einer quantensicheren SSI-Architektur für kritische Infrastrukturen auf unterschiedlichen Analyseebenen adressieren:

\textbf{FF1 -- Systemarchitektur \& Compliance:}
Wie kann ein blockchain-basiertes \ac{SSI}-System unter Einsatz von \ac{PQC} gestaltet werden, um die regulatorischen und technischen Anforderungen von \ac{KRITIS} nachhaltig zu erfüllen?

FF1 operationalisiert die Frage auf architektonischer Ebene, indem sie das Design der Gesamtstruktur adressiert und darauf abzielt, architektonischen Muster, Schichten und Integrationspunkte zu ermitteln, um Quantensicherheit und regulatorische Compliance simultan zu realisieren.

\textbf{FF2 -- Algorithmenauswahl \& Sicherheitsbewertung:}
Welche \ac{PQC}-Algorithmen eignen sich für die Integration in \ac{SSI}-Systeme hinsichtlich Sicherheit und Interoperabilität, insbesondere im Kontext von \ac{KRITIS}?

FF2 konkretisiert die Frage auf algorithmischer Ebene, indem sie die Wahl konkreter kryptografischer Standards adressiert und untersucht, welche Algorithmen und Parametermengen unter realistischen Integrationsbedingungen praktikabel und sicher sind.

\textbf{FF3 -- Kryptografische Agilität:} 
Welche kryptografischen Agilitätsmechanismen sind erforderlich, um zukünftige \ac{PQC}-Algorithmenupdates ohne Systemunterbrechung zu ermöglichen?

FF3 adressiert die Frage auf der Ebene der Systemevolution, indem sie untersucht, welche Mechanismen in die Architektur eingebaut werden müssen, um langfristige Wartbarkeit und Anpassungsfähigkeit an zukünftige kryptografische Standards zu sichern.

\subsection{Methodisches Vorgehen} \label{sec:Methodisches Vorgehen}

Die vorliegende Masterarbeit folgt dem \ac{DSR}-Paradigma \parencite[S. 75]{hevner_DesignScienceInformationsystemsresearch_2004}, das auf die Erschaffung und wissenschaftliche Evaluation innovativer Artefakte ausgerichtet ist (Kapitel~\ref{sec:Design Science Research}). Das methodische Vorgehen wird durch drei zentrale Komponenten strukturiert. Als Fundament dient eine systematische Literaturrecherche (Kapitel~\ref{sec:Systematische Literaturrecherche}) nach \ac{PRISMA} 2020 Standards \parencite[S. 1]{page_PRISMA2020Statementupdatedguidelinereportingsystematicreviews_2021} in zwei zeitlich versetzten Phasen, die sowohl initiale Problemidentifikation als auch kontinuierliche Wissensaktualisierung unter Berücksichtigung des dynamischen Forschungsfeldes gewährleistet. Dies sichert Transparenz, Reproduzierbarkeit und wissenschaftliche Nachvollziehbarkeit.

Kernaktivität ist die iterative Artefaktentwicklung nach dem \ac{DSRM} Prozessmodell \parencite[S. 46, 54]{peffers_DesignScienceResearchmethodologyinformationsystemsresearch_2007}, das einen strukturierten sechsphasigen Rahmen bereitstellt (Kapitel~\ref{sec:DSRM Prozessmodell}). Der Research Entry erfolgt als Problem-Centered Initiation, motiviert durch spezifische technologische Bedrohungen und regulatorische Anforderungen. Das Entwicklungsverfahren organisiert sich als zweizyklischer Iterationsprozess, wobei jeder Zyklus Lösungsdefinition, Designentwicklung, technische Demonstration und Evaluation umfasst. Diese Struktur ermöglicht systematisch, Erkenntnisse aus einer Implementierungsrunde zur iterativen Verbesserung der nachfolgenden Phase einzubeziehen.

Die Evaluation folgt dem \ac{FEDS} \parencite[S. 1]{venable_FEDSFrameworkEvaluationDesignScienceResearch_2016} mit einer Technical Risk Efficacy Strategy, die mit formativen künstlichen Tests beginnt und in summativer Evaluation endet (Kapitel~\ref{sec:FEDS-Framework}). Da das primäre Entwicklungsrisiko technischer Natur ist, erfolgen Tests in kontrollierten Umgebungen, bevor Feldtests aus Sicherheitsgründen vertretbar sind. Die Evaluation untergliedert sich in diskrete Episoden parallel zu den Entwicklungszyklen und überprüft einzelne Komponenten und Subsysteme unter kontrollierten Bedingungen.

Die Operationalisierung der Forschungsfragen erfolgt bidirektional. Auf normativer Ebene werden explizite Designziele pro Iterationsphase (Kapitel~\ref{sec:Designziele_Iteration_1} und \ref{sec:Designziele_Iteration_2}) etabliert, die unmittelbar aus den übergeordneten Forschungsfragen ableiten und als Vorgaben für Architektur- und Implementierungsentscheidungen fungieren. Auf operativer Ebene erfolgt systematische Konkretisierung in funktionale Anforderungen (Kapitel~\ref{sec:Funktionale Anforderungen}), die den System-Lebenszyklus abdecken, sowie in spezifische Compliance-Anforderungen (Kapitel~\ref{sec:KRITIS-spezifische Compliance-Anforderungen}) aus sicherheitsrelevanten Standards und Richtlinien. Diese strukturierte, bidirektionale Operationalisierung garantiert wissenschaftliche Nachvollziehbarkeit und Evidenzbasierung des gesamten methodischen Vorgehens.

\subsection{Scope und Limitationen} \label{sec:Scope und Limitationen}

Der begrenzte Rahmen einer Masterarbeit erfordert eine klare Fokussierung und Priorisierung der Forschungsaktivitäten. Die vorliegende Arbeit konzentriert sich daher auf die \enquote{Technical Feasibility} und \enquote{Compliance Readiness} eines blockchain-basierten \ac{SSI}-Systems mit integrierter \ac{PQC} für \ac{KRITIS}. Diese Schwerpunktsetzung leitet sich direkt aus der in Kapitel~\ref{sec:Stand der Forschung und Identifikation der Forschungslücke} identifizierten Diskrepanz zwischen theoretischer \ac{PQC}-Forschung und fehlender praktischer Implementierungsexpertise ab \parencite[S. 14--15]{barrett-danes_QuantumComputingCybersecurityrigoroussystematicreviewemergingthreatspostquantumsolutionsresearchdirections_2025}. Während mathematische Primitive bereits standardisiert vorliegen \parencite{nationalinstituteofstandardsandtechnologyus_ModulelatticebasedDigitalSignaturestandard_2024,nationalinstituteofstandardsandtechnologyus_ModulelatticebasedKeyencapsulationMechanismstandard_2024,nationalinstituteofstandardsandtechnologyus_StatelessHashbasedDigitalsignaturestandard_2024}, ist deren architektonische Integration in verteilte Identitätssysteme bisher kaum empirisch validiert. Das primäre Forschungsziel ist folglich der Nachweis, dass eine solche Integration unter Erhalt der Systemintegrität prinzipiell möglich ist, bevor in zukünftigen Arbeiten eine Optimierung auf Effizienz erfolgen kann.

Der methodische Rahmen umfasst zwei \ac{DSR}-Iterationen, die jeweils durch formative Evaluationen begleitet werden und in einer summativen Gesamtevaluation münden. Diese Beschränkung auf zwei Entwicklungszyklen erlaubt eine sorgfältige Dokumentation und wissenschaftlich fundierte Reflexion der Designentscheidungen, limitiert jedoch naturgemäß die Anzahl möglicher Architekturvarianten und experimenteller Optimierungsschritte. Der Forschungsschwerpunkt liegt hierbei explizit auf der \ac{SSI}-Agent-Schicht, während die zugrundeliegende \ac{DLT}- und Revocations-Infrastruktur bewusst als reine Labor-Komponenten ohne Anspruch auf Produktivreife eingesetzt werden. Diese Entscheidung ist methodisch begründet, da der Erkenntnisfokus auf der Integration von Post-Quantum-Algorithmen in die Identitätsverwaltungsschicht liegt und nicht auf der Optimierung der Blockchain-Konsensverfahren oder der skalierten Revocation-Infrastruktur. Die gewählte Architektur ermöglicht reproduzierbare Experimente unter kontrollierten Bedingungen, erlaubt jedoch keine Aussagen über das Verhalten in produktiven, verteilten Netzwerken mit hunderten Validierungsknoten oder unter realen Lastbedingungen.

Die Anforderungsanalyse konzentriert sich auf funktionale Anforderungen, die den vollständigen \ac{SSI}-Lebenszyklus abdecken, sowie auf technische Compliance-Anforderungen, die aus \ac{BSI}-Richtlinien, \ac{DSGVO} und internationalen Standards abgeleitet werden. Nichtfunktionale Anforderungen wie Performance, Durchsatz, Latenzzeiten oder Ressourcenverbrauch werden explizit ausgeklammert. Diese Delimitation ist methodisch begründet dadurch, dass PQC-Algorithmen sowie deren Integration teilweise noch in frühen Standardisierungsphasen befinden, sodass Performance-Messungen an Referenzimplementierungen von begrenzter Aussagekraft für zukünftige produktive Systeme wären \parencite[S. 3--5]{demir_PerformanceAnalysisIndustryDeploymentPostQuantumCryptographyAlgorithms_2025}. Zusätzlich erfolgt die Evaluation bewusst in einer artifiziellen Laborumgebung, um isolierte funktionale Nachweise ohne Störfaktoren realer Netzwerke zu erbringen, wodurch Rückschlüsse auf die Performance in produktiven Systemen methodisch nicht valide wären.

Die Evaluation der Compliance-Anforderungen erfolgt aus einer rein technisch-informatischen Perspektive und adressiert ausschließlich technologische Umsetzungsaspekte. Prozessuale und organisatorische Compliance-Dimensionen liegen außerhalb des Untersuchungsrahmens. Dies entspricht der Zielsetzung, \enquote{Compliance Readiness} als technische Grundlage nachzuweisen, nicht jedoch vollständige organisatorische Compliance zu gewährleisten, die umfassende Governance-Strukturen und operative Prozesse erfordert.

Die Arbeit untersucht die Integration von Post-Quantum-Kryptografie in bestehende \ac{SSI}-Frameworks, entwickelt jedoch keine neuen kryptografischen Primitive. Die Forschungsfrage adressiert somit die architektonische Integration und Anwendbarkeit standardisierten Algorithmen in dezentralen Identitätssystemen, nicht deren mathematische Sicherheit oder algorithmische Optimierung. Da sich sowohl die PQC-Standardisierungsprozesse als auch die regulatorischen Rahmenbedingungen für digitale Identitäten in kontinuierlicher Weiterentwicklung befinden, bildet diese Arbeit den Wissensstand zum Zeitpunkt der Erstellung ab und kann zukünftige Änderungen in Standards oder Regularien nicht antizipieren.

\subsection{Aufbau der Arbeit} \label{sec:Aufbau der Arbeit}

Die vorliegende Masterarbeit gliedert sich in acht Kapitel, die eine kohärente Argumentation vom Problem über die Lösung bis zur Reflexion der Ergebnisse aufbauen. Nach der Einleitung behandelt Kapitel~\ref{sec:Theoretische Grundlagen} die theoretischen Grundlagen zu \ac{SSI}, Blockchain-Technologie, \ac{PQC} sowie \ac{KRITIS} und Compliance-Anforderungen. Kapitel~\ref{sec:Methodik} dokumentiert das methodische Vorgehen bestehend aus systematischer Literaturrecherche nach \ac{PRISMA} 2020 Standards, der \ac{DSR}-Methodik und dem \ac{FEDS}-Framework zur Evaluation. Die Kapitel~\ref{sec:Erste Iteration der Artefaktentwicklung} und Kapitel~\ref{sec:Zweite Iteration der Artefaktentwicklung} präsentieren zwei iterative Zyklen der Artefaktentwicklung mit jeweiligen formativen Evaluationen. Kapitel~\ref{sec:Erste Iteration der Artefaktentwicklung} fokussiert auf die Basis-Implementierung der \ac{SSI}-Architektur mit Post-Quantum-Kryptografie auf Transport-Ebene, während Kapitel~\ref{sec:Zweite Iteration der Artefaktentwicklung} eine Verfeinerung auf Applikations-Ebene durch Plugin-basierte Integration durchführt. Kapitel~\ref{sec:Summative Evaluation} führt eine summative Evaluation des finalen Artefakts durch und validiert funktionale Anforderungen, KRITIS-Compliance-Anforderungen sowie die erreichte Kryptoagilität. Kapitel~\ref{sec:Ergebnisse und Diskussion} beantwortet die Forschungsfragen, reflektiert kritisch die Ergebnisse und artikuliert wissenschaftliche sowie praktische Beiträge. Kapitel~\ref{sec:Fazit und Ausblick} fasst die zentralen Ergebnisse zusammen, erörtert Implikationen und identifiziert zukünftige Forschungsrichtungen.