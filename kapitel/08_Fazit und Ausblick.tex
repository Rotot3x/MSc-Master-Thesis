\newpage
\section{Fazit und Ausblick} \label{sec:Fazit und Ausblick}

\subsection{Zusammenfassung der Ergebnisse} \label{sec:Zusammenfassung der Ergebnisse}

Die vorliegende Masterarbeit entwickelt und evaluiert im Rahmen des \ac{DSR} einen funktionsfähigen Prototypen für eine quantensichere dezentrale Identitätsverwaltung in kritischen Infrastrukturen. Ausgangspunkt ist das Spannungsfeld zwischen den architektonischen Sicherheitsvorteilen von \ac{SSI} und der Verwundbarkeit aktueller Implementierungen durch zukünftige Quantencomputer. Die systematische Literaturrecherche nach \ac{PRISMA}-Standards zeigt auf, dass die praktische Integration \ac{NIST}-standardisierter \ac{PQC} unter gleichzeitiger Wahrung von Dezentralisierungsprinzipien und \ac{KRITIS}-Compliance bislang eine Forschungslücke darstellt.

Die iterative Artefaktentwicklung liefert als zentrales architektonisches Ergebnis die Erkenntnis, dass zur Schließung dieser Lücke zwei komplementäre Integrationsmuster erforderlich sind. Auf der Transportebene ermöglicht ein \gls{Sidecar-Proxy}-Ansatz mit \ac{PQC}-zertifizierten TLS-1.3-Verbindungen eine transparente Quantenverschlüsselung ohne Änderungen an der \ac{SSI}-Agent-Logik. Auf der Applikationsebene erlaubt ein Plugin-basiertes System mit \gls{Monkey-Patching} die Integration von \ac{PQC}-Operationen in den Credential-Lebenszyklus ohne invasive Code-Modifikationen. 

Ein zweites wesentliches Ergebnis betrifft die Kryptoagilität. Die Masterarbeit verdeutlicht, dass langfristige Quantensicherheit weniger einen statischen Zustand als vielmehr die architektonische Fähigkeit erfordert, auf die dynamische Entwicklung kryptografischer Standards reagieren zu können. Dies wird technisch durch Abstraktionsschichten gelöst, welche den Austausch kryptografischer Subroutinen zur Laufzeit ermöglichen. Neue Algorithmen lassen sich so mittels Dockerfile-Modifikationen nahtlos integrieren, ohne die Verfügbarkeit der Systeme zu beeinträchtigen.

Ein drittes Kernresultat betrifft die KRITIS-Compliance. Die Arbeit operationalisiert regulatorische Vorgaben aus nationaler Cybersicherheit, Datenschutzrecht und internationalen Sicherheitsstandards in ein konsistentes technisches Anforderungsprofil. Die summative Evaluation bestätigt dessen vollständige Erfüllung, insbesondere durch hybride Kryptoverfahren zur Risikominimierung und granulare Audit-Mechanismen als Basis für Angriffserkennungssysteme. Damit wird der Nachweis erbracht, dass Quantensicherheit und regulatorische Compliance auch innerhalb dezentraler Architekturen gemeinsam operationalisierbar sind.

In der Gesamtschau bestätigt die Masterarbeit die technische Realisierbarkeit einer quantenresistenten Transformation dezentraler Identitätssysteme für kritische Infrastrukturen. Die erarbeiteten Architekturmuster und Agilitätsmechanismen fundieren ein übertragbares Referenzmodell für die Praxis. Dieses befähigt Organisationen zur proaktiven Härtung ihrer Infrastrukturen, indem es einen validierten Pfad zur Quantensicherheit aufzeigt, der Dezentralisierungsprinzipien und regulatorische Compliance in Einklang bringt.

\subsection{Zukünftige Forschungsrichtungen} \label{sec:Zukünftige Forschungsrichtungen}

Anknüpfend an den hier erbrachten Nachweis der technischen Machbarkeit fokussiert ein erster und dringlicher Forschungsstrang auf die quantitative Performance-Evaluation (Efficiency). Da Post-Quantum-Algorithmen wie ML-DSA und ML-KEM signifikant größere Schlüssel- und Signaturlängen aufweisen als klassische Verfahren, ist eine systematische Messung von Latenzzeiten, Durchsatzraten und Ressourcenverbrauch unter Last erforderlich. Zukünftige Studien sollten untersuchen, wie sich diese kryptografischen Overheads in ressourcenbeschränkten IoT-Umgebungen, die für kritische Infrastrukturen im OT-Bereich typisch sind, auf die Echtzeitfähigkeit der Kommunikation auswirken und welche Optimierungspotenziale durch Hardware-Beschleunigung realisierbar sind.

Ein zweites Forschungsfeld betrifft den Transfer der Evaluation von der kontrollierten Laborumgebung (Artificial Evaluation) in reale Anwendungsszenarien (Naturalistic Evaluation). Während dieser Prototyp in einer isolierten Docker-Umgebung validiert wurde, erfordert der produktive Einsatz in kritischen Infrastrukturen die Untersuchung von Interdependenzen mit heterogenen Legacy-Systemen und komplexen Netzwerktopologien. Folgeforschung sollte sich daher auf Pilotierungen in realen KRITIS-Sektoren konzentrieren, um die Robustheit der \gls{Sidecar-Proxy}-Architektur gegenüber realen Angriffsvektoren, Netzwerklatenzen und Ausfallszenarien empirisch zu validieren.

Schließlich erfordert die ganzheitliche Betrachtung eine Erweiterung des Forschungsfokus von der technischen Compliance Readiness hin zur organisatorischen Governance. Während die vorliegende Masterarbeit die technischen Voraussetzungen für Datenschutz und Angriffserkennung validiert hat, bedarf die prozessuale Verankerung dieser Artefakte in etablierte Sicherheitsmanagementsysteme weiterführender Untersuchung. Zukünftige Arbeiten sollten daher evaluieren, welche adaptierten Prozessmodelle und Audit-Verfahren notwendig sind, um dezentrale Identitätsarchitekturen und dynamische Kryptoagilität nahtlos und auditsicher in die betrieblichen Governance-Strukturen von KRITIS-Betreibern zu integrieren.