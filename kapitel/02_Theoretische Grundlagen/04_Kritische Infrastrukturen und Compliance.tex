\subsection{Kritische Infrastrukturen und Compliance} \label{sec:Kritische Infrastrukturen und Compliance}

- KRITIS-Anforderungen nach BSI \\
- Privacy by Design und DSGVO-Konformität \\
- Regulatorische Rahmenbedingungen





\fixme{Die unsystematische Literaturrecherche während des DSR Artefaktentwicklung orientiert sich am \enquote{iterative Review-Ansatz} nach \textcite[S. 208--209]{brocke_StandingShouldersGiantsChallengesRecommendationsLiteratureSearchInformationSystemsResearch_2015}, der mit einer initialen Recherche startet und sich iterativ vertieft abhängig von der aktuellen Herausforderung während der Artefaktentwicklung.}



\subsection{Kryptoagilität} \label{sec:Kryptoagilität}

Die Validierung der Kryptoagilität am Transportlayer demonstriert die Fähigkeit des Systems, kryptographische Algorithmen ohne grundlegendeArchitekturänderungen zu wechseln.