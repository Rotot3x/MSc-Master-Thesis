\subsection{Framework- und Technologie-Evaluation} \label{sec:Evaluation und Auswahl des SSI-Frameworks}

\subsubsection{Evaluationsmethodik und -kriterien} \label{Evaluationsmethodik und -kriterien}
\subsubsection{SSI-Framework-Vergleich} \label{SSI-Framework-Vergleich}
\subsubsection{Blockchain-Plattform-Auswahl} \label{Blockchain-Plattform-Auswahl}
\subsubsection{Technologie-Stack-Integration} \label{Technologie-Stack-Integration}
\subsubsection{Auswahlentscheidungen und Begründung} \label{Auswahlentscheidungen und Begründung}

- Kriterien für die Framework-Auswahl (Open Source, PQC-Kompatibilität) \\
- Vergleich und Entscheidung (z.B. Hyperledger Indy/Aries) \\
- Anpassungen und Erweiterungen für die ermittelten Anforderungen


Infrastructure ==> indy on besu \\
Identifier + Cryptographic material ==> Indy DID \\
Credential \& Presentations ==> AnonCreds W3C VCs \\
Agent ==> ACA-Py


- Open Source
- Modular aufgebaut sein


F

Begründung der Kategorien und wissenschaftliche Quellen
Architektur/Blockchain

    Die Wahl der Architektur bestimmt Skalierbarkeit, Transaktionsmodell, Dezentralität und die regulatorische Einwirkbarkeit (permissioned vs. permissionless). Frameworks wie Hyperledger Indy, Besu und ION werden in der Literatur gezielt nach Blockchain-Typ, Governance und Infrastruktur verglichen.

Quelle: „Self-sovereign identity on the blockchain: contextual analysis and ...“ (Frontiers in Blockchain, 2024); Härer \& Fill, 2020.
Offenheit \& Transparenz

    Open-Source-Frameworks ermöglichen Auditierbarkeit, Anpassung und Community-getriebene Innovation – zentral für Trust und Security in KRITIS.

Quelle: Frontiers in Blockchain, 2024; Fill \& Härer, 2020.
Interoperabilität

    Die Fähigkeit eines Systems, über W3C-DID-, VC- und andere Standards mit verschiedensten Ökosystemen und Registern zu kommunizieren, wird als Schlüsselanforderung klassifiziert.

Quelle: Fraunhofer \& Universität Bayreuth Diskussionspapier zu SSI, 2020; Grüner et al. „Analyzing Interoperability and Portability Concepts...“ (HPI).
SSI-Prinzipien-Compliance

    Die 10 Grundprinzipien nach Christopher Allen (Existenz, Kontrolle, Interoperabilität, Transparenz, Portabilität, etc.) sind das wissenschaftliche Fundament für die Bewertung von SSI-Systemen und werden in praktisch jedem SSI-Vergleich herangezogen.

Quelle: Christopher Allen (Blog; 2017), überführt in diverse Whitepaper und Publikationen, z.B. Societybyte/SFI.
PQC-Unterstützung

    Post-Quanten-Kryptografie ist ein verpflichtender Baustein für KRITIS-Sicherheit – das BSI und Fraunhofer empfehlen die Integration quantenresistenter Algorithmen speziell für Identitäts- und Signaturlösungen.

Quelle: BSI, „Post-Quanten-Kryptografie“; Fraunhofer Cybersecurity Blog, „Post-Quanten-Kryptografie in der Praxis“.
Eignung für KRITIS

    Die Tauglichkeit für kritische Infrastrukturen erfordert geprüfte Governance-, Compliance- und Skalierungsmechanismus, wie vom BSI und in wissenschaftlichen Evaluationsframeworks für Public-Key-Infrastruktur und Identitätssysteme festgelegt.

Quelle: ETH Zürich, „Evaluierungs-Framework und Kriterienkatalog für Public-Key-Infrastrukturen“; Fraunhofer FIT (SSI \& KRITIS Fokus).


\begin{longtable}{L{1cm}L{2cm}L{2.5cm}L{2cm}L{1.5cm}L{1.5cm}L{1.5cm}}
    \caption{Vergleich ausgewählter SSI-Frameworks für blockchain-basierte, KRITIS-taugliche Prototypen mit PQC-Perspektive}
    \label{tab:ssi-frameworks} \\
    \toprule
    \textbf{Framework} & \textbf{Architektur/\newline Blockchain} & \textbf{Offenheit \& Transparenz} & \textbf{Interoperabilität} & \textbf{SSI-Prinzipien-\newline Compliance} & \textbf{PQC-\newline Unterstützung} & \textbf{Eignung für KRITIS} \\
    \midrule
    \endfirsthead
    \multicolumn{7}{l}{\textit{Tabelle \thetable\ (Fortsetzung)}} \\
    \toprule
    \textbf{Framework} & \textbf{Architektur/\newline Blockchain} & \textbf{Offenheit \& Transparenz} & \textbf{Interoperabilität} & \textbf{SSI-Prinzipien-\newline Compliance} & \textbf{PQC-\newline Unterstützung} & \textbf{Eignung für KRITIS} \\
    \midrule
    \endhead
    \midrule
    \multicolumn{7}{r}{\textit{Fortsetzung auf nächster Seite}} \\
    \endfoot
    \bottomrule
    \multicolumn{7}{p{\linewidth}}{\textit{Anmerkung.} Quellen: Eigene Darstellung nach \cite{frontiersssi2024,lfbesublog2025,marketanalysis2025}} \\
    \endlastfoot
    Hyperledger Indy &
    Federated Indy Blockchain &
    Open Source, starker Audit-Trail &
    W3C VC/DID, hohe Interoperabilität &
    Sehr hoch (Existenz, Kontrolle, Transparenz) &
    Experimentell (kein nativer PQC-Support) &
    Sehr hoch (Governance, Auditierbarkeit) \\
    \midrule
    ION (Microsoft/DIF) &
    Bitcoin, Sidetree DPKI &
    Open Source, breite Transparenz &
    Dezentral, weltweit, W3C DIDs &
    Hoch (Portabilität, Kontrolle, Skalierung) &
    Erweiterbar, keine native PQC &
    Mittel (abhängig von Bitcoin, schwer reg. steuerbar) \\
    \midrule
    Indy on Besu &
    Besu Ethereum (Permissioned) &
    Open Source, modular, Enterprise-tauglich &
    W3C VC/DID, did:indy:besu, Legacy-Migration &
    Sehr hoch (volle SSI-Rollen, flexible Governance) &
    PQC-Addons via Smart Contracts, nativ in Entwicklung &
    Sehr hoch (10x Durchsatz, Trusted List, flexible Governance) \\
\end{longtable}

indy besu ==> \parencite{shcherbakov_HyperledgerIndyBesupermissionedledgerSelfsovereignIdentity_2024}



