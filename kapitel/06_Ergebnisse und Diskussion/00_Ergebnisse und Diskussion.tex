\newpage
\section{Ergebnisse und Diskussion} 
\label{sec:Ergebnisse und Diskussion}

\subsection{Beantwortung der Forschungsfragen} 
\label{sec:Beantwortung der Forschungsfragen}

Die vorliegende Arbeit adressiert durch \ac{DSR} Methodologie drei zentrale Forschungsfragen an der Schnittstelle von \ac{SSI}, \ac{PQC} und \ac{KRITIS}. Die in Kapitel~\ref{sec:Summative Evaluation} durchgeführte summative Evaluation unter Anwendung des FEDS-Frameworks demonstriert, dass das entwickelte Artefakt diese Fragen nicht nur theoretisch sondern empirisch durch kontrollierte Evaluation beantwortet.

\textbf{FF1 -- Systemarchitektur \& Compliance:}
Wie kann ein blockchain-basiertes \ac{SSI}-System unter Einsatz von \ac{PQC} gestaltet werden, um die regulatorischen und technischen Anforderungen von \ac{KRITIS} nachhaltig zu erfüllen?

Diese Forschungsfrage wird durch die Konzeption und Implementierung einer zweischichtigen Architektur beantwortet, die quantenresistente Kryptografie orthogonal und nicht-invasiv in bestehende \ac{SSI}-Infrastrukturen integriert und dem mehrschichtigen Sicherheitsansatz \enquote{Defense in Depth} nach \textcite[S. 242--243]{alsaqour_DefenseDepthMultilayersecurity_2021} folgt. 

Die erste Schicht realisiert Quantensicherheit auf der Transportebene (\gls{Data-In-Motion}) mittels eines \gls{Sidecar Proxy} Patterns mit nginx, das \ac{TLS} 1.3 als Protokoll-Standard durchsetzt und ein \glslink{Hybride Schemata}{hybrides Schlüsseleinigungsverfahren} (X25519 + ML-KEM-768) implementiert. Parallel dazu werden digitale Zertifikate mit ML-DSA-65-Signaturen ausgestellt und validiert, was die Post-Quantum-Authentifizierung von Netzwerk-Endpunkten sicherstellt. Diese nicht-invasive Architektur erfüllt die Transition-Anforderung nach Backward-Kompatibilität \parencite[S. 3]{mamatha_PostQuantumCryptographySecuringDigitalCommunicationQuantumEra_2024}. SSI-Agenten (ACA-Py), Blockchain-Knoten (Hyperledger Indy) und Wallet-Applikationen benötigen keine Modifikation ihrer Anwendungslogik, da die Quantensicherung transparent auf der Netzwerk-Infrastruktur-Ebene erfolgt.

Die zweite Schicht erweitert die Quantensicherheit auf die Applikationsebene (\gls{Data-At-Rest} und Credential-Verarbeitung) durch native PQC-Integration in Hyperledger \ac{ACA-Py}. Mittels einer \gls{Monkey-Patching}-Strategie werden zur Laufzeit kritische Komponenten überschrieben. Die \ac{DID}-Verarbeitung ermöglicht es dezentrale Identifikatoren mit ML-DSA-65-Signaturen zu erzeugen und zu validieren. Gleichzeitig unterstützt die \gls{DIDComm}-Envelope-Verarbeitung ML-KEM-768-basierte Schlüsselkapseln für die Nachrichtenverschlüsselung. Diese duale Implementierung folgt dem Sicherheitsprinzip der Defense-in-Depth, bei welcher gemäß \textcite[S. 242--243]{alsaqour_DefenseDepthMultilayersecurity_2021} durch mehrschichtige Sicherheitsmaßnahmen ein Ausfallschutz garantiert wird, sodass der Kompromiss einer einzelnen Schicht nicht zum Systemversagen führt.

Die Validierung aller neun Compliance-Anforderungen (CR1--CR9) in Kapitel~\ref{sec:Validierung der KRITIS-Compliance-Anforderungen} demonstriert empirisch, dass die Architektur sowohl die kryptografischen Parameter für ML-DSA-65 und ML-KEM-768 \parencite[Kap. 2.4.3, 5.3.4.2]{bsi_BSITR021021KryptographischeVerfahrenEmpfehlungenundSchluessellaengenVersion202501_2025}, als auch die organisatorischen Anforderungen der \ac{DSGVO} an Privacy by Design \parencite[Art. 25]{daseuropaeischeparlamentundderratdereuropaeischenunion_VerordnungEU2016679EuropaeischenParlamentsundRatesvom27April2016_2016}, den Grundsatz der Datenminimierung \parencite[Art. 5]{daseuropaeischeparlamentundderratdereuropaeischenunion_VerordnungEU2016679EuropaeischenParlamentsundRatesvom27April2016_2016} und das Recht auf Löschung \parencite[Art. 17]{daseuropaeischeparlamentundderratdereuropaeischenunion_VerordnungEU2016679EuropaeischenParlamentsundRatesvom27April2016_2016} erfüllt (CR7--CR9). Insbesondere werden die Anforderungen zur strikten Netzsegmentierung \parencite[Control A.8.22]{iso/iec_ISOIEC270012022InformationsecuritycybersecurityprivacyprotectionInformationsecuritymanagement_2022} (CR6) und die Protokollierung sicherheitsrelevanter Ereignisse \parencite[Nr. 101, 103]{bsi_KonkretisierungKRITISAnforderungen8aAbsatz1undAbsatz1aBSIG_2024} (CR5) vollständig operationalisiert. Das Artefakt demonstriert somit, dass quantenresistente Kryptografie für die Anwendungsdomäne \ac{KRITIS} nicht nur ein technisches Problem, sondern ein systemisches Problem ist, das Compliance-, Architektur- und Governance-Aspekte simultan adressieren muss.

\textbf{FF2 -- Algorithmenauswahl \& Sicherheitsbewertung:}
Welche \ac{PQC}-Algorithmen eignen sich für die Integration in \ac{SSI}-Systeme hinsichtlich Sicherheit und Interoperabilität, insbesondere im Kontext von \ac{KRITIS}?

Diese Forschungsfrage wird durch die empirische Validierung der \ac{NIST}-standardisierten \enquote{Security-Kategorie 3} Algorithmen ML-DSA-65 \parencite[Tab. 1]{nationalinstituteofstandardsandtechnologyus_ModulelatticebasedDigitalSignaturestandard_2024} und ML-KEM-768 \parencite[S. 14]{nationalinstituteofstandardsandtechnologyus_ModulelatticebasedKeyencapsulationMechanismstandard_2024} beantwortet. Die bewusste Auswahl dieser Parameter-Sets entspricht der direkten Empfehlung des \ac{BSI} \parencite[S. 41, 58]{bsi_BSITR021021KryptographischeVerfahrenEmpfehlungenundSchluessellaengenVersion202501_2025}.

Die empirische Evaluation in Kapitel~\ref{sec:Validierung der funktionalen Anforderungen} und Kapitel~\ref{sec:Validierung der KRITIS-Compliance-Anforderungen} zeigt, dass diese Algorithmen nicht primär durch intrinsische Sicherheitsdefizite limitiert sind, sondern durch technische Integrationschallenges in bestehende Hyperledger-Aries-Infrastrukturen, die originär auf elliptischen Kurven (X25519, Ed25519) basieren und keine native Post-Quantum-Kryptografie-Unterstützung bieten \parencite[S. 4733, 4735]{badertscher_WhatDidComeOutItAnalysisImprovementsDIDCommMessaging_2024}. Beide Verfahren basieren auf der rechnerischen Schwierigkeit gitterbasierter Probleme, insbesondere dem Shortest Vector Problem und dem Module Learning With Errors Problem, deren mathematische Härte rigoros analysiert ist \parencite[S. 1]{zong_MathematicalFoundationPostQuantumCryptography_2025}. Die Infrastruktur-Kompatibilität adressiert die Arbeit durch drei komplementäre Mechanismen.

Erstens ermöglicht \ac{TLS} 1.3-Algorithmen-Aushandlung es Endpunkten, zwischen hybriden Cipher-Suites und rein klassischen Verfahren einen Fallback zu nutzen \parencite[S. 26]{rescorla_TransportLayerSecurityTLSProtocolVersion13_2018}, was die schrittweise Migration für heterogene Deployments vereinfacht. Dies wird durch eine konfigurationsbasierte Fallback-Chain in Listing~A-\ref{lst:nginx_holder.conf} realisiert, die automatisch auf rein X25519 zurückfällt.

Zweitens detektiert die erweiterte Envelope-Verarbeitung anhand von Multicodec-Präfixen automatisch, ob \ac{PQC} oder klassische Algorithmen erforderlich sind. Der OutOfBandManager evaluiert hierfür den keytype-Parameter via create\_did\_peer\_4\_conditional\_pqc und delegiert bei ed25519 zur ursprünglichen ACA-Py-Implementierung, während das System standardmäßig ML-DSA-65 und ML-KEM-768 aktiviert (Listing~A-\ref{lst:base_manager_patch.py}, Listing~A-\ref{lst:Jupyter-Notebook-Cell-9-Demonstration-Kryptoagilität-ed25519} und Listing~A-\ref{lst:Jupyter-Notebook-Cell-9-Demonstration-Kryptoagilität-ed25519-output}). Dies ist eine notwendige Erweiterung, da \gls{DIDComm}-v1 keine native Kryptoagilität bietet und ausschließlich auf X25519 für Key Exchange sowie Ed25519 für Signing festgelegt ist \parencite[S. 4735]{badertscher_WhatDidComeOutItAnalysisImprovementsDIDCommMessaging_2024}. Bedingt dadurch, dass \gls{DIDComm} keine Algorithmen-Aushandlung vorsieht \parencite[S. 4733]{badertscher_WhatDidComeOutItAnalysisImprovementsDIDCommMessaging_2024} stellt die PQC-Integration eine nicht-standardkonforme Erweiterung dar, die durch Monkey-Patching der Verarbeitungslogik realisiert werden muss (Listing~A-\ref{lst:base_manager_patch.py} und Listing~A-\ref{lst:Jupyter-Notebook-Cell-9-Demonstration-Kryptoagilität-ed25519}). 

Drittens schafft die Multikey-Kodierung mittels Multicodec-Registry die kryptographische Identifikation neuer Algorithmen ohne Schema-Änderungen im \ac{DID}-Dokument. Das Aries-Askar-Wallet kodiert ML-DSA-65- und ML-KEM-768-Schlüsselpaare mit den definierten Multicodec-Präfixen Listing~A-\ref{lst:pqc_multicodec.py} und konvertiert diese zu Multikey-Strings Listing~A-\ref{lst:pqc_multikey.py}, die in der SQLite-Datenbank persistiert werden, ohne dass die DID-Dokumentstruktur angepasst werden muss.

Die Hybrid-Strategie, die klassische und quantenresistente Algorithmen kombiniert, ist ein Best-Practice für die aktuelle Übergangswelt \parencite[S. 22]{bsi_BSITR021021KryptographischeVerfahrenEmpfehlungenundSchluessellaengenVersion202501_2025}. Die Kombination von X25519 (ECC, etabliert sicher) mit ML-KEM-768 (PQC, mathematisch neu) garantiert, dass die Gesamtsicherheit mindestens so stark wie der stärkere der beiden Algorithmen ist.

\textbf{FF3 -- Kryptografische Agilität:} 
Welche kryptografischen Agilitätsmechanismen sind erforderlich, um zukünftige \ac{PQC}-Algorithmenupdates ohne Systemunterbrechung zu ermöglichen?

Diese Forschungsfrage wird durch die Implementierung zweier orthogonaler, schichtenspezifischer Agilitätsmechanismen beantwortet, die das Open-Closed-Prinzip nach \textcite[S. 99]{martin_AgileSoftwareDevelopmentprinciplespatternspractices_2003} realisieren und damit die Eigenschaften Extensibility, Removability und Fungibility nach \textcite[S. 102--103]{mehrez_CryptoAgilityProperties_2018} operationalisieren. Die Architektur adressiert kryptografische Agilität nicht als nachträgliche Erweiterung, sondern als fundamentales Designprinzip, das durch systematische Separation of Concerns auf beiden Schichten verankert wird.

Auf der Transportebene wird Agilität durch die Protokoll-inhärente Aushandlungslogik von \ac{TLS} 1.3 realisiert. Durch die Konfiguration einer präferierten Cipher-Suite-Reihenfolge im \gls{Sidecar Proxy} wird ein automatischer Fallback-Mechanismus etabliert, der hybride Verfahren (X25519+ML-KEM-768) priorisiert, jedoch transparent auf klassische Verfahren (X25519) zurückfällt, sofern die Gegenseite keine \ac{PQC}-Unterstützung signalisiert. Diese Entkopplung ermöglicht den Austausch kryptografischer Primitive durch reine Konfigurationsanpassungen ohne Re-Kompilierung oder Systemunterbrechung, was die Anforderungen an Extensibility und Removability nach \textcite[S. 102]{mehrez_CryptoAgilityProperties_2018} erfüllt.

Auf der Applikationsebene adressiert das entwickelte Plugin-System die fehlende native Agilität von \gls{DIDComm}-v1 \parencite[S. 4733]{badertscher_WhatDidComeOutItAnalysisImprovementsDIDCommMessaging_2024} durch eine modulare Architektur nach dem Software Stability Model von \textcite[S. 1--2]{hamza_SeparationConcernsEvolvingsystemsstabilitydrivenapproach_2005}. Die Stratifizierung in stabile abstrakte Operationen (\ac{EBT}), adaptierbare Geschäftsobjekte (\ac{BO}) wie die Multicodec-Registry und volatile Implementierungen (\ac{IO}) erlaubt es, Algorithmen austauschbar zu gestalten. Mittels \gls{Monkey-Patching} (\ac{IO}) werden zur Laufzeit kritische ACA-Py-Funktionen überschrieben, während die \ac{BO}-Schicht die Koexistenz verschiedener Schlüsseltypen (z.\,B. Ed25519 neben ML-DSA-65) ohne Schema-Änderungen an \ac{DID}-Dokumenten ermöglicht.

Die empirische Evaluation in Kapitel~\ref{sec:Validierung der Kryptoagilität} bestätigt, dass durch diese Mechanismen ein heterogener Mischbetrieb möglich ist: Das System etabliert erfolgreich PQC-gesicherte Verbindungen, während es für Legacy-Agenten automatisch und unterbrechungsfrei auf klassische Algorithmen (Ed25519) zurückfällt (Listing~A-\ref{lst:Jupyter-Notebook-Cell-9-Demonstration-Kryptoagilität-ed25519-output}).

\subsection{Kritische Reflexion}
\label{sec:Kritische Reflexion}

Die vorliegende Arbeit demonstriert erfolgreich, dass die Entwicklung eines blockchain-basierten \ac{SSI}-Prototypen unter Integration von \ac{PQC} technisch machbar und unter Laborbedingungen validierbar ist. Die iterative Anwendung der \ac{DSR}-Methodik sowie die umfassende Evaluation gegen funktionale und Compliance-Anforderungen stellen einen substantiellen Beitrag dar, der die Forschungslücke zwischen abstrakten \ac{PQC}-Standards und ihrer praktischen Umsetzung in dezentralen Identitätssystemen verringert.

Gleichwohl verdient eine kritische Reflexion bestimmte Designentscheidungen zu beleuchten, um sowohl die Stärken als auch die bewussten Limitation dieser Arbeit transparent darzulegen. Die Nutzung von \gls{Monkey-Patching} zur Laufzeitmodifikation von \ac{ACA-Py}-Klassen ist pragmatisch und methodisch im Kontext eines akademischen Prototypen gerechtfertigt, da das Framework zum Erhebungszeitpunkt keine nativen Mechanismen für flexible Kryptografieauswahl auf der Applikationsebene bietet. Diese Implementierungstechnik erlaubt es, PQC-Algorithmen zu injizieren, ohne den ursprünglichen Quellcode zu verändern. Für eine potenzielle Produktionisierung währe jedoch eine saubere Architekturrefaktorisierung oder die Integration nativer Plugin-Schnittstellen erforderlich, um Anforderungen der Code-Transparenz und statischen Analyse, die für KRITIS-Zertifizierungen zentral sind, vollständig zu erfüllen. Dies stellt weniger eine Schwäche des Prototypen, welcher im begrenzten Rahmen dieser Masterarbeit entstand, dar, sondern verdeutlicht vielmehr die Notwendigkeit, dass die Hyperledger-Community zukünftig Kryptoagilität als Design-Prinzip in ihre Kernarchitktur integriert.

Ein zweiter Aspekt betrifft die Protokollebene und die Wahl von DIDComm v1. Diese Entscheidung beruhte auf der zum Zeitpunkt der Entwicklung verfügbaren Stabilität und Reife der Bibliotheken. Da DIDComm v1 jedoch kryptografisch strikt auf elliptische Kurven normiert ist, musste die Kryptoagilität durch eine hybride Kapselung auf der Transport-Ebene (TLS 1.3 mit Sidecar-Proxies) realisiert werden. Das Resultat ist architektonisch konsistent und funktioniert zuverlässig in der isolierten Laborumgebung. Die Interoperabilität mit Standard-SSI-Wallets, die diese Erweiterungen nicht verstehen, ist jedoch begrenzt. Die Arbeit trägt damit nicht nur zu einem funktionierenden Artefakt bei, sondern identifiziert auch, wo Standards weiterentwickelt werden müssen, um zukünftige Anforderungen zu erfüllen.

Bezüglich der Evaluationsstrategie wurde bewusst der Fokus auf Efficacy (Wirksamkeit und funktionale Korrektheit) gelegt, während Efficiency-Metriken (Performance, Latenz, Speicheroverhead) nicht erhoben wurden. Diese Abgrenzung ist methodisch begründet, da bei sicherheitskritischen Systemen die funktionale Integrität Vorrang hat. Nichtsdestotrotz ist anzumerken, dass die größeren Schlüssel und Signaturen der \ac{PQC}-Algorithmen (insbesondere ML-DSA und ML-KEM) in skalierten Blockchain-Szenarien zu Speicher- und Netzwerk-Engpässen führen könnten. Eine zukünftige Erweiterung dieser Arbeit sollte systematisch untersuchen, wie sich diese \ac{PQC}-Overhead-Faktoren auf die Throughput und Latenz verteilter Ledger auswirken und welche Optimierungsstrategien (z. B. Batch-Signierung, Compression) möglich sind.

Schließlich ist eine Limitation in der Evaluationsdimension zu reflektieren. Die Validierung der Compliance-Anforderungen erfolgte primär auf technischer Ebene, etwa durch den Nachweis, dass ML-DSA-Schlüssellängen den NIST-Vorgaben entsprechen oder dass TLS 1.3 erzwungen wird. Echte KRITIS-Compliance ist jedoch ein soziotechnischer Prozess, der Audits, Rollenkonzepte, Incident-Response-Verfahren und organisatorische Kontrollen umfasst. Der entwickelte Prototyp demonstriert technische Compliance Readiness, nicht die volle organisatorische Compliance. Dies ist jedoch eine erwartete Limitation einer akademischen Arbeit und keine Schwäche des Artefakts.

\subsection{Wissenschaftliche und praktische Beiträge}
\label{sec:Wissenschaftliche und praktische Beiträge}

Diese Arbeit leistet einen ersten umfassenden Beitrag zur Integration der drei bislang isoliert betrachteten Forschungsdomänen \ac{SSI}, \ac{PQC} und \ac{KRITIS} in einem kohärenten und empirisch validierten Systemdesign. Die existierende Literatur adressiert typischerweise zwei der drei Domänen \fixme{SIEHE KAPITEL 1.2 STAND DER FORSCHUNG}, während die Formulierung einer ganzheitlichen, quanten-sicheren SSI-Architektur speziell für \ac{KRITIS}-Kontexte bislang nicht konsistent durchgeführt wurde. Ein zentrales Ergebnis ist die Einsicht, dass quantenresistente Kryptografie nicht als bloße Cryptography-Substitution verstanden werden darf, sondern als ein Architektur-Problem, das fundamentale Anforderungen an die Systemgestaltung stellt. Die Integration größerer Schlüssel ohne Netzwerk-Overhead, die Gewährleistung von Interoperabilität zwischen PQC-fähigen und Legacy-Systemen, die Realisierung von Kryptoagilität sowie die Wahrung von Privacy-by-Design bei veränderten Signatur-Algorithmen erfordern spezifische technische Lösungsmuster wie Sidecar Proxies für transparente Transport-Layer-Sicherung, Monkey-Patching für Applikations-Layer-Agilität, Multicodec-Extensibilität und sorgfältige Governance-Modelle.

Ein weiterer wissenschaftlicher Beitrag ist die formale Charakterisierung von schichtenspezifischer Kryptoagilität gemäß dem Software Stability Model nach Hamza. 
Während Transport-Layer-Agilität durch TLS 1.3 mittels Mechanismen wie Cipher Suite Negotiation und Post-Handshake Key Updates adressiert wird, erfordert Applikations-Layer-Agilität die Implementierung von Rückwärtskompatibilität auf mehreren Ebenen. Insbesondere müssen DIDComm-Envelope-Verarbeitung, Credential-Verifikation und Revocation-Mechanismen so gestaltet werden, dass Systeme mit unterschiedlichen Algorithmus-Generationen interoperieren können, ohne bestehende Verifikationsketten zu invalidieren. Das Plugin-Modell demonstriert, dass durch Separation von stabilen Abstraktionsschichten (\ac{EBT}), adaptiven Konkretisierungsschichten (\ac{BO}) und volatilen Implementierungsdetails (\ac{IO}) nach \textcite[S. 2]{hamza_SeparationConcernsEvolvingsystemsstabilitydrivenapproach_2005} Algorithmen-Updates nicht als Breaking Changes kodifiziert werden müssen. Neue Algorithmen werden stattdessen als \ac{BO}-Erweiterungen behandelt, welche die stabilen \ac{EBT}-Interfaces respektieren, wodurch Kryptoagilität nicht länger als architektonische Belastung, sondern als natürliche Konsequenz strukturierter Modularität entsteht.

Zu den praktischen Beiträgen zählt das entwickelte Artefakt, bestehend aus Sidecar-Proxy-Templates, ACA-Py-Plugin-Code, vollständiger Docker-Compose-Orchestrierung und ausführlichen Jupyter-Notebooks. Die Dokumentation basierend auf realistischen \ac{KRITIS}-Szenarios ermöglicht es, die Technologie-Anwendbarkeit unabhängig zu bewerten. Über die bloße Bereitstellung des Artefakts hinaus leistet die modulare Architektur einen praktischen Beitrag durch die Ermöglichung inkrementeller Adoptionsstrategien, die der organisatorischen Realität entsprechen. Eine Organisation könnte zunächst nur die Sidecar-Proxy-Layer deployieren, um ihre bestehende SSI-Infrastruktur unmittelbar quantensicher zu gestalten, und später zur Applikations-Layer-Integration übergehen. Diese Strategie vermeidet Big-Bang-Deployments und ermöglicht gradueller, reversibel-testbare Übergänge, die dem organisatorischen Änderungsmanagement entsprechen.

Das Prinzip \enquote{Defense in Depth} von \textcite[S. 242--243]{alsaqour_DefenseDepthMultilayersecurity_2021} motivierte ein iteratives Design-Refinement mit wissenschaftlicher und praktischer Relevanz. Dieses Vorgehen spiegelt authentische Entwicklungsprozesse wider und fungiert als Blaupause für andere Projekte im Bereich der Kryptografie-Migration. Auf technologischer Ebene wird durch die Konstruktion des liboqs-Python-Bindings mit Multicodec-Integration gezeigt, dass moderne Kryptografiebibliotheken in bestehende Identitäts-Ökosysteme integrierbar sind. Die Implementierung der didpeer4-DID-Methode mit \ac{PQC}-Unterstützung erweitert die W3C-Spezifikationen um eine quantenresistente Variante, ohne die bestehenden ed25519-Verfahren zu beeinträchtigen. Diese Strategie folgt dem Prinzip adaptiver Modularität, bei dem neue Algorithmen als Erweiterungen bestehender Schnittstellen implementiert werden, anstatt fundamentale Architekturänderungen zu erzwingen.