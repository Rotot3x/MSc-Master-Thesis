\newglossaryentry{CRQC}{
    name={Cryptographically Relevant Quantum Computer (CRQC)},
    description={beschreibt das Stadium, in dem Quantencomputer aktuelle Kryptosysteme brechen können \parencite[S. 338]{geremew_PreparingCriticalInfrastructurePostQuantumCryptographyStrategiesTransitioningAheadCryptanalyticallyRelevantQuantumComputing_2024}}
}
\newglossaryentry{Kryptoagilität}{
    name={Kryptoagilität},
    description={gewährleistet die langfristige Sicherheit von Kryptografie. Kryptoagile Systeme sind offen für Innovationen in der Kryptografie und verbessern die Widerstandsfähigkeit bei Angriffen auf einzelne Verfahren. Kryptoagilität ist eine Voraussetzung für Innovationen in der Kryptografie und für langfristige Sicherheit. Sie fördert die technologische Souveränität und ihre kluge Umsetzung bringt Wettbewerbsvorteile \parencite[S. 1]{kreutzer_Kryptoagilitaet_2024}}
}
\newglossaryentry{EBSCO}{
    name={EBSCO},
    description={ist eine umfassende, multidisziplinäre Datenbank, die eine breite Palette von akademischen Zeitschriften, Magazinen und anderen Publikationen abdeckt. Sie bietet Zugang zu einer Vielzahl von Fachgebieten, einschließlich Informatik, Cybersicherheit und Politikwissenschaften. Dies ist besonders wertvoll für das Thema, da es die Schnittstelle zwischen Technologie und nationaler Sicherheit betrifft \parencite{_EBSCOInformationServices_a}}
}
\newglossaryentry{SzA}{
    name={SzA},
    description={Systeme zur Angriffserkennung sind im Sinne des \textcite[§ 2 Absatz 9b]{bundesministeriumderjustiz_GesetzUeberBundesamtfuerSicherheitInformationstechnikBSIGesetzBSIG_2009} durch technische Werkzeuge und organisatorische Einbindung unterstützte Prozesse zur Erkennung von Angriffen auf informationstechnische Systeme. Die Angriffserkennung erfolgt dabei durch Abgleich der in einem informationstechnischen System verarbeiteten Daten mit Informationen und technischen Mustern, die auf Angriffe hindeuten}
}
\newglossaryentry{Privacy by Design}{
    name={Privacy by Design},
    description={\fixme{ergänzen}}
}
\newglossaryentry{Privacy by Default}{
    name={Privacy by Default},
    description={\fixme{ergänzen}}
}
\newglossaryentry{Sidecar Proxy}{
    name={Sidecar Proxy},
    description={
    ist ein zusätzlicher Microservice, der neben Business-Microservices in einem Pod bereitgestellt wird, um Proxy-Funktionen bereitzustellen, welche sämtliche ein- und ausgehende Kommunikation eines Pods abfangen und kryptografisch schützen, ohne die zugrundeliegende Anwendungslogik zu modifizieren \parencite[S. 1--2]{berlato_WorkinProgressSidecarProxyUsablePerformanceAdaptableEndtoEndProtectionCommunicationsCloudNativeApplications_2024}
    }
}
\newglossaryentry{Hybride Schemata}{
    name={Hybride Schemata},
    description={
    kombinieren zwei oder mehr Algorithmen derselben Kategorie, sodass das Gesamtsystem seine Sicherheit behält, solange mindestens eine der verwendeten Komponenten als sicher gilt \parencite[S. 3]{bindel_HybridKeyEncapsulationMechanismsAuthenticatedKeyExchange_2019}
    }
}
\newglossaryentry{Data-In-Motion}{
    name={Data-In-Motion},
    description={
    bezieht sich auf Daten, die aktiv von einem Ort zu einem anderen über ein Netzwerk oder von einem lokalen Speichergerät zu einem Cloud-Speicher übertragen werden \parencite[S. 62]{swanzy_DataSecurityFrameworkProtectingDataTransitDataRestCloud_2024}
    }
}
\newglossaryentry{Data-At-Rest}{
    name={Data-At-Rest},
    description={
    bezieht sich auf Daten, die sich auf einem Computer, einem Server oder irgendwo in der Cloud befinden \parencite[S. 62]{swanzy_DataSecurityFrameworkProtectingDataTransitDataRestCloud_2024}
    }
}
\newglossaryentry{Monkey-Patching}{
    name={Monkey-Patching},
    description={
    bezeichnet eine Technik im Software-Engineering, bei der zur Laufzeit bestehende Klassen oder Methoden dynamisch verändert oder erweitert werden, ohne den ursprünglichen Quellcode zu modifizieren \parencite[S. 2]{luo_EasyVitessceAutomagicallyAddinginteractivityScversesinglecellspatialbiologyplots_2025}
    }
}
\newglossaryentry{DIDComm}{
    name={DIDComm},
    description={
    bezeichnet nach \textcite[S. 4732]{badertscher_WhatDidComeOutItAnalysisImprovementsDIDCommMessaging_2024} ein kryptographisches Framework zur Etablierung sicherer Kommunikation zwischen Entitäten mit Decentralized Identifiers, das Sender-Anonymität und Authentizität kombiniert. DIDComm v1 spezifiziert ausschließlich X25519 für Key Exchange und Ed25519 für digitale Signaturen ohne Algorithmen-Aushandlung, wodurch keine native Kryptoagilität vorliegt \parencite[S. 4733, 4735]{badertscher_WhatDidComeOutItAnalysisImprovementsDIDCommMessaging_2024}.
    }
}
