\newglossaryentry{CRQC}{
    name={Cryptographically Relevant Quantum Computer (CRQC)},
    description={beschreibt das Stadium, in dem Quantencomputer aktuelle Kryptosysteme brechen können \parencite[S. 338]{geremew_PreparingCriticalInfrastructurePostQuantumCryptographyStrategiesTransitioningAheadCryptanalyticallyRelevantQuantumComputing_2024}}
}
\newglossaryentry{Kryptoagilität}{
    name={Kryptoagilität},
    description={gewährleistet die langfristige Sicherheit von Kryptografie. Kryptoagile Systeme sind offen für Innovationen in der Kryptografie und verbessern die Widerstandsfähigkeit bei Angriffen auf einzelne Verfahren. Kryptoagilität ist eine Voraussetzung für Innovationen in der Kryptografie und für langfristige Sicherheit. Sie fördert die technologische Souveränität und ihre kluge Umsetzung bringt Wettbewerbsvorteile \parencite[S. 1]{kreutzer_Kryptoagilitaet_2024}}
}
\newglossaryentry{EBSCO}{
    name={EBSCO},
    description={ist eine umfassende, multidisziplinäre Datenbank, die eine breite Palette von akademischen Zeitschriften, Magazinen und anderen Publikationen abdeckt. Sie bietet Zugang zu einer Vielzahl von Fachgebieten, einschließlich Informatik, Cybersicherheit und Politikwissenschaften. Dies ist besonders wertvoll für das Thema, da es die Schnittstelle zwischen Technologie und nationaler Sicherheit betrifft \parencite{_EBSCOInformationServices_a}}
}
\newglossaryentry{SzA}{
    name={SzA},
    description={Systeme zur Angriffserkennung sind im Sinne des \textcite[§ 2 Absatz 9b]{bundesministeriumderjustiz_GesetzUeberBundesamtfuerSicherheitInformationstechnikBSIGesetzBSIG_2009} durch technische Werkzeuge und organisatorische Einbindung unterstützte Prozesse zur Erkennung von Angriffen auf informationstechnische Systeme. Die Angriffserkennung erfolgt dabei durch Abgleich der in einem informationstechnischen System verarbeiteten Daten mit Informationen und technischen Mustern, die auf Angriffe hindeuten}
}
\newglossaryentry{Privacy by Design}{
    name={Privacy by Design},
    description={\fixme{ergänzen}}
}
\newglossaryentry{Privacy by Default}{
    name={Privacy by Default},
    description={\fixme{ergänzen}}
}