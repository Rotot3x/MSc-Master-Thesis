\section*{Abstract} \label{sec:Abstract}
\addcontentsline{toc}{section}{Abstract}
\small

\fixme{Die vorliegende Seminararbeit befasst sich mit der Ableitung eines Maßnahmenkataloges auf Basis der NIS-2-Richtlinie (EU) 2022/2555 unter Berücksichtigung der bestehenden Controls der ISO 27001:2022 mit dem Ziel die Anforderungen der NIS-2-Richtlinie zu analysieren und mit den bereits etablierten Controls der ISO 27001:2022 abzugleichen. Auf diese Weise sollen etwaige Lücken identifiziert und notwendige Ergänzungen vorgenommen werden. Im Rahmen einer detaillierten GAP-Analyse werden die Unterschiede zwischen den Anforderungen der NIS-2 und den bestehenden ISO-27001:2022-Controls ermittelt. Auf Basis der ermittelten Ergebnisse werden spezifische neue Controls entwickelt, um die identifizierten GAPs zu schließen und somit einen umfassenden Maßnahmenkatalog zu erstellen, der als praxisorientierte Leitlinie für Unternehmen dient. Die vorliegende Arbeit basiert auf einem \acl*{DSR}-Ansatz, welcher durch die Entwicklung und Evaluierung von Artefakten zur Lösung spezifischer Probleme beiträgt. In der abschließenden Betrachtung werden die entwickelten Maßnahmen einer kritischen Reflexion unterzogen. Zudem werden die Herausforderungen sowie potenzielle zukünftige Entwicklungen im Bereich der Cybersicherheit diskutiert. Die Ergebnisse dieser Arbeit bieten Unternehmen eine fundierte Basis, um den gestiegenen Anforderungen der NIS-2-Richtlinie gerecht zu werden und ihre Informationssicherheitsstandards gemäß den internationalen Normen zu verbessern.}

\normalsize