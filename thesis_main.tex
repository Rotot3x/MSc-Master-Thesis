%-----------------------------------
% Define document and include general packages
%-----------------------------------
% Tabellen- und Abbildungsverzeichnis stehen normalerweise nicht im
% Inhaltsverzeichnis. Gleiches gilt für das Abkürzungsverzeichnis (siehe unten).
% Manche Dozenten bemängeln das. Die Optionen 'listof=totoc,bibliography=totoc'
% geben das Tabellen- und Abbildungsverzeichnis im Inhaltsverzeichnis (toc=Table
% of Content) aus.
% Da es aber verschiedene Regelungen je nach Dozent geben kann, werden hier
% beide Varianten dargestellt.
\documentclass[12pt,oneside,titlepage,listof=totoc,bibliography=totoc]{scrartcl}
%\documentclass[12pt,oneside,titlepage]{scrartcl}

%-----------------------------------
% Dokumentensprache
%-----------------------------------
%\def\FOMEN{}% Auskommentieren um die Dokumentensprache auf englisch zu ändern
\newif\ifde
\newif\ifen

%-----------------------------------
% Meta informationen
%-----------------------------------
%-----------------------------------
% Meta Informationen zur Arbeit
%-----------------------------------

% Autor
\newcommand{\myAutor}{Ferris Florian Menzel}


% Adresse
\newcommand{\myAdresse}{München}

% Titel der Arbeit
\newcommand{\myTitel}{Design Science Research-geleitete Entwicklung und Evaluation eines blockchain-basierten Self-Sovereign-Identity-Prototypen mit Post-Quantum-Kryptografie für kritische Infrastrukturen}

% Betreuer
\newcommand{\myBetreuer}{Prof. Dr. Martin Rupp}

% Lehrveranstaltung
\newcommand{\myLehrveranstaltung}{MODUL}

% Matrikelnummer
\newcommand{\myMatrikelNr}{718680}

% Ort
\newcommand{\myOrt}{MyORT in Meta leer}

% Datum der Abgabe
\newcommand{\myAbgabeDatum}{06.01.2026}

% Semesterzahl
\newcommand{\mySemesterZahl}{5}

% Name der Hochschule
\newcommand{\myHochschulName}{FOM Hochschule für Oekonomie \& Management}

% Standort der Hochschule
\newcommand{\myHochschulStandort}{DLS}

% Studiengang
\newcommand{\myStudiengang}{Cyber Security Management}

% Art der Arbeit
\newcommand{\myThesisArt}{Master-Thesis}

% Zu erlangender akademische Grad
\newcommand{\myAkademischerGrad}{Master of Science (M.Sc.)}

% Firma
\newcommand{\myFirma}{KOMM IN DIE GRUPPE GmbH}


\ifdefined\FOMEN
%Englisch
\entrue
\usepackage[english]{babel}
\else
%Deutsch
\detrue
\usepackage[ngerman]{babel}
\fi

% Definition eigene Farbe und Einrücken von Text für das Mapping
\newcommand{\gapdetail}[1]{\hangindent=1em \hangafter=0 \textcolor{darkblue_gap}{#1}}


\newcommand{\langde}[1]{%
   \ifde\selectlanguage{ngerman}#1\fi}
\newcommand{\langen}[1]{%
   \ifen\selectlanguage{english}#1\fi}
\newcommand{\fixme}[1]{\color{red} FIXME: #1 \color{black}}
\usepackage{booktabs}
\usepackage{longtable}
\usepackage[utf8]{luainputenc}
\langde{\usepackage[babel,german=quotes]{csquotes}}
\langen{\usepackage[babel,english=british]{csquotes}}
\usepackage[T1]{fontenc}
\usepackage{fancyhdr}
\usepackage{fancybox}
\usepackage[a4paper, left=4cm, right=2cm, top=4cm, bottom=2cm]{geometry}
\usepackage{graphicx}
\usepackage{colortbl}
\usepackage[capposition=top]{floatrow}
\usepackage{array}
\usepackage{float}      %Positionierung von Abb. und Tabellen mit [H] erzwingen
\usepackage{footnote}
% Darstellung der Beschriftung von Tabellen und Abbildungen (Leitfaden S. 44)
% singlelinecheck=false: macht die Caption linksbündig (statt zentriert)
% labelfont auf fett: (Tabelle x.y:, Abbildung: x.y)
% font auf fett: eigentliche Bezeichnung der Abbildung oder Tabelle
% Fettschrift laut Leitfaden 2018 S. 45
%\usepackage[singlelinecheck=false, labelfont=bf, font=bf]{caption} % für IEEE
\usepackage{caption}
\captionsetup{ 					% für APA
    format=plain,
	labelsep=newline,
    labelfont=bf,
    textfont=it,
    singlelinecheck=false,
    justification=raggedright,
    position=above
}
\usepackage{enumitem}
\usepackage{dirtree}
\usepackage{amssymb}
\usepackage{mathptmx}
%\usepackage{minted} %Kann für schöneres Syntax Highlighting genutzt werden. ACHTUNG: Python muss installiert sein.
\usepackage[scaled=0.9]{helvet} % Behebt, zusammen mit Package courier, pixelige Überschriften. Ist, zusammen mit mathptx, dem times-Package vorzuziehen. Details: https://latex-kurs.de/fragen/schriftarten/Times_New_Roman.html
\usepackage{courier}
\usepackage{amsmath}
\usepackage[table]{xcolor}
\usepackage{marvosym}			% Verwendung von Symbolen, z.B. perfektes Eurozeichen
\definecolor{darkblue_gap}{HTML}{026873}

\definecolor{mainblue}{RGB}{42,87,141}
\definecolor{lightblue}{RGB}{204,224,245}
\definecolor{milestone}{RGB}{235,87,87}

\renewcommand\familydefault{\sfdefault}
\usepackage{ragged2e}

% Mehrere Fussnoten nacheinander mit Komma separiert
\usepackage[hang,multiple]{footmisc}
\setlength{\footnotemargin}{1em}

% todo Aufgaben als Kommentare verfassen für verschiedene Editoren
\usepackage{todonotes}

% Verhindert, dass nur eine Zeile auf der nächsten Seite steht
\setlength{\marginparwidth}{2cm}
\usepackage[all]{nowidow}


\usepackage[ngerman]{babel}
\usepackage{pgfgantt}
\usepackage{translator} % Für deutsche Monatsnamen

%-----------------------------------
% Schusterjungen und Hurenkinder
%-----------------------------------
\widowpenalty=10000
\clubpenalty=10000

%-----------------------------------
% Farbdefinitionen
%-----------------------------------
\definecolor{darkblack}{rgb}{0,0,0}
\definecolor{dunkelgrau}{rgb}{0.8,0.8,0.8}
\definecolor{hellgrau}{rgb}{0.0,0.7,0.99}
\definecolor{mauve}{rgb}{0.58,0,0.82}
\definecolor{dkgreen}{rgb}{0,0.6,0}

%-----------------------------------
% Pakete für Tabellen
%-----------------------------------
\usepackage{epstopdf}
\usepackage{nicefrac} % Brüche
\usepackage{multirow}
\usepackage{rotating} % vertikal schreiben
\usepackage{mdwlist}
\usepackage{tabularx}% für Breitenangabe

%-----------------------------------
% sauber formatierter Quelltext
%-----------------------------------
\usepackage{listings}
% JavaScript als Sprache definieren:
\lstdefinelanguage{JavaScript}{
	keywords={break, super, case, extends, switch, catch, finally, for, const, function, try, continue, if, typeof, debugger, var, default, in, void, delete, instanceof, while, do, new, with, else, return, yield, enum, let, await},
	keywordstyle=\color{blue}\bfseries,
	ndkeywords={class, export, boolean, throw, implements, import, this, interface, package, private, protected, public, static},
	ndkeywordstyle=\color{darkgray}\bfseries,
	identifierstyle=\color{black},
	sensitive=false,
	comment=[l]{//},
	morecomment=[s]{/*}{*/},
	commentstyle=\color{purple}\ttfamily,
	stringstyle=\color{red}\ttfamily,
	morestring=[b]',
	morestring=[b]"
}

\lstset{
	%language=JavaScript,
	numbers=left,
	numberstyle=\tiny,
	numbersep=5pt,
	breaklines=true,
	showstringspaces=false,
	frame=l ,
	xleftmargin=5pt,
	xrightmargin=5pt,
	basicstyle=\ttfamily\scriptsize,
	stepnumber=1,
	keywordstyle=\color{blue},          % keyword style
  	commentstyle=\color{dkgreen},       % comment style
  	stringstyle=\color{mauve}         % string literal style
}

%-----------------------------------
%Literaturverzeichnis Einstellungen
%-----------------------------------

% Biblatex

\usepackage{url}
\def\UrlBreaks{\do\/\do-}
\urlstyle{same}

%%%% Neuer Leitfaden (2018)
\usepackage[
backend=biber,
style=apa,
maxcitenames=3,	% mindestens 3 Namen ausgeben bevor et. al. kommt
maxbibnames=999,
date=iso,
seconds=true, %werden nicht verwendet, so werden aber Warnungen unterdrückt.
urldate=iso,
dashed=false,
autocite=inline,
useprefix=true, % 'von' im Namen beachten (beim Anzeigen)
mincrossrefs = 1
]{biblatex}%iso dateformat für YYYY-MM-DD

%weitere Anpassungen für BibLaTex
%\input{skripte/modsBiblatex2018}
%% et al. anstatt u. a. bei mehr als drei Autoren.
\DefineBibliographyStrings{ngerman}{ 
	andothers = {{et\,al\adddot}},             
}
\DefineBibliographyStrings{english}{ 
	andothers = {{et\,al\adddot}},             
}

%% Vorher 
%%%% Neuer Leitfaden (2018)
%\usepackage[
%backend=biber,
%style=ext-authoryear-ibid, % Auskommentieren und nächste Zeile einkommentieren, falls "Ebd." (ebenda) nicht für sich-wiederholende Fussnoten genutzt werden soll.
%%style=ext-authoryear,
%maxcitenames=3,	% mindestens 3 Namen ausgeben bevor et. al. kommt
%maxbibnames=999,
%mergedate=false,
%date=iso,
%seconds=true, %werden nicht verwendet, so werden aber Warnungen unterdrückt.
%urldate=iso,
%innamebeforetitle,
%dashed=false,
%autocite=footnote,
%doi=false,
%useprefix=true, % 'von' im Namen beachten (beim Anzeigen)
%mincrossrefs = 1
%]{biblatex}%iso dateformat für YYYY-MM-DD
%
%%weitere Anpassungen für BibLaTex
%\input{skripte/modsBiblatex2018}
%
%%


%%%%% Alter Leitfaden. Ggf. Einkommentieren und Bereich hierüber auskommentieren
%\usepackage[
%backend=biber,
%style=numeric,
%citestyle=authoryear,
%url=false,
%isbn=false,
%notetype=footonly,
%hyperref=false,
%sortlocale=de]{biblatex}

%weitere Anpassungen für BibLaTex
%\input{skripte/modsBiblatex}

%%%% Ende Alter Leitfaden

%Bib-Datei einbinden
\addbibresource{literatur/literatur.bib}

% Zeilenabstand im Literaturverzeichnis ist Einzeilig
% siehe Leitfaden S. 14
\AtBeginBibliography{\singlespacing}

%-----------------------------------
% Silbentrennung
%-----------------------------------
\usepackage{hyphsubst}
\HyphSubstIfExists{ngerman-x-latest}{%
\HyphSubstLet{ngerman}{ngerman-x-latest}}{}

%-----------------------------------
% Pfad fuer Abbildungen
%-----------------------------------
\graphicspath{{./}{./abbildungen/}}

%-----------------------------------
% Weitere Ebene einfügen
%-----------------------------------
\input{skripte/weitereEbene}

%-----------------------------------
% Paket für die Nutzung von Anhängen
%-----------------------------------
\usepackage{appendix}


%-----------------------------------
% Eigenes Inhaltsverzeichnis für den Anhang (BABEL-SAFE)
%-----------------------------------
\makeatletter

% Inhaltsverzeichnis des Anhangs drucken
\newcommand{\listofappendices}{%
  \section*{Inhaltsverzeichnis des Anhangs}%
  \addcontentsline{toc}{subsection}{Inhaltsverzeichnis des Anhangs}%
  \@starttoc{apx}%
  \vspace{1em}%
}

% TOC-Formatierung: SECTION FETT
\newcommand{\l@appsection}[2]{%
  \addpenalty\@secpenalty
  \addvspace{0.01em \@plus\p@}%
  \setlength\@tempdima{5.5em}%
  \begingroup
    \parindent \z@ \rightskip \@pnumwidth
    \parfillskip -\@pnumwidth
    \leavevmode \bfseries
    \advance\leftskip\@tempdima
    \hskip -\leftskip
    #1\nobreak\dotfill\nobreak\hb@xt@\@pnumwidth{\hss #2}\par
  \endgroup
}

% TOC-Formatierung: SUBSECTION normal (OHNE tocdepth-Check)
\newcommand{\l@appsubsection}[2]{%
  \addvspace{0.01em \@plus\p@}%
  \setlength\@tempdima{6.5em}%
  \begingroup
    \parindent \z@ \rightskip \@pnumwidth
    \parfillskip -\@pnumwidth
    \leavevmode
    \advance\leftskip 1.0em
    \advance\leftskip\@tempdima
    \hskip -\leftskip
    #1\nobreak\dotfill\nobreak\hb@xt@\@pnumwidth{\hss #2}\par
  \endgroup
}

% TOC-Formatierung: SUBSUBSECTION normal (OHNE tocdepth-Check)
\newcommand{\l@appsubsubsection}[2]{%
  \addvspace{0.01em \@plus\p@}%
  \setlength\@tempdima{7.5em}%
  \begingroup
    \parindent \z@ \rightskip \@pnumwidth
    \parfillskip -\@pnumwidth
    \leavevmode
    \advance\leftskip 2.0em
    \advance\leftskip\@tempdima
    \hskip -\leftskip
    #1\nobreak\dotfill\nobreak\hb@xt@\@pnumwidth{\hss #2}\par
  \endgroup
}

% Original-Befehle speichern
\let\origsection\section
\let\origsubsection\subsection
\let\origsubsubsection\subsubsection

% Neue Appendix-Befehle
\newcommand{\appsection}[1]{%
  \origsection{#1}%
  \addcontentsline{apx}{appsection}{\protect\numberline{\thesection}#1}%
}

\newcommand{\appsubsection}[1]{%
  \origsubsection{#1}%
  \addcontentsline{apx}{appsubsection}{\protect\numberline{\thesubsection}#1}%
}

\newcommand{\appsubsubsection}[1]{%
  \origsubsubsection{#1}%
  \addcontentsline{apx}{appsubsubsection}{\protect\numberline{\thesubsubsection}#1}%
}

\makeatother











%-----------------------------------
% Zeilenabstand 1,5-zeilig
%-----------------------------------
\usepackage{setspace}
\onehalfspacing

%-----------------------------------
% Absätze durch eine neue Zeile
%-----------------------------------
\setlength{\parindent}{0mm}
\setlength{\parskip}{0.8em plus 0.5em minus 0.3em}

\sloppy					%Abstände variieren
\pagestyle{headings}

%----------------------------------
% Präfix in das Abbildungs- und Tabellenverzeichnis aufnehmen, statt nur der Nummerierung (siehe Issue #206).
%----------------------------------
\KOMAoption{listof}{entryprefix} % Siehe KOMA-Script Doku v3.28 S.153
\BeforeStartingTOC[lof]{\renewcommand*\autodot{:}} % Für den Doppelpunkt hinter Präfix im Abbildungsverzeichnis
\BeforeStartingTOC[lot]{\renewcommand*\autodot{:}} % Für den Doppelpunkt hinter Präfix im Tabellenverzeichnis

%-----------------------------------
% Abkürzungsverzeichnis
%-----------------------------------
\usepackage[printonlyused]{acronym}

%-----------------------------------
% Symbolverzeichnis
%-----------------------------------
% Quelle: https://www.namsu.de/Extra/pakete/Listofsymbols.pdf
\usepackage[final]{listofsymbols}

%-----------------------------------
% Glossar
%-----------------------------------
\usepackage{glossaries}
\glstoctrue %Auskommentieren, damit das Glossar nicht im Inhaltsverzeichnis angezeigt wird.
\newglossaryentry{CRQC}{
    name={Cryptographically Relevant Quantum Computer (CRQC)},
    description={beschreibt das Stadium, in dem Quantencomputer aktuelle Kryptosysteme brechen können \parencite[S. 338]{geremew_PreparingCriticalInfrastructurePostQuantumCryptographyStrategiesTransitioningAheadCryptanalyticallyRelevantQuantumComputing_2024}}
}
\newglossaryentry{Kryptoagilität}{
    name={Kryptoagilität},
    description={gewährleistet die langfristige Sicherheit von Kryptografie. Kryptoagile Systeme sind offen für Innovationen in der Kryptografie und verbessern die Widerstandsfähigkeit bei Angriffen auf einzelne Verfahren. Kryptoagilität ist eine Voraussetzung für Innovationen in der Kryptografie und für langfristige Sicherheit. Sie fördert die technologische Souveränität und ihre kluge Umsetzung bringt Wettbewerbsvorteile \parencite[S. 1]{kreutzer_Kryptoagilitaet_2024}}
}
\newglossaryentry{EBSCO}{
    name={EBSCO},
    description={ist eine umfassende, multidisziplinäre Datenbank, die eine breite Palette von akademischen Zeitschriften, Magazinen und anderen Publikationen abdeckt. Sie bietet Zugang zu einer Vielzahl von Fachgebieten, einschließlich Informatik, Cybersicherheit und Politikwissenschaften. Dies ist besonders wertvoll für das Thema, da es die Schnittstelle zwischen Technologie und nationaler Sicherheit betrifft \parencite{_EBSCOInformationServices_a}}
}
\newglossaryentry{SzA}{
    name={SzA},
    description={Systeme zur Angriffserkennung sind im Sinne des \textcite[§ 2 Absatz 9b]{bundesministeriumderjustiz_GesetzUeberBundesamtfuerSicherheitInformationstechnikBSIGesetzBSIG_2009} durch technische Werkzeuge und organisatorische Einbindung unterstützte Prozesse zur Erkennung von Angriffen auf informationstechnische Systeme. Die Angriffserkennung erfolgt dabei durch Abgleich der in einem informationstechnischen System verarbeiteten Daten mit Informationen und technischen Mustern, die auf Angriffe hindeuten}
}
\newglossaryentry{Privacy by Design}{
    name={Privacy by Design},
    description={\fixme{ergänzen}}
}
\newglossaryentry{Privacy by Default}{
    name={Privacy by Default},
    description={\fixme{ergänzen}}
}
\newglossaryentry{Sidecar Proxy}{
    name={Sidecar Proxy},
    description={
    ist ein zusätzlicher Microservice, der neben Business-Microservices in einem Pod bereitgestellt wird, um Proxy-Funktionen bereitzustellen, welche sämtliche ein- und ausgehende Kommunikation eines Pods abfangen und kryptografisch schützen, ohne die zugrundeliegende Anwendungslogik zu modifizieren \parencite[S. 1--2]{berlato_WorkinProgressSidecarProxyUsablePerformanceAdaptableEndtoEndProtectionCommunicationsCloudNativeApplications_2024}
    }
}
\newglossaryentry{Hybride Schemata}{
    name={Hybride Schemata},
    description={
    kombinieren zwei oder mehr Algorithmen derselben Kategorie, sodass das Gesamtsystem seine Sicherheit behält, solange mindestens eine der verwendeten Komponenten als sicher gilt \parencite[S. 3]{bindel_HybridKeyEncapsulationMechanismsAuthenticatedKeyExchange_2019}
    }
}
\makenoidxglossaries

%-----------------------------------
% PDF Meta Daten setzen
%-----------------------------------
\usepackage[hyperfootnotes=false]{hyperref} %hyperfootnotes=false deaktiviert die Verlinkung der Fußnote. Ansonsten inkompaibel zum Paket "footmisc"
% Behebt die falsche Darstellung der Lesezeichen in PDF-Dateien, welche eine Übersetzung besitzen
% siehe Issue 149
\makeatletter
\pdfstringdefDisableCommands{\let\selectlanguage\@gobble}
\makeatother

\hypersetup{
    pdfinfo={
        Title={\myTitel},
        Subject={\myStudiengang},
        Author={\myAutor},
        Build=1.1
    }
}

%-----------------------------------
% PlantUML
%-----------------------------------
%\usepackage{plantuml}

%-----------------------------------
% Umlaute in Code korrekt darstellen
% siehe auch: https://en.wikibooks.org/wiki/LaTeX/Source_Code_Listings
%-----------------------------------
\lstset{literate=
	{á}{{\'a}}1 {é}{{\'e}}1 {í}{{\'i}}1 {ó}{{\'o}}1 {ú}{{\'u}}1
	{Á}{{\'A}}1 {É}{{\'E}}1 {Í}{{\'I}}1 {Ó}{{\'O}}1 {Ú}{{\'U}}1
	{à}{{\`a}}1 {è}{{\`e}}1 {ì}{{\`i}}1 {ò}{{\`o}}1 {ù}{{\`u}}1
	{À}{{\`A}}1 {È}{{\'E}}1 {Ì}{{\`I}}1 {Ò}{{\`O}}1 {Ù}{{\`U}}1
	{ä}{{\"a}}1 {ë}{{\"e}}1 {ï}{{\"i}}1 {ö}{{\"o}}1 {ü}{{\"u}}1
	{Ä}{{\"A}}1 {Ë}{{\"E}}1 {Ï}{{\"I}}1 {Ö}{{\"O}}1 {Ü}{{\"U}}1
	{â}{{\^a}}1 {ê}{{\^e}}1 {î}{{\^i}}1 {ô}{{\^o}}1 {û}{{\^u}}1
	{Â}{{\^A}}1 {Ê}{{\^E}}1 {Î}{{\^I}}1 {Ô}{{\^O}}1 {Û}{{\^U}}1
	{œ}{{\oe}}1 {Œ}{{\OE}}1 {æ}{{\ae}}1 {Æ}{{\AE}}1 {ß}{{\ss}}1
	{ű}{{\H{u}}}1 {Ű}{{\H{U}}}1 {ő}{{\H{o}}}1 {Ő}{{\H{O}}}1
	{ç}{{\c c}}1 {Ç}{{\c C}}1 {ø}{{\o}}1 {å}{{\r a}}1 {Å}{{\r A}}1
	{€}{{\EUR}}1 {£}{{\pounds}}1 {„}{{\glqq{}}}1
}

%-----------------------------------
% Kopfbereich / Header definieren
%-----------------------------------
\pagestyle{fancy}
\fancyhf{}
% Seitenzahl oben, mittig, mit Strichen beidseits
% \fancyhead[C]{-\ \thepage\ -}

% Seitenzahl oben, mittig, entsprechend Leitfaden ohne Striche beidseits
\fancyhead[C]{\thepage}
%\fancyhead[L]{\leftmark}							% kein Footer vorhanden
% Waagerechte Linie unterhalb des Kopfbereiches anzeigen. Laut Leitfaden ist
% diese Linie nicht erforderlich. Ihre Breite kann daher auf 0pt gesetzt werden.
\renewcommand{\headrulewidth}{0.4pt}
%\renewcommand{\headrulewidth}{0pt}

%-----------------------------------
% Damit die hochgestellten Zahlen auch auf die Fußnote verlinkt sind (siehe Issue 169)
%-----------------------------------
\hypersetup{colorlinks=true, breaklinks=true, linkcolor=darkblack, citecolor=darkblack, menucolor=darkblack, urlcolor=darkblack, linktoc=all, bookmarksnumbered=false, pdfpagemode=UseOutlines, pdftoolbar=true}
\urlstyle{same}%gleiche Schriftart für den Link wie für den Text

%-----------------------------------
% Start the document here:
%-----------------------------------
\begin{document}

\pagenumbering{Roman}								% Seitennumerierung auf römisch umstellen
%\newcolumntype{C}{>{\centering\arraybackslash}X}	% Neuer Tabellen-Spalten-Typ:
%Zentriert und umbrechbar

\newcolumntype{L}[1]{>{\raggedright\arraybackslash}p{#1}} % Linksbündig
\newcolumntype{C}[1]{>{\centering\arraybackslash}p{#1}}   % Zentriert

\newcolumntype{M}{>{\centering\arraybackslash}X}	% Neuer Tabellen-Spalten-Typ:

%-----------------------------------
% Textcommands
%-----------------------------------
\input{skripte/textcommands}

%-----------------------------------
% Titlepage
%-----------------------------------
\begin{titlepage}
	\newgeometry{left=2cm, right=2cm, top=2cm, bottom=2cm}
	\begin{center}
    \includegraphics[width=2.3cm]{abbildungen/fomLogo} \\
    \vspace{.5cm}
		\begin{Large}\textbf{\myHochschulName}\end{Large}\\
    \vspace{.5cm}
		\begin{Large}\langde{Hochschulzentrum}\langen{university location} \myHochschulStandort\end{Large}\\
		\vspace{1.5cm}
    \begin{Large}\textbf{\myThesisArt}\end{Large}\\
    \vspace{.5cm}
		% \langde{Berufsbegleitender Studiengang}
		% \langen{part-time degree program}\\
		% \mySemesterZahl. Semester\\
    \langde{im Studiengang}\langen{in the study course}\\
	\vspace{.5cm}
	\myStudiengang
		\vspace{1.5cm}

		\langde{zur Erlangung des Grades eines}\langen{to obtain the degree of}\\
    \vspace{.5cm}
		\begin{Large}{\myAkademischerGrad}\end{Large}\\
		% Oder für Hausarbeiten:
		%\textbf{im Rahmen der Lehrveranstaltung}\\
		%\textbf{\myLehrveranstaltung}\\
		\vspace{1.5cm}
		\langde{über das Thema}
		\langen{on the subject}\\
    \vspace{.5cm}
		\large{\textbf{\myTitel}}\\
		\vspace{1.5cm}
    \langde{von}\langen{by}\\
    \vspace{.5cm}
    \begin{Large}{\myAutor}\end{Large}\\
	\end{center}
	\normalsize
	\vfill
    \begin{tabular}{ l l }
        \langde{Betreuer:} % für Hausarbeiten
        %\langde{Erstgutachter} % für Bachelor- / Master-Thesis
        \langen{Advisor:} & \myBetreuer\\
        \langde{Matrikelnummer:}
        \langen{Matriculation Number:} & \myMatrikelNr\\
        \langde{Abgabedatum:}
        \langen{Submission:} & \myAbgabeDatum
    \\
    \end{tabular}
\end{titlepage}


%-----------------------------------
% Vorwort (optional; bei Verwendung beide Zeilen entkommentieren und unter Inhaltsverzeichnis setcounter entsprechend anpassen)
%-----------------------------------
\section*{Vorwort}
\fixme{Die vorliegende Masterarbeit baut auf dem Exposé auf, das im Rahmen des Moduls 'Projektseminar' (Betreuer: Prof. Dr. Martin Rupp) im August 2025 erstellt und bewertet wurde. Die dort entwickelte Problemstellung, Forschungsfrage, methodische Konzeption und initiale systematische Literaturrecherche bilden die Grundlage dieser Arbeit und wurden für die Masterarbeit vertieft, erweitert und aktualisiert.

Abzüglich der Abbildungen, Tabellen und Listings entspricht der Umfang des Hauptteils 75 Seiten.

Der Anhang ist aufgrund der hohen Transparenz- und Nachvollziehbarkeitsanforderungen wissenschaftlicher Gütekriterien sehr umfangreich gestaltet, da der Source Code des entwickelten Prototypen als auch die summative Evaluation großteilig dokumentiert wurden, um Reproduzierbarkeit und Prüfbarkeit der Ergebnisse zu gewährleisten.

in der Scientific Community anerkannten APA-Zitierstil verwendet.

Die vollständige Dokumentation der Literaturrecherche beider Iterationen nach PRISMA 2020 ist als Anhang dieser Arbeit integriert, um Transparenz, Reproduzierbarkeit und wissenschaftliche Nachvollziehbarkeit zu gewährleisten, damit die vorliegende Masterarbeit ein in sich geschlossenes und eigenständiges wissenschaftliches Dokument darstellt.
- Ich verwende APA Zitierstil.
- Leitfaden zur formalen Gestaltung von Seminar- \& Abschlussarbeiten (Stand 01/24) erstellt von Prof. Dr. Dr. habil. Clemens Jäger, Prof. Dr. Thomas Kümpel, Prof. Dr. Anja Seng.
- Alle Internetquellen als ZIP mit abgeben! \\
- Code Repository mit abgeben! \\
- Leitfaden heraussuchen und Referenzieren an den ich mich gehalten habe. \\}
\\[1cm]
München, \the\day.\the\month.\the\year

{\myAutor}
\newpage

%-----------------------------------
% Genderhinweis
%-----------------------------------
\newpage
\section*{Inklusionshinweis}
Zur besseren Lesbarkeit wird in dieser Arbeit das generische Maskulinum verwendet. Die in dieser Arbeit verwendeten Personenbezeichnungen beziehen sich - sofern nicht anders kenntlich gemacht - auf alle Geschlechter.
\newpage

%-----------------------------------
% Inhaltsverzeichnis
%-----------------------------------
% Um das Tabellen- und Abbbildungsverzeichnis zu de/aktivieren ganz oben in Documentclass schauen
% \setcounter{page}{2}
\addtocontents{toc}{\protect\enlargethispage{-20mm}}% Die Zeile sorgt dafür, dass das Inhaltsverzeichnisseite auf die zweite Seite gestreckt wird und somit schick aussieht. Das sollte eigentlich automatisch funktionieren. Wer rausfindet wie, kann das gern ändern.
\setcounter{tocdepth}{4}
\tableofcontents
\newpage

%-----------------------------------
% Abbildungsverzeichnis
%-----------------------------------
\listoffigures
\newpage

%-----------------------------------
% Listingverzeichnis 
%-----------------------------------
\renewcommand{\lstlistlistingname}{Listingverzeichnis}
\newcounter{manualListingCounter}

% Caption-Format für Listings: Nummer und Titel auf separaten Zeilen
\makeatletter
\renewcommand\lst@makecaption[2]{%
  \vskip\abovecaptionskip
  {\textbf{Listing \themanualListingCounter}\par\textit{#2}\par}%
  \vskip\belowcaptionskip
}
\makeatother


\renewcommand{\thelstlisting}{Listing \arabic{lstlisting}:}
\lstlistoflistings
\newpage

%-----------------------------------
% Tabellenverzeichnis
%-----------------------------------
\listoftables
\newpage
%-----------------------------------
% Abkürzungsverzeichnis
%-----------------------------------
% Falls das Abkürzungsverzeichnis nicht im Inhaltsverzeichnis angezeigt werden soll
% dann folgende Zeile auskommentieren.
\addcontentsline{toc}{section}{\abbreHeadingName}

\section*{\langde{Abkürzungsverzeichnis}\langen{List of Abbreviations}}

\begin{acronym}[WYSIWYG]\itemsep0pt %der Parameter in Klammern sollte die längste Abkürzung sein. Damit wird der Abstand zwischen Abkürzung und Übersetzung festgelegt
  \acro{AM}{Access Management}
  \acro{BSI}{Bundesamt für Sicherheit in der Informationstechnik}
  \acro{CA}{Certification Authority}
  \acro{CSPs}{Credential Service Providers}
  \acro{DID}{Decentralized Identifiers}
  \acro{DLT}{Decentralized Ledger Technology}
  \acro{DSR}{Design Science Research}
  \acro{DPKI}{Decentralized Public Key Infrastructure}
  \acro{ECC}{Elliptic Curve Cryptography}
  \acro{ECDSA}{Elliptic Curve Digital Signature Algorithm}
  \acro{FEDS}{Framework for Evaluation in Design Science}
  \acro{IAM}{Identity Access Management}
  \acro{IdM}{Identity Management}
  \acro{IoT}{Internet of Things}
  \acro{KRITIS}{Kritische Infrastrukturen}
  \acro{NIST}{National Institute of Standards and Technology}
  \acro{PKI}{Public Key Infrastructure}
  \acro{PQC}{Post-Quantum Cryptography}
  \acro{PRISMA}{Preferred Reporting Items for Systematic Reviews and Meta-Analyses}
  \acro{RSA}{Rivest–Shamir–Adleman}
  \acro{SSI}{Self‐Sovereign Identity}
  \acro{VC}{Verifiable Credentials}
  \acro{VDR}{Verifiable Data Registry}
  \acro{VP}{Verifiable Presentation}
  \acro{ZKP}{Zero Knowledge Proof}
\end{acronym}
\newpage

%-----------------------------------
% Symbolverzeichnis
%-----------------------------------
% In Overleaf führt der Einsatz des Symbolverzeichnisses zu einem Fehler, der aber ignoriert werdne kann
% Falls das Symbolverzeichnis nicht im Inhaltsverzeichnis angezeigt werden soll
% dann folgende Zeile auskommentieren.
%\phantomsection\addcontentsline{toc}{section}{\symheadingname}
%\input{skripte/symbolDef}
%\listofsymbols
%\newpage

%-----------------------------------
% Glossar
%-----------------------------------
\printnoidxglossaries
\newpage

%-----------------------------------
% Abstract
%-----------------------------------
\section*{Abstract} \label{sec:Abstract}
\addcontentsline{toc}{section}{Abstract}
\small

\fixme{GEWÜNSCHT ODER NICHT????.}

\normalsize
\newpage

%-----------------------------------
% Sperrvermerk
%-----------------------------------
%\input{kapitel/anhang/sperrvermerk}

%-----------------------------------
% Seitennummerierung auf arabisch und ab 1 beginnend umstellen
%-----------------------------------
\pagenumbering{arabic}
\setcounter{page}{1}

%-----------------------------------
% Kapitel / Inhalte
%-----------------------------------
% Die Kapitel werden über folgende Datei eingebunden
%-----------------------------------
% Kapitel / Inhalte
%-----------------------------------

% 01_Einleitung                                     // 06-08 Seiten
\fixme{VORWORT}

\fixme{Die vorliegende Masterarbeit baut auf dem Exposé auf, das im Rahmen des Moduls 'Projektseminar' (Betreuer: Prof. Dr. Martin Rupp) im August 2025 erstellt und bewertet wurde. Die dort entwickelte Problemstellung, Forschungsfrage, methodische Konzeption und initiale systematische Literaturrecherche bilden die Grundlage dieser Arbeit und wurden für die Masterarbeit vertieft, erweitert und aktualisiert.

Die vollständige Dokumentation der Literaturrecherche beider Iterationen nach PRISMA 2020 ist als Anhang dieser Arbeit integriert, um Transparenz, Reproduzierbarkeit und wissenschaftliche Nachvollziehbarkeit zu gewährleisten, damit die vorliegende Masterarbeit ein in sich geschlossenes und eigenständiges wissenschaftliches Dokument darstellt.}

\newpage
\fixme{ABSTRACT}

\newpage
\fixme{Genderhinweis?}

\newpage
\fixme{KI-Hilfsmittelverzeichnis?}

\newpage
\section{Einleitung} \label{sec:Einleitung}


\subsection{Problemstellung und Motivation} \label{sec:Problemstellung und Motivation}

- Quantenbedrohung für aktuelle SSI-Systeme
- Herausforderungen in KRITIS
- Notwendigkeit kryptoagiler Lösungen
\subsection{Stand der Forschung und Identifikation der Forschungslücke} \label{sec:Stand der Forschung und Identifikation der Forschungslücke}

\fixme{Mix aus Expose und Masterarbeit auf Basis der beiden Iterationen der Literaturrecherchen.}

\fixme{

Die systematische Literaturrecherche zum gegenwärtigen Stand der Forschung wurde nach PRISMA-2020-Standards durchgeführt und in zwei iterativen Phasen organisiert. Dieser Ansatz ermöglicht es, sowohl die initiale Problemidentifikation als auch die Aktualisierung der Wissensbasis unter Berücksichtigung des dynamischen Charakters des Forschungsfeldes abzubilden. Die erste Iteration am 30. Mai 2025 wurde im Rahmen des Exposés durchgeführt und etablierte die methodische Grundlage für das gesamte Forschungsprojekt, indem sie 61 relevante Quellen identifizierte und systematisch klassifizierte. Die zweite Iteration am 02. November 2025 ergänzte diese Erkenntnisbasis um aktualisierte Literatur aus dem Zeitraum Mai bis November 2025 und prägte damit ein umfassendes Verständnis des gegenwärtigen Forschungsstands mit zusätzlich 34 identifizierten Quellen.

Die Entwicklung von Blockchain-Technologien und dezentralen Identitätskonzepten bildet die Grundlage für Self-Sovereign Identity (SSI). Giannopoulou und Wang definieren SSI als ein Identitätsmanagementsystem ohne zentrale Autoritäten, das auf dezentralen Architekturen basiert und Sicherheit, Privatsphäre sowie individuelle Selbstbestimmung betont. Die systematische Recherche adressierte vier zentrale, interdisziplinäre Domänen, die das Kernforschungsinteresse dieser Arbeit konstituieren. Die erste Domäne, Self-Sovereign Identity, umfasst dezentrale Identitätssysteme, in denen Nutzer die Kontrolle über ihre Identitätsdaten behalten und deren Weitergabe eigenverantwortlich regeln können. Dies wird technisch durch Standards wie Decentralized Identifiers (DIDs), Verifiable Credentials (VCs) und Verifiable Presentations (VPs) realisiert. Die zweite Domäne, Blockchain-Technologie als Identitätsinfrastruktur, adressiert die Nutzung von Distributed-Ledger-Technology (DLT) zur Sicherstellung von Integrität, Transparenz und Manipulationssicherheit im dezentralen Identitätsmanagement, insbesondere durch permissioned und permissionless Ledger-Architekturen. Die dritte Domäne, Post-Quantum Cryptography (PQC), untersucht kryptografische Verfahren, die gegen Angriffe durch leistungsstarke Quantencomputer resistent sind, einschließlich der aktuellen NIST-Standards (ML-DSA, ML-KEM, SLH-DSA) und ihrer kryptoagilen Implementierung. Die vierte Domäne schließlich, kritische Infrastrukturen (KRITIS), konzentriert sich auf die spezifischen Sicherheits-, Compliance- und Resilienzanforderungen von Systemen mit essentieller Gemeinwohlfunktion gemäß BSI- und internationalen Standards.

Die Bewertung der identifizierten Literatur zeigt, dass lediglich fünf Publikationen aus der ersten Iteration und zwei Publikationen aus der zweiten Iteration eine hohe Relevanz besitzen und gleichzeitig innovative Beiträge zu mindestens zwei Kernbereichen liefern, während die Mehrzahl der Quellen entweder thematisch breiter aufgestellt ist oder nur einzelne Teilaspekte abdeckt. Die sachliche Synthese dieser Hochrelevanz-Arbeiten offenbart ein differenziertes Bild der gegenwärtigen Forschungslandschaft.

Aus der ersten Iteration zeigen Alam, Hoffstein und Cambou konkrete Ansätze zur praktischen Erzeugung und Verteilung quantensicherer Schlüsselpaare auf Basis von Crystals-Dilithium in verteilten Netzwerken und adressieren dabei wesentlich die Integration von Multi-Faktor-Authentifizierung und Challenge-Response-Mechanismen, was sowohl für die Sicherheit verteilter Infrastrukturen als auch für zukünftige SSI-basierte Identitätslösungen zentral ist. Szymanski demonstriert die technische Machbarkeit quantensicherer, hardwaregestützter Kommunikationssysteme für das Industrial Internet of Things im KRITIS-Kontext und behandelt dabei Zero-Trust-Architekturen sowie KI-gestützte Sicherheitsmechanismen. Der Beitrag zeigt explizit den Einsatz quantensicherer Verschlüsselungsmechanismen und QKD-Netzwerke zur Gewährleistung von Resilienz und Sicherheitsanforderungen in kritischen Infrastrukturen. Nouma und Yavuz adressieren die praktische Integration quantensicherer, hybrider Hardware-assistierter Signaturen in Digital Twins und IoT-Infrastrukturen, was für ressourcenbeschränkte SSI-Lösungen unmittelbare Relevanz besitzt. Sharif et al. bieten eine umfassende Übersicht der europäischen regulatorischen Entwicklung hin zu dezentralisierten eID-Lösungen im eIDAS-Rahmenwerk und beleuchten die in eIDAS 2.0 antizipierte Entwicklung hin zu dezentralisierten Identitätsarchitekturen, die als Grundlage künftiger SSI-Lösungen dienen. Radanliev liefert eine vergleichende Analyse internationaler Cybersecurity-Standards (NIST, ISO 27001, MiCA) und deren Anwendung auf blockchain-basierte Systeme, mit explizitem Bezug zu regulatorischen Anforderungen für kritische Infrastrukturen sowie systematischer Prüfung neuerer Regularien unter Einbeziehung technologieübergreifender Aspekte wie PQC, Cloud Security und IoT.

Die zweite Iteration identifizierte zwei weitere hochrelevante Quellen. Barrett-Danes und Ahmad setzen sich systematisch, methodisch und interdisziplinär mit den Bedrohungen durch Quantum Computing für klassische Kryptosysteme auseinander und behandeln ausdrücklich Post-Quantum-Kryptografie, hybride Frameworks einschließlich QKD, Umsetzungsherausforderungen und deren Auswirkungen insbesondere für IoT-Umgebungen. Die Arbeit analysiert sowohl den Forschungsstand als auch Implementierungs- und Migrationspfade anhand konkreter Pilotstudien und adressiert Skalierbarkeit und Wirtschaftlichkeit. Die Nutzung eines PRISMA-basierten Review-Frameworks und der Fokus auf praktische Handlungs- und Politikempfehlungen verleihen der Arbeit hohe wissenschaftliche und praxisnahe Relevanz für die Entwicklung, Migration und Absicherung quantensicherer sowie zukunftsfähiger Systeme mit starker Anschlussfähigkeit an KRITIS, PQC und angrenzende sicherheitskritische Domänen. Feng et al. beschreiben eine umfassende und systematische Übersicht zu Blockchain-basierten Identity Management Systems mit explizitem Fokus auf Self-Sovereign Identity, dezentralen Identifikatoren, Interoperabilität und Sicherheitsanalyse über den gesamten Identitätslebenszyklus hinweg. Die Arbeit adressiert technologische und organisatorische Herausforderungen wie Revokation, Übertragbarkeit, Interoperabilität und Quantum-Resilienz für nutzerkontrollierte Identitätsmodelle. Die PRISMA-basierte Methodik sowie die sektorübergreifende Taxonomisierung und die systematische Ableitung aktuell ungelöster Probleme und künftiger Forschungsrichtungen machen die Quelle höchst wertvoll im Umfeld SSI, Blockchain, PQC und KRITIS.

Arbeiten mittlerer Relevanz vertiefen beispielsweise Sicherheit in IoT-Kontexten oder analysieren Blockchain in der digitalen Forensik. Atlam et al. bieten eine umfassende Analyse von Blockchain-Technologien im digitalen Forensik-Kontext mit Schwerpunkt auf Untersuchungsmethodik und Anwendungsfeldern, adressieren insbesondere die Herausforderungen beim Nachweis und der Verfolgung von Aktivitäten auf Blockchain-Systemen und beleuchten regulatorische Aspekte, jedoch ohne explizite Behandlung von SSI, PQC oder den speziellen Anforderungen kritischer Infrastrukturen. Enaya, Fernando und Kashef betrachten zentrale Aspekte von Blockchain-Technologien und Sicherheit mit Schwerpunkt auf IoT-Anwendungen und branchenspezifische Implementierungen, jedoch fehlen explizite Bezüge zu SSI, PQC und KRITIS, was die Anwendbarkeit auf alle vier Domänen einschränkt. Weniger relevante Quellen bieten überwiegend nur ergänzende Einblicke, etwa zu Cloud-Performance oder KI-gestützter Datenanalyse. Kumar et al. fokussieren auf die Weiterentwicklung von Big-Data-Analyse und Management durch künstliche Intelligenz, mit Betonung neuer Rahmenwerke für die Charakterisierung und Handhabung großer, komplexer Datensätze, jedoch ohne explizite Behandlung oder Integration von SSI, Blockchain-Technologien, PQC oder besonderen Anforderungen kritischer Infrastrukturen.

Trotz dieser differenzierten Forschungslandschaft offenbart die systematische Analyse eine signifikante Forschungslücke. Blockchain-basierte SSI-Ansätze werden als Alternative zu zentralisierten Systemen erforscht, doch die aktuelle Forschung weist weiterhin erhebliche Lücken auf. Vor allem mangelt es an der Integration quantenresistenter Algorithmen in SSI-Architekturen. Lösungen sind meist nicht auf PQC ausgelegt, wie Dutto et al. beschreiben. Die Entwicklung und Bewertung PQC-basierter Ansätze ist daher ein offenes Feld. Ebenso bestehen Defizite in der spezifischen Anpassung dieser Lösungen an die regulatorischen und technischen Anforderungen von KRITIS-Sektoren, wie vom Bundesamt für Sicherheit in der Informationstechnik dargestellt. Der Bundesbeauftragte für den Datenschutz und die Informationsfreiheit fordert, dass digitale Identitätslösungen Datenschutz und Compliance gewährleisten müssen. Bezüglich der technischen Anforderungen wird ein rein softwarebasierter Schutz als unzureichend angesehen.

Bislang existiert kein wissenschaftlicher Ansatz, der SSI, PQC, hardwarebasierte Sicherheit und regulatorische Konformität domänenspezifisch für KRITIS integriert und evaluiert. Die identifizierten Hochrelevanz-Arbeiten adressieren die Domänen SSI, Blockchain und PQC vorwiegend isoliert, ohne ihre systemische Zusammenführung methodisch und implementierungstechnisch zu klären. Darüber hinaus besteht ein Defizit an designwissenschaftlich fundierten Artefaktentwicklungen, die unter Anwendung von Design Science Research Methodologie solche integrierten Systeme systematisch konzipieren, prototypisch umsetzen und validieren. Hinsichtlich der spezifischen KRITIS-Anforderungen zeigt sich, dass einzelne Technologien adressiert werden, ohne jedoch ihre Vereinbarkeit mit BSI-Standards und Compliance-Anforderungen kritischer Infrastrukturen systematisch zu analysieren. Besonders kritisch ist die Unterrepräsentation von Mechanismen für kryptografische Agilität, also die Fähigkeit eines Systems, künftige Algorithmenupdates ohne Systemunterbrechung zu absorbieren, im Kontext von blockchain-basierten SSI-Systemen, obwohl diese für die Langzeitresilienz gegen die Quantenbedrohung fundamental sind.

Diese Lücke in der wissenschaftlichen Literatur bildet das Fundament dieser Masterarbeit. Sie wird durch die vier in Kapitel 1.3 formulierten Forschungsfragen strukturiert und motiviert einen Design-Science-Forschungsansatz, der sowohl die theoretische Erkenntnis als auch die praktische Prototypentwicklung in den Blick nimmt. Die vorliegende Arbeit adressiert diese Lücke durch die iterative Konzeption, Implementierung und Evaluation eines blockchain-basierten, quantenresistenten SSI-Prototyps für kritische Infrastrukturen, wodurch eine wissenschaftlich fundierte Grundlage für zukünftige Deployments in echten KRITIS-Umgebungen geschaffen wird.

}



\subsection{Zielsetzung und Forschungsfragen} \label{sec:Zielsetzung und Forschungsfragen}

- Konkretisierung der vier Hauptforschungsfragen \\
- Erwartete Beiträge zur Wissenschaft und Praxis



Angesichts dessen sind die zentralen Forschungsfragen dieser Arbeit in \autoref{tab:forschungsfragen} dargestellt.

\begin{longtable}{L{0.5cm}L{3.5cm}L{10cm}}
    \caption{Forschungsfragen}
    \label{tab:forschungsfragen} \\
    \toprule
    \textbf{Nr.} & \textbf{Kategorie} & \textbf{Forschungsfrage} \\
    \midrule
    \endfirsthead
    \multicolumn{3}{l}{\textit{Tabelle \thetable\ (Fortsetzung)}} \\
    \toprule
    \textbf{Nr.} & \textbf{Kategorie} & \textbf{Forschungsfrage} \\
    \midrule
    \endhead
    \midrule
    \multicolumn{3}{r}{\textit{Fortsetzung auf nächster Seite}} \\
    \endfoot
    \bottomrule
    \multicolumn{3}{p{\linewidth}}{\textit{Anmerkung.} Eigene Darstellung.} \\
    \endlastfoot
    1 & Systemarchitektur \newline \& \newline Compliance &
    Wie kann ein blockchain-basiertes \ac{SSI}-System unter Einsatz von \ac{PQC} gestaltet werden, um die regulatorischen und technischen Anforderungen von \ac{KRITIS} nachhaltig zu erfüllen? \\
    \midrule
    2 & Algorithmenauswahl \newline \& \newline Sicherheitsbewertung &
    Welche \ac{PQC}-Algorithmen eignen sich für die Integration in \ac{SSI}-Systeme hinsichtlich Sicherheit und Interoperabilität, insbesondere im Kontext von \ac{KRITIS}? \\
    \midrule
    3 & Kryptografische Agilität &
    Welche kryptografischen Agilitätsmechanismen sind erforderlich, um zukünftige \ac{PQC}-Algorithmenupdates ohne Systemunterbrechung zu ermöglichen? \\
\end{longtable}
\subsection{Methodisches Vorgehen} \label{sec:Methodisches Vorgehen}

- Design Science Research Ansatz \\
- Abgrenzung zu realen KRITIS-Umgebungen


\textbf{Systematische Literaturrecherche nach PRISMA 2020 und unsystematische Recherche im DSR-Kontext:} Die Arbeit folgt einem hybriden Review-Ansatz, der zeitlich und funktional klar differenziert ist. Die initiale, systematische Literaturrecherche nach ausgewählten Methoden der PRISMA 2020-Richtlinien (Kapitel 3.1) dient ausschließlich der \textbf{initialen Problemidentifikation, Ergrundung des Forschungsstands und Ableitung der Forschungslücke}. Diese systematische Vorgehensweise gewährleistet einen transparenten reproduzierbaren Prozess und verbessert die Berichtqualität \fixme{PRISMA-Quelle}. \fixme{Vertiefung relevanter Themenbereiche basierend auf den Erkenntnissen aus der Artefakt-Entwicklung erfolgt unsystematisch...} Abgrenzung Herleitung des Themas und Relevanz systematisch vs. DSR unsystematische Literaturrecherche bedarfsspezifisch?
\subsection{Aufbau der Arbeit} \label{sec:Aufbau der Arbeit}

- Aufbau und Struktur der Masterarbeit



\subsection{ABGRENZUNG} \label{ABGRENZUNG}


\fixme{Klare Abgrenzung zu realen KRITIS-Umgebungen

Limitationen und Übertragbarkeit der Erkenntnisse}

- Keine prozessualen oder organisatorischen Compliance Anforderungen, nur technisch!
- Compliance Anforderungen eingeschränkt => ausrichtung KRTISI aber technische Umsetzung im Labor.


Limitation der ARbeit durch Umsetzungsherausforderungen

2 Implementierungszyklen

ebd.

Listings verzeichnis ==> Quellcode verzeichnis

Wegen begrenztem Umfang der Masterarbeit
    - Fokus auf SSI (von-network und indy tails server für labor only!) \\
    - 2 DSR Zyklen \\
    - funktionale Anforderungen \\ 
    - Compliance Anforderungen \\



\textbf{Abgrenzung und Limitationen der Evaluation}

Im Rahmen des FEDS-Frameworks fokussiert sich diese Arbeit bewusst auf das Evaluationsziel der \textit{Efficacy} (Wirksamkeit) und der \textit{Fidelity} (Übereinstimmung mit Anforderungen). Ziel ist der Nachweis der technischen Machbarkeit (Technical Feasibility) und der regulatorischen Konformität (Compliance Readiness) der entwickelten Artefakte.

Eine quantitative Evaluierung der \textit{Efficiency} (Performance, Latenzzeiten, Durchsatz) ist explizit \textbf{nicht} Bestandteil dieser Untersuchung. Diese Abgrenzung erfolgt aus folgenden Gründen:

\begin{itemize}
    \item \textbf{Primat der Korrektheit:} Im Kontext von KRITIS und Post-Quantum-Kryptografie ist die funktionale Korrektheit der Implementierung und die Einhaltung spezifizierter Sicherheitsparameter (z.\,B. NIST Security Levels) die notwendige Vorbedingung für jeden operativen Einsatz. Performance-Optimierungen sind erst sinnvoll, wenn die funktionale Integrität zweifelsfrei bewiesen ist.
    \item \textbf{Status der Standardisierung:} Da die verwendeten Algorithmen (ML-KEM, ML-DSA) und deren Integration in SSI-Protokolle (DIDComm v2) sich teilweise noch in der Standardisierungs- oder frühen Adoptionsphase befinden, liegt der wissenschaftliche Mehrwert primär in der Erforschung der Integrationsfähigkeit und Interoperabilität, nicht in der Laufzeitmessung von Referenzimplementierungen, die sich noch signifikant ändern können.
    \item \textbf{Artifizielle Evaluationsumgebung:} Die Evaluation findet bewusst in einer rein artifiziellen Laborumgebung statt, um isolierte funktionale Nachweise ohne Störfaktoren realer Netzwerke zu erbringen. Rückschlüsse auf die Performance in produktiven, verteilten Systemen wären auf Basis dieser Umgebung von begrenzter Validität.
\end{itemize}


% 02_Theoretische Grundlagen                        // 15-18 Seiten
\newpage
\section{Theoretische Grundlagen} \label{sec:Theoretische Grundlagen}
\subsection{Blockchain-Technologie für Identitätsmanagement} \label{sec:Blockchain-Technologie für Identitätsmanagement}

- Permissioned vs. Permissionless Ledger
- Smart Contracts für SSI
- Konsensverfahren und Skalierbarkeit
\subsection{Self-Sovereign Identity} \label{sec:Self-Sovereign Identity}

- Grundprinzipien und Architekturkomponenten
- DID und VC
- Blockchain-basierte SSI-Implementierungen
\subsection{Post-Quantum Kryptografie} \label{sec:Post-Quantum Kryptografie}

- NIST-Standards (ML-KEM, ML-DSA, SLH-DSA)
- Lattice-based und Hash-based Verfahren
- Kryptoagilität und Migrationsstategien
\subsection{Kritische Infrastrukturen und Compliance} \label{sec:Kritische Infrastrukturen und Compliance}

- KRITIS-Anforderungen nach BSI \\
- Privacy by Design und DSGVO-Konformität \\
- Regulatorische Rahmenbedingungen





\fixme{Die unsystematische Literaturrecherche während des DSR Artefaktentwicklung orientiert sich am \enquote{iterative Review-Ansatz} nach \textcite[S. 208--209]{brocke_StandingShouldersGiantsChallengesRecommendationsLiteratureSearchInformationSystemsResearch_2015}, der mit einer initialen Recherche startet und sich iterativ vertieft abhängig von der aktuellen Herausforderung während der Artefaktentwicklung.}

% 03_Systemarchitektur und Design                   // 12-15 Seiten
\newpage
\section{Methodik} \label{sec:Methodik}

\fixme{2,3333 Seiten BILDER/TABELLEN ==> kann erweitert werden}

\subsection{Systematische Literaturrecherche} \label{sec:Systematische Literaturrecherche}

Die systematische Literaturrecherche dieser Masterarbeit folgt ausgewählten Methoden der \ac{PRISMA} 2020 Richtlinien zur strukturierten Identifikation, Selektion, Bewertung und Synthese einschlägiger Studien, wodurch eine belastbare Grundlage für die Analyse der Forschungslücke und die Ableitung der Forschungsfragen geschaffen wird \parencite[S. 1]{page_PRISMA2020Statementupdatedguidelinereportingsystematicreviews_2021}. Dieses Vorgehen bietet einen strukturierten Ansatz zur Durchführung und Dokumentation von Literaturrecherchen, der die Qualität und Vollständigkeit der Berichterstattung verbessert und einen transparenten und reproduzierbaren Prozess gewährleistet \parencite[S. 1]{page_PRISMA2020Statementupdatedguidelinereportingsystematicreviews_2021}.

Die systematische Literaturrecherche wurde in zwei zeitlich getrennten Iterationen durchgeführt, um den dynamischen Charakter des Forschungsfeldes zu adressieren und die Aktualität der Wissensbasis sicherzustellen. Beide Iterationen sind ausführlich dokumentiert in \ref{sec:Anhang_Dokumentation_der_systematischen_Literaturrecherche} und dienen ausschließlich der Problemidentifikation, der Ergründung des Forschungsstands sowie der Ableitung der Forschungslücke und initialen Forschungsfragen.

Der konsolidierte Suchprozess in der Datenbank EBSCO führte über beide Phasen hinweg zur Identifikation von insgesamt 95 potenziellen Quellen. Davon entfielen 61 Publikationen auf die erste Iteration im Mai 2025 (siehe \autoref{fig:PRISMA_Flussdiagramm_iteration1}) und 34 Publikationen auf die zweite Iteration im November 2025 (siehe \autoref{fig:PRISMA_Flussdiagramm_Iteration2}). Die methodische Konsistenz wurde dabei durch die Anwendung identischer Suchstrategien und Selektionskriterien in beiden Zeiträumen sichergestellt.

Nach der Bereinigung von Duplikaten und der Anwendung der Ein- und Ausschlusskriterien verblieb ein fokussierter Bestand an hochrelevanten Studien. Eine detaillierte Übersicht der identifizierten Quellen findet sich in \autoref{tab:quellenuebersicht_Iteration1} für die erste und \autoref{tab:quellenuebersicht_iteration2} für die zweite Phase. Die vollständige Aufschlüsselung der Selektionsschritte sowie die zugehörigen Flussdiagramme sind zudem gesammelt in \ref{sec:Anhang_Dokumentation_der_systematischen_Literaturrecherche} aufgeführt.

\fixme{Anhangskapitel fixen ==> Transparente Dokumentation // Hat noch komische Texte}

\subsection{Design Science Research} \label{Design Science Research}

\ac{DSR} bildet den methodischen Rahmen dieser Masterarbeit. \ac{DSR} stellt neben der verhaltenswissenschaftlichen Forschung ein eigenständiges Forschungsparadigma dar \parencite[S. 75]{hevner_DesignScienceInformationsystemsresearch_2004}. Während verhaltenswissenschaftliche Ansätze Theorien zur Erklärung oder Vorhersage entwickeln, fokussiert DSR auf die Erschaffung innovativer Artefakte zur Erweiterung menschlicher und organisatorischer Fähigkeiten \parencite[S. 75]{hevner_DesignScienceInformationsystemsresearch_2004}. Fundamental ist \ac{DSR} ein lösungsorientiertes Paradigma \parencite[S. 76]{hevner_DesignScienceInformationsystemsresearch_2004}, dessen Prinzip darin besteht, Wissen durch den Bau und die Anwendung eines Artefakts zu gewinnen \parencite[S. 82]{hevner_DesignScienceInformationsystemsresearch_2004}.

Das Information Systems Research Framework (\autoref{fig:hevner2004_framework}) nach \textcite[S. 80]{hevner_DesignScienceInformationsystemsresearch_2004} strukturiert den Forschungsprozess durch drei Hauptkomponenten.

\begin{figure}[H]
    \centering
    \includegraphics[width=\linewidth]{dsr_ISR_Framework.png}
    \caption{Information Systems Research Framework}
    \begin{flushleft}
    \textit{Anmerkung.} Aus \textcite[S. 80]{hevner_DesignScienceInformationsystemsresearch_2004}.
    \end{flushleft}
    \label{fig:hevner2004_framework}
\end{figure}

Die \textbf{Environment}-Komponente definiert den Problemraum \parencite[S. 108]{simon_SciencesArtificial_1996} mit Menschen, Organisationen und Technologien sowie den Geschäftsanforderungen \parencite[S. 7--11]{silver_InformationTechnologyInteractionModelFoundationMBACoreCourse_1995}. Für diese Arbeit bilden kritische Infrastrukturen die Environment mit ihren Anforderungen an Post-Quantum-Sicherheit und selbstbestimmte Identitätsverwaltung. Die \textbf{IS Research}-Komponente umfasst Build/Evaluate-Aktivitäten für Artefakte \parencite[S. 80]{hevner_DesignScienceInformationsystemsresearch_2004}. Hier erfolgt die Instanziierung und Evaluation eines PQC-fähigen SSI-Prototypen, der speziell für den Einsatz in kritischen Infrastrukturen konzipiert ist. Die \textbf{Knowledge Base} liefert Foundations (Theorien, Frameworks, Konstrukte) und Methodologies (Evaluationsmethoden) \parencite[S. 80]{hevner_DesignScienceInformationsystemsresearch_2004}. Sie umfasst für diese Arbeit SSI-Standards, NIST-PQC-Algorithmen und KRITIS-Anforderungen. Rigor wird durch angemessene Anwendung dieser Wissensbasis erreicht \parencite[S. 80]{hevner_DesignScienceInformationsystemsresearch_2004}.

\subsubsection{Zyklen}

\textcite[S. 88]{hevner_ThreeCycleViewDesignScienceResearch_2007} identifiziert drei eng verbundene Aktivitätszyklen, die in jedem DSR-Projekt präsent sein müssen (\autoref{fig:3-cycle-model}).

\begin{figure}[H]
    \centering
    \includegraphics[width=\linewidth]{3-cycle.png}
    \caption{Design Science Research Zyklen}
    \begin{flushleft}
    \textit{Anmerkung.} Aus \textcite[S. 88]{hevner_ThreeCycleViewDesignScienceResearch_2007}.
    \end{flushleft}
    \label{fig:3-cycle-model}
\end{figure}

Der \textbf{Relevance Cycle} initiiert DSR mit Anforderungen aus der Anwendungsdomäne und fordert Field Testing des Outputs \parencite[S. 88-89]{hevner_ThreeCycleViewDesignScienceResearch_2007}. In dieser Arbeit manifestiert sich dies durch die iterative Identifikation von KRITIS-Anforderungen an an Sicherheit, Compliance und Kryptoagilität in zwei Implementierungszyklen. Dazu gehören quantenresistente Kommunikations- und Signaturverfahren, die Einhaltung relevanter Regulierungsvorgaben sowie die Wahrung von Datenschutz und technischer Resilienz. Die entwickelten Artefakte werden kontinuierlich in Laborumgebungen getestet, wobei Erkenntnisse aus der ersten Iteration die Designziele der zweiten Iteration prägen. Der \textbf{Rigor Cycle} verbindet DSR-Aktivitäten mit der Wissensbasis, um Innovation zu gewährleisten \parencite[S. 88-90]{hevner_ThreeCycleViewDesignScienceResearch_2007}. In dieser Arbeit wird dieser Zyklus durch die kontinuierliche Integration wissenschaftlicher Grundlagen etablierter SSI-Frameworks, NIST-standardisierter PQC-Algorithmen und KRITIS-Standards operationalisiert. Der \textbf{Design Cycle} strukturiert die Artefaktentwicklung als iterativen Prozess zwischen Konstruktion und Evaluation \parencite[S. 90--91]{hevner_ThreeCycleViewDesignScienceResearch_2007}. In dieser Arbeit erfolgt dies in zwei aufeinanderfolgenden Iterationen, wobei jede Iteration Feedback für die Designverbesserung liefert und die Erkenntnisse in die nächste Iteration einfließen.

\subsubsection{Artefakte}

\textcite[S. 255]{march_DesignNaturalScienceresearchinformationtechnology_1995} identifizieren vier Artefakttypen: Constructs, Models, Methods und Instantiations. \textbf{Constructs} stellen die Sprache bereit, in der Probleme und Lösungen definiert werden, und beeinflussen die Problemkonzeption \parencite[S. 78, 83]{hevner_DesignScienceInformationsystemsresearch_2004}. \textbf{Models} repräsentieren das Designproblem und den Lösungsraum, unterstützen das Problemverständnis und ermöglichen die Erkundung von Designentscheidungen \parencite[S. 78-79]{hevner_DesignScienceInformationsystemsresearch_2004}. \textbf{Methods} definieren Prozesse zur Problemlösung, von formalen Algorithmen bis zu Best-Practice-Beschreibungen \parencite[S. 79]{hevner_DesignScienceInformationsystemsresearch_2004}. \textbf{Instantiations} demonstrieren Machbarkeit durch Implementierung in einem funktionierenden System und liefern Beweis durch Konstruktion \parencite[S. 79, 84]{hevner_DesignScienceInformationsystemsresearch_2004}.

Diese Arbeit entwickelt eine \textit{Instantiation} in Form eines funktionsfähigen Prototyps eines blockchain-basierten SSI-Systems mit integrierter Post-Quantum-Kryptographie für KRITIS, welcher die technische Machbarkeit und praktische Anwendbarkeit des Ansatzes demonstriert.

\subsubsection{Richtlinien}

\textcite[S. 82]{hevner_DesignScienceInformationsystemsresearch_2004} formulieren sieben Richtlinien für \ac{DSR}, die jeweils in einer qualitätsvollen Forschungsarbeit adressiert werden sollten. Diese Richtlinien bilden einen strukturierten Rahmen zur Durchführung und Bewertung von Design-Science-Forschung und gewährleisten, dass sowohl wissenschaftliche Rigor als auch praktische Relevanz erreicht werden.

\textbf{Guideline 1: Design as an Artifact} - Design-Science-Forschung muss ein brauchbares Artefakt produzieren \parencite[S. 82]{hevner_DesignScienceInformationsystemsresearch_2004}. Diese Arbeit erfüllt dies durch eine funktionsfähige Instantiation eines blockchain-basierten SSI-Systems mit integrierter Post-Quantum-Kryptographie.

\textbf{Guideline 2: Problem Relevance} - Design-Science-Forschung zielt auf technologiebasierte Lösungen für relevante Geschäftsprobleme \parencite[S. 84--85]{hevner_DesignScienceInformationsystemsresearch_2004}. Die Relevanz ergibt sich aus der Quantenbedrohung für KRITIS-Kryptographie und dem Bedarf an datenschutzfreundlichen Identitätslösungen.

\textbf{Guideline 3: Design Evaluation} - Die Nützlichkeit und Wirksamkeit eines Design-Artefakts müssen rigoros demonstriert werden \parencite[S. 85]{hevner_DesignScienceInformationsystemsresearch_2004}. Diese Arbeit validiert Funktionalität, Compliance und Kryptoagilität.

\textbf{Guideline 4: Research Contributions} - \ac{DSR} muss klare Beiträge zu Design-Artefakt, Foundations oder Methodologies liefern \parencite[S. 87]{hevner_DesignScienceInformationsystemsresearch_2004}. Der Beitrag liegt in der neuartigen Integration von PQC in SSI-Systeme für KRITIS-Kontexte.

\textbf{Guideline 5: Research Rigor} - \ac{DSR} beruht auf rigoroser Anwendung von Methoden in Konstruktion und Evaluation \parencite[S. 87]{hevner_DesignScienceInformationsystemsresearch_2004}. Diese Arbeit nutzt etablierte Frameworks (Hyperledger Indy, ACA-Py, Aries), NIST-standardisierte PQC-Algorithmen und wissenschaftlich hergeleitete Konzepte der Kryptoagilität.

\textbf{Guideline 6: Design as a Search Process} - Die Suche nach einem effektiven Artefakt erfordert iterative Exploration unter Berücksichtigung der Problemumgebung \parencite[S. 87--88]{hevner_DesignScienceInformationsystemsresearch_2004}. Der Design Cycle dieser Arbeit manifestiert sich als iterativer Prozess, bei dem Erkenntnisse aus einer Iteration die Designziele der nachfolgenden Iteration prägen.

\textbf{Guideline 7: Communication of Research} - Design-Science-Forschung muss effektiv sowohl für technologie-orientierte als auch für management-orientierte Audiences präsentiert werden \parencite[S. 90]{hevner_DesignScienceInformationsystemsresearch_2004}. Diese Arbeit adressiert dies durch eine umfassende Dokumentation des entwickelten Prototypen, systematische Darstellung der Evaluationsergebnisse sowie die explizite Ableitung \fixme{praktischer Implikationen} für kritische Infrastrukturen.

\subsection{FEDS-Framework} \label{sec:FEDS-Framework}

Nach \textcite[S. 2]{venable_FEDSFrameworkEvaluationDesignScienceResearch_2016} ist die Evaluation von Design-Artefakten eine Schlüsselaktivität des \ac{DSR}, ohne die die Forschung auf der Ebene theoretischer Annahmen über die Utility eines Artefakts verbleibt, ohne Evidenz für dessen tatsächliche Funktionsfähigkeit zu liefern. Um diese Lücke zu schließen und Rigor sicherzustellen, folgt diese Arbeit dem \ac{FEDS}-Framework nach \textcite[S. 2, 6]{venable_FEDSFrameworkEvaluationDesignScienceResearch_2016}, welches den Evaluationsprozess in vier iterative Schritte unterteilt, um die Strategie passgenau auf die spezifischen Projektrisiken abzustimmen.

Das \ac{FEDS}-Framework unterscheidet dabei fundamental zwischen zwei Dimensionen der Evaluation: der funktionalen Absicht (Why to evaluate) und dem Paradigma (How to evaluate). Hinsichtlich der Absicht wird zwischen formativer Evaluation, die der kontinuierlichen Verbesserung während der Entwicklung dient, und summativer Evaluation, die eine abschließende Bewertung der Zielerreichung vornimmt, differenziert. Bezüglich des Paradigmas unterscheidet das Framework zwischen Artificial Evaluation, die in kontrollierten Laborumgebungen stattfindet, und Naturalistic Evaluation, welche die Artefakte in realen organisatorischen Umfeldern unter echten Bedingungen prüft \parencite[S. 2, 6]{venable_FEDSFrameworkEvaluationDesignScienceResearch_2016}. Im Folgenden wird der vierstufige FEDS-Prozess für die vorliegende Arbeit erläutert.

\subsubsection{Schritt 1: Explikation der Evaluationsziele}
Der erste Schritt des Frameworks verlangt die Explikation der Evaluationsziele, um Konflikte zwischen konkurrierenden Anforderungen wie Rigor, Risikominimierung und Effizienz aufzulösen \parencite[S. 6--7]{venable_FEDSFrameworkEvaluationDesignScienceResearch_2016}. Für die Entwicklung eines blockchain-basierten \ac{SSI}-Protypen mit \ac{PQC} im \ac{KRITIS}-Kontext ergeben sich hieraus spezifische Prioritäten:

\textbf{Rigour (Efficacy vs. Effectiveness):} Das primäre Ziel dieser Arbeit ist der Nachweis der \textit{Efficacy}. Dies bedeutet den rigorosen Beleg, dass das instanziierte Artefakt (die \ac{PQC}-Integration in SSI) den beobachteten Effekt (sichere, quantenresistente Identitätsverifikation) kausal verursacht und nicht externe Störfaktoren \parencite[S. 6]{venable_FEDSFrameworkEvaluationDesignScienceResearch_2016}. Da es sich bei \ac{KRITIS}-Komponenten um sicherheitskritische Infrastruktur handelt, muss vor einem Feldtest (\enquote{Effectiveness} in realer Umgebung) zwingend die funktionale Korrektheit in einer kontrollierten Umgebung bewiesen werden, um \enquote{False Positives}, eine fälschliche Annahme der Sicherheit, auszuschließen \parencite[S. 3]{venable_FEDSFrameworkEvaluationDesignScienceResearch_2016}.
    
\textbf{Risikoreduktion (Technical Risk):} Das Hauptrisiko dieser Arbeit ist technischer Natur. Es besteht die Unsicherheit, ob die \ac{PQC}-Algorithmen (ML-DSA, ML-KEM) technisch in bestehende SSI-Frameworks (Hyperledger Aries) integriert werden können, ohne deren Kernfunktionen zu brechen. Soziale Risiken (z.B. Benutzerakzeptanz der Wallet-App) werden als nachrangig eingestuft und nicht evaluiert.
    
\textbf{Ausschluss von Effizienz-Zielen:} In Anlehnung an die Design-Science-Leitlinien wird explizit darauf hingewiesen, dass eine quantitative Performance-Evaluation (\enquote{Efficiency}) nicht Ziel dieser Arbeit ist. Die verwendeten \ac{PQC}-Referenzimplementierungen befinden sich in einem frühen Stadium, weshalb Laufzeitmessungen keine valide Aussagekraft für zukünftige produktive Systeme hätten.

\subsubsection{Schritt 2: Wahl der Evaluationsstrategie}
Basierend auf den identifizierten Zielen und Risiken wird für diese Arbeit die \textbf{Technical Risk \& Efficacy Strategy} gewählt, welche mit formativen, artifiziellen Tests beginnt und in einer summativen, artifiziellen Evaluation endet (\autoref{fig:feds_strategy}).

\begin{figure}[H]
    \centering
    \includegraphics[width=\linewidth]{FEDS_eval_strat}
    \caption{Gewählte Evaluationsstrategie im FEDS-Framework}
    \begin{flushleft}
    \textit{Anmerkung.} Eigene Darstellung in Anlehnung an \textcite[S. 4]{venable_FEDSFrameworkEvaluationDesignScienceResearch_2016}.
    \end{flushleft}
    \label{fig:feds_strategy}
\end{figure}

Diese Strategie ist nach \textcite[S. 6]{venable_FEDSFrameworkEvaluationDesignScienceResearch_2016} dann indiziert, wenn das primäre Entwicklungsrisiko technischer Natur ist und Evaluationen in realen Umgebungen aus Sicherheits- oder Kostengründen nicht durchführbar sind.
Ein naturalisitischer Ansatz (\enquote{Human Risk \& Effectiveness}) wird verworfen, da der Zugriff auf reale \ac{KRITIS}-Netzwerke für experimentelle kryptografische Prototypen ethisch und regulatorisch nicht vertretbar ist und die Technologie noch nicht den Reifegrad für Endanwendertests besitzt.

\subsubsection{Schritt 3: Bestimmung der zu evaluierenden Eigenschaften}
Der dritte Schritt definiert die konkreten Eigenschaften, die evaluiert werden sollen \parencite[S. 7--8]{venable_FEDSFrameworkEvaluationDesignScienceResearch_2016}. Für das entwickelte Artefakt leiten sich diese direkt aus den in Kapitel~\ref{sec:Anforderungsanalyse} definierten Anforderungen ab:

\begin{enumerate}
    \item \textbf{Fidelity (Funktionale Korrektheit):} Das System muss in der Lage sein, den vollständigen SSI-Lebenszyklus (Issuance, Verification, Revocation) unter Verwendung von PQC-Signaturen fehlerfrei durchzuführen. Es wird geprüft, ob die Artefakte (Agenten, Ledger, Wallet) spezifikationsgemäß interagieren.
    \item \textbf{KRITIS-Compliance (Sicherheit):} Es wird evaluiert, ob die implementierten Mechanismen den regulatorischen Vorgaben entsprechen. Dies umfasst die strikte Durchsetzung von TLS 1.3, die Verwendung hybrider Verfahren und die Netzwerksegmentierung.
    \item \textbf{Interoperabilität:} Die Fähigkeit des modifizierten Systems, Standard-DIDComm-Nachrichten trotz der veränderten kryptografischen Payload zu verarbeiten, ist ein kritisches Kriterium für die Efficacy.
\end{enumerate}

\subsubsection{Schritt 4: Design der individuellen Evaluationsepisoden}
\label{sec:Schritt4-Design der individuellen Evaluationsepisoden}

Der vierte Schritt umfasst das Design konkreter Evaluationsepisoden \parencite[S. 8]{venable_FEDSFrameworkEvaluationDesignScienceResearch_2016}. Für diese Arbeit wurden drei diskrete Episoden definiert, die parallel zu den Entwicklungszyklen verlaufen und in \autoref{tab:eval_episodes} dem jeweiligen Arbeitsfortschritt zugeordnet sind.

\begin{longtable}{L{1.5cm}L{5cm}L{8.5cm}}
    \caption{Evaluationsepisoden nach \ac{FEDS}}
    \label{tab:eval_episodes} \\
    \toprule
    \textbf{Ep.} & \textbf{Phase \& Art (FEDS)} & \textbf{Fokus / Methode} \\
    \midrule
    \endfirsthead
    \multicolumn{3}{l}{\textit{Tabelle \thetable\ (Fortsetzung)}} \\
    \toprule
    \textbf{Ep.} & \textbf{Phase \& Art (FEDS)} & \textbf{Fokus / Methode} \\
    \midrule
    \endhead
    \midrule
    \multicolumn{3}{r}{\textit{Fortsetzung auf nächster Seite}} \\
    \endfoot
    \bottomrule
    \multicolumn{3}{p{\linewidth}}{\textit{Anmerkung.} Eigene Darstellung der Evaluationsstrategie in Anlehnung an das \ac{FEDS}-Framework von \textcite{venable_FEDSFrameworkEvaluationDesignScienceResearch_2016}.} \\
    \endlastfoot
    1 & 
    Iteration 1 (Kapitel~\ref{sec:formative_evaluation_iteration1}) \newline 
    \textit{Formativ, Artifiziell} &
    White-Box Tests und Log-Analyse der Transportverschlüsselung sowie der Infrastrukturkomponenten. \\
    \midrule
    2 & 
    Iteration 2 (Kapitel~\ref{sec:formative_evaluation_iteration2}) \newline
    \textit{Formativ, Artifiziell} &
    Integrationstests der Plugin-Architektur, DIDComm-Verarbeitung und Validierung der PQC-Signaturen. \\
    \midrule
    3 & 
    Final (Kapitel~\ref{sec:Summative Evaluation}) \newline
    \textit{Summativ, Artifiziell} &
    Requirement Tracing und Validierung der \ac{KRITIS}-Compliance am integrierten Gesamtsystem. \\
\end{longtable}

\textbf{Episode 1 (Formativ):} Diese Episode erfolgt parallel zur ersten Iteration und fokussiert sich auf die Validierung der \textit{Transport-Layer-Security}. Da in dieser Phase fundamentale Infrastrukturkomponenten wie \glslink{Sidecar Proxy}{Sidecar Proxies} entwickelt werden, dient die formative Evaluation primär dazu, Designfehler so früh wie möglich zu identifizieren \parencite[S. 6, 7]{venable_FEDSFrameworkEvaluationDesignScienceResearch_2016}. Methodisch erfolgt dies durch die Analyse von Handshake-Protokollen und Cipher-Suites.

\textbf{Episode 2 (Formativ):} In der zweiten Iteration verlagert sich der Fokus auf den \textit{Application-Layer}. Hier wird formativ geprüft, ob die entwickelten Python-Plugins korrekt in den ACA-Py-Core geladen werden und ob die erweiterte Kryptografiebibliothek (liboqs) korrekt angesprochen wird.

\textbf{Episode 3 (Summativ):} Die abschließende Evaluation in Kapitel~\ref{sec:Summative Evaluation} führt alle Komponenten zusammen. Sie dient dem rigorosen Nachweis, dass das Gesamtsystem die eingangs definierten Forschungsfragen beantwortet. Hierbei wird geprüft, ob die Efficacy des \ac{PQC}-\ac{SSI}-Prototypen für \ac{KRITIS}-Anwendungsfälle gegeben ist, ohne reale Risiken einzugehen.

\subsection{DSRM Prozessmodell} \label{DSRM Prozessmodell}

Zur Sicherstellung einer rigorosen methodischen Fundierung orientiert sich der Forschungsablauf dieser Arbeit am \ac{DSRM} Prozessmodell nach \textcite[S. 46]{peffers_DesignScienceResearchmethodologyinformationsystemsresearch_2007}, welches einen Rahmen für die Durchführung und Präsentation von \ac{DSR} bereitstellt um die wissenschaftliche Validität von Design-Artefakten zu gewährleisten.

\fixme{Da die vorliegende Arbeit durch die spezifischen Bedrohungen des Quantencomputings für bestehende Infrastrukturen und die daraus resultierenden regulatorischen Anforderungen für \ac{KRITIS} motiviert ist} (Kapitel~\ref{sec:Problemstellung und Motivation}), folgt das Forschungsvorhaben einer \enquote{Problem-Centered Initiation}. Dies entspricht dem klassischen Einstieg in den \ac{DSRM}-Prozess über die erste Phase (\autoref{fig:peffers_dsrm}) \parencite[S. 56]{peffers_DesignScienceResearchmethodologyinformationsystemsresearch_2007}.

Neben der \enquote{Objective-Centered Solution}, die durch einen Bedarf in der Industrie oder Forschung ausgelöst wird (Phase 2), definieren die Autoren eine \enquote{Design \& Development Centered Initiation} auf Basis existierender Artefakte für explizite Problembereiche (Phase 3), sowie eine \enquote{Client-/Context-Initiated Solution}, welche auf der Beobachtung einer praktischen Lösung basiert (Phase 4), als drei weitere Einstiegspunkte \parencite[S. 56]{peffers_DesignScienceResearchmethodologyinformationsystemsresearch_2007}.

\begin{figure}[H]
    \centering
    \includegraphics[width=\linewidth]{DSRM Process Model_red.png}
    \caption{DSRM Process Model}
    \begin{flushleft}
    \textit{Anmerkung.} Adaptiert aus \textcite[S. 54]{peffers_DesignScienceResearchmethodologyinformationsystemsresearch_2007}.
    \end{flushleft}
    \label{fig:peffers_dsrm}
\end{figure}

Die Operationalisierung der sechs Phasen des \ac{DSRM} beginnt mit der Phase \enquote{Problem Identification and Motivation} \parencite[S. 52--55]{peffers_DesignScienceResearchmethodologyinformationsystemsresearch_2007}. Die Definition des spezifischen Forschungsproblems und dessen Relevanz für kritische Infrastrukturen wurde hierfür in Kapitel~\ref{sec:Problemstellung und Motivation} dargelegt. 

Darauf aufbauend werden in der Phase \enquote{Define the Objectives for a Solution} \parencite[S. 55]{peffers_DesignScienceResearchmethodologyinformationsystemsresearch_2007} aus der Problemanalyse konkrete Forschungsfragen (Kapitel~\ref{sec:Zielsetzung und Forschungsfragen}) und designrelevante Ziele für jede Iteration (Kapitel~\ref{sec:Designziele_Iteration_1} und Kapitel~\ref{sec:Designziele_Iteration_2}) abgeleitet, welche als Grundlage für die späteren Evaluationsphasen dienen.

Den Kern der Arbeit bildet die Phase \enquote{Design and Development} \parencite[S. 55]{peffers_DesignScienceResearchmethodologyinformationsystemsresearch_2007}, welche die Konzeption der Architektur sowie die technische Implementierung des \ac{PQC}-\ac{SSI}-Prototypen in beiden Iterationen umfasst (Kapitel~\ref{sec:Iterative Artefaktentwicklung}). Diese Phase operationalisiert die systematische Umsetzung der in den vorherigen Phasen definierten Anforderungen in ein funktionsfähiges Artefakt.

Die Eignung dieses Artefakts zur Problemlösung wird im Rahmen der Phase \enquote{Demonstration} \parencite[S. 55]{peffers_DesignScienceResearchmethodologyinformationsystemsresearch_2007} validiert. Diese erfolgt durch formative Evaluationsepisoden während der beiden Iterationen, welche schrittweise die Funktionsfähigkeit des Artefakts durch modulare Tests unter kontrollierten Bedingungen (Kapitel~\ref{sec:formative_evaluation_iteration1} und \ref{sec:formative_evaluation_iteration2}), bis hin zum vollständigen Prototypen (Kapitel~\ref{sec:Summative Evaluation}) nachweisen. 

Daran anschließend wird in der Phase \enquote{Evaluation} \parencite[S. 56]{peffers_DesignScienceResearchmethodologyinformationsystemsresearch_2007}strukturiert nach dem \ac{FEDS}-Framework (Kapitel~\ref{sec:FEDS-Framework}) in mehreren diskrete Evaluationsepisoden eine systematische Bewertung des Artefakts vorgenommen. Formative Episoden während beider Iterationen (Kapitel~\ref{sec:formative_evaluation_iteration1} und \ref{sec:formative_evaluation_iteration2}) adressieren technische Einzelfunktionen und Designfehler frühzeitig, die abschließende summative Evaluation in Kapitel~\ref{sec:Summative Evaluation} validiert systematisch die funktionale Korrektheit, \ac{KRITIS}-Compliance und Kryptoagilität des integrierten Gesamtsystems.

Den Abschluss bildet die Phase \enquote{Communication} \parencite[S. 56]{peffers_DesignScienceResearchmethodologyinformationsystemsresearch_2007}, in der die Ergebnisse, der Designprozess und die Evaluation durch die vorliegende schriftliche Ausarbeitung dokumentiert und der wissenschaftlichen Gemeinschaft zur Verfügung gestellt werden.

Obwohl das Modell sequenziell dargestellt ist, handelt es sich bei der Entwicklung in Kapitel~\ref{sec:Iterative Artefaktentwicklung} um einen iterativen Prozess, der Rücksprünge von der Evaluation zur Design-Phase erlaubt, um das Artefakt schrittweise zu verfeinern \parencite[S. 56]{peffers_DesignScienceResearchmethodologyinformationsystemsresearch_2007}.

\newpage
\section{Erste Iteration der Artefaktentwicklung}
\label{sec:Erste Iteration der Artefaktentwicklung}

\subsection{Designziele dieser Iteration} \label{sec:Designziele_Iteration_1}

Die Designziele dieser Iteration leiten sich unmittelbar aus den in Kapitel~\ref{sec:Zielsetzung und Forschungsfragen} definierten Forschungsfragen ab und werden operationalisiert durch eine schichtenbasierte Architektur, die die Quantensicherheit auf der Transportebene verankert. 

Im Fokus von FF1 (Systemarchitektur \& Compliance) steht die Etablierung einer modularen Architektur mit klarer Separation zwischen SSI-Agenten, DLT-Infrastruktur und quantensicherer Transportschicht auf Basis ausgewählter Frameworks und Technologien. Das Design soll die Kernherausforderung, Post-Quantum-Kryptografie nicht-invasiv und modular zu integrieren, adressieren.

Bezüglich FF2 (Algorithmenauswahl und Sicherheitsbewertung) liegt das Designziel auf der Erprobung quantenresistenter kryptografischer Primitive für die Transportebene auf Basis standardisierter Algorithmen und hybrider Schlüsselaustauschverfahren. Die formative Evaluierung soll dabei die technische Machbarkeit dieser Algorithmen in der Infrastruktur validieren und Erkenntnisse für die nachgelagerte Iteration generieren.

Für FF3 (Kryptografische Agilität) zielt diese Iteration auf die architektonische Vorbereitung für Austauschbarkeit kryptografischer Algorithmen ab. Das Design soll durch modulare Infrastrukturkomponenten die Voraussetzungen für zukünftige Algorithmenupdates ohne grundlegende Systemumgestaltung schaffen.

\subsection{Anforderungsanalyse} \label{sec:Anforderungsanalyse}

\subsubsection{Funktionale Anforderungen} \label{sec:Funktionale Anforderungen}

Basierend auf der Analyse von \textcite[S. 130]{nokhbehzaeem_BlockchainBasedSelfSovereignIdentitySurveyRequirementsUseCasesComparativeStudy_2021} lassen sich für das SSI-System sechs zentrale funktionale Anforderungen identifizieren, die den den vollständigen Lebenszyklus digitaler Identitätsnachweise ab decken (\autoref{tab:functional_requirements}).

\begin{longtable}{L{1cm}L{4cm}L{9cm}}
    \caption{Funktionale Anforderungen an SSI-Systeme}
    \label{tab:functional_requirements} \\
    \toprule
    \textbf{Nr.} & \textbf{Funktionale Anforderung} & \textbf{Beschreibung} \\
    \midrule
    \endfirsthead
    \multicolumn{3}{l}{\textit{Tabelle \thetable\ (Fortsetzung)}} \\
    \toprule
    \textbf{Nr.} & \textbf{Funktionale Anforderung} & \textbf{Beschreibung} \\
    \midrule
    \endhead
    \midrule
    \multicolumn{3}{r}{\textit{Fortsetzung auf nächster Seite}} \\
    \endfoot
    \bottomrule
    \multicolumn{3}{p{\linewidth}}{\textit{Anmerkung.} Eigene Darstellung auf Basis der Auflistungen und des Sequenzdiagramms in Anlehnung an \textcite[S. 130-132]{nokhbehzaeem_BlockchainBasedSelfSovereignIdentitySurveyRequirementsUseCasesComparativeStudy_2021}.} \\
    \endlastfoot
    1 & Issuer Discovery &
    Das System muss die Auffindbarkeit von publizierten Credential Schemata des Issuers digitaler Identitätsnachweise ermöglichen. \\
    \midrule
    2 & Connection Creation &
    Das System muss Verbindungen zwischen den Akteuren des SSI-Ökosystems etablieren können. \\
    \midrule
    3 & Credential Creation &
    Das System muss Funktionalität zur Erstellung und Ausstellung digitaler Credentials bereitstellen. \\
    \midrule
    4 & Verification with Credentials &
    Das System muss einen Verifikationsprozess zwischen Identity Holder, Verifier und Blockchain-basierter \ac{VDR} durch Validierung eines Identitätsnachweises ermöglichen. \\
    \midrule
    5 & Credential Revocation &
    Das System muss die Funktionalität zum Widerruf von Credentials unterstützen. \\
    \midrule
    6 & Credential Deletion &
    Das System muss die Funktionalität zur Löschung von Credentials unterstützen. \\
\end{longtable}

\subsubsection{KRITIS-spezifische Compliance-Anforderungen} \label{sec:KRITIS-spezifische Compliance-Anforderungen}

Die in \autoref{tab:compliance_requirements} konsolidierten Anforderungen definieren den normativen Rahmen für die Gestaltung und Evaluierung des \ac{PQC}-\ac{SSI}-Prototypen im Kontext KRITIS.

Im Bereich der \textbf{kryptografischen Verfahren} (Nr. 1-4) basieren die Vorgaben primär auf den Technischen Richtlinien des BSI. Für die Migration auf Post-Quantum-Kryptografie ist insbesondere die Wahl spezifischer Parameter-Sets für ML-DSA (NIST Level 3/5) und ML-KEM (Level 3/5) sowie die zwingende Implementierung hybrider Schlüsseleinigung vorgeschrieben, um sowohl Integrität als auch langfristige Vertraulichkeit gegen Quantencomputer-Angriffe zu gewährleisten \parencite[Kap. 2.2, 2.4, 5.3.4.2]{bsi_BSITR021021KryptographischeVerfahrenEmpfehlungenundSchluessellaengenVersion202501_2025}. Ergänzend fordert die TR-02102-2 den Einsatz moderner Transportverschlüsselung via TLS 1.3, um durch Perfect Forward Secrecy (PFS) die Kommunikationskanäle abzusichern \parencite[Kap. 3.2]{bsi_TechnischeRichtlinieTR021022KryptographischeVerfahrenEmpfehlungenundSchlussellangenTeil2VerwendungTransport_2025}.

Hinsichtlich der \textbf{Betriebssicherheit} (Nr. 5-6) leiten sich die Anforderungen direkt aus dem IT-Sicherheitsgesetz 2.0 (BSIG) und internationalen Standards ab. Essenziell für KRITIS-Betreiber ist hierbei die Implementierung effektiver \gls{SzA} durch umfassende Protokollierung sicherheitsrelevanter Ereignisse gemäß § 8a BSIG \parencite[Nr. 101, 103]{bsi_KonkretisierungKRITISAnforderungen8aAbsatz1undAbsatz1aBSIG_2024}. Flankierend schreibt die ISO/IEC 27001 eine strikte logische Netzsegmentierung vor, um die Ausbreitung potenzieller Sicherheitsvorfälle innerhalb der Infrastruktur zu begrenzen \parencite[Control A.8.22]{iso/iec_ISOIEC270012022InformationsecuritycybersecurityprivacyprotectionInformationsecuritymanagement_2022}.

Die dritte Säule bildet der \textbf{Datenschutz} (Nr. 7-9) auf Basis der DSGVO. Hierbei stehen Prinzipien wie \textit{Privacy by Design} gemäß Art. 25 und Datenminimierung nach Art. 5 im Fokus. Zudem muss das Recht auf Löschung nach Art. 17 durch geeignete Architekturmuster, etwa die Trennung von Identifikatoren und Inhaltsdaten, technisch gewährleistet werden \parencite[Art. 5, 17, 25]{daseuropaeischeparlamentundderratdereuropaeischenunion_VerordnungEU2016679EuropaeischenParlamentsundRatesvom27April2016_2016}.

\begin{longtable}{L{1cm}L{4cm}L{9cm}}
    \caption{Compliance Anforderungen an SSI-Systeme im KRITIS-Kontext}
    \label{tab:compliance_requirements} \\
    \toprule
    \textbf{Nr.} & \textbf{Compliance Anforderung} & \textbf{Beschreibung} \\
    \midrule
    \endfirsthead
    \multicolumn{3}{l}{\textit{Tabelle \thetable\ (Fortsetzung)}} \\
    \toprule
    \textbf{Nr.} & \textbf{Compliance Anforderung} & \textbf{Beschreibung} \\
    \midrule
    \endhead
    \midrule
    \multicolumn{3}{r}{\textit{Fortsetzung auf nächster Seite}} \\
    \endfoot
    \bottomrule
    \multicolumn{3}{p{\linewidth}}{\textit{Anmerkung.} Eigene Darstellung auf Basis von \textcite{bsi_BSITR021021KryptographischeVerfahrenEmpfehlungenundSchluessellaengenVersion202501_2025,bsi_TechnischeRichtlinieTR021022KryptographischeVerfahrenEmpfehlungenundSchlussellangenTeil2VerwendungTransport_2025,bsi_KonkretisierungKRITISAnforderungen8aAbsatz1undAbsatz1aBSIG_2024,iso/iec_ISOIEC270012022InformationsecuritycybersecurityprivacyprotectionInformationsecuritymanagement_2022,daseuropaeischeparlamentundderratdereuropaeischenunion_VerordnungEU2016679EuropaeischenParlamentsundRatesvom27April2016_2016}.} \\
    \endlastfoot
    1 & Einhaltung spezifischer Parameter-Sets für ML-DSA &
    Zur Gewährleistung der vom BSI geforderten Sicherheitsniveaus dürfen für das Verfahren ML-DSA ausschließlich die Parameter-Sets verwendet werden, die den NIST Security Strength Categories 3 oder 5 entsprechen. Konkret sind dies ML-DSA-65 oder ML-DSA-87 \parencite[Kap. 5.3.4.2]{bsi_BSITR021021KryptographischeVerfahrenEmpfehlungenundSchluessellaengenVersion202501_2025} \\
    \midrule
    2 & Einhaltung spezifischer Parameter-Sets für ML-KEM &
    Für den langfristigen Schutz vertraulicher Informationen mittels des gitterbasierten Schlüsselkapselungsverfahrens ML-KEM dürfen gemäß BSI-Einschätzung ausschließlich Parametersätze verwendet werden, die den NIST Security Strength Categories 3 oder 5 entsprechen. Zulässig sind demnach ML-KEM-768 sowie ML-KEM-1024 \parencite[Kap. 2.4.3]{bsi_BSITR021021KryptographischeVerfahrenEmpfehlungenundSchluessellaengenVersion202501_2025}. \\
    \midrule
    3 & Implementierung hybrider Schlüsseleinigung &
    Um langfristige Vertraulichkeit (Schutz vor \textit{Store Now, Decrypt Later}) zu gewährleisten, muss für die Schlüsseleinigung zwingend ein hybrides Verfahren implementiert werden, das ein anerkanntes klassisches Verfahren mit einem empfohlenen PQC-KEM kombiniert \parencite[Kap. 2.2, 2.4]{bsi_BSITR021021KryptographischeVerfahrenEmpfehlungenundSchluessellaengenVersion202501_2025}. \\
    \midrule
    4 & Bevorzugte Verwendung von TLS 1.3 &
    Für die Absicherung der Transportebene wird gemäß \textcite[Kap. 3.2]{bsi_TechnischeRichtlinieTR021022KryptographischeVerfahrenEmpfehlungenundSchlussellangenTeil2VerwendungTransport_2025} vorrangig das Protokoll TLS 1.3 empfohlen, da es PFS erzwingt und auf unsichere Cipher-Suites verzichtet. \\
    \midrule
    5 & Protokollierung sicherheitsrelevanter Ereignisse &
    Sicherheitsrelevante Ereignisse müssen auf System- und Netzebene zentral protokolliert werden, um eine zeitnahe Erkennung von Angriffen zu ermöglichen \parencite[Nr. 101, 103]{bsi_KonkretisierungKRITISAnforderungen8aAbsatz1undAbsatz1aBSIG_2024}. \\
    \midrule
    6 & Logische Netzsegmentierung &
    Gruppen von Informationsdiensten, Benutzern und Informationssystemen sollten in den Netzwerken der Organisation getrennt werden \parencite[Control A.8.22]{iso/iec_ISOIEC270012022InformationsecuritycybersecurityprivacyprotectionInformationsecuritymanagement_2022}. \\
    \midrule
    7 & Datenschutz durch Technikgestaltung (Privacy by Design) &
    Gemäß \textcite[Art. 25]{daseuropaeischeparlamentundderratdereuropaeischenunion_VerordnungEU2016679EuropaeischenParlamentsundRatesvom27April2016_2016} sind bereits bei der Entwicklung des Systems geeignete technische Maßnahmen zu treffen, die die Datenschutzgrundsätze addressieren. \\
    \midrule
    8 & Grundsatz der Datenminimierung &
    Personenbezogene Daten müssen dem Zweck angemessen und auf das notwendige Maß beschränkt sein. \parencite[Art. 5]{daseuropaeischeparlamentundderratdereuropaeischenunion_VerordnungEU2016679EuropaeischenParlamentsundRatesvom27April2016_2016}. \\
    \midrule
    9 & Recht auf Löschung &
    Die betroffene Person hat das Recht, von dem Verantwortlichen die unverzügliche Löschung sie betreffender personenbezogener Daten zu verlangen. Der Verantwortliche ist verpflichtet, personenbezogene Daten unverzüglich zu löschen \parencite[Art. 17]{daseuropaeischeparlamentundderratdereuropaeischenunion_VerordnungEU2016679EuropaeischenParlamentsundRatesvom27April2016_2016}. \\
\end{longtable}

\subsection{Framework- und Technologie-Auswahl}

Wie von \textcite[S. 3]{ghosh_DecentralizedCrossNetworkIdentityManagementBlockchainInteroperation_2021} demonstriert, setzt auch die vorliegende Arbeit auf existierende Konzepte und Tools für dezentrale Identitätsverwaltung als Bausteine für die Entwicklung des \ac{PQC}-\ac{SSI}-Prototypen. Die systematische Auswahl der Frameworks und Technologien -- bestehend aus der \ac{DLT}-Plattform, dem SSI-Framework, der Kryptografiebibliothek, der Revocation-Infrastruktur und dem \gls{Sidecar Proxy} -- wird transparent in \ref{sec:Anhang_Framework- und Technologie-Auswahl} dokumentiert und begründet.

\subsection{Architekturentwurf}

\subsubsection{Gesamtarchitektur}
\label{sec:Gesamtarchitektur_Iteration1}

Der Architekturentwurf dieser Iteration operationalisiert die in Kapitel~\ref{sec:Designziele_Iteration_1} definierten Designziele durch eine dreischichtige, containerbasierte Systemarchitektur, die Post-Quantum-Kryptografie auf der Transportebene verankert. Abbildung~\ref{fig:Architektur_Iteration1} visualisiert die Zielarchitektur, bestehend aus der DLT-Schicht mit vier Hyperledger-Indy-Validator-Nodes samt Ledger Browser, der Revocation-Schicht mit dediziertem Tails-Server sowie der SSI-Agenten-Schicht mit drei ACA-Py-Instanzen in den Rollen Issuer, Holder und Verifier.

\begin{figure}[H]
    \centering
    \includegraphics[width=\linewidth]{Architektur_Iteration1}
    \caption{Architekturentwurf der ersten Iteration}
    \begin{flushleft}
    \textit{Anmerkung.} Eigene Darstellung.
    \end{flushleft}
    \label{fig:Architektur_Iteration1}
\end{figure}

Das zentrale Architekturprinzip bildet das Sidecar Proxy Pattern (\ref{sec:Anhang_Sidecar Proxy}). Jede extern erreichbare Komponente wird durch einen NGINX-basierten \ac{PQC} Sidecar Proxy geschützt, der TLS-1.3-Verbindungen mit \glslink{Hybride Schemata}{hybrider Schlüsseleinigung} terminiert und im Fallback \ac{ECC} (X25519MLKEM768:x25519) unterstützt. Die Authentifizierung erfolgt über ML-DSA-65-signierte X.509-Zertifikate. In \autoref{fig:Architektur_Iteration1} wird diese verschlüsselte Kommunikation durch die rot-gestrichelten Verbindungspfeile zwischen den Sidecar Proxies visualisiert.

Die Backend-Dienste (Indy-Nodes, Webserver, Tails-Server und ACA-Py-Agents) verbleiben innerhalb isolierter interner Docker-Netzwerke und kommunizieren ausschließlich über unverschlüsseltes HTTP. Externe Zugriffe erfolgen ausschließlich über das gemeinsame Netzwerk sidecar\_proxy, das als quantensichere Kommunikationsdomäne fungiert und eine strikte Netzsegmentierung gemäß der sechsten \ac{KRITIS}-Anforderung (\autoref{tab:compliance_requirements}) gewährleistet.

\subsubsection{ACA-Py Applikationsarchitektur}
\label{sec:ACA-Py Applikationsarchitektur_Iteration1}

Die Architektur von \ac{ACA-Py} folgt einem modularen Designansatz, der durch eine strikte Trennung zwischen der generischen Protokollebene und der anwendungsspezifischen Geschäftslogik gekennzeichnet ist. Dieses Architekturmuster entkoppelt die kryptografischen Kernfunktionen und das DIDComm-Messaging (Agent) von der Steuerungslogik (Controller) \parencite{openwallet-foundation_AcapyREADMEmdMainopenwalletfoundationacapyGitHub_}. \autoref{fig:ACAPY_Application_Architecture_Iteration 1} visualisiert diese High-Level-Applikationsarchitektur und verdeutlicht die Interaktion der Schichten.

\begin{figure}[H]
    \centering
    \includegraphics[width=\linewidth]{ACAPY Application Architecture_Iteration 1}
    \caption{ACA-Py High Level Applikationsarchitektur}
    \begin{flushleft}
    \textit{Anmerkung.} Eigene Darstellung auf Basis von \textcite{openwallet-foundation_AcapyREADMEmdMainopenwalletfoundationacapyGitHub_,openwallet-foundation_AcapyAcapy_agentAdminserverpymainopenwalletfoundationacapyGitHub_,openwallet-foundation_AcapyAcapy_agentProtocolsout_of_bandv1_0routespymainopenwalletfoundationacapyGitHub_,openwallet-foundation_AcapyAcapy_agentProtocolsissue_credentialv2_0routespymainopenwalletfoundationacapyGitHub_,decentralized-identity_AriesrfcsFeatures0434outofbandREADMEmdmaindecentralizedidentityariesrfcs_,decentralized-identity_AriesrfcsFeatures0023didexchangemaindecentralizedidentityariesrfcs_,decentralized-identity_AriesrfcsFeatures0453issuecredentialv2maindecentralizedidentityariesrfcsGitHub_,openwallet-foundation_AcapyAcapy_agentWalletaskarpymainopenwalletfoundationacapyGitHub_,openwallet-foundation_AcapyAcapy_agentProtocolsout_of_bandv1_0managerpymainopenwalletfoundationacapyGitHub_,openwallet-foundation_AcapyAcapy_agentWalletkey_typepymainopenwalletfoundationacapyGitHub_,openwallet-foundation_AcapyAcapy_agentAskarstorepymainopenwalletfoundationacapyGitHub_,openwallet-foundation_AcapyAcapy_agentAskardidcommv1pymainopenwalletfoundationacapyGitHub_,openwallet-foundation_AcapyAcapy_agentTransportinboundhttppymainopenwalletfoundationacapyGitHub_,openwallet-foundation_AcapyAcapy_agentTransportoutboundhttppymainopenwalletfoundationacapyGitHub_}.
    \end{flushleft}
    \label{fig:ACAPY_Application_Architecture_Iteration 1}
\end{figure}

Layer 1 bildet eine HTTP-REST-API (Admin API), die über \enquote{server.py} implementiert wird \parencite{openwallet-foundation_AcapyAcapy_agentAdminserverpymainopenwalletfoundationacapyGitHub_}. Sie ermöglicht es Controller-Anwendungen den Agent über standardisierte Endpoints wie \enquote{/out-of-band/create-invitation} \parencite{openwallet-foundation_AcapyAcapy_agentProtocolsout_of_bandv1_0routespymainopenwalletfoundationacapyGitHub_} oder \enquote{/issue-credential-2.0/send} \parencite{openwallet-foundation_AcapyAcapy_agentProtocolsissue_credentialv2_0routespymainopenwalletfoundationacapyGitHub_} zu steuern, ohne direkt mit der Python-Codebasis zu interagieren. Die Endpunkte werden hierbei durch mehrere \enquote{routes.py} umgesetzt.

Layer 2 implementiert die SSI-Geschäftslogik als Protocol Handler durch mehrere \enquote{manager.py}, die Aries \ac{RFC}s umsetzen. Das Out-of-Band Protocol (RFC 0434) \parencite{decentralized-identity_AriesrfcsFeatures0434outofbandREADMEmdmaindecentralizedidentityariesrfcs_} generiert Invitation-Nachrichten mit \enquote{did\:peer\:4-DIDs} \parencite{openwallet-foundation_AcapyAcapy_agentProtocolsout_of_bandv1_0managerpymainopenwalletfoundationacapyGitHub_}, das DID Exchange Protocol (RFC 0023) \parencite{decentralized-identity_AriesrfcsFeatures0023didexchangemaindecentralizedidentityariesrfcs_} authentifiziert Agent-Verbindungen mittels ED25519-Signaturen \parencite{openwallet-foundation_AcapyAcapy_agentProtocolsdidexchangev1_0managerpymainopenwalletfoundationacapyGitHub_}, und das Issue Credential Protocol (RFC 0453) \parencite{decentralized-identity_AriesrfcsFeatures0453issuecredentialv2maindecentralizedidentityariesrfcsGitHub_} erstellt AnonCreds-Credentials \parencite{openwallet-foundation_AcapyAcapy_agentProtocolsissue_credentialv2_0managerpymainopenwalletfoundationacapyGitHub_}.

Layer 3 abstrahiert die Schlüsselverwaltung durch das Aries-Askar-Wallet \parencite{openwallet-foundation_AcapyAcapy_agentWalletaskarpymainopenwalletfoundationacapyGitHub_}, das ED25519-Schlüsselpaare für Signaturen und X25519-Schlüsselpaare für Key Agreement generiert \parencite{openwallet-foundation_AcapyAcapy_agentWalletkey_typepymainopenwalletfoundationacapyGitHub_}, diese mit Multicodec-Präfixen kodiert \parencite{openwallet-foundation_AcapyAcapy_agentWalletkey_typepymainopenwalletfoundationacapyGitHub_} und verschlüsselt in einer SQLite-Datenbank ablegt \parencite{openwallet-foundation_AcapyAcapy_agentAskarstorepymainopenwalletfoundationacapyGitHub_}, wobei ChaCha20-Poly1305-Authenticated-Encryption verwendet wird \parencite{openwallet-foundation_AcapyAcapy_agentAskardidcommv1pymainopenwalletfoundationacapyGitHub_}.

Layer 4 realisiert das DIDComm-Messaging über \enquote{pack\_message()} und \enquote{unpack\_message()} \parencite{openwallet-foundation_AcapyAcapy_agentAskardidcommv1pymainopenwalletfoundationacapyGitHub_}, welches anschließend mittels HTTP-basierter Transport-Mechanismen für Inbound- \parencite{openwallet-foundation_AcapyAcapy_agentTransportinboundhttppymainopenwalletfoundationacapyGitHub_} und Outbound-Kommunikation \parencite{openwallet-foundation_AcapyAcapy_agentTransportoutboundhttppymainopenwalletfoundationacapyGitHub_} übertragen wird, wobei in der klassischen Architektur kein quantensicheres TLS verwendet wird.

\subsection{Implementierung}

Die Implementierung realisiert die Zertifikatsinfrastruktur mit ML-DSA-Signaturen, PQC-Sidecar-Proxies mit TLS~1.3 und hybrider Schlüsseleinigung, die Hyperledger-Indy-DLT-Schicht, die Revocation Registry sowie drei ACA-Py-Agenten, die mittels Docker-Compose mit expliziter Netzsegmentierung orchestriert werden. Die hierfür genutzte Entwicklungsumgebung ist in \ref{sec:Anhang_Setup der Entwicklungsumgebung} dokumentiert.

\subsubsection{Zertifikatsstruktur}
\label{sec:Zertifikatsstruktur}

Die Zertifikatsinfrastruktur für die PQC-Sidecar-Proxies basiert auf einer selbstsignierten Root \ac{CA}, die mit dem Post-Quantum-Signaturalgorithmus ML-DSA-87 erstellt wurde. Die Root \ac{CA} dient als Trust Anchor für alle in der Architektur verwendeten TLS-Zertifikate und gewährleistet, dass sämtliche Zertifikatssignaturen quantenresistent nach der höchsten \ac{NIST} Sicherheitsstufe 5 sind \parencite[S. 15]{nationalinstituteofstandardsandtechnologyus_ModulelatticebasedDigitalSignaturestandard_2024} .

Das Zertifikatserstellungsverfahren folgt einem fünfschrittigen Workflow, wie in \autoref{fig:Zertifikatserstellungsworkflow} dargestellt.

\begin{figure}[H]
    \centering
    \includegraphics[width=\linewidth]{Zertifikatserstellungsworkflow}
    \caption{Zertifikatserstellungsworkflow für PQC-basierte Sidecar-Proxies}
    \begin{flushleft}
    \textit{Anmerkung.} Eigene Darstellung.
    \end{flushleft}
    \label{fig:Zertifikatserstellungsworkflow}
\end{figure}

Zunächst wird der Root-CA-Schlüssel generiert (Step 1), gefolgt von der Erstellung des selbstsignierten Root-CA-Zertifikats mit einer Gültigkeit von zehn Jahren (Step 2). Anschließend werden für jeden Sidecar-Proxy dedizierte Schlüssel mit ML-DSA-65 generiert (Step 3), die ein besseres Verhältnis zwischen Sicherheit und Speicherbedarf bieten \parencite[S. 16]{nationalinstituteofstandardsandtechnologyus_ModulelatticebasedDigitalSignaturestandard_2024}. Für jeden Proxy wird ein Certificate Signing Request erstellt, der die erforderlichen Subject Alternative Names enthält (Step 4), bevor die Zertifikate abschließend von der Root CA mit ML-DSA-65 und SHA3-256 signiert werden (Step 5). Die detaillierte Implementierung dieses Workflows ist in \ref{sec:Anhang_Zertfikatserstellungsworkflow} dokumentiert.

\subsubsection{Sidecar Proxy nginx}
\label{sec:Sidecar Proxy nginx}

Die Implementierung der post-quanten-kryptographischen Absicherung auf Transport-Layer-Ebene basiert auf einer modifizierten Version des NGINX-Dockerfiles von \textcite{open-quantum-safe_OpenquantumsafeOqsdemosNginxDockerfile_2025} sowie spezifischen NGINX-Konfigurationsdateien für jeden \gls{Sidecar Proxy}. Beide Komponenten realisieren ein konsistentes Konzept bestehend aus Hybrid-Key-Exchange mit \ac{ECC} als Fallback, welches auf Transport-Layer-Ebene Anwendung findet.

Das modifizierte Dockerfile folgt dem Multi-Stage-Build-Prinzip nach \textcite[S. 1]{rosa_MiningMeasuringImpactchangepatternsimprovingsizebuildtimedockerimages_2025} und integriert OpenSSL 3.5.4, liboqs und den oqs-provider. \autoref{fig:Sidecar_Proxy_nginx_Dockerfile} visualisiert die zwei zentralen Phasen (Build-Stage und Runtime-Stage) sowie die Modifikationen gegenüber dem Original-Dockerfile (rote Markierungen). Die Hybrid-Key-Exchange mit \ac{ECC}-Fallback wird durch die DEFAULT-GROUPS-Konfiguration X25519MLKEM768:X25519 realisiert, eine Konvention, die sich konsistent durch die gesamte Implementierung zieht und in \ref{sec:Anhang_Dockerfile Sidecar Proxy nginx} ausführlich erläutert wird.

\begin{figure}[H]
    \centering
    \includegraphics[width=\linewidth]{2 Stage Sidecarproxy Dockerfile}
    \caption{Sidecar Proxy NGINX Dockerfile Multi-Stage Build}
    \begin{flushleft}
    \textit{Anmerkung.} Eigene Darstellung.
    \end{flushleft}
    \label{fig:Sidecar_Proxy_nginx_Dockerfile}
\end{figure}

Die NGINX-Konfigurationsdateien implementieren ein standardisiertes \gls{Sidecar Proxy} Pattern, das diese Architektur operationalisiert. Sie folgen einem konsistenten Schichtenmodell aus globalen Parametern, internen Service-Abstraktionen über Upstream-Blöcke und externen HTTPS-Endpunkten über Server-Blöcke. TLS~1.3 wird durch die Direktive ssl\_ecdh\_curve X25519MLKEM768:X25519 konfiguriert (Listing~\ref{lst:nginx_holder.conf}), wodurch Schlüsselaustausch und Fallback auf klassische \ac{ECC}-basierte Kurven ermöglicht werden. Die aktivierte Direktive ssl\_protocols TLSv1.3 wird durch SSL-Zertifikate mit ML-DSA-65-Signaturalgorithmus ergänzt, sodass sowohl der Schlüsselaustausch als auch die Server-Authentifikation post-quantensicher erfolgen. Die gesamte Konfiguration wird in \ref{sec:Anhang_nginx_holder.conf} erläutert.

\subsubsection{DLT-Infrastruktur}

Für die \ac{PQC}-Integration Transport-Layer-Ebene wurde die von-network-Architektur um einen PQC Nginx Sidecar Proxy erweitert.
Diese Modifikation stellt die zentrale Anpassung gegenüber dem Original-Quellcode \parencite{bcgov_GitHubBcgovVonnetworkportabledevelopmentlevelIndyNodenetwork_} dar und betrifft primär die docker-compose.yml-Konfigurationsdatei (Listing \ref{lst:docker-compose.yml-DLT-Infrastruktur}), sowie die Hinzufügung eines neuen Verzeichnisses \enquote{pqc\_sidecarproxy\_nginx/}, welches die in Kapitel~\ref{sec:Sidecar Proxy nginx} und Kapitel~\ref{sec:Zertifikatsstruktur} vorgestellten Dockerfile-, Nginx-Konfigurationsdatei und Zertifikate für den quantensicheren \gls{Sidecar Proxy} enthält. Der Webserver-Container, der im Original-Setup direkt auf Port 9000 exponiert ist \parencite{bcgov_VonnetworkDockercomposeymlMainbcgovvonnetworkGitHub_}, verbleibt in der modifizierten Architektur ausschließlich im internen Docker-Netzwerk \enquote{von}. Stattdessen terminiert der neu hinzugefügte pqc-sidecarproxy-webserver-Container alle eingehenden \ac{TLS}-1.3-Verbindungen auf Port 8000 und leitet die Anfragen nach erfolgreicher \glslink{Hybride Schemata}{hybrider Schlüsselvereinbarung} und ML-DSA-65-Zertifikatsverifikation als unverschlüsseltes HTTP an den internen Webserver-Container weiter. Die unmodifizierte von-network-Implementierung wird in Kapitel~\ref{sec:Anhang_DLT-Infrastruktur} näher erläutert.

Die Integration des Sidecar Proxies erforderte die Definition eines zusätzlichen, extern zugänglichen Docker-Netzwerks \enquote{von\_sidecarproxy}, das als gemeinsame Kommunikationsebene für alle PQC-Proxies der Gesamtarchitektur dient. Dieses Netzwerk wird in der docker-compose.yml (Listing \ref{lst:docker-compose.yml-DLT-Infrastruktur}) als \enquote{external: true} deklariert. Der pqc-sidecarproxy-webserver-Container ist sowohl mit dem internen von-Netzwerk (für Backend-Kommunikation) als auch mit dem externen von\_sidecarproxy-Netzwerk (für Client-Zugriffe) verbunden, wodurch eine strikte Netzwerksegmentierung zwischen interner und externer Kommunikation gewährleistet wird.

\subsubsection{Revocation Registry}

Für die Integration in die Post-Quantum-gesicherte Gesamtarchitektur wurde der indy-tails-server analog zur DLT-Infrastruktur um einen PQC Nginx Sidecar Proxy erweitert. Diese Modifikation betrifft primär die Docker-Compose-Konfigurationsdatei (Listing \ref{lst:docker-compose.yml-Revocation-Registry}), in der ein zusätzlicher Service \enquote{pqc-sidecarproxy-tails-server} definiert wurde, sowie die Hinzufügung eines neuen Verzeichnisses \enquote{pqc\_sidecarproxy\_nginx/}, welches die in Kapitel~\ref{sec:Sidecar Proxy nginx} und Kapitel~\ref{sec:Zertifikatsstruktur} vorgestellten Dockerfile-, Nginx-Konfigurationsdatei und Zertifikate für den quantensicheren \gls{Sidecar Proxy} enthält. Der ursprüngliche Tails-Server-Container verbleibt im internen Docker-Netzwerk \enquote{tails-server} und exponiert Port 6543 ausschließlich innerhalb dieses Netzwerks. Der neu hinzugefügte PQC-Proxy-Container terminiert alle externen \ac{TLS}-1.3-Verbindungen auf Port 6543 und leitet die Anfragen nach erfolgreicher erfolgreicher \glslink{Hybride Schemata}{hybrider Schlüsselvereinbarung} und ML-DSA-65-Zertifikatsverifikation als unverschlüsseltes HTTP an den internen Tails-Server weiter. Die unmodifizierte indy-tails-server-Implementierung wird in Kapitel~\ref{sec:Anhang_Revocation Registry} näher erläutert.

Die Netzwerk-Integration folgt dem etablierten \gls{Sidecar Proxy} Pattern. Der \enquote{pqc-sidecarproxy-tails-server}-Container ist sowohl mit dem internen \enquote{tails-server}-Netzwerk (für Backend-Kommunikation) als auch mit dem externen, manuell erstellten \enquote{von\_sidecarproxy}-Netzwerk (für Client-Zugriffe) verbunden. Diese Dual-Network-Architektur erzwingt, dass alle externen Zugriffe auf den Tails-Server über den quantensicheren \gls{Sidecar Proxy} geleitet werden.

\subsubsection{SSI-Agenten} \label{SSI-Agenten}

Die SSI-Agent-Schicht bildet die Anwendungsebene der Gesamtarchitektur und implementiert die drei klassischen Rollen des SSI-Ökosystems Issuer, Holder und Verifier (Kapitel~\ref{sec:Self-Sovereign Identity}) samt ihrer dedizierten PQC Nginx Sidecar Proxies. Die Architektur folgt dem in Kapitel~\ref{sec:ACA-Py Applikationsarchitektur} beschriebenen Referenzmodell, alle drei Agenten basieren auf dem unmodifizierten \ac{ACA-Py} Base-Image (Listing \ref{lst:Dockerfile-acapy-base}) und werden ausschließlich über Kommandozeilen-Parameter konfiguriert, ohne Änderungen am ACA-Py-Quellcode vorzunehmen.

Listing \ref{lst:docker-compose.yml-SSI-Agenten} zeigt die Docker-Compose-Konfiguration der drei ACA-Py-Agenten innerhalb der Gesamtarchitektur. Die Agent-Konfiguration erfolgt hierbei vollständig deklarativ über Docker-Compose-Service-Definitionen, die jeweils den \enquote{start}-Befehl von \ac{ACA-Py} mit rollenspezifischen Parametern aufrufen. Jeder Agent wird in einem dedizierten Docker-Netzwerk isoliert betrieben und über einen PQC Nginx Sidecar Proxy mit quantensicherer TLS-1.3-Verschlüsselung nach außen exponiert. Die Wallet-Konfiguration nutzt persistente Docker-Volumes zur Speicherung von DIDs, Credentials und Connections über Container-Neustarts hinweg. Die Agents verbinden sich über PQC-gesicherte Proxies mit der DLT-Infrastruktur und der Revocation Registry. Auto-Response-Features ermöglichen vollständig scriptgesteuerte SSI-Workflows ohne manuelle Interaktion. Die Netzwerk-Architektur folgt einem strikten Isolation-Prinzip, sodass Agents nur über das externe \enquote{von\_sidecarproxy}-Netzwerk untereinander kommunizieren können. Health-Checks und Service-Dependencies orchestrieren deterministische Startup-Sequenzen und eliminieren Race-Conditions während des Deployments.
Eine ausführliche Darstellung der Agent-Konfigurationsparameter und der Sidecar-Proxy-Architektur findet sich in \ref{sec:Anhang_SSI-Agenten}.

\subsubsection{Docker Orchestrierung der Gesamtarchitektur} \label{sec:Docker Orchestrierung der Gesamtarchitektur}

Die in den vorangegangenen Abschnitten beschriebenen Einzelkomponenten -- Zertifikatsinfrastruktur, PQC-Sidecar-Proxies, DLT-Infrastruktur, Revocation Registry und SSI-Agenten -- werden mittels einer mehrstufigen Docker-Compose-Orchestrierung zu einem funktionsfähigen Gesamtsystem integriert. Der Startprozess der Gesamtarchitektur folgt einer deterministischen Sequenz, die in Listing~\ref{lst:Docker-Compose-Start-der-Gesamtarchitektur} dokumentiert ist und die korrekten Abhängigkeiten zwischen den Infrastrukturschichten gewährleistet.

Die Orchestrierung gliedert sich in drei sequenzielle Phasen, die jeweils durch separate Docker-Compose-Konfigurationen gesteuert werden. In der ersten Phase wird die DLT-Infrastruktur über das von-network-Management-Skript (Listing~\ref{lst:von-network-manage-script}) initialisiert, wodurch die vier Indy-Validator-Nodes, der Genesis-Webserver sowie der zugehörige PQC-Sidecar-Proxy gestartet werden. Diese Phase erzeugt das gemeinsame externe Docker-Netzwerk \enquote{von\_sidecarproxy}, das als zentrale Kommunikationsschicht für alle quantensicheren Verbindungen dient. Die zweite Phase umfasst die Initialisierung der Revocation Registry mittels des indy-tails-server-Management-Skripts (Listing~\ref{lst:indy-tails-server-manage-script}), wodurch der Tails-Server-Container sowie dessen PQC-Proxy-Frontend bereitgestellt werden. In der dritten Phase werden schließlich die SSI-Agenten gemeinsam mit ihren jeweiligen PQC-Sidecar-Proxies über die projektspezifische Docker-Compose-Konfiguration (Listing~\ref{lst:docker-compose.yml-SSI-Agenten}) gestartet.

Die in \autoref{fig:Docker-Compose-Übersicht-Iteration-1} visualisierte Containerarchitektur verdeutlicht die resultierende Systemtopologie. Die Gesamtarchitektur umfasst insgesamt 14 Container, die sich auf die drei funktionalen Schichten verteilen. Die DLT-Schicht besteht aus insgesamt sechs Containern, den vier Validator-Nodes, einem Webserver und einem PQC Sidecar Proxy. Die Revocation-Schicht umfasst insgesamt zwei Container, einen Tails-Server und dessen PQC Sidecar Proxy. Die Agent-Schicht besteht aus insgesamt sechs Containern für die drei SSI-Agenten und ihre jeweiligen PQC Sidecar Proxies.

Die Netzwerktopologie nutzt sechs dedizierte Docker-Bridges, um die logische Separation der Schichten zu realisieren. Das hope\_hope-holder-, hope\_hope-issuer- und hope\_hope-verifier-Netzwerk verbinden jeweils die SSI-Agenten mit der DLT-Schicht. Das docker\_tails-server-Netzwerk isoliert die Revocation-Infrastruktur, während das von\_sidecarproxy-Netzwerk die PQC-Sidecar-Proxies konnektiviert. Das von\_von-Netzwerk integriert die Validator-Nodes untereinander und mit der Revocation-Schicht.

Die Datenpersistenz wird durch elf Docker Volumes realisiert. Während docker\_nginx-logs und von\_nginx-logs die Webserver-Logs aggregieren, speichern hope\_holder-data, hope\_issuer-data und hope\_verifier-data die lokalen Wallets und kryptografischen Materialien der SSI-Agenten. Die Validator-Node-Datenbank wird durch von\_node1-data bis von\_node4-data persistiert, während von\_webserver-cli und von\_webserver-ledger die Ledger-Zustandsdaten und CLI-Konfigurationen verwalten. Diese Volumenstruktur entkoppelt den Containern-Lebenszyklus vom Datenschicksal und ermöglicht die Wiederaufnahme der Systemzustände über Container-Neustarts hinweg.

Die \enquote{depends\_on}-Direktiven in den Docker-Compose-Konfigurationen (Listing~\ref{lst:docker-compose.yml-DLT-Infrastruktur}, Listing~\ref{lst:docker-compose.yml-Revocation-Registry} und Listing~\ref{lst:docker-compose.yml-SSI-Agenten}) definieren explizite Startup-Abhängigkeiten, die Race-Conditions während des Deployment-Prozesses eliminieren. Die PQC Sidecar Proxies werden vor den ihnen zugeordneten Backend-Services gestartet, und die SSI-Agenten warten auf die vollständige Initialisierung der Infrastruktur-Services, bevor sie ihre Genesis-Transaktionsdatei abrufen. Diese Orchestrierung gewährleistet eine deterministische Startup-Sequenz und stellt sicher, dass alle Komponenten beim Erreichen ihres operativen Zustands auf vollständig verfügbare Abhängigkeiten zugreifen können.

Die \enquote{depends\_on}-Direktiven in den Docker-Compose-Konfigurationen (Listing~\ref{lst:docker-compose.yml-DLT-Infrastruktur}, Listing~\ref{lst:docker-compose.yml-Revocation-Registry} und Listing~\ref{lst:docker-compose.yml-SSI-Agenten}) definieren explizite Startup-Abhängigkeiten, die über reine Ausführungsreihenfolgen hinausgehen. Ohne Synchronisation führen verfrühte Interaktionen abhängiger Services zu \enquote{faulty interactions} während der Boot-Phase \parencite[S. 25]{deiasio_FrameworkMicroservicesSynchronization_2021}. Die implementierte Konfiguration eliminiert diese Race-Conditions, indem sie sicherstellt, dass die PQC Sidecar Proxies nicht gestartet werden, bevor die ihnen zugeordneten Backend-Services den Status (\enquote{ready}) erreichen. Diese Orchestrierung erzwingt eine deterministische Startup-Sequenz, die verhindert, dass abhängige Komponenten auf Dienste zugreifen, die sich zwar im Status \textit{Running}, aber noch nicht im Status \textit{Ready} befinden \parencite[S. 29]{deiasio_FrameworkMicroservicesSynchronization_2021}.

\begin{figure}[H]
    \centering
    \includegraphics[width=\linewidth]{docker_compose_übersicht.png}
    \caption{Docker-Compose-Übersicht der ersten Iteration}
    \begin{flushleft}
    \textit{Anmerkung.} Eigene Darstellung.
    \end{flushleft}
    \label{fig:Docker-Compose-Übersicht-Iteration-1}
\end{figure}

\subsection{Formative Evaluation}
\label{sec:formative_evaluation_iteration1}

In Übereinstimmung mit dem in Kapitel~\ref{sec:Schritt4-Design der individuellen Evaluationsepisoden} entworfenen Evaluationsdesign setzt dieser Abschnitt die erste definierte Evaluationsepisode (\autoref{tab:eval_episodes}) um. Charakterisiert als formative und artifizielle Untersuchung, liegt der Fokus dieser Phase exklusiv auf der Validierung der Transport-Layer-Security sowie der korrekten Konfiguration der Infrastrukturkomponenten. Ziel ist es, gemäß den Prinzipien von \textcite[S. 6, 7]{venable_FEDSFrameworkEvaluationDesignScienceResearch_2016} Designfehler in der Sidecar-Architektur frühzeitig zu identifizieren.

Methodisch erfolgt dies primär durch White-Box-Tests. Da bei diesem Verfahren die internen Strukturen und Implementierungsdetails der Software bekannt sind und gezielt in die Prüfung einbezogen werden \parencite[S. 10]{myers_ArtSoftwareTesting_2012}, eignet es sich besonders gut, um die korrekte Konfiguration der kryptographischen Primitive innerhalb der Container-Architektur durch eine detaillierte Analyse der Handshake-Logs zu validieren.

Für die technische Durchführung dieser Analysen war die Bereitstellung eines \ac{PQC}-fähigen Browsers zwingend erforderlich, da aktuelle Produktivbrowser noch keine Post-Quanten-Kryptographie in ihren \ac{TLS}-Implementierungen unterstützen. Diese Limitation führt bei Verbindungsversuchen zu \ac{PQC}-fähigen Servern unweigerlich zu einem Cipher-Mismatch, wie in \autoref{fig:Cipher-Mismatch-Blockchain-Webserver} veranschaulicht. Um diese Inkompatibilität zu überwinden, wurde ein Chromium-basierter Browser mit integrierter \ac{PQC}-Unterstützung kompiliert (siehe \ref{sec:Anhang_Eigenkompilation eines Chromium-Browsers mit PQC-Unterstützung}). Dieser ermöglicht die Durchführung von TLS-Handshakes mit hybriden sowie rein \ac{PQC}-basierten Algorithmen und dient als fundamentale Testplattform für die experimentelle Analyse der implementierten Verfahren.

\begin{figure}[H]
    \centering
    \includegraphics[width=\linewidth]{validate_blockchain_webserver_CIPHER_MISMATCH.png}
    \caption{Cipher Mismatch bei Validierung der TLS-1.3-Verbindung des Blockchain-Webservers}
    \begin{flushleft}
    \textit{Anmerkung.} Eigene Darstellung.
    \end{flushleft}
    \label{fig:Cipher-Mismatch-Blockchain-Webserver}
\end{figure}

\subsubsection{Validierung der Zertifikatskette und ML-DSA-Signaturen}

Zur Verifikation der kryptographischen Integrität der implementierten \ac{PKI} wurde die Zertifikatskette der Sidecar-Proxies mittels openssl-Diagnosewerkzeugen analysiert. Ziel war der Nachweis, dass die ausgelieferten X.509-Zertifikate korrekt auf den spezifizierten Post-Quanten-Signaturalgorithmen basieren. Die Inspektion des vom Issuer-Agenten-Proxy bereitgestellten Zertifikats (\autoref{fig:Successful-Validation-Issuer-MLDSA-Cert}) bestätigt, dass der öffentliche Schlüssel des Leaf-Zertifikats (\enquote{pqc reverseproxy issuer agent}) den Algorithmus \enquote{ML-DSA-65} verwendet. Des Weiteren belegt der Signaturalgorithmus \enquote{ML-DSA-87}, dass die Zertifikatskette valide durch die \enquote{Master Thesis PQC Root CA} signiert wurde, was die erfolgreiche Generierung und Einbindung der Dilithium-basierten Zertifikate in den TLS-Handshake beweist.

\begin{figure}[H]
    \centering
    \includegraphics[width=\linewidth]{validate_issuer_mldsa_cert.png}
    \caption{Erfolgreiche Validierung des ML-DSA-Zertifikats des Issuer-Agenten}
    \begin{flushleft}
    \textit{Anmerkung.} Eigene Darstellung.
    \end{flushleft}
    \label{fig:Successful-Validation-Issuer-MLDSA-Cert}
\end{figure}

\subsubsection{Validierung der TLS 1.3 Algorithmen-Aushandlung}

Die erfolgreiche Integration der PQC-Algorithmen in das Transportprotokoll wurde durch einen Verbindungsaufbau mittels \enquote{openssl s\_client} verifiziert. Wie in \autoref{fig:Successful-Validation-Issuer-TLS1.3} dargestellt, konnte erfolgreich eine TLS-1.3-Sitzung etabliert werden. Die Analyse der Handshake-Parameter bestätigt die Verwendung der hybrid-post-quanten Schlüsselaustauschgruppe \enquote{X25519MLKEM768}, welche den klassischen elliptischen Kurvenalgorithmus X25519 mit dem KEM-Verfahren ML-KEM-768 kombiniert. Zudem wird für die Authentifizierung des Peer-Zertifikats der Signaturalgorithmus \enquote{mldsa65} ausgewiesen. Diese Ergebnisse validieren die korrekte Konfiguration der OQS-Provider-Bibliothek innerhalb der Proxy-Komponenten und belegen die praktische Funktionsfähigkeit des hybriden Schlüsselaustauschs im Zusammenspiel mit PQC-Signaturen.

\begin{figure}[H]
    \centering
    \includegraphics[width=\linewidth]{validate_issuer_TLS1.3.png}
    \caption{Erfolgreiche Validierung der TLS-1.3-Verbindung des Issuer-Agenten}
    \begin{flushleft}
    \textit{Anmerkung.} Eigene Darstellung.
    \end{flushleft}
    \label{fig:Successful-Validation-Issuer-TLS1.3}
\end{figure}

\subsubsection{Validierung der Ledger-Initialisierung}

Die operative Funktionsfähigkeit des Hyperledger Indy Netzwerks wurde primär über das Web-Interface des Blockchain-Servers validiert. Wie in \autoref{fig:Successful-Validation-Blockchain-Webserver} dargestellt, zeigen die Statusindikatoren aller vier Validator-Nodes eine aktive Beteiligung am Konsensus-Protokoll (Status Node1-4), womit der Distributed Ledger erfolgreich initialisiert ist. Simultan belegt diese Abbildung die korrekte PQC-Absicherung der Webserver-Komponente. Der Zugriff erfolgt über den eigens kompilierten PQC-Chromium-Browser, dessen Security-Panel explizit eine authentifizierte \ac{TLS}-1.3-Verbindung unter Verwendung der hybriden Schlüsselaustauschgruppe \enquote{X25519MLKEM768} ausweist.

\begin{figure}[H]
    \centering
    \includegraphics[width=\linewidth]{validate_blockchain_webserver.png}
    \caption{Erfolgreiche Validierung der TLS-1.3-Verbindung des Blockchain-Webservers}
    \begin{flushleft}
    \textit{Anmerkung.} Eigene Darstellung.
    \end{flushleft}
    \label{fig:Successful-Validation-Blockchain-Webserver}
\end{figure}

Als zweite notwendige Bedingung für die spätere Anbindung der SSI-Agenten wurde die Verfügbarkeit der Genesis-Datei verifiziert. \autoref{fig:Successful-Validation-Genesis-File-Blockchain-Webserver} dokumentiert den Abruf des Endpunkts \enquote{/genesis} mittels \enquote{curl}. Die erfolgreiche Rückgabe der JSON-formatierten Genesis-Transaktionen bestätigt, dass die für die Anbindung externer Clients erforderlichen Netzwerkinformationen korrekt publiziert werden.

\begin{figure}[H]
    \centering
    \includegraphics[width=\linewidth]{validate_genesis.png}
    \caption{Erfolgreiche Validierung der Genesis-Datei des Blockchain-Webservers}
    \begin{flushleft}
    \textit{Anmerkung.} Eigene Darstellung.
    \end{flushleft}
    \label{fig:Successful-Validation-Genesis-File-Blockchain-Webserver}
\end{figure}

\subsubsection{Validierung der ACA-Py API-Verfügbarkeit}

Die funktionale Erreichbarkeit der SSI-Agenten wurde durch eine systematische Analyse der Initialisierungsphase und der anschließenden API-Verfügbarkeit validiert. Die Logging-Ausgabe des Issuer-Agenten (Listing~\ref{lst:Issuer-Agent-Boot-Logs}) dokumentiert den erfolgreichen Abruf der Genesis-Datei und die vollständige Ledger-Konfiguration. Die Erstellung eines neuen Wallet-Profils mit Askar-Backend und die erfolgreiche Initialisierung der Inbound- und Outbound-Transports demonstrieren die korrekte Konfiguration des ACA-Py-Agents. Die durchgeführten Health-Checks über den Endpunkt \enquote{/status/ready} bestätigen die vollständige Initialisierung und Bereitschaft des Agenten.

Die Visualisierung der Swagger-basierten Admin-Oberfläche (\autoref{fig:Successful-Validation-Issuer-Agent-ACA-Py-Swagger-API}) ergänzt diese technischen Log-Daten durch den Nachweis, dass die Admin-API über den PQC-Reverse-Proxy fehlerfrei erreichbar ist und alle administrativen Endpunkte zur Steuerung der Agenten-Komponente bereitstellt. Die Tatsache, dass die Swagger-Oberfläche unter dem PQC-gesicherten HTTPS-Endpoint vollständig funktionsfähig ist, belegt die korrekte TLS-Terminierung am Proxy sowie die fehlerfreie Weiterleitung der HTTP-Anfragen an den ACA-Py-Container.

\begin{figure}[H]
    \centering
    \includegraphics[width=\linewidth]{validate_Issuer_Agent_Swagger_API.png}
    \caption{Erfolgreiche Validierung der Issuer Agent ACA-Py Swagger Admin API}
    \begin{flushleft}
    \textit{Anmerkung.} Eigene Darstellung.
    \end{flushleft}
    \label{fig:Successful-Validation-Issuer-Agent-ACA-Py-Swagger-API}
\end{figure}

\subsubsection{Validierung der Netzwerkisolation}

Die integrale Sicherheitseigenschaft der Netzwerksegmentierung wurde durch eine Inspektion der Docker-Netzwerktopologie und systematische Erreichbarkeitstests validiert. \autoref{fig:Darstellung-Network-Isolation} belegt diese Topologie anhand des \enquote{docker network inspect}-Outputs: Der Issuer-Agent befindet sich exklusiv im Netzwerksegment \enquote{hope\_hope-issuer}, während der Holder-Agent im Segment \enquote{hope\_hope-holder} isoliert ist. In jedem dieser Segmente fungiert der zugehörige PQC-Sidecar-Proxy als einziger Ingress-Punkt.

\begin{figure}[H]
    \centering
    \includegraphics[width=\linewidth]{validate_network_isolation.png}
    \caption{Darstellung der Netzwerkisolation innerhalb der Gesamtarchitektur}
    \begin{flushleft}
    \textit{Anmerkung.} Eigene Darstellung.
    \end{flushleft}
    \label{fig:Darstellung-Network-Isolation}
\end{figure}

\autoref{fig:Successful-Validation-Network-Isolation-Through-Tests} demonstriert die Wirksamkeit dieser Isolation auf zwei Ebenen. Erstens zeigt die Prozessliste (\enquote{docker ps}), dass lediglich die Sidecar-Proxies externe Ports (8020, 8030, 8040) an das Host-System binden, während die Ports der ACA-Py-Container nicht exponiert sind. Zweitens beweisen die Inter-Container-Verbindungstests die logische Trennung. Ein direkter Zugriffsversuch aus dem \enquote{issuer-agent}-Container auf den \enquote{holder-agent} schlägt mit einem DNS-Auflösungsfehler (\enquote{Could not resolve host}) fehl, da keine Routing-Route zwischen den isolierten Netzwerkbrücken existiert. Im Gegensatz dazu ist der lokale Zugriff des \enquote{pqc-sidecarproxy-holder} auf seinen zugehörigen Agenten erfolgreich möglich.

\begin{figure}[H]
    \centering
    \includegraphics[width=\linewidth]{validate_network_isolation_through_tests.png}
    \caption{Erfolgreiche Validierung der Netzwerkisolation innerhalb der Gesamtarchitektur}
    \begin{flushleft}
    \textit{Anmerkung.} Eigene Darstellung.
    \end{flushleft}
    \label{fig:Successful-Validation-Network-Isolation-Through-Tests}
\end{figure}

\subsection{Erkenntnisse und Anpassungsbedarfe}
\label{subsec:findings_adaptation_needs}

Die erste Iteration bildet das fundamentale technologische Fundament der Forschungsarbeit. Die formative Evaluation (Kapitel~\ref{sec:formative_evaluation_iteration1}) validierte die operative Integrität der entwickelten Architektur. Die verteilten Micro-Services, der Hyperledger Indy Ledger und die PQC-Sidecar-Proxies interagieren funktional korrekt. Diese Initialphase generierte jedoch spezifische Erkenntnisse, die eine gezielte Weiterentwicklung in der zweiten Iteration motivieren. Diese werden nachfolgend in Bezug auf die Designziele analysiert.

\subsubsection{Abgleich mit den Designzielen und kritische Erkenntnisse}
\label{subsubsec:design_goals_critique}

Das primäre Designziel, die Absicherung der Transportebene in einem SSI-Ökosystem mittels Post-Quanten-Kryptographie, wurde vollständig erreicht. Die erfolgreiche Validierung des Sidecar-Musters belegt die Machbarkeit einer transparenten PQC-Migration für Legacy-Systeme (ACA-Py, Indy Node) ohne Eingriffe in deren Kerncode. Die implementierte Micro-Segmentation erfüllt zudem die architektonischen Anforderungen an logische Netzsegmentierung in \ac{KRITIS}-Umgebungen (\autoref{tab:compliance_requirements}).

Durch die strikte Entkopplung der kryptographischen Terminierung von der Business-Logik durch die Sidecar-Proxy-Architektur konnte eine PQC-Integration realisiert werden, die die Kernprozesse der Identitätsverwaltung funktional nicht beeinträchtigt. Diese architektonische Entscheidung ermöglicht es, sicherheitskritische Updates an der Krypto-Komponente vorzunehmen, ohne die Integrität der komplexen SSI-Logik zu gefährden. Die empirische Validierung der Zertifikatsketten und TLS-1.3-Handshake-Protokolle (\autoref{sec:formative_evaluation_iteration1}) bestätigt die technische Reife dieser Architekturentscheidung.

Bezüglich der Algorithmenauswahl und Sicherheitsbewertung operationalisiert die erste Iteration die in Tabelle~\ref{tab:compliance_requirements} definierten BSI-Vorgaben für Post-Quantum-Kryptografie durch die Implementierung der NIST-standardisierten Algorithmen ML-DSA-65 und ML-KEM-768, welche explizit die NIST Security Strength Categories 3/5 erfüllen \parencite[Kap.~2.4, 5.3.4.2]{bsi_BSITR021021KryptographischeVerfahrenEmpfehlungenundSchluessellaengenVersion202501_2025}. Die formative Evaluation bestätigte die technische Machbarkeit dieser Algorithmen in der Sidecar-Proxy-Infrastruktur durch die erfolgreiche Validierung hybrider Zertifikatsketten und TLS-1.3-Handshake-Protokolle mit der Schlüsselaustauschgruppe \enquote{X25519MLKEM768}.

Bezüglich der kryptografischen Agilität zeigt die erste Iteration, dass das Designziel einer architektonischen Vorbereitung auf Algorithmenaustauschbarkeit erreicht wurde. Die containerbasierte Sidecar-Architektur ermglicht es, kryptografische Bibliotheken durch Rolling Updates der Proxy-Images auszutauschen, ohne ACA-Py oder die übrige Systemlogik anzupassen, was direkt die von \textcite[S. 102]{mehrez_CryptoAgilityProperties_2018} geforderte \enquote{Extensibility} kryptoagiler Systeme adressiert. Gleichzeitig nutzt die TLS-1.3-Integration eine konfigurationsbasierte Fallback-Kette (\enquote{DEFAULT\_GROUPS:X25519:ML-KEM-768:mlkem768x25519:mlkem1024}), sodass hybride und klassische Verfahren orthogonal ausgehandelt und bei Inkompatibilitäten automatisch gewechselt werden können \parencite[S. 26]{rescorla_TransportLayerSecurityTLSProtocolVersion13_2018}. Diese Protokoll- und Infrastrukturmechanismen realisieren damit zentrale Kryptoagilitäts-Eigenschaften wie \enquote{Fungibility} und \enquote{Updateability} \parencite[S. 102--103]{mehrez_CryptoAgilityProperties_2018} und entsprechen dem Erfordernis, kryptografische Komponenten ohne grundlegende Systemumgestaltung migrieren zu können \parencite[S. 670]{kreutzer_Kryptoagilitaet_2024a}.

Eine zentrale Limitation identifizierte jedoch die Analyse der Sicherheitsmodelle. Während die Transportebene durch die Sidecar-Proxies vollständig quantensicher abgesichert ist (\gls{Data-In-Motion}), verbleiben Verifiable Credentials und DID-Dokumente (\gls{Data-At-Rest}) mittels klassischer Kryptografie verschlüsselt. Diese Diskrepanz zwischen Transportschutz (PQC-gesichert) und Datenpersistierung (klassische Kryptografie) widerspricht dem mehrschichtigen Sicherheitsansatz \enquote{Defense in Depth} von \textcite[S. 242--243]{alsaqour_DefenseDepthMultilayersecurity_2021}, bei dem konsistente, sich gegenseitig verstärkende Kontrollen auf mehreren Ebenen implementiert werden, um Ressourcen und Assets umfassend zu schützen. Diese Erkenntnisse folgern eine Erweiterung des Sicherheitsmodells von der Transportebene auf die Applikationsschicht in der zweiten Iteration.

\subsubsection{Design-Refinements und Operationalisierung der zweiten Iteration}
\label{subsubsec:design_refinements_iteration2}

Die systematische Analyse der Evaluationsergebnisse führt zu zwei konvergenten Design-Refinements, die die identifizierten Sicherheitslücken adressieren und die zweite Iteration strukturieren.

\textbf{Refinement 1: Applikationsebenen-Integration der Post-Quanten-Kryptografie.} Dieses Refinement adressiert die identifizierte Sicherheitslücke durch technische Erweiterung der PQC-Integration von der Transportebene auf die Applikationsebene. Die direkte Einbindung der \emph{liboqs}-Bibliothek in die ACA-Py-Agenten ermöglicht ML-DSA-65-Signaturen für Verifiable Credentials und DIDComm-Nachrichten, wodurch die kritische Diskrepanz zwischen quantensicherer Transportverschlüsselung und ungeschützter Datenpersistierung geschlossen wird. Die Implementierung als modulares Plugin-System realisiert das Open-Closed-Prinzip \parencite[S. 99]{martin_AgileSoftwareDevelopmentprinciplespatternspractices_2003} und ermöglicht eine nicht-invasive Erweiterung der ACA-Py-Kernarchitektur, ohne deren Codebasis zu modifizieren.

\textbf{Refinement 2: Hybride Sicherheitsarchitektur mit redundanter Tiefenstaffelung.} Dieses Refinement etabliert ein übergeordnetes Defense-in-Depth-Modell durch beibehaltene Sidecar-Proxies als erste Verteidigungslinie (Transport Layer Security) bei gleichzeitiger Quantensicherung der Datenobjekte auf Anwendungsebene (Application Layer Security). Diese redundante Absicherung schafft eine tiefengestaffelte Sicherheitsarchitektur, bei der ein Bruch einzelner Schichten, nicht automatisch zum Kollaps der Gesamtsicherheit führt. Das Refinement implementiert damit die von \textcite[S. 242--243]{alsaqour_DefenseDepthMultilayersecurity_2021} geforderte Prinzipienkonsistenz über alle Architekturebenen hinweg und gewährleistet die Resilienz des Systems gegen hybride Angriffszenarien.



\newpage
\section{Zweite Iteration der Artefaktentwicklung}
\label{Zweite Iteration der Artefaktentwicklung}

\subsection{Designziele dieser Iteration} \label{sec:Designziele_Iteration_2}

Die zweite Iteration der Artefaktentwicklung baut auf der in Iteration 1 erfolgreich validierten Basisarchitektur auf und korrespondiert erneut mit der DSRM-Phase 2 \textit{Objectives} nach \textcite[S. 54]{peffers_DesignScienceResearchmethodologyinformationsystemsresearch_2007}. Der Fokus dieser Iteration liegt auf der Erweiterung des Prototyps um eine tiefgreifende PQC-Integration auf der Anwendungsebene (Application Layer). Im Kontext des Drei-Zyklen-Modells nach \textcite[S. 88]{hevner_ThreeCycleViewDesignScienceResearch_2007} wird der Design Cycle intensiviert, um die kryptografische Sicherheit von der reinen Transportsicherung (TLS) auf die tatsächlichen Nutzdaten (Verifiable Credentials und DID-Dokumente) auszuweiten und somit eine Ende-zu-Ende-Sicherheit zu gewährleisten.

Die Designziele dieser Iteration leiten sich konsistent aus den in Kapitel~\ref{sec:Zielsetzung und Forschungsfragen} definierten Forschungsfragen ab, wobei eine inhaltliche Vertiefung der technischen Anforderungen erfolgt.

Bezüglich FF1 (Systemarchitektur \& Compliance) wird das Ziel verfolgt, die SSI-Kernprozesse so zu modifizieren, dass sie quantenresistente Signaturen und Schlüsselformate nativ unterstützen. Das Design muss sicherstellen, dass die Unveränderlichkeit und Authentizität von Identitätsnachweisen unabhängig vom Transportkanal auch langfristig gegenüber Quantencomputer-Angriffen gewährleistet bleibt, was eine zentrale Anforderung für den Einsatz in KRITIS-Umgebungen darstellt \parencite[.S 25]{bsi_BSITR021021KryptographischeVerfahrenEmpfehlungenundSchluessellaengenVersion202501_2025}.

Hinsichtlich FF2 (Algorithmenauswahl \& Sicherheitsbewertung) wird das Ziel verfolgt, die praktische Machbarkeit von NIST-standardisierten Post-Quantum-Algorithmen in den Kernkomponenten des SSI-Systems nachzuweisen. Der Fokus liegt auf der Integration quantenresistenter Signatur- und Verschlüsselungsverfahren auf der Applikationsebene, um die digitale Authentizität und Integrität von Identitätsnachweisen langfristig gegen Quantencomputer-Angriffe zu schützen.

Für FF3 (Kryptografische Agilität) zielt diese Iteration auf die Implementierung von Agilitätsmechanismen direkt in den Datenstrukturen ab. Das System soll so gestaltet werden, dass es hybride Szenarien unterstützt und eine Koexistenz sowie den nahtlosen Wechsel zwischen klassischen (z.\,B. Ed25519) und post-quanten Kryptografieverfahren innerhalb der DID-Methoden und Credential-Definitionen ermöglicht, ohne die Interoperabilität grundlegend zu gefährden.

\subsection{Architekturentwurf}

\subsubsection{Gesamtarchitektur}

Die Gesamtarchitektur der zweiten Iteration erweitert die in Kapitel~\ref{sec:Gesamtarchitektur_Iteration1} etablierte dreischichtige Containerarchitektur um eine zusätzliche Kryptoebene auf der Anwendungsschicht. Während die in der ersten Iteration implementierte Sidecar-Proxy-Architektur mit TLS~1.3 und hybrider Schlüsseleinigung (X25519 + ML-KEM-768) vollständig beibehalten wird und weiterhin die Transportverschlüsselung zwischen den Komponenten gewährleistet, wird in dieser Iteration die kryptografische Absicherung auf die SSI-Agenten-Schicht ausgeweitet.

\autoref{fig:Gesamtarchitektur_Iteration2} visualisiert diese Erweiterung durch die Integration von \enquote{PQC-PLUGIN}-Modulen in die drei ACA-Py-Instanzen (Issuer, Holder, Verifier). Diese Plugin-Module ermöglichen die Verwendung quantenresistenter Signaturalgorithmen (ML-DSA-65) innerhalb von Verifiable Credentials und Decentralized Identifiers sowie die Verwendung quantenresistenter Schlüsselkapselungsverfahren (ML-KEM-768) für die DIDComm-Messaging-Verschlüsselung. Durch diese duale Architektur mit Sidecar-Proxies für die Transportebene und Plugin-Module für die Applikationsebene wird eine durchgängige Ende-zu-Ende-Quantenresistenz realisiert, die sowohl \gls{Data-In-Motion} als auch \gls{Data-At-Rest} schützt. Die DLT-Infrastruktur (vier Hyperledger-Indy-Validator-Nodes, Ledger-Browser, Webserver), die Revocation Registry (Tails-Server) sowie die zugrundeliegende Docker-Netzwerktopologie bleiben strukturell unverändert und gewährleisten die Kontinuität der in Iteration~1 validierten Infrastrukturkomponenten.

\begin{figure}[H]
    \centering
    \includegraphics[width=\linewidth]{Gesamtarchitektur_Iteration2}
    \caption{Gesamtarchitekturentwurf der zweiten Iteration}
    \begin{flushleft}
    \textit{Anmerkung.} Eigene Darstellung.
    \end{flushleft}
    \label{fig:Gesamtarchitektur_Iteration2}
\end{figure}

\subsubsection{ACA-Py Applikationsarchitektur}

Während Kapitel~\ref{sec:ACA-Py Applikationsarchitektur_Iteration1} die klassische Applikationsarchitektur als geschichtete, unveränderliche Referenzimplementierung vorstellt, demonstriert die zweite Iteration die transparente Applikationsschicht-Integration von PQC durch ein modulares Plugin-System, das bestehendes Kerncode-Design respektiert und durch gezieltes Patching erweitert. ACA-Py-Plugins ermöglichen hierbei eine standardisierte Erweiterbarkeit, ohne die ACA-Py-Codebasis zu überlasten \parencite{_ACAPyPluginsACAPyDocs_}. 

\autoref{fig:ACAPY_Application_Architecture_Iteration2_PQC} visualisiert diese erweiterte Architektur und verdeutlicht die Integration des PQC-Plugins als zentrale Interceptions- und Delegationsschicht. Die roten Markierungen heben die Unterschiede zur klassischen ACA-Py Applikationsarchitektur (\autoref{fig:ACAPY_Application_Architecture_Iteration_1}) hervor. Das Plugin ist seitlich an alle vier klassischen Schichten (Protocol Handlers, Wallet Interface, Transport Layer, externe Business Logic) angebunden, woraus sich bidirektionale Pfeile ergeben. Diese Bidirektionalität symbolisiert die transparente Interception. Wenn eine klassische Schicht eine Operation initiiert, wird diese zunächst vom Plugin abgefangen. Das Plugin führt eine PQC-äquivalente Operation durch und gibt das Ergebnis an die ursprüngliche Schicht zurück. Diese Interceptions-Delegation ermöglicht es dem Plugin, als Zwischenschicht zu fungieren, ohne die übergeordnete Geschäftslogik zu modifizieren. Die Erweiterungen manifestieren sich somit nicht nur auf vier Architekturebenen, sondern auch in der Rolle des Plugins als verteilte, querschichtliche Integrationsfläche, die nachfolgend charakterisiert werden.

Die Protocol Handlers-Schicht wird durch eine PQC-Interception-Delegation erweitert. Während die klassische Architektur DID Exchange, Credential Issuance und Presentation Proof ausschließlich mit Signaturen klassischer Kryptografie durchführt \parencite{openwallet-foundation_AcapyAcapy_agentWalletkey_typepymainopenwalletfoundationacapyGitHub_}, delegiert die erweiterte Variante diese Operationen an das Plugin, das ML-DSA-65 als Signaturverfahren verwendet.

Die Wallet Interface-Schicht wird nicht ersetzt, sondern erweitert. Das klassische Aries-Askar-Backend bleibt vollständig funktionsfähig und verwaltet weiterhin klassische Kryptografie \parencite{openwallet-foundation_AcapyAcapy_agentWalletkey_typepymainopenwalletfoundationacapyGitHub_}. Parallel dazu registriert das Plugin zusätzliche Operationsmethoden für die PQC-Schlüssel ML-DSA-65-Signaturen, ML-KEM-768-Key-Encapsulation und angepasste Speicherfunktionen, die die größeren Schlüssellängen in der bestehenden SQLite-Struktur persistieren. Diese Erweiterung ist nicht-invasiv und erhält die ChaCha20-Poly1305-Verschlüsselung der Wallet-Datenbank.

Die Key-Type-Registry und DIDComm-Verschlüsselung werden dynamisch erweitert. Das Plugin registriert ML-DSA-65 und ML-KEM-768 als vollwertige Schlüsseltypen neben den klassischen Varianten ed25519 und x25519 \parencite{openwallet-foundation_AcapyAcapy_agentWalletkey_typepymainopenwalletfoundationacapyGitHub_}. Die DIDComm-Verschlüsselung wird orthogonal erweitert, sodass PQC-Nachrichten ML-KEM-768-Key-Encapsulation nutzen.

Die Integration-Patching-Schicht realisiert die transparente Einbettung der PQC-Funktionalität durch gezieltes \gls{Monkey-Patching}, das kritische Funktionen der bestehenden Schichten zur Laufzeit überschreibt, ohne den ACA-Py-Quellcode zu modifizieren. Dies ermöglicht die Einbindung von PQC-Operationen (Schlüsselgenerierung, Signatur, Key-Encapsulation) in bestehende Workflows und erlaubt hybride Betriebsweisen, bei denen Agenten je nach Plugin-Ladezustand klassische oder quantenresistente Schlüssel erzeugen.

\begin{figure}[H]
    \centering
    \includegraphics[width=\linewidth]{ACA-Py Applikationsarchitektur mit PQC-Integration}
    \caption{ACA-Py High Level Applikationsarchitektur mit PQC-Integration}
    \begin{flushleft}
    \textit{Anmerkung.} Eigene Darstellung auf Basis von \autoref{fig:ACAPY_Application_Architecture_Iteration 1}.
    \end{flushleft}
    \label{fig:ACAPY_Application_Architecture_Iteration2_PQC}
\end{figure}

\subsection{Implementierung}

Die Implementierung der PQC-Unterstützung auf Application-Ebene folgt einem zweistufigen Ansatz. In der ersten Stufe wird das pqc\_didpeer4\_fm-Plugin entwickelt, das sich in den ACA-Py Plugin-Mechanismus nach \textcite{_ACAPyPluginsACAPyDocs_,_ACAPyPluginsACAPyPlugins_,openwallet-foundation_OpenwalletfoundationAcapyplugins_2025} einfügt. Das Plugin ist als Python-Paket strukturiert und definiert einen setup-Entrypoint, der beim Agent-Start aufgerufen wird. Über den PluginContext erhält das Plugin unmittelbaren Zugriff auf zentrale Agent-Komponenten wie die Wallet, das DIDComm-System und die Protocol Registry. Diese Architektur ermöglicht es Plugins, neue Protokolle zu registrieren oder bestehende Funktionen durch Monkey-Patching zu erweitern, ohne den ACA-Py Kern zu modifizieren. Die zweite Stufe integriert das entwickelte Plugin in die containerisierte Deployment-Infrastruktur. Dabei wird das Plugin als Abhängigkeit in dem Dockerfile definiert, sodass es während des Container-Builds installiert wird. Anschließend erfolgt die Konfiguration über docker-compose, indem der ACA-Py Service mit den erforderlichen Umgebungsvariablen und Plugin-Parametern initialisiert wird.

Der Aufbau des PQC-Plugins gliedert sich in drei funktionale Schichten, in die die in \autoref{fig:pqc_didpeer4_fm_directory_structure} dargestellten 15 Module eingeteilt sind. Diese Architektur garantiert vollständige PQC-Funktionalität ohne Änderungen am ACA-Py-Quellcode.

Die Kryptografische Abstraktionsschicht bildet die Grundlage und abstrahiert die Komplexität der liboqs-C-Bibliothek durch ein Python-Wrapper-Modul (liboqs\_wrapper.py). Sie stellt einheitliche Operationen für ML-DSA-65 (digitale Signaturen) und ML-KEM-768 (Schlüsselencapsulation) bereit und garantiert, dass kryptografisches Schlüsselmaterial als Byte-Sequenzen serialisiert wird, was eine Voraussetzung für Wallet-Persistierung und Multi-Codec-Kodierungen darstellt.

Die DID-Verarbeitungsschicht orchestriert die vollständige Lebenszyklusbearbeitung von PQC-fähigen did:peer:4-Identifikatoren. Dazu zählen, die Erzeugung und Auflösung (pqc\_peer4\_creator.py, pqc\_peer4\_resolver.py), W3C-konforme Multicodecs (pqc\_multicodec.py, pqc\_multikey.py) und standardisierte DIDComm-Nachrichtenverschlüsselung (pqc\_didcomm\_v1.py). Diese Schicht verbindet kryptografische Primitive mit Identity-Protokollen und ermöglicht hybrid-sichere Kommunikation zwischen PQC- und klassischen Agenten.

Die Integration-Patching-Schicht implementiert die transparente Einbettung in ACA-Py durch gezieltes \gls{Monkey-Patching} und Registry-Erweiterungen. Neun spezialisierte Module patchen Wallet-Operationen (askar\_pqc\_patch.py, wallet\_patch.py), Connection-Management (base\_manager\_patch.py, connection\_target\_patch.py), Schlüsseltyp-Infrastruktur (key\_types.py, key\_type\_patches.py), Validierungslogik (validator\_patch.py), Multicodec-Registries (multicodec\_patch.py) und koordinieren deren Installation (monkey\_patches.py).

\begin{figure}[H]
  \flushleft
  \caption{Verzeichnisstruktur des Projekts}
  \label{fig:pqc_didpeer4_fm_directory_structure}
\dirtree{%
.1 pqc\_didpeer4\_fm/.
.2 pqc\_didpeer4\_fm/.
.3 v1\_0/.
.4 askar\_pqc\_patch.py.
.4 base\_manager\_patch.py.
.4 connection\_target\_patch.py.
.4 key\_type\_patches.py.
.4 key\_types.py.
.4 liboqs\_wrapper.py.
.4 monkey\_patches.py.
.4 multicodec\_patch.py.
.4 pqc\_didcomm\_v1.py.
.4 pqc\_multicodec.py.
.4 pqc\_multikey.py.
.4 pqc\_peer4\_creator.py.
.4 pqc\_peer4\_resolver.py.
.4 validator\_patch.py.
.4 wallet\_patch.py.
.3 \_\_init\_\_.py.
.2 README.md.
.2 setup.py.
}
\begin{flushleft}
    \textit{Anmerkung.} Eigene Darstellung.
\end{flushleft}
\end{figure}

\subsubsection{Pluginentwicklung: Kryptografie-Abstraktionsschicht}

Die Kryptografie-Abstraktionsschicht bildet die unterste Ebene der PQC-Integration und wird durch das Modul liboqs\_wrapper.py (\autoref{fig:pqc_didpeer4_fm_directory_structure_Kryptografie-Abstraktionsschicht}) realisiert. Dieses Modul kapselt die nativen Operationen der C-basierten liboqs-Bibliothek und stellt eine Python-API für die NIST-standardisierten PQC-Algorithmen ML-DSA-65 und ML-KEM-768 bereit.

\begin{figure}[H]
  \flushleft
  \caption{Verzeichnisstruktur des Projekts}
  \label{fig:pqc_didpeer4_fm_directory_structure_Kryptografie-Abstraktionsschicht}
\dirtree{%
.1 pqc\_didpeer4\_fm/.
.2 pqc\_didpeer4\_fm/.
.3 v1\_0/.
.4 askar\_pqc\_patch.py.
.4 base\_manager\_patch.py.
.4 connection\_target\_patch.py.
.4 key\_type\_patches.py.
.4 key\_types.py.
.4 \textbf{liboqs\_wrapper.py}.
.4 monkey\_patches.py.
.4 multicodec\_patch.py.
.4 pqc\_didcomm\_v1.py.
.4 pqc\_multicodec.py.
.4 pqc\_multikey.py.
.4 pqc\_peer4\_creator.py.
.4 pqc\_peer4\_resolver.py.
.4 validator\_patch.py.
.4 wallet\_patch.py.
.3 \_\_init\_\_.py.
.2 README.md.
.2 setup.py.
}
\begin{flushleft}
    \textit{Anmerkung.} Eigene Darstellung.
\end{flushleft}
\end{figure}

Die Implementierung (Listing~\ref{lst:liboqs_wrapper.py}) definiert eine LibOQSWrapper-Klasse mit sechs Kernmethoden. Die ersten beiden Methoden generate\_ml\_dsa\_65\_keypair() und generate\_ml\_kem\_768\_keypair() erzeugen kryptografische Schlüsselpaare, sign\_ml\_dsa\_65() und verify\_ml\_dsa\_65() implementieren digitale Signaturen, während encapsulate\_ml\_kem\_768() und decapsulate\_ml\_kem\_768() die Key Encapsulation für sichere Schlüsselvereinbarung realisieren.

Die Wrapper-Architektur abstrahiert die komplexen \ac{FFI}-Aufrufe an die liboqs-C-Bibliothek und stellt sicher, dass Schlüsselmaterial ausschließlich als Byte-Arrays serialisiert wird, was eine Voraussetzung für die Persistierung in der Aries-Askar-Wallet \parencite{_EntryAries_askarEntryRust_} und die Kodierung in Multicodec-Formaten \parencite{multiformats_MultiformatsMulticodec_2025} darstellt. Das Singleton-Pattern (get\_liboqs()) gewährleistet eine einzige globale Instanz zur Vermeidung redundanter Initialisierungen. Diese Abstraktionsschicht ermöglicht es den höheren Modulen (DID-Generierung, DIDComm-Verschlüsselung), PQC-Operationen durchzuführen, ohne direkte Abhängigkeiten zur liboqs-C-API zu haben.

\subsubsection{Pluginentwicklung: DID-Verarbeitungsschicht}

Die DID-Verarbeitungsschicht (siehe Abbildung~\ref{fig:pqc_didpeer4_fm_directory_structure_DID-Verarbeitungsschicht}) orchestriert die Erzeugung, Auflösung und Kodierung von PQC-fähigen did:peer:4-Identifikatoren sowie die DIDComm-Nachrichtenverschlüsselung. Diese Schicht umfasst sechs funktional gekoppelte Module, die gemeinsam eine standardkonforme Integration von Post-Quantum-Kryptografie in das did:peer:4-Ökosystem realisieren.

\begin{figure}[H]
  \flushleft
  \caption{Verzeichnisstruktur des Projekts}
  \label{fig:pqc_didpeer4_fm_directory_structure_DID-Verarbeitungsschicht}
\dirtree{%
.1 pqc\_didpeer4\_fm/.
.2 pqc\_didpeer4\_fm/.
.3 v1\_0/.
.4 askar\_pqc\_patch.py.
.4 base\_manager\_patch.py.
.4 connection\_target\_patch.py.
.4 key\_type\_patches.py.
.4 key\_types.py.
.4 liboqs\_wrapper.py.
.4 monkey\_patches.py.
.4 multicodec\_patch.py.
.4 \textbf{pqc\_didcomm\_v1.py}.
.4 \textbf{pqc\_multicodec.py}.
.4 \textbf{pqc\_multikey.py}.
.4 \textbf{pqc\_peer4\_creator.py}.
.4 \textbf{pqc\_peer4\_resolver.py}.
.4 validator\_patch.py.
.4 wallet\_patch.py.
.3 \_\_init\_\_.py.
.2 README.md.
.2 setup.py.
}
\begin{flushleft}
    \textit{Anmerkung.} Eigene Darstellung.
\end{flushleft}
\end{figure}

Das Modul pqc\_peer4\_creator.py (Listing~\ref{lst:pqc_peer4_creator.py}) implementiert die Funktion create\_pqc\_peer4\_did(), die aus zwei PQC-Schlüsselpaaren (ML-DSA-65 für authentication/assertionMethod, ML-KEM-768 für keyAgreement) einen did:peer:4-Long-Form-Identifier generiert. Die Schlüssel werden über die Wallet-API erzeugt, in Multikey-Format transformiert und als KeySpec-Objekte in einem did:peer:4-Input-Dokument strukturiert, wobei die Reihenfolge der Schlüssel deren Fragment-IDs determiniert (\#key-0 für Signaturen, \#key-1 für Verschlüsselung in recipientKeys). Das Gegenstück pqc\_peer4\_resolver.py (Listing~\ref{lst:pqc_peer4_resolver.py}) registriert einen DID-Resolver für die peer-Methode, der did:peer:4-Long-Form-DIDs in DID-Dokumente auflöst und dabei PQC-Multicodec-Präfixe korrekt dekodiert.

Die Multiformat-Kodierung nach \parencite[Kap. 5.6]{w3c_ControlledIdentifiersV11_2025} wird durch drei Module realisiert. Das Modul pqc\_multicodec.py (Listing~\ref{lst:pqc_multicodec.py}) definiert eine Multicodec-Registry mit provisorischen Präfixen (ML-DSA-65: 0xd065, ML-KEM-768: 0xe018) und stellt Wrapper-Funktionen (wrap\_pqc(), unwrap\_pqc()) für Präfix-Operationen bereit. Das Modul pqc\_multikey.py (Listing~\ref{lst:pqc_multikey.py}) transformiert Schlüsselinformationen in das Multikey-Format durch Verkettung von Multicodec-Präfix und Schlüsselmaterial sowie Base58-Kodierung mit Multibase-Präfix \enquote{z} (Base58btc), wodurch Multikeys wie \enquote{z6MNxxx...} (ML-DSA-65) oder \enquote{z6MK768xxx...} (ML-KEM-768) entstehen. Das Modul pqc\_didcomm\_v1.py (Listing~\ref{lst:pqc_didcomm_v1.py}) erweitert die DIDComm-v1-Envelope-Verarbeitung um PQC-Unterstützung. Hier detektieren die beiden Methoden pack\_message\_pqc() und unpack\_message\_pqc() automatisch anhand der unterschiedlichen Schlüssellängen ob PQC- oder klassische Kryptografie verwendet werden muss, und generieren JWE-Envelopes mit angepassten Algorithmus-Headern. Die Content Encryption erfolgt weiterhin mit XChaCha20-Poly1305,  während der Content Encryption Key mittels ML-KEM-768 Key Encapsulation für jeden Empfänger verschlüsselt wird. Diese Schicht ermöglicht eine hybride Betriebsweise, bei der PQC- und klassische Agenten koexistieren können, solange jeweils homogene Verschlüsselungsmodi verwendet werden.

\subsubsection{Pluginentwicklung: Integration-Patching-Schicht}

Die Integration-Patching-Schicht (Abbildung~\ref{fig:pqc_didpeer4_fm_directory_structure_Integration-Patching-Schicht}) implementiert die transparente Einbettung der PQC-Funktionalität in den ACA-Py-Kern durch gezieltes Monkey-Patching kritischer Funktionen und Erweiterung globaler Registries. Diese Schicht umfasst neun Module, die gemeinsam eine vollständig transparente PQC-Integration ohne Modifikation des ACA-Py-Quellcodes ermöglichen, sodass existierende Workflows und API-Aufrufe unverändert funktionsfähig bleiben.

\begin{figure}[H]
  \flushleft
  \caption{Verzeichnisstruktur des Projekts}
  \label{fig:pqc_didpeer4_fm_directory_structure_Integration-Patching-Schicht}
\dirtree{%
.1 pqc\_didpeer4\_fm/.
.2 pqc\_didpeer4\_fm/.
.3 v1\_0/.
.4 \textbf{askar\_pqc\_patch.py}.
.4 \textbf{base\_manager\_patch.py}.
.4 \textbf{connection\_target\_patch.py}.
.4 \textbf{key\_type\_patches.py}.
.4 \textbf{key\_types.py}.
.4 liboqs\_wrapper.py.
.4 \textbf{monkey\_patches.py}.
.4 \textbf{multicodec\_patch.py}.
.4 pqc\_didcomm\_v1.py.
.4 pqc\_multicodec.py.
.4 pqc\_multikey.py.
.4 pqc\_peer4\_creator.py.
.4 pqc\_peer4\_resolver.py.
.4 \textbf{validator\_patch.py}.
.4 \textbf{wallet\_patch.py}.
.3 \_\_init\_\_.py.
.2 README.md.
.2 setup.py.
}
\begin{flushleft}
    \textit{Anmerkung.} Eigene Darstellung.
\end{flushleft}
\end{figure}

Das zentrale Orchestrierungsmodul monkey\_patches.py (Listing~\ref{lst:monkey_patches.py}) koordiniert die Installation aller Patches durch die Funktion apply\_all\_patches(), die beim Plugin-Setup aufgerufen wird. Dieses Modul überschreibt Methoden der Klasse BaseConnectionManager (z.\,B. create\_did\_peer\_4(), \_extract\_key\_material\_in\_base58\_format(), long\_did\_peer\_4\_to\_short()) und delegiert deren Implementierung an spezialisierte Patch-Module, wobei die ursprünglichen Methoden als Fallback-Referenzen gespeichert werden. Das Modul base\_manager\_patch.py (Listing~\ref{lst:base_manager_patch.py}) stellt die PQC-Implementierungen dieser BaseConnectionManager-Methoden bereit: create\_did\_peer\_4\_pqc\_complete() generiert did:peer:4-DIDs mit ML-DSA-65- und ML-KEM-768-Schlüsseln anstelle klassischer ED25519/X25519-Schlüssel, \_extract\_key\_material\_in\_base58\_format\_pqc() extrahiert PQC-Schlüsselmaterial aus DID-Dokumenten unter Berücksichtigung der größeren Schlüssellängen, und record\_keys\_for\_resolvable\_did\_pqc() persistiert beide PQC-Schlüssel (Signatur- und Verschlüsselungsschlüssel) in der Wallet-Datenbank.

Die Wallet-Integration erfolgt durch drei Module: askar\_pqc\_patch.py (Listing~\ref{lst:askar_pqc_patch.py}) patcht die Aries-Askar-Funktionen create\_keypair() zur Unterstützung von PQC-Schlüsselgenerierung mittels liboqs sowie pack\_message() und unpack\_message() zur Integration der PQC-DIDComm-v1-Implementierung aus pqc\_didcomm\_v1.py. wallet\_patch.py (Listing~\ref{lst:wallet_patch.py}) erweitert die Methode get\_local\_did\_for\_verkey() der AskarWallet-Klasse, um ML-KEM-768-Verkeys korrekt in der Datenbank zu lokalisieren. Dies stellt eine kritische Anpassung dar, da klassische Verkey-Lookups, wie in der Originalmethode get\_local\_did\_for\_verkey() \parencite{openwallet-foundation_AcapyAcapy_agentWalletaskarpymainopenwalletfoundationacapyGitHub_} demonstriert, nur für 32-Byte-ED25519-Schlüssel ausgelegt sind. connection\_target\_patch.py (Listing~\ref{lst:connection_target_patch.py}) passt das Marshmallow-Schema der ConnectionTarget-Klasse an, indem die Validierungsregeln für recipient\_keys PQC-konforme Schlüssellängen akzeptieren.

Die Erweiterung der Schlüsseltyp-Infrastruktur erfolgt durch zwei Module: key\_types.py (Listing~\ref{lst:key_types.py}) definiert neue KeyType-Konstanten (ML\_DSA\_65, ML\_KEM\_768) mit Metadaten wie NIST-FIPS-Referenzen, Schlüssellängen und Multicodec-Präfixen. key\_type\_patches.py (Listing~\ref{lst:key_type_patches.py}) registriert diese KeyTypes in der globalen ACA-Py-Registry durch register\_pqc\_key\_types(), erweitert die Admin-API-Schemata (patch\_api\_key\_type\_schemas()) zur Akzeptanz von PQC-KeyType-Strings in JSON-Requests, und patcht Algorithmus-Mappings (patch\_alg\_mappings\_for\_pqc()) für JWS/JWE-Header-Generierung. multicodec\_patch.py (Listing~\ref{lst:multicodec_patch.py}) erweitert die globale SupportedCodecs-Enumeration durch dynamisches Hinzufügen von ML-DSA-65- und ML-KEM-768-Multicodec-Einträgen, sodass Multicodec-Dekodierungsfunktionen aus multiformats-Bibliotheken PQC-Präfixe verarbeiten können.

Das Modul validator\_patch.py (Listing~\ref{lst:validator_patch.py}) patcht die JWSHeaderKid-Validierungsklasse, die standardmäßig nur klassische DID-Formate (did:key, did:sov) in JWS-Header-kid-Feldern akzeptiert, um did:peer:4-Identifier zu unterstützen. Dies ist eine Voraussetzung für ML-DSA-65-signierte DID-Exchange-AttachDecorators. 

Diese neun Module bilden gemeinsam eine Patch-Architektur, die durch sequenzielle Installation beim Plugin-Setup (orchestriert in \_\_init\_\_.py) eine vollständige PQC-Funktionalität in ACA-Py injiziert, ohne dass Änderungen an Controllern, Admin-API-Endpunkten oder externen Business-Logic-Schichten erforderlich sind.

\subsubsection{Dockerfile-Modifikation}

In der zweiten Iteration wurde das ACA-Py Docker-Base-Image (Listing~\ref{lst:Dockerfile-acapy-base}) aus der ersten Iteration (Kapitel~\ref{SSI-Agenten}) auf einen vier-stufigen Multi-Stage-Build erweitert (Listing~\ref{lst:Dockerfile-acapy-base-pqc} visualisiert dargestellt in \autoref{fig:Iteration2_Acapy_Multi_Stage_Build}).

\begin{figure}[H]
    \centering
    \includegraphics[width=\linewidth]{Iteration2_Acapy_Multi_Stage Build.png}
    \caption{ACA-Py Multi-Stage Build Dockerfile mit PQC-Integration (Iteration 2)}
    \begin{flushleft}
    \textit{Anmerkung.} Eigene Darstellung.
    \end{flushleft}
    \label{fig:Iteration2_Acapy_Multi_Stage_Build}
\end{figure}

Stage~1 kompiliert OpenSSL~3.5.4 mit FIPS-Modul und nativer ML-KEM/ML-DSA-Unterstützung, Stage~2 baut liboqs~0.14.0 als Shared Library, Stage~3 bleibt identisch zur ersten Iteration (Poetry-basiertes ACA-Py-Wheel), und Stage~4 integriert alle Artefakte durch \enquote{COPY --from}-Direktiven aus den Builder-Stages. Die Runtime-Stage überschreibt System-OpenSSL-Symlinks mittels \enquote{ln -sf}, aktualisiert Shared-Library-Pfade via \enquote{ldconfig}, importiert das PQC-Root-CA-Zertifikat in den System-Trust-Store (\enquote{update-ca-certificates}), und installiert das pqc\_didpeer4\_fm-Plugin mithilfe von \enquote{pip} direkt in das Container-Image.

\subsubsection{Deployment in docker-compose.yml}

Das Deployment der PQC-fähigen SSI-Agenten erfolgt innerhalb der in der ersten Iteration (Kapitel~\ref{sec:Docker Orchestrierung der Gesamtarchitektur}) entwickelten docker-compose.yml-Orchestrierung, deren Evolution vom klassischen Setup (erste Iteration, Listing~\ref{lst:docker-compose.yml-SSI-Agenten}) zur PQC-Integration (zweite Iteration, Listing~\ref{lst:docker-compose.yml-SSI-Agenten-mit-acapy-base-pqc-und-plugin}) zwei zentrale Anpassungen umfasst. Während die Konfiguration der ersten Iteration noch das klassische ACA-Py-Base-Image ohne PQC-Unterstützung und Plugin-Aktivierung verwendet, wurde in der zweiten Iteration im Rahmen der ersten Anpassung die docker-compose.yml so modifiziert, dass alle drei Agent-Services (issuer, holder, verifier) das neue \texttt{acapy-base-pqc}-Image nutzen in welchem das pqc\_didpeer4\_fm-Plugin enthalten ist. Diese Änderung kann durch den Vergleich von \autoref{fig:Docker-Compose-Übersicht-Iteration-2} mit \autoref{fig:Docker-Compose-Übersicht-Iteration-1} nachvollzogen werden. Die zweite Anpassung erweitert die command-Direktive aller drei Agenten um den Parameter \enquote{--plugin pqc\_didpeer4\_fm}, der beim Agent start das pqc\_didpeer4\_fm-Plugin lädt.

\begin{figure}[H]
    \centering
    \includegraphics[width=\linewidth]{docker_compose_übersicht_pqc_plugin.png}
    \caption{Docker-Compose-Übersicht der zweiten Iteration}
    \begin{flushleft}
    \textit{Anmerkung.} Eigene Darstellung.
    \end{flushleft}
    \label{fig:Docker-Compose-Übersicht-Iteration-2}
\end{figure}

\subsection{Formative Evaluation} \label{sec:formative_evaluation_iteration2}

\subsubsection{Validierung des Plugin-Ladevorgangs bei Agent-Start}

Die erste formative Evaluationsmaßnahme bestand in der Validierung des korrekten Plugin-Ladevorgangs beim Start eines ACA-Py-Agenten. Dieser Test diente der Sicherstellung, dass die PQC-Integration transparent und ohne Beeinträchtigung der Standard-ACA-Py-Funktionalität erfolgt.

Listing~\ref{lst:Issuer-Agent-Boot-Logs} zeigt den Boot-Prozess eines Standard-ACA-Py-Agenten ohne PQC-Plugin. Nach der Registrierung der Default- und Askar-Plugins wird direkt mit der Ledger-Konfiguration und Wallet-Initialisierung fortgefahren. Im Vergleich dazu zeigt Listing~\ref{lst:Issuer-Agent-Boot-Logs-mit-PQC-Plugin} den erweiterten Boot-Prozess mit geladenem pqc\_didpeer4\_fm-Plugin. Zwischen der Askar-Plugin-Registrierung und der Ledger-Konfiguration erfolgt nun die Plugin-Initialisierung mit mehreren charakteristischen Schritten.

\begin{enumerate}
\item Askar-Patching: Die \_create\_keypair-Funktion wird durch eine PQC-fähige Variante ersetzt, die ML-DSA-65 und ML-KEM-768 unterstützt. Zusätzlich werden Session-Methoden (insert\_key, fetch\_key, update\_key) und AskarWallet.assign\_kid\_to\_key() gepatcht.
\item KeyType-Registry-Erweiterung: Die neuen Schlüsseltypen ml-dsa-65 und ml-kem-768 werden in der ACA-Py KeyTypes-Registry registriert und die API-Schemas zur Laufzeit erweitert.
\item did:peer:4-Erweiterung: Die unterstützten Schlüsseltypen für did:peer:4 werden von ['ed25519', 'x25519'] auf ['ed25519', 'x25519', 'ml-dsa-65', 'ml-kem-768'] erweitert.
\item Multicodec-Patching: Die SupportedCodecs-Klasse wird für PQC-Multicodec-Präfixe erweitert (ML-DSA-65: 0xd065, ML-KEM-768: 0xe018).
\item DIDComm-Patching: AskarWallet.pack\_message() und unpack\_message() werden für ML-KEM-768-basierte Verschlüsselung angepasst. Die AttachDecorator-Klasse wird für ML-DSA-65-JWS-Signaturen erweitert.
\item Monkey-Patches: Die BaseConnectionManager-Methoden (create\_did\_peer\_4, record\_keys\_for\_resolvable\_did, etc.) werden durch PQC-fähige Varianten ersetzt.
\end{enumerate}

\subsubsection{Validierung der Pluginfunktionalität: Out-of-Band Invitation mit did:peer:4}

Die zweite formative Evaluationsmaßnahme validierte die Kernfunktionalität des Plugins: die transparente Erstellung von PQC-fähigen did:peer:4-DIDs während des Out-of-Band-Invitation-Prozesses.

Listing~\ref{lst:Iteration2_Validierung-der-Pluginfunktionalität-Wallet-DID-Abfrage-vor-OOB-Invitation} zeigt die initiale Wallet-Abfrage eines frisch gestarteten Issuer-Agenten. Das leere results-Array bestätigt, dass noch keine DIDs im Wallet vorhanden sind.

\refstepcounter{manualListingCounter}
\label{lst:Iteration2_Validierung-der-Pluginfunktionalität-Wallet-DID-Abfrage-vor-OOB-Invitation}
\begin{lstlisting}[language=bash, caption={Zweite Iteration - Wallet DID Abfrage vor Out-of-Band Invitation}, numbers=left, frame=single]
ferris@blockchain-ssi-pqc:~$ curl -X GET https://host.docker.internal:8021/wallet/did | jq
  % Total    % Received % Xferd  Average Speed   Time    Time     Time  Current
                                 Dload  Upload   Total   Spent    Left  Speed
100    15  100    15    0     0   1083      0 --:--:-- --:--:-- --:--:--  1153
{
  "results": []
}
\end{lstlisting}

Anschließend wurde mittels \enquote{POST /out-of-band/create-invitation} mit dem Parameter \enquote{use\_did\_method: 'did:peer:4'} eine Einladung erstellt (Listing~\ref{lst:Issuer-Agent-Boot-Logs-mit-PQC-Plugin}). Die API-Response enthält eine vollständige did:peer:4-Langform-DID im services-Array der Invitation, erkennbar am charakteristischen Format \enquote{did:peer:4zQm...:z25g...}.

\refstepcounter{manualListingCounter}
\label{lst:Issuer-Agent-Boot-Logs-mit-PQC-Plugin}
\begin{lstlisting}[language=bash, caption={Zweite Iteration - Out-of-Band Invitation}, numbers=left, frame=single]
ferris@blockchain-ssi-pqc:~$ curl -X POST https://host.docker.internal:8021/out-of-band/create-invitation     -H "Content-Type: application/json"     -d '{
      "handshake_protocols": ["https://didcomm.org/didexchange/1.1"],
      "use_did_method": "did:peer:4",
      "my_label": "Issuer Test"
    }' | jq
  % Total    % Received % Xferd  Average Speed   Time    Time     Time  Current
                                 Dload  Upload   Total   Spent    Left  Speed
100 16198  100 16051  100   147  91160    834 --:--:-- --:--:-- --:--:-- 91514
{
  "state": "initial",
  "trace": false,
  "invi_msg_id": "89e9cc87-318f-49aa-a61a-fc805706cd8d",
  "oob_id": "70998122-5a5b-4020-8b5f-ae5884af20b3",
  "invitation": {
    "@type": "https://didcomm.org/out-of-band/1.1/invitation",
    "@id": "89e9cc87-318f-49aa-a61a-fc805706cd8d",
    "label": "Issuer Test",
    "handshake_protocols": [
      "https://didcomm.org/didexchange/1.1"
    ],
    "services": [
      "did:peer:4zQmYFdntsqaiZcU9PMf4dVshmxyTu5yk3NnkA28VjHqaySm:z25gYmQoBS9XWQbLxdKXKizWUz5MxCWwLc..."
    ]
  },
  "invitation_url": "https://host.docker.internal:8020?oob=eyJAdHlwZSI6ICJodHR..."
}
\end{lstlisting}

Die entscheidende Validierung erfolgt in Listing~\ref{lst:Iteration2_Validierung-der-Pluginfunktionalität-Wallet-DID-Abfrage-nach-OOB-Invitation} durch eine erneute Wallet-Abfrage nach der Invitation-Erstellung. Die Response zeigt nun die automatisch generierte PQC-DID mit folgenden charakteristischen Merkmalen:

\begin{itemize}
\item Dual-Key-Struktur: Das key\_type-Feld weist den Wert ml-dsa-65 auf, während die Metadata zusätzlich kem\_verkey (ML-KEM-768) enthält. Dies bestätigt die erfolgreiche Implementierung der Hybrid-Kryptografie mit getrennten Schlüsseln für digitale Signaturen und Schlüsselvereinbarung.
\item PQC-Metadata: Die Metadaten enthalten explizite Marker (pqc\_enabled: true, signature\_algorithm: "ml-dsa-65", key\_agreement\_algorithm: "ml-kem-768"), die eine eindeutige Identifikation PQC-fähiger DIDs zur Laufzeit ermöglichen.
\item Key Identifier: Das kem\_key\_kid-Feld referenziert den KEM-Schlüssel über den DID-URL-Fragment-Identifier \#key-1, was der did:peer:4-Spezifikation entspricht, bei der Verification-Methods sequenziell nummeriert werden (\#key-0 für Authentication, \#key-1 für Key Agreement).
\end{itemize}

\refstepcounter{manualListingCounter}
\label{lst:Iteration2_Validierung-der-Pluginfunktionalität-Wallet-DID-Abfrage-nach-OOB-Invitation}
\begin{lstlisting}[language=bash, caption={Zweite Iteration - Wallet DID Abfrage nach Out-of-Band Invitation}, numbers=left, frame=single]
ferris@blockchain-ssi-pqc:~$ curl -X GET https://host.docker.internal:8021/wallet/did | jq
  % Total    % Received % Xferd  Average Speed   Time    Time     Time  Current
                                 Dload  Upload   Total   Spent    Left  Speed
100 17778  100 17778    0     0  1655k      0 --:--:-- --:--:-- --:--:-- 1736k
{
  "results": [
    {
      "did": "did:peer:4zQmYFdntsqaiZcU9PMf4dVshmxyTu5yk3NnkA28VjHqaySm:z25gYmQoBS9XWQbLxdKXKizWUz5MxCWwLc...",
      "verkey": "2BvJSsMeLjejWKygFBC1qFPLqUvvTzfed7y2Btp...",
      "posture": "wallet_only",
      "key_type": "ml-dsa-65",
      "method": "did:peer:4",
      "metadata": {
        "invitation_reuse": "true",
        "pqc_enabled": true,
        "signature_algorithm": "ml-dsa-65",
        "key_agreement_algorithm": "ml-kem-768",
        "kem_key_kid": "did:peer:4zQmYFdntsqaiZcU9PMf4dVshmxyTu5yk3NnkA28VjHqaySm:z25gYmQoBS9XWQbLxdKXKizWUz5MxCWwLc...D6SUGP43VJWg#key-1",
        "kem_verkey": "h6ngVfG9n2qF1SY5gM3DaDhK9iiwhvnW555QtodD1sgvEcg5...",
        "plugin": "pqc_didpeer4_fm",
        "version": "0.1.0"
      }
    }
  ]
}
\end{lstlisting}

\subsection{Finales Artefakt}

Das finale Artefakt der zweiten Iteration repräsentiert einen funktionsfähigen SSI-Prototypen mit vollständiger Post-Quantum-Kryptografie-Integration auf Application-Layer-Ebene. Die Architektur vereint die in Iteration~1 etablierte Transport-Layer-Sicherung mittels PQC-Sidecar-Proxies mit einer tiefgreifenden Anwendungsschicht-Integration durch das entwickelte pqc\_did\_peer4\_fm-Plugin.

Die Kernkomponente bildet das ACA-Py-Plugin mit dreischichtiger Architektur. Die Kryptografie-Abstraktionsschicht kapselt native liboqs-Operationen und exponiert eine Python-API für ML-DSA-65 und ML-KEM-768. Die DID-Verarbeitungsschicht orchestriert Generierung, Auflösung und Kodierung PQC-fähiger did:peer:4-Identifikatoren. Die Integration-Patching-Schicht realisiert transparentes Monkey-Patching kritischer ACA-Py-Kernfunktionen ohne Modifikation des Framework-Quellcodes.

Hinsichtlich der in Kapitel~4.2.1 definierten Designziele erfüllt das finale Artefakt sämtliche Anforderungen. Das Designziel zu FF1 (Systemarchitektur \& Compliance) wird durch die native Unterstützung quantenresistenter Signaturen in den DID-Dokumenten adressiert, wodurch die Authentizität von Identitätsnachweisen unabhängig vom Transportkanal langfristig gegenüber Quantencomputer-Angriffen gewährleistet bleibt. Das Designziel zu FF2 (Algorithmenauswahl \& Sicherheitsbewertung) manifestiert sich in der erfolgreichen Integration von ML-DSA-65 für digitale Signaturen innerhalb der did:peer:4-Strukturen, was die praktische Machbarkeit von PQC-Signaturen in dezentralen Identifikatoren nachweist. Das Designziel zu FF3 (Kryptografische Agilität) wird durch die Erweiterung der Multicodec-Registry um provisorische Präfixe für ML-DSA-65 (0xd065) und ML-KEM-768 (0xe018) sowie die abstrahierte Kryptografie-Schicht realisiert, welche die Koexistenz klassischer und post-quanten Verfahren innerhalb der DID-Methoden ermöglicht.

Das Multi-Stage-Build-Dockerfile integriert alle Abhängigkeiten in ein kohärentes Container-Image: Stage~1 kompiliert OpenSSL~3.5.4 mit nativer ML-KEM/ML-DSA-Unterstützung, Stage~2 baut liboqs~0.14.0, Stage~3 generiert das ACA-Py-Wheel, und Stage~4 fusioniert alle Artefakte in ein produktionsfähiges Runtime-Image. Die formative Evaluation validierte die funktionale Korrektheit durch erfolgreiche Plugin-Registrierung beim Agent-Start sowie korrekte Generierung von did:peer:4-Long-Form-DIDs mit PQC-Schlüsselmaterial im Out-of-Band-Invitation-Workflow.












\newpage
\section{Summative Evaluation} \label{sec:Summative Evaluation}
        - Evaluationsmethodik (FEDS)


Im Rahmen der summativen Evaluation wurde das finale Artefakt aus \fixme{KAPITEL} mithilfe des in \ref{sec:Anhang_Summative Evaluation} dargestellten Jupyter Notebooks evaluiert.

KRITIS Szenario ...  Ziel der Evaluation ist die Validierung der funktionalen Anforderungen gemäß Kapitel~\ref{sec:Funktionale Anforderungen}.

- Erst Initialisierung des Artefakts durch:

==> \ref{sec:Anhang_Teil1-Setup-Verbindungstests}
    Variablendeklaration und Helper Funktionen ==> Listing~\ref{lst:Jupyter-Notebook-Cell-1}

    Infrastrukturcheck und Zeigen der Ledgerinitialisierung durch Abruf der Ledgertransaktionen (Validator-Node-Registrierung) ==> Listing~\ref{lst:Jupyter-Notebook-Cell-2-output}

==> \ref{sec:Anhang_Teil2-DID-Setup-Ledger-Registration-KRITIS-Identitäten}
    - ANlegen von Issuer DID lokal im acapy agent ==> Listing~\ref{lst:Jupyter-Notebook-Cell-3-output}
    - Registrieren als ENDORSER auf Ledger ==> Listing~\ref{lst:Jupyter-Notebook-Cell-4-output}
    - Walletansicht indy ed25519 posted mit true

\subsection{Validierung der funktionalen Anforderungen}

\subsubsection{Issuer Discovery}

Die funktionale Anforderung FR1 fordert, dass das System die Auffindbarkeit von publizierten Credential-Schemata des Issuers digitaler Identitätsnachweise ermöglichen muss. Die Erfüllung dieser Anforderung an das finale Artefakt wird anhand eines dreiphasigen, Ledger-basierten Discovery-Mechanismus demonstriert (Listing~\ref{lst:Jupyter-Notebook-Cell-8} und Listing~\ref{lst:Jupyter-Notebook-Cell-8-output}). 

Phase 1 extrahiert alle TRUST\_ANCHOR-Identitäten (Role \texttt{'101'}) aus NYM-Transaktionen des Domain Ledgers, wobei im KRITIS-Szenario der Issuer \enquote{Energienetzbetreiber} mit DID \texttt{9pbXiFBZZGwXKp61HQBz3J} identifiziert wird (Listing~\ref{lst:Jupyter-Notebook-Cell-8-output}, Zeilen 7--17). 

Phase 2 verifiziert sechs kryptographische Eigenschaften (DID-Identifier, Ed25519-Verkey, TRUST\_ANCHOR-Role, Endorser, On-Ledger-Aktivitäten, Registrierungszeitpunkt) mittels der Funktion \texttt{verify\_issuer\_identity()}, wobei für den identifizierten Issuer alle Eigenschaften erfolgreich validiert werden (Listing~\ref{lst:Jupyter-Notebook-Cell-8-output}, Zeilen 24--29). 

Phase 3 filtert SCHEMA-Transaktionen nach dem Schema-Namen \texttt{kritis\_emergency\_maintenance\_cert}, extrahiert den Issuer-DID aus dem Schema-Identifier-Format \texttt{<issuer\_did>:2:<schema\_name>:<version>} und führt eine Cross-Referenzierung mit den TRUST\_ANCHOR-Identitäten durch (Listing~\ref{lst:Jupyter-Notebook-Cell-8-output}, Zeilen 39--60).


\subsubsection{Connection Creation}

Die funktionale Anforderung FR2 fordert, dass das System Verbindungen zwischen den Akteuren des SSI-Ökosystems etablieren muss. Die Erfüllung dieser Anforderung an das finale Artefakt wird anhand eines dreiphasigen Out-of-Band-Invitation-Protokolls mit did:peer:4-basierter Post-Quantum-Kryptographie demonstriert (Listing~\ref{lst:Jupyter-Notebook-Cell-9}, Listing~\ref{lst:Jupyter-Notebook-Cell-9-output}, Listing~\ref{lst:Jupyter-Notebook-Cell-10} und Listing~\ref{lst:Jupyter-Notebook-Cell-10-output}).

Phase~1 implementiert einen Pre-Check existierender Connections via \texttt{GET /connections} auf beiden Agenten, um redundante Connection-Erstellungen zu vermeiden, wobei im KRITIS-Szenario keine existierenden Connections gefunden werden und eine neue Etablierung ausgelöst wird (Listing~\ref{lst:Jupyter-Notebook-Cell-9-output}, Zeilen~5--10).

Phase~2 realisiert die Connection-Etablierung mittels Aries RFC~0434 Out-of-Band Protocol: Der Inviter erstellt eine Invitation mit \texttt{POST /out-of-band/create-invitation} unter Verwendung von \texttt{use\_did\_method: "did:peer:4"}, wobei die Response eine \texttt{invitation\_msg\_id} als eindeutigen Identifier enthält (Listing~\ref{lst:Jupyter-Notebook-Cell-9-output}, Zeile~14). Der Invitee akzeptiert die Invitation via \texttt{POST /out-of-band/receive-invitation}, wodurch das DIDComm DIDExchange-Protokoll initiiert und did:peer:4-DIDs mit ML-DSA-65-Schlüsselmaterial generiert werden (Listing~\ref{lst:Jupyter-Notebook-Cell-12-output} zeigt die resultierenden DIDs mit Metadata \texttt{pqc\_enabled: true}, \texttt{signature\_algorithm: ml-dsa-65}, \texttt{key\_agreement\_algorithm: ml-kem-768}). Der Inviter identifiziert seine Connection anhand der \texttt{invitation\_msg\_id} durch Iteration über alle Connections, wobei die erfolgreiche Zuordnung mit übereinstimmender \texttt{invitation\_msg\_id} und State \texttt{active} validiert wird (Listing~\ref{lst:Jupyter-Notebook-Cell-9-output}, Zeilen~23--27).

Phase~3 validiert den Connection-Status durch Abruf detaillierter Connection-Informationen via \texttt{GET /connections/\{conn\_id\}} auf beiden Seiten, wobei die Konsistenz durch Vergleich der \texttt{invitation\_msg\_id}, komplementäre \texttt{their\_role}-Werte (\texttt{inviter}/\texttt{invitee}) und beidseitigen State \texttt{active} verifiziert wird (Listing~\ref{lst:Jupyter-Notebook-Cell-9-output}, Zeilen~31--45). Die Connection-Übersicht (Listing~\ref{lst:Jupyter-Notebook-Cell-11-output}) gruppiert Connections anhand der \texttt{invitation\_msg\_id} und zeigt zwei aktive Connection-Paare: Issuer<-->Holder (Connection~Group~1) und Holder<-->Verifier (Connection~Group~2), wodurch die vollständige Konnektivität des SSI-Dreiecks validiert wird.

\subsubsection{Credential Creation}

Die funktionale Anforderung FR3 fordert, dass das System Funktionalität zur Erstellung und Ausstellung digitaler Credentials bereitstellen muss. Die Erfüllung dieser Anforderung an das finale Artefakt wird anhand eines mehrstufigen Credential-Issuance-Workflows mit Revocation-Registry-Integration demonstriert (Listing~\ref{lst:Jupyter-Notebook-Cell-13}, Listing~\ref{lst:Jupyter-Notebook-Cell-13-output} und Listing~\ref{lst:Jupyter-Notebook-Cell-14-output}).

Der Issuer initiiert die Credential-Ausstellung durch Versenden eines Credential Offers via \texttt{POST /issue-credential-2.0/send-offer} mit einer Credential Preview, die neun KRITIS-spezifische Attribute enthält (Identität: \texttt{first\_name}, \texttt{name}, \texttt{organisation}; Berechtigung: \texttt{cert\_type}, \texttt{facility\_type}, \texttt{security\_clearance\_level}; Zeitgültigkeit: \texttt{epoch\_valid\_from}, \texttt{epoch\_valid\_until}; Rolle: \texttt{role}), wobei die Ausstellung über die in FR2 etablierte Connection (\texttt{connection\_id}) und die in FR1 identifizierte Credential Definition (\texttt{cred\_def\_id}) erfolgt (Listing~\ref{lst:Jupyter-Notebook-Cell-13}, Zeilen~8--31). Die Response enthält eine Exchange ID zur Nachverfolgung des Issuance-Prozesses, wobei der initiale State \texttt{offer-sent} den erfolgreichen Versand bestätigt (Listing~\ref{lst:Jupyter-Notebook-Cell-13-output}, Zeilen~3--5).

Der Holder akzeptiert das Credential Offer automatisch (\texttt{auto-store=true} Konfiguration), wodurch das Aries RFC~0453 Issue Credential v2.0 Protocol den vollständigen State-Machine-Durchlauf (\texttt{offer-sent} → \texttt{request-sent} → \texttt{credential-issued} → \texttt{done}) ausführt und das Credential im Holder Wallet persistiert. Nach einer Wartezeit von 5 Sekunden (Listing~\ref{lst:Jupyter-Notebook-Cell-13}, Zeile~55) zeigt der Status-Check auf Issuer-Seite den finalen State \texttt{done} (Listing~\ref{lst:Jupyter-Notebook-Cell-13-output}, Zeile~11), während der Holder-Wallet-Abruf via \texttt{GET /credentials} das gespeicherte Credential mit Referent \texttt{39ac5fc4-efc2-45eb-9a21-01c589757b65} und allen neun Attributen bestätigt (Listing~\ref{lst:Jupyter-Notebook-Cell-13-output}, Zeilen~14--22).

Die Revocation-Registry-Integration extrahiert zwei kritische Identifier aus der Issuer-Exchange-Response: Die Revocation Registry ID (\texttt{9pbXiFBZZGwXKp61HQBz3J:4:...:CL\_ACCUM:...}) identifiziert die auf dem Indy Ledger publizierte Revocation Registry (Type~113 REVOC\_REG\_DEF Transaction), während die Credential Revocation ID (\texttt{1}) die Position des Credentials im Revocation-Accumulator spezifiziert (Listing~\ref{lst:Jupyter-Notebook-Cell-13-output}, Zeilen~26--28). Diese IDs werden für den Revocation-Workflow (FR5) benötigt und demonstrieren die Integration von Credential Issuance und Revocation Management im Gesamtsystem.

Die Holder-Credentials-Übersicht (Listing~\ref{lst:Jupyter-Notebook-Cell-14-output}) zeigt das vollständig ausgestellte KRITIS-Notfall-Wartungszertifikat mit allen Attributen, dem Schema-Identifier \texttt{...kritis\_emergency\_maintenance\_cert:1.1} aus FR1, der Credential-Definition-ID aus dem Schema-basierten Discovery-Prozess und dem initialen Revoked-Status \texttt{False}, der die Gültigkeit des Credentials bestätigt (Zeilen~6--19). Der Issuer Credential Registry Check via \texttt{GET /issue-credential-2.0/records} validiert die serverseitige Persistierung des Exchange Records (Listing~\ref{lst:Jupyter-Notebook-Cell-13-output}, Zeilen~35--38), wobei die Verfügbarkeit des Records die Aktivierung des \texttt{--preserve-exchange-records}-Flags bestätigt, das für Audit-Zwecke und Revocation-Management in KRITIS-Kontexten erforderlich ist.

\subsubsection{Verification with Credentials}

Die funktionale Anforderung FR4 fordert, dass das System einen Verifikationsprozess zwischen Identity Holder, Verifier und Blockchain-basierter Verifiable Data Registry (VDR) durch Validierung eines Identitätsnachweises ermöglichen muss. Die Erfüllung dieser Anforderung an das finale Artefakt wird anhand eines vierstufigen Privacy-Preserving-Verification-Workflows mit Zero-Knowledge-Proofs, Revocation-Detection und Zeitgültigkeitsprüfung demonstriert (Listing~\ref{lst:Jupyter-Notebook-Cell-15} bis Listing~\ref{lst:Jupyter-Notebook-Cell-18-Output}).

Der Verifier initiiert den Verifikationsprozess durch Versenden eines Proof Requests via \texttt{POST /present-proof-2.0/send-request} mit einer Indy-Proof-Request-Struktur, die fünf offengelegte Attribute (\texttt{requested\_attributes}: \texttt{cert\_type}, \texttt{facility\_type}, \texttt{epoch\_valid\_from}, \texttt{epoch\_valid\_until}, \texttt{role}) und ein Zero-Knowledge-Predicate (\texttt{requested\_predicates}: \texttt{security\_clearance\_level >= 2}) fordert, während drei Identitätsattribute (\texttt{first\_name}, \texttt{name}, \texttt{organisation}) durch Selective Disclosure geschützt bleiben (Listing~\ref{lst:Jupyter-Notebook-Cell-15}, Zeilen~21--56). Alle Attribute und Predicates enthalten eine \texttt{non\_revoked}-Constraint mit Zeitintervall \texttt{\{from: 0, to: current\_timestamp\}}, die eine Ledger-basierte Echtzeit-Revocation-Prüfung gegen die Revocation Registry erzwingt (Zeilen~24, 30, 36, 42, 48, 54). Der initiale State \texttt{request-sent} bestätigt die erfolgreiche Übermittlung des Proof Requests über die in FR2 etablierte Connection (Listing~\ref{lst:Jupyter-Notebook-Cell-15-Output}, Zeilen~11--12).

Der Holder empfängt den Proof Request (State \texttt{request-received}) und ruft via \texttt{GET /present-proof-2.0/records/\{pres\_ex\_id\}/credentials} alle Credentials ab, die die Proof-Request-Anforderungen erfüllen, wobei die Schema-ID und Credential-Definition-ID aus FR1 und FR3 zur Filterung verwendet werden (Listing~\ref{lst:Jupyter-Notebook-Cell-16-Output}, Zeilen~4--19). Der Holder konstruiert ein \texttt{requested\_credentials}-Objekt durch Mapping der fünf Attribute-Referents (\texttt{attr1\_referent} bis \texttt{attr5\_referent}) und des Predicate-Referents (\texttt{pred1\_clearance}) auf die Credential-ID \texttt{39ac5fc4...}, wobei Attribute mit \texttt{revealed: true} gekennzeichnet werden, während das Predicate ohne Offenlegung des Attributwerts evaluiert wird (Listing~\ref{lst:Jupyter-Notebook-Cell-17-Output}, Zeilen~6--11). Der Versand der Presentation via \texttt{POST .../send-presentation} erzeugt einen Zero-Knowledge-Proof, der kryptographisch beweist, dass der Holder ein Credential mit den geforderten Attributen und erfülltem Predicate besitzt, ohne die unrevealed Attribute offenzulegen (Zeilen~18--26).

Der Verifier empfängt die Presentation (State \texttt{done}, \texttt{verified: true}) und extrahiert die revealed Attributes durch dreistufiges Mapping: (1) Abruf der Attribute-Namen aus \texttt{by\_format.pres\_request.indy.requested\_attributes}, (2) Abruf der Attribut-Werte aus \texttt{by\_format.pres.indy.requested\_proof.revealed\_attrs}, (3) Konstruktion eines Name-Value-Mappings (Listing~\ref{lst:Jupyter-Notebook-Cell-18}, Zeilen~30--44), wodurch fünf offengelegte Attribute (\texttt{cert\_type: Notfall-Wartungsberechtigung}, \texttt{facility\_type: Umspannwerk Nord-Ost}, \texttt{epoch\_valid\_from: 1765026000}, \texttt{epoch\_valid\_until: 1765033200}, \texttt{role: Notfalltechniker}) extrahiert werden (Listing~\ref{lst:Jupyter-Notebook-Cell-18-Output}, Zeilen~5--9). Die drei Identitätsattribute bleiben durch Selective Disclosure geschützt und werden als \enquote{NICHT offengelegt (Zero-Knowledge-Proof)} ausgewiesen, wodurch Privacy by Design gemäß DSGVO Art.~25 realisiert wird (Listing~\ref{lst:Jupyter-Notebook-Cell-18-Output}, Zeilen~11--14).

Die Blockchain-basierte Revocation-Prüfung validiert den Credential-Status durch Vergleich des Indy-Proof-Timestamps mit dem Revocation-Registry-Delta auf dem Ledger, wobei State \texttt{done} und \texttt{verified: true} bestätigen, dass das Credential zum Verifikationszeitpunkt nicht revoked war (Listing~\ref{lst:Jupyter-Notebook-Cell-18-Output}, Zeile~18). Die Zeitgültigkeitsprüfung vergleicht einen aktuellen Epoch-Timestamp (Beispielwert \texttt{1765029600}) mit den extrahierten Zeitgrenzen (\texttt{epoch\_valid\_from: 1765026000}, \texttt{epoch\_valid\_until: 1765033200}), wobei die Bedingung \texttt{epoch\_valid\_from <= current\_epoch <= epoch\_valid\_until} die zeitliche Gültigkeit bestätigt (Zeilen~20--27). Die Zero-Knowledge-Predicate-Auswertung extrahiert das erfüllte Predicate \texttt{security\_clearance\_level >= 2} aus \texttt{requested\_proof.predicates}, wobei die exakte Clearance-Stufe (ob Ü2 oder Ü3) durch die ZKP-Eigenschaft verborgen bleibt (Zeilen~29--33).

Die finale Zugriffsentscheidung kombiniert drei Validierungsergebnisse in einer logischen UND-Verknüpfung (\texttt{not is\_revoked AND is\_time\_valid AND has\_required\_clearance}), die im demonstrierten KRITIS-Szenario zum Ergebnis \enquote{ZUGANG GEWÄHRT} führt, da alle Bedingungen erfüllt sind (Listing~\ref{lst:Jupyter-Notebook-Cell-18-Output}, Zeilen~36--42). Die Verifiable Data Registry (VDR)-Integration manifestiert sich durch drei Ledger-basierte Validierungsschritte: (1) Schema-Validation via Schema-ID aus FR1, (2) Credential-Definition-Validation via Cred-Def-ID, (3) Revocation-Registry-Validation via \texttt{non\_revoked}-Constraint, wodurch alle kryptographischen Artefakte (Schema, Cred-Def, RevReg-Def, RevReg-Delta) vom Hyperledger Indy Ledger abgerufen und validiert werden.

\subsubsection{Credential Revocation}

Die funktionale Anforderung zur Credential Revocation fordert, dass das System die Ungültigkeitserklärung ausgestellter Credentials ermöglichen muss, wobei Verifier die Gültigkeit kryptographisch überprüfen können. Die Erfüllung dieser Anforderung an das finale Artefakt wird anhand eines dreiphasigen Revocation-Workflows demonstriert: Registry-Management, Two-Phase-Revocation (Staging + Publishing) sowie Non-Revocation-Proof Verification.

Phase~1 implementiert die Verwaltung von Revocation Registries durch den Issuer. Listing~\ref{lst:Jupyter-Notebook-Cell-19} demonstriert den Abruf aktiver Revocation Registries via \texttt{GET /revocation/registries/created?state=active}. Das Output (Listing~\ref{lst:Jupyter-Notebook-Cell-19-output}) zeigt zwei aktive Registries mit jeweils 100 Credential-Kapazität. Jede Registry enthält kritische Metadaten: \texttt{rev\_reg\_id} (eindeutige Registry-Identifier), \texttt{tails\_hash} (kryptographischer Hash der Tails-File für Zero-Knowledge Non-Revocation-Proofs), \texttt{tails\_local\_path} (lokaler Speicherort der Tails-File) sowie \texttt{issuer\_did} (did:indy des Issuers). Die Tails-File ist essentiell für die kryptographische Accumulator-basierte Revocation-Prüfung nach dem CL-Signature-Schema und wird vom Holder benötigt, um Non-Revocation-Proofs zu generieren.

Phase~2 realisiert die Two-Phase-Revocation durch Staging und Ledger-Publishing. Listing~\ref{lst:Jupyter-Notebook-Cell-20} demonstriert die Staging-Phase via \texttt{POST /revocation/revoke} mit \texttt{publish: false}. Die Revocation-Request referenziert das Credential durch \texttt{rev\_reg\_id} (Registry-Identifier) und \texttt{cred\_rev\_id} (Credential-spezifische Revocation-ID, hier: \texttt{"1"}). Das Output (Listing~\ref{lst:Jupyter-Notebook-Cell-20-output}) bestätigt den Status \texttt{"Pending"} -- die Revocation ist lokal staged, jedoch noch nicht auf dem Ledger publiziert. Listing~\ref{lst:Jupyter-Notebook-Cell-21} führt die Publishing-Phase aus via \texttt{POST /revocation/publish-revocations}. Das Output (Listing~\ref{lst:Jupyter-Notebook-Cell-21-output}) zeigt die erfolgreiche Ledger-Transaktion: Typ \texttt{114 (REVOC\_REG\_ENTRY)}, Sequence Number \texttt{14}, mit kryptographischen Accumulator-Updates (\texttt{prevAccum}, \texttt{accum}) und der Liste revokierter Credential-IDs (\texttt{revoked: [1]}). Die Ledger-Transaktion ist durch ED25519-Signatur des Issuers authentifiziert und via Byzantine Fault Tolerance konsistent repliziert.

Phase~3 validiert die Revocation-Enforcement durch Proof-Verification. Listing~\ref{lst:Jupyter-Notebook-Cell-14-output-revocation} zeigt das Holder-Wallet nach Revocation mit \texttt{Revoked Status: True}. Listings~\ref{lst:Jupyter-Notebook-Cell-15-output-revocation} bis \ref{lst:Jupyter-Notebook-Cell-17-output-revocation} demonstrieren den Proof-Request- und Presentation-Workflow: Der Holder wählt das revokierte Credential aus (Auto-Select via \texttt{auto\_present: true}), versucht einen Non-Revocation-Proof zu generieren, jedoch schlägt die Proof-Generierung fehl, da der kryptographische Accumulator das Credential als revoked markiert. Listing~\ref{lst:Jupyter-Notebook-Cell-18-output-revocation} zeigt die finale Verifier-Entscheidung: \texttt{Verified: false}, \texttt{Credential ist REVOKED}, \texttt{ZUGANG VERWEIGERT}. Trotz zeitlicher Gültigkeit (Epoch-Check erfüllt: \texttt{1765026000 <= 1765029600 <= 1765033200}) und erfülltem ZKP-Predicate (\texttt{security\_clearance\_level >= 2}) wird der Zugang verweigert, da die Revocation-Prüfung fehlschlägt. Dies demonstriert die Enforcement-Priorität: Revocation-Status dominiert alle anderen Validierungskriterien und stellt sicher, dass kompromittierte Credentials sofort unwirksam werden -- eine kritische Sicherheitsanforderung für KRITIS-Infrastrukturen.

\subsubsection{Credential Deletion}

Die funktionale Anforderung zur Credential Deletion fordert, dass Holder die Möglichkeit besitzen müssen, Credentials lokal aus ihrem Wallet zu entfernen, wobei diese Operation ausschließlich die lokale Datenhaltung betrifft und vom Ledger-basierten Revocation-Mechanismus zu unterscheiden ist. Die Erfüllung dieser Anforderung an das finale Artefakt wird anhand eines dreiphasigen Deletion-Workflows demonstriert: Pre-Deletion Inventory, Credential Deletion sowie Post-Deletion Verification.

Phase~1 implementiert das Pre-Deletion Inventory durch Abruf aller im Holder-Wallet gespeicherten Credentials (Listing~\ref{lst:Jupyter-Notebook-Cell-22}, Zeilen~7--43). Via \texttt{GET /credentials} werden alle Credential-Metadaten abgerufen, wobei für jedes Credential der \texttt{referent} (eindeutige Wallet-Identifier), \texttt{schema\_id}, \texttt{cred\_def\_id} sowie der Revocation-Status via \texttt{GET /credential/revoked/\{referent\}} ermittelt wird. Das Output (Listing~\ref{lst:Jupyter-Notebook-Cell-22-output}, Zeilen~1--19) zeigt ein Credential mit \texttt{Revoked: True} -- dieses Credential wurde zuvor via Ledger-Revocation ungültig erklärt (Cell~20--21), befindet sich jedoch weiterhin im lokalen Wallet. Die Auflistung aller Attribute (\texttt{security\_clearance\_level: 2}, \texttt{role: Notfalltechniker}, \texttt{facility\_type: Umspannwerk Nord-Ost}) demonstriert, dass revokierte Credentials im Wallet persistieren, bis der Holder sie explizit löscht. Die \texttt{referent}-Identifier werden in der Liste \texttt{credentials\_to\_delete} gespeichert (Zeile~34) für die nachfolgende Deletion-Phase.

Phase~2 realisiert die Credential Deletion durch iterative Deletion aller erfassten Credentials (Listing~\ref{lst:Jupyter-Notebook-Cell-22}, Zeilen~45--75). Für jeden \texttt{credential\_id} wird via \texttt{DELETE /credential/\{credential\_id\}} die lokale Wallet-Entfernung ausgeführt. Die \texttt{api\_delete()}-Hilfsfunktion behandelt HTTP~204~No~Content Responses korrekt (erfolgreiches Löschen ohne Response-Body). Das Output (Listing~\ref{lst:Jupyter-Notebook-Cell-22-output}, Zeilen~21--30) zeigt die erfolgreiche Deletion: \texttt{Gelöscht: 1/1}, wobei die Deletion Summary die Erfolgsrate dokumentiert. Diese Phase demonstriert die Holder-Autonomie über lokale Wallet-Daten -- der Holder kann Credentials unilateral entfernen, ohne Issuer-Interaktion oder Ledger-Transaktion.

Phase~3 validiert die Deletion-Enforcement durch Post-Deletion Verification (Listing~\ref{lst:Jupyter-Notebook-Cell-22}, Zeilen~77--115). Ein erneuter Abruf via \texttt{GET /credentials} bestätigt das leere Wallet: \texttt{Holder Wallet ist jetzt LEER (alle Credentials gelöscht)} (Listing~\ref{lst:Jupyter-Notebook-Cell-22-output}, Zeilen~32--36). Die explizite Unterscheidung zwischen lokaler Deletion und Ledger-Revocation wird durch Hinweise dokumentiert (Listing~\ref{lst:Jupyter-Notebook-Cell-22-output}, Zeilen~38--43): \texttt{Credential ist LOKAL im Wallet gelöscht}, \texttt{Credential ist NICHT auf dem Ledger revoked}, \texttt{Issuer kann das Credential weiterhin sehen}. Diese Differenzierung ist kritisch für das Verständnis des SSI-Sicherheitsmodells: Lokale Deletion entfernt das Credential aus der Holder-Verfügungsgewalt (kein Proof mehr generierbar), jedoch bleibt die Ledger-basierte Revocation-Historie intakt (\texttt{--preserve-exchange-records} beim Issuer aktiv). Für KRITIS-konforme Audit-Trails bedeutet dies: Die Credential-Issuance-History ist unveränderlich auf dem Ledger gespeichert, während der Holder die Privacy-wahrende Möglichkeit besitzt, lokale Credential-Kopien zu entfernen.

%       - KRITIS-Szenarien (Energie, Gesundheit, Wasser)
% \subsection{Performance-Analyse}
%        - Latenz-Messungen (Baseline, PQC, Hybrid) \\
%        - Durchsatz-Analyse \\
%        - Speicher- und Rechenaufwand \\
%        - Skalierbarkeitstest



\subsection{Validierung der KRITIS-Compliance-Anforderungen}

\subsubsection{Einhaltung spezifischer Parameter-Sets für ML-DSA}

Die Compliance-Anforderung zur Einhaltung BSI-konformer ML-DSA Parameter-Sets (NIST Security Strength Category 3 oder 5) wird durch strategische Verwendung zweier Sicherheitsstufen erfüllt: ML-DSA-65 (Category 3) für operationale Signaturen sowie ML-DSA-87 (Category 5) für die Root Certificate Authority. Das finale Artefakt implementiert ML-DSA in drei Schichten: (1)~TLS~1.3 Server-Zertifikate für alle fünf Nginx Sidecar Proxies (hopE-Agenten: Issuer/Holder/Verifier, VON-Network Webserver, Tails Server) werden mittels ML-DSA-65 signiert, konfiguriert via Build-Argument (SIG\_ALG: mldsa65) in den Docker-Infrastruktur-Definitionen (Listing~\ref{lst:docker-compose.yml-DLT-Infrastruktur}, Listing~\ref{lst:docker-compose.yml-Revocation-Registry}, Listing~\ref{lst:docker-compose.yml-SSI-Agenten}), wobei die Root CA als langfristiger Trust-Anchor mit der höheren Sicherheitsstufe ML-DSA-87 geschützt ist (Listing~\ref{lst:Zertifikatserstellungsworkflow}). (2)~did:peer:4 Signing Keys für dezentrale Agent-to-Agent Authentifizierung werden mit ML-DSA-65 generiert (Listing~\ref{lst:key_types.py}). (3)~DIDComm~v1 Authcrypt Message-Signierung nutzt ML-DSA-65 via LibOQS-Integration (Listing~\ref{lst:liboqs_wrapper.py}) für kryptographisch verifizierbare Sender-Authentifizierung in verschlüsselten Nachrichten.

Die erfolgreiche operationale Integration von ML-DSA-65 wird zum einen durch \autoref{fig:Successful-Validation-Issuer-TLS1.3} demonstriert, welches die TLS~1.3 Verbindung zum Issuer-Agenten mit ML-DSA-65 signiertem Server-Zertifikat validiert, und zum anderen durch die Wallet-Übersicht in Listing~\ref{lst:Jupyter-Notebook-Cell-12-output} demonstriert, in welcher alle drei SSI-Agenten key\_type: "ml-dsa-65" für did:peer:4 DIDs nutzen.

\subsubsection{Einhaltung spezifischer Parameter-Sets für ML-KEM}

Diese Anforderung wird durch systemweite Implementierung von ML-KEM-768 (Category 3) erfüllt. Das finale Artefakt implementiert ML-KEM-768 in zwei Schichten: (1)~TLS~1.3 Key Exchange für alle fünf Nginx Sidecar Proxies nutzt ML-KEM-768 konfiguriert via \texttt{DEFAULT\_GROUPS=X25519MLKEM768} in den Docker-Infrastruktur-Definitionen (Listing~\ref{lst:docker-compose.yml-DLT-Infrastruktur}, Listing~\ref{lst:docker-compose.yml-Revocation-Registry}, Listing~\ref{lst:docker-compose.yml-SSI-Agenten}) sowie via \texttt{ssl\_ecdh\_curve X25519MLKEM768} in den Nginx-Konfigurationen (Listing~\ref{lst:nginx-pqc-konfiguration}). (2)~did:peer:4 Key Agreement für dezentrale Agent-to-Agent Verschlüsselung wird mit ML-KEM-768 realisiert (Listing~\ref{lst:key_types.py}), womit DIDComm-Nachrichten mittels Post-Quantum-resistenter Schlüsselkapselung verschlüsselt werden.

Die erfolgreiche operationale Integration von ML-KEM-768 wird zum einen durch \autoref{fig:Successful-Validation-Issuer-TLS1.3} demonstriert, welches die TLS~1.3 Verbindung zum Issuer-Agenten mit ML-DSA-65 signiertem Server-Zertifikat validiert, und zum anderen durch die Wallet-Übersicht in Listing~\ref{lst:Jupyter-Notebook-Cell-12-output} demonstriert, in welcher alle drei SSI-Agenten key\_type: "ml-dsa-65" für did:peer:4 DIDs nutzen.

\subsubsection{Implementierung hybrider Schlüsseleinigung}

Die BSI-Anforderung zur hybriden Schlüsseleinigung (Kombination klassisches Verfahren mit PQC-KEM) wird durch X25519+ML-KEM-768 Hybrid-Modus erfüllt. Das finale Artefakt implementiert hybride Schlüsseleinigung systemweit via \texttt{DEFAULT\_GROUPS=X25519MLKEM768:mlkem768:x25519} in allen Docker-Infrastruktur-Definitionen (Listing~\ref{lst:docker-compose.yml-DLT-Infrastruktur}, Listing~\ref{lst:docker-compose.yml-Revocation-Registry}, Listing~\ref{lst:docker-compose.yml-SSI-Agenten}), wobei die Priorisierung X25519MLKEM768 als primären Hybrid-Modus sicherstellt. Die TLS~1.3 Verbindungen aller fünf Nginx Sidecar Proxies kombinieren elliptische Kurven-Kryptographie (X25519, klassisch) mit gitterbasiertem ML-KEM-768 (Post-Quantum) für Perfect Forward Secrecy, wobei OpenSSL~3.5.4 mit nativer PQC-Unterstützung die kryptographische Verknüpfung beider Shared Secrets via Key Derivation Function durchführt (Listing~\ref{lst:nginx-pqc-konfiguration}). Zusätzlich implementiert did:peer:4 hybride Key Agreement zwischen X25519- und ML-KEM-768-Keys aus DID Documents für DIDComm Message Encryption (Listing~\ref{lst:key_types.py}). Die Fallback-Strategie \texttt{mlkem768:x25519} gewährleistet Interoperabilität mit Peers ohne Hybrid-Unterstützung, wobei reine PQC-Verschlüsselung via ML-KEM-768 Vorrang vor klassischem X25519 hat. Diese Architektur entspricht BSI-TR-02102-1 Kapitel~2.2 und 2.4 zur Absicherung gegen kryptanalytische Durchbrüche sowohl im klassischen als auch im Quantum-Computing-Bereich.

\subsubsection{Bevorzugte Verwendung von TLS 1.3}
  
Die BSI-Empfehlung zur vorrangigen Verwendung von TLS~1.3 wird durch systemweite TLS~1.3-Enforcement erfüllt. Das finale Artefakt erzwingt TLS~1.3 in allen fünf Nginx Sidecar Proxies via \texttt{ssl\_protocols TLSv1.3;} in den Konfigurationsdateien (Listing~\ref{lst:nginx_holder.conf}).

\begin{figure}[H]
    \centering
    \includegraphics[width=\linewidth]{summative_evaluation_TLS1.2_error.png}
    \caption{TLS 1.2 Verbindungsversuch zum Issuer-Agenten schlägt fehl (TLS 1.3 Enforcement)}
    \begin{flushleft}
    \textit{Anmerkung.} Eigene Darstellung.
    \end{flushleft}
    \label{fig:Summative_Evaluation_TLS1.2_Error}
\end{figure}

\autoref{fig:Summative_Evaluation_TLS1.2_Error} demonstriert das Fehlschlagen des TLS~1.2 Verbindungsversuchs zum Issuer-Agenten.

Für \gls{SzA}

\subsubsection{Protokollierung Sicherheitsrelevanter Ereignisse}

Das finale Artefakt implementiert Protokollierung auf drei Ebenen:

(1)~ACA-Py Agent-Level Logging protokolliert alle Authentifizierungsversuche (Connection Requests, Credential Issuance, Proof Presentations), Zustandsübergänge (State Machines für Issue-Credential/Present-Proof Protokolle) sowie Fehlerzustände via \texttt{--log-level info} in den Docker-Infrastruktur-Definitionen (Listing~\ref{lst:docker-compose.yml-SSI-Agenten}). Listing~\ref{lst:issuer-acapy-agent-logs} demonstriert das Logging sicherheitsrelevanter kryptographischer Events wie ML-DSA-65 Signature Verification, Schlüsselabruf via DID-Resolution und erfolgreiche/fehlgeschlagene DIDcomm-Decryption während des Credential-Issuance-Workflows.

(2)~Nginx Access \& Error Logs auf allen fünf Sidecar Proxies protokollieren TLS-Handshakes, HTTP-Requests sowie fehlgeschlagene Verbindungsversuche, persistiert in Docker Volumes \texttt{nginx-logs} (Listing~\ref{lst:docker-compose.yml-DLT-Infrastruktur}, Listing~\ref{lst:docker-compose.yml-Revocation-Registry}). Listing~\ref{lst:issuer-nginx-logs} demonstriert das Logging der HTTP-Security-Events auf Proxy-Ebene, Connection-Authentifizierung, Request-Authentifizierung via User-Agent, sowie Request-Body-Buffering für PQC-Schlüsselmaterialien.

(3)~VON-Network Ledger Transaction Logs erfassen alle Blockchain-Operationen (NYM-Transaktionen für DID-Registrierung, SCHEMA/CRED\_DEF-Publikationen, REVOC\_REG\_ENTRY für Revocations) mit Zeitstempeln und Sequence Numbers für unveränderlichen Audit-Trail (demonstriert in Listing~\ref{lst:Jupyter-Notebook-Cell-4-output}, Listing~\ref{lst:Jupyter-Notebook-Cell-21-output}).

\subsubsection{Logische Netzsegmentierung}

Das finale Artefakt implementiert strikte Netzwerk-Segmentierung mittels dedizierter Docker Networks: 

(1)~hopE-Agenten nutzen isolierte Networks \texttt{hope-issuer}, \texttt{hope-holder}, \texttt{hope-verifier} pro Agent, wobei nur die zugehörigen Sidecar Proxies Zugriff haben (Listing~\ref{lst:docker-compose.yml-SSI-Agenten}). 

(2)~Shared Network \texttt{von\_sidecarproxy} verbindet ausschließlich die PQC Sidecar Proxies untereinander für Agent-to-Agent Kommunikation, während interne Agent-Container isoliert bleiben. 

(3)~VON-Network Blockchain-Nodes operieren im dedizierten \texttt{von} Network mit separatem \texttt{sidecarproxy} Network für Webserver-Zugriff (Listing~\ref{lst:docker-compose.yml-DLT-Infrastruktur}). 

(4)~Tails Server nutzt isoliertes \texttt{tails-server} Network mit kontrolliertem Zugang via \texttt{von\_sidecarproxy} (Listing~\ref{lst:docker-compose.yml-Revocation-Registry}). 

\autoref{fig:Docker-Compose-Übersicht-Iteration-1} und \autoref{fig:Darstellung-Network-Isolation} demonstrieren die logische Netzwerksegmentierung.

\subsubsection{Datenschutz durch Technikgestaltung (Privacy by Design)}
% Die Architektur realisiert dies durch den konsequenten Verzicht auf die Speicherung personenbezogener Daten (PII) auf dem unveränderlichen Ledger (Off-Chain-Architektur) und die Nutzung von Zero-Knowledge Proofs

Die DSGVO-Anforderung zu Privacy by Design (Art.~25) wird durch architektonische Trennung von öffentlichen Identifikatoren und personenbezogenen Daten erfüllt. Das finale Artefakt realisiert konsequente Off-Chain-Architektur: (1)~Personenbezogene Daten (PII) wie Name, Organisation, Sicherheitsfreigabe werden ausschließlich in verschlüsselten Verifiable Credentials gespeichert, die lokal im Holder-Wallet persistiert sind (demonstriert in Listing~\ref{lst:Jupyter-Notebook-Cell-14-output}), niemals auf dem unveränderlichen Blockchain-Ledger. (2)~Ledger-Transaktionen enthalten ausschließlich kryptographische Identifikatoren (Schema-IDs, Credential-Definition-IDs, Revocation-Registry-IDs) sowie Public Keys, jedoch keine personenbezogenen Attribute (Listing~\ref{lst:Jupyter-Notebook-Cell-4-output}, Listing~\ref{lst:Jupyter-Notebook-Cell-21-output}). (3)~Zero-Knowledge Proofs ermöglichen selektive Offenlegung: Der Verifier erhält nur explizit angeforderte Attribute (\texttt{revealed\_attrs}), während sensible Daten wie Vorname, Nachname, Organisation durch \texttt{unrevealed}-Status geschützt bleiben, sowie Predicate-basierte Proofs (\texttt{security\_clearance\_level >= 2}) ohne Offenlegung exakter Werte (Listing~\ref{lst:Jupyter-Notebook-Cell-18-output-revocation}). Diese Architektur stellt Privacy by Default sicher, da personenbezogene Daten per Design dezentral beim Holder verbleiben und nur kryptographisch verifizierbare Proofs ausgetauscht werden, womit DSGVO Art.~25 Konformität erreicht wird.

\subsubsection{Grundsatz der Datenminimierung}
% Durch den Einsatz von \textit{Pairwise DIDs} (did:peer) für jede Interaktion statt einer globalen ID wird die Korrelierbarkeit von Daten minimiert und Profilbildung technisch unterbunden

Das finale Artefakt addressiert Datenminimierung in drei Dimensionen:

(1)~Pairwise did:peer:4 DIDs eliminieren globale Identifikatoren: Für jede Agent-to-Agent Connection wird eine dedizierte DID generiert (Listing~\ref{lst:Jupyter-Notebook-Cell-12-output} zeigt pro Agent 3--4 verschiedene did:peer:4 DIDs), wodurch Transaktionskorrelation über verschiedene Verifier hinweg technisch unterbunden wird -- ein kompromittierter Verifier kann keine Aktivitäten des Holders bei anderen Verifiern nachverfolgen. 

(2)~Selective Disclosure in Proof Presentations: Der Holder offenbart ausschließlich die vom Verifier angeforderten Attribute (\texttt{revealed\_attrs}: \texttt{cert\_type, facility\_type, epoch\_valid\_from/until, role}), während Identitätsdaten (\texttt{first\_name, name, organisation}) unrevealed bleiben (Listing~\ref{lst:Jupyter-Notebook-Cell-18-output-revocation}, Zeilen~12--19), wodurch nur zweckgebundene Minimaldaten übermittelt werden. 

(3)~Predicate-basierte Zero-Knowledge Proofs reduzieren Datenoffenlegung weiter: Statt exakte Sicherheitsfreigabe-Stufe zu übermitteln, beweist der Holder kryptographisch \texttt{security\_clearance\_level >= 2} ohne Preisgabe, ob Stufe 2 oder 3 vorliegt (Listing~\ref{lst:Jupyter-Notebook-Cell-18-output-revocation}, Zeilen~21--26). 

Diese mehrstufige Datenminimierung verhindert Profilbildung und unnötige Datensammlung, womit DSGVO-konforme Zweckbindung technisch durchgesetzt wird.

\subsubsection{Recht auf Löschung}
% Durch die strikte Trennung von öffentlichen Identifikatoren (Ledger) und privaten Daten (lokale Wallet) ist eine Löschung technisch vollständig realisierbar, indem die lokale Wallet und die zugehörigen kryptografischen Schlüssel vernichtet werden (Crypto-Shredding)

Die DSGVO-Anforderung zum Recht auf Löschung (Art.~17) wird durch strikte Trennung von Ledger-Identifikatoren und Wallet-Daten erfüllt. Das finale Artefakt realisiert vollständige Löschbarkeit personenbezogener Daten mittels Crypto-Shredding: (1)~Lokale Credential-Deletion entfernt Credentials aus dem Holder-Wallet via \texttt{DELETE /credential/\{credential\_id\}}, demonstriert in Listing~\ref{lst:Jupyter-Notebook-Cell-22}: Nach Deletion ist das Wallet leer (\texttt{Holder Wallet ist jetzt LEER}), womit alle personenbezogenen Attribute (Name, Organisation, Sicherheitsfreigabe) irreversibel gelöscht sind. (2)~Crypto-Shredding durch Wallet-Destruction: Da Credentials im Holder-Wallet mit dem Wallet-Key verschlüsselt persistiert sind (\texttt{holder\_wallet\_key}), führt die Vernichtung des Wallet-Keys zur kryptographischen Unlesbarkeit aller Credential-Daten, selbst wenn Backups existieren -- ohne Key ist Decryption unmöglich. (3)~Ledger-Immutabilität ohne Privacy-Verletzung: Während Blockchain-Transaktionen (Schema-IDs, Cred-Def-IDs, Revocation-Entries) unveränderlich auf dem Ledger verbleiben, enthalten diese per Design keine personenbezogenen Daten, sondern ausschließlich kryptographische Identifier, wodurch Art.~17 DSGVO erfüllt wird (Listing~\ref{lst:Jupyter-Notebook-Cell-22-output}, Zeilen~38--43: \texttt{Credential ist LOKAL im Wallet gelöscht, NICHT auf dem Ledger revoked}). Diese Architektur gewährleistet vollständige Löschung von PII bei gleichzeitiger Beibehaltung der Audit-Trail-Integrität für KRITIS-Compliance, womit die Balance zwischen DSGVO-Rechten und regulatorischen Anforderungen erreicht wird.

\subsection{Validierung der Kryptoagilität}

\subsubsection{Transportlayer}

Die Implementierung realisiert Kryptoagilität mithilfe der eingebauten Mechanismen in TLS 1.3, welche explizit entwickelt wurden, um eine modulare Austauschbarkeit kryptografischer Primitive zu gewährleisten. Im Gegensatz zu früheren Protokollversionen entkoppelt TLS 1.3 die Aushandlung von Cipher Suite, Schlüsselaustauschverfahren und Signaturalgorithmen vollständig voneinander. Diese Parameter werden orthogonal ausgehandelt, wodurch jeder Mechanismus unabhängig modifiziert werden kann \cite[S. 26]{rescorla_TransportLayerSecurityTLSProtocolVersion13_2018}.
Diese Trennung wird technisch durch die \texttt{supported\_groups}-Erweiterung realisiert, die es Endpunkten erlaubt, präferierte Schlüsselaustauschverfahren unabhängig vom symmetrischen Verschlüsselungsverfahren (AEAD) auszuhandeln \cite[S. 47]{rescorla_TransportLayerSecurityTLSProtocolVersion13_2018}.

Dieser Ansatz korrespondiert direkt mit den von \textcite[S. 102]{mehrez_CryptoAgilityProperties_2018} identifizierten Kryptoagilitätseigenschaften. Spezifisch adressiert die hier gewählte Architektur die Eigenschaft der \enquote{Extensibility}, die Fähigkeit, neue Algorithmen effizient hinzuzufügen, sowie die \enquote{Removability}, das elegante Außerbetriebnehmen veralteter Verfahren, ohne Gefährdung der Systemintegrität.

Die konkrete Umsetzung dieser Agilität in der vorliegenden Arbeit nutzt diese Protokollstruktur, um eine nahtlose Migration zu ermöglichen. Konkret wurde eine konfigurationsbasierte Algorithm-Fallback-Chain via \texttt{DEFAULT\_GROUPS=X25519MLKEM768:mlkem768:x25519:mlkem1024} (Listing~\ref{lst:Dockerfile-Sidecar-Proxy-nginx}) implementiert. Diese Konfiguration instruiert den TLS-Handshake, primär hybride Verfahren zu nutzen, bietet jedoch eine automatische Rückfalloption (Fallback) auf rein klassische Verfahren (\texttt{x25519}) im Falle einer Inkompatibilität. Damit erfüllt die Lösung die Anforderung der \enquote{Fungibility} nach \textcite[S. 102]{mehrez_CryptoAgilityProperties_2018}, die es Systemen erlaubt, Algorithmen auszutauschen. Ebenfalls wird die erste Eigenschaft nach \textcite[S. 19]{cyberresilienceworkshopseriescommittee_CryptographicAgilityInteroperabilityProceedingsWorkshop_2017} adressiert Algorithmen in Echtzeit basierend auf ihrer kombinierten Sicherheitsfunktion auszuwählen.

\autoref{fig:Summative_Evaluation_TLS1.3_Kryptoagilität} demonstriert den kryptoagilen Fallback-Prozess von X25519+ML-KEM-768 zu X25519 im TLS 1.3 Protokoll.

\begin{figure}[H]
    \centering
    \includegraphics[width=\linewidth]{summative_evaluation_TLS1.3_kryptoagilität.png}
    \caption{TLS 1.3 Kryptoagiler Fallback von X25519+ML-KEM-768 zu X25519}
    \begin{flushleft}
    \textit{Anmerkung.} Eigene Darstellung.
    \end{flushleft}
    \label{fig:Summative_Evaluation_TLS1.3_Kryptoagilität}
\end{figure}

\subsubsection{Applikationslayer}

Die Kryptoagilität des SSI-Systems auf Applikationsebene manifestiert sich in der Erfüllung der ersten und zweiten Anforderung von \textcite[S. 19--20]{cyberresilienceworkshopseriescommittee_CryptographicAgilityInteroperabilityProceedingsWorkshop_2017} Sicherheitsalgorithmen in Echtzeit auf Basis ihrer kombinierten Sicherheitsfunktionen auszuwählen, und die Möglichkeit zu eröffnen, neue kryptographische Funktionen bzw. Algorithmen in bestehende Hard- und Software zu integrieren.

Dafür nutzt das SSI-System die Plugin-Architektur von Aries Cloud Agent Python (ACA-Py), die es ermöglicht, bestehende kryptographische Workflows durch gezielte Eingriffe zu modifizieren, ohne den Kerncode der Agenten zu verändern. Dieses Designprinzip entspricht dem \enquote{Open/Closed Principle} der Softwareentwicklung, welches besagt, dass Softwaremodule offen für Erweiterungen, jedoch geschlossen für Modifikationen sein sollten \cite[S. 99]{martin_AgileSoftwareDevelopmentprinciplespatternspractices_2003}.

Dafür implementiert das Plugin einen metadatengesteuerten Ansatz zur Algorithmenauswahl. Wie in Listing~\ref{lst:Jupyter-Notebook-Cell-9-Demonstration-Kryptoagilität-ed25519} demonstriert, kann der gewünschte Schlüsseltyp durch Übergabe des Parameters \enquote{metadata: \{key\_type : ed25519\}} explizit spezifiziert werden.

Bei der Verarbeitung einer Out-of-Band (OOB) Invitation durchläuft das System folgenden Entscheidungsbaum: Der \enquote{OutOfBandManager} empfängt die Metadata-Parameter aus der API-Anfrage und propagiert diese an die DID-Erstellungslogik. In der Funktion \enquote{create\_did\_peer\_4\_conditional\_pqc} (Listing~\ref{lst:base_manager_patch.py}) erfolgt eine Auswertung des \enquote{key\_type}-Parameters. Wird der Wert \enquote{ed25519} erkannt, delegiert das Plugin die gesamte DID-Erstellung an die ursprüngliche ACA-Py-Implementierung (\enquote{\_original\_create\_did\_peer\_4}), wodurch ein vollständiger Fallback auf klassische Kryptographie ohne Plugin-Interferenz gewährleistet wird. Fehlt die Metadata-Spezifikation aktiviert das System standardmäßig die Post-Quantum-Kryptographie mit den Algorithmen ML-DSA-65 für digitale Signaturen und ML-KEM-768 für Schlüsselkapselung.

Die erfolgreiche Validierung dieser Kryptoagilität zeigt Listing~\ref{lst:Jupyter-Notebook-Cell-9-Demonstration-Kryptoagilität-ed25519-output}. Trotz aktiviertem Plugin etabliert sich eine vollständig ED25519-basierte Verbindung zwischen Issuer- und Holder-Agent. Die Verbindung erreicht den Status \enquote{active} auf beiden Seiten, was die korrekte Durchführung des DID Exchange-Protokolls mit klassischen Algorithmen bestätigt.

Listing~\ref{lst:Jupyter-Notebook-Cell-11-Demonstration-Kryptoagilität-ed25519-output} verifiziert die persistierte Kryptographie-Konfiguration auf Wallet-Ebene. Die Inspektion der gespeicherten DIDs zeigt konsistent den Wert \enquote{{key\_type}:{ed25519}}. Insbesondere die Abwesenheit von PQC-spezifischen Metadata-Markern wie \enquote{pqc\_enabled}, \enquote{signature\_algorithm} oder \enquote{kem\_verkey} in den DID-Metadaten bestätigt, dass das Plugin keinerlei modifizierende Eingriffe in den ED25519-Workflow vorgenommen hat.

% \newpage
% \section{Systemarchitektur und Design} \label{sec:Systemarchitektur und Design}
\subsection{Anforderungsanalyse} \label{sec:Anforderungsanalyse}

- Funktionale Anforderungen aus den Forschungsfragen
- Nicht-funktionale Anforderungen (Performance, Sicherheit)
- KRITIS-spezifische Compliance-Anforderungen
\subsection{Framework- und Technologie-Evaluation} \label{sec:Evaluation und Auswahl des SSI-Frameworks}

\subsubsection{Evaluationsmethodik und -kriterien} \label{Evaluationsmethodik und -kriterien}
\subsubsection{SSI-Framework-Vergleich} \label{SSI-Framework-Vergleich}
\subsubsection{Blockchain-Plattform-Auswahl} \label{Blockchain-Plattform-Auswahl}
\subsubsection{Technologie-Stack-Integration} \label{Technologie-Stack-Integration}
\subsubsection{Auswahlentscheidungen und Begründung} \label{Auswahlentscheidungen und Begründung}

- Kriterien für die Framework-Auswahl (Open Source, PQC-Kompatibilität) \\
- Vergleich und Entscheidung (z.B. Hyperledger Indy/Aries) \\
- Anpassungen und Erweiterungen für die ermittelten Anforderungen


Infrastructure ==> indy on besu \\
Identifier + Cryptographic material ==> Indy DID \\
Credential \& Presentations ==> AnonCreds W3C VCs \\
Agent ==> ACA-Py


- Open Source
- Modular aufgebaut sein


F

Begründung der Kategorien und wissenschaftliche Quellen
Architektur/Blockchain

    Die Wahl der Architektur bestimmt Skalierbarkeit, Transaktionsmodell, Dezentralität und die regulatorische Einwirkbarkeit (permissioned vs. permissionless). Frameworks wie Hyperledger Indy, Besu und ION werden in der Literatur gezielt nach Blockchain-Typ, Governance und Infrastruktur verglichen.

Quelle: „Self-sovereign identity on the blockchain: contextual analysis and ...“ (Frontiers in Blockchain, 2024); Härer \& Fill, 2020.
Offenheit \& Transparenz

    Open-Source-Frameworks ermöglichen Auditierbarkeit, Anpassung und Community-getriebene Innovation – zentral für Trust und Security in KRITIS.

Quelle: Frontiers in Blockchain, 2024; Fill \& Härer, 2020.
Interoperabilität

    Die Fähigkeit eines Systems, über W3C-DID-, VC- und andere Standards mit verschiedensten Ökosystemen und Registern zu kommunizieren, wird als Schlüsselanforderung klassifiziert.

Quelle: Fraunhofer \& Universität Bayreuth Diskussionspapier zu SSI, 2020; Grüner et al. „Analyzing Interoperability and Portability Concepts...“ (HPI).
SSI-Prinzipien-Compliance

    Die 10 Grundprinzipien nach Christopher Allen (Existenz, Kontrolle, Interoperabilität, Transparenz, Portabilität, etc.) sind das wissenschaftliche Fundament für die Bewertung von SSI-Systemen und werden in praktisch jedem SSI-Vergleich herangezogen.

Quelle: Christopher Allen (Blog; 2017), überführt in diverse Whitepaper und Publikationen, z.B. Societybyte/SFI.
PQC-Unterstützung

    Post-Quanten-Kryptografie ist ein verpflichtender Baustein für KRITIS-Sicherheit – das BSI und Fraunhofer empfehlen die Integration quantenresistenter Algorithmen speziell für Identitäts- und Signaturlösungen.

Quelle: BSI, „Post-Quanten-Kryptografie“; Fraunhofer Cybersecurity Blog, „Post-Quanten-Kryptografie in der Praxis“.
Eignung für KRITIS

    Die Tauglichkeit für kritische Infrastrukturen erfordert geprüfte Governance-, Compliance- und Skalierungsmechanismus, wie vom BSI und in wissenschaftlichen Evaluationsframeworks für Public-Key-Infrastruktur und Identitätssysteme festgelegt.

Quelle: ETH Zürich, „Evaluierungs-Framework und Kriterienkatalog für Public-Key-Infrastrukturen“; Fraunhofer FIT (SSI \& KRITIS Fokus).


\begin{longtable}{L{1cm}L{2cm}L{2.5cm}L{2cm}L{1.5cm}L{1.5cm}L{1.5cm}}
    \caption{Vergleich ausgewählter SSI-Frameworks für blockchain-basierte, KRITIS-taugliche Prototypen mit PQC-Perspektive}
    \label{tab:ssi-frameworks} \\
    \toprule
    \textbf{Framework} & \textbf{Architektur/\newline Blockchain} & \textbf{Offenheit \& Transparenz} & \textbf{Interoperabilität} & \textbf{SSI-Prinzipien-\newline Compliance} & \textbf{PQC-\newline Unterstützung} & \textbf{Eignung für KRITIS} \\
    \midrule
    \endfirsthead
    \multicolumn{7}{l}{\textit{Tabelle \thetable\ (Fortsetzung)}} \\
    \toprule
    \textbf{Framework} & \textbf{Architektur/\newline Blockchain} & \textbf{Offenheit \& Transparenz} & \textbf{Interoperabilität} & \textbf{SSI-Prinzipien-\newline Compliance} & \textbf{PQC-\newline Unterstützung} & \textbf{Eignung für KRITIS} \\
    \midrule
    \endhead
    \midrule
    \multicolumn{7}{r}{\textit{Fortsetzung auf nächster Seite}} \\
    \endfoot
    \bottomrule
    \multicolumn{7}{p{\linewidth}}{\textit{Anmerkung.} Quellen: Eigene Darstellung nach \cite{frontiersssi2024,lfbesublog2025,marketanalysis2025}} \\
    \endlastfoot
    Hyperledger Indy &
    Federated Indy Blockchain &
    Open Source, starker Audit-Trail &
    W3C VC/DID, hohe Interoperabilität &
    Sehr hoch (Existenz, Kontrolle, Transparenz) &
    Experimentell (kein nativer PQC-Support) &
    Sehr hoch (Governance, Auditierbarkeit) \\
    \midrule
    ION (Microsoft/DIF) &
    Bitcoin, Sidetree DPKI &
    Open Source, breite Transparenz &
    Dezentral, weltweit, W3C DIDs &
    Hoch (Portabilität, Kontrolle, Skalierung) &
    Erweiterbar, keine native PQC &
    Mittel (abhängig von Bitcoin, schwer reg. steuerbar) \\
    \midrule
    Indy on Besu &
    Besu Ethereum (Permissioned) &
    Open Source, modular, Enterprise-tauglich &
    W3C VC/DID, did:indy:besu, Legacy-Migration &
    Sehr hoch (volle SSI-Rollen, flexible Governance) &
    PQC-Addons via Smart Contracts, nativ in Entwicklung &
    Sehr hoch (10x Durchsatz, Trusted List, flexible Governance) \\
\end{longtable}

indy besu ==> \parencite{shcherbakov_HyperledgerIndyBesupermissionedledgerSelfsovereignIdentity_2024}




% \subsection{Systemarchitektur} \label{sec:Systemarchitektur}

% - Gesamtarchitektur des SSI-Prototyps basierend auf ausgewählten Frameworks \\
% - Komponentendiagramm, Schnittstellen und Erweiterungspunkte \\
% - Integration von PQC-Algorithmen \\
% (- Integration der ausgewählten Technologien) 



% \subsection{Kryptoagiles Design} \label{sec:Kryptoagiles Design}

% - Algorithmus-Abstraktion und Flexibilität \\
% - Update-Mechanismen für zukünftige PQC-Standards
% \subsection{Blockchain-Integration} \label{sec:Blockchain-Integration}

% - Auswahl der Blockchain-Plattform \\
% - Smart Contract Design für DID und VC \\
% - Konsensverfahren und Governance

% 04_Implementierung des Prototypen                 // 12-15 Seiten
% \newpage
% \section{Implementierung des Prototypen} \label{sec:Implementierung des Prototypen}
% \subsection{Technologie-Stack und Entwicklungsumgebung} \label{sec:Technologie-Stack und Entwicklungsumgebung}

% - Laborumgebung Setup (Hyper-V, Ubuntu 24.04 LTS, Docker, Github) \\
% - Entwicklungstools und Frameworks
\subsection{SSI-Kern-Komponenten} \label{sec:SSI-Kern-Komponenten}

- Kryptografische Schlüsselverwaltung
- Dezentrale Identifikatoren (DIDs)
- Verifiable Credentials (VCs)
- Verifiable Presentations (VPs)
- Wallet (Agenten-Software)
- Credential Issuer, Holder, Verifier (Rollenmodell)
\subsection{Blockchain-Backend} \label{sec:Blockchain-Backend}

- Distributed Ledger (Blockchain/DLT-Layer)
- Konsensmechanismus
- DID-Registry / DID-Storage
- VC-Schema- und Definitions-Registry
- Revocation Registry
- Zugriffs- und Berechtigungsmanagement (Governance Layer)
\subsection{PQC-Algorithmen-Integration} \label{sec:PQC-Algorithmen-Integration}

- Schlüsselerzeugung und -verwaltung (SSI \& Blockchain) \\
- Digitale Signaturen (SSI \& Blockchain) \\
- Verschlüsselung von Daten (SSI \& Blockchain) \\
- Blockchain-Transaktionen \& Konsensmechanismus (Blockchain) \\
- Smart Contracts (Blockchain) \\
- Verifiable Credentials \& Präsentationen (SSI) \\
- DID-Methoden und DID-Dokumente (SSI) \\
- Agenten-Software (Wallets, Clients) (SSI) \\
- Credential Issuer, Holder, Verifier (SSI) \\
- Umsetzung der Kryptoagilität (SSI \& Blockchain)

% 05_Evaluation und Validierung                     // 08-10 Seiten
\newpage
\section{Evaluation und Validierung} \label{sec:Evaluation und Validierung}

Use Cases? ==> \parencite[S. 130]{nokhbehzaeem_BlockchainBasedSelfSovereignIdentitySurveyRequirementsUseCasesComparativeStudy_2021}
% \subsection{Evaluationsmethodik} \label{sec:Evaluationsmethodik}

% - FEDS-Framework Anwendung \\
% - Metriken und Bewertungskriterien
\subsection{Funktionalitätstests} \label{sec:Funktionalitätstests}

- Identity Lifecycle Management Szenarien \\
- Use Case Validierung
\subsection{Performance-Analyse} \label{sec:Performance-Analyse}

- Durchsatz und Latenz-Messungen
- Speicher- und Rechenaufwand
- Skalierbarkeitsanalyse
% \subsection{Sicherheitsbewertung} \label{sec:Sicherheitsbewertung}

% - Kryptografische Stärke \\
% - Angriffsvektoren und Resilience \\
% - Compliance-Validierung

% 06_Ergebnisse und Diskussion                      // 06-08 Seiten
\newpage
\section{Ergebnisse und Diskussion} 
\label{sec:Ergebnisse und Diskussion}

\subsection{Beantwortung der Forschungsfragen} 
\label{sec:Beantwortung der Forschungsfragen}

Die vorliegende Arbeit adressiert durch Design Science Research Methodologie drei zentrale Forschungsfragen an der Schnittstelle von Self-Sovereign Identity, Post-Quantum Cryptography und kritischen Infrastrukturen. Die in Kapitel 6 durchgeführte summative Evaluation unter Anwendung des FEDS-Frameworks demonstriert, dass das entwickelte Artefakt diese Fragen nicht nur theoretisch sondern empirisch durch kontrollierte Evaluation beantwortet.

\textbf{Forschungsfrage FF1}: \enquote{Wie kann ein blockchain-basiertes SSI-System unter Einsatz von Post-Quantum-Kryptografie gestaltet werden, um die regulatorischen und technischen Anforderungen von kritischen Infrastrukturen nachhaltig zu erfüllen?}

Diese Forschungsfrage wird durch die Konzeption und Implementierung einer zweischichtigen Architektur beantwortet, die quantenresistente Kryptografie orthogonal und nicht-invasiv in bestehende SSI-Infrastrukturen integriert. Die erste Schicht realisiert Quantensicherheit auf der Transportebene (Data-In-Motion) mittels eines Sidecar-Proxy-Patterns mit nginx, das TLS 1.3 als Protokoll-Standard durchsetzt und ein hybrides Schlüsseleinigungsverfahren (X25519 + ML-KEM-768) implementiert. Parallel dazu werden digitale Zertifikate mit ML-DSA-65-Signaturen ausgestellt und validiert, was die Post-Quantum-Authentifizierung von Netzwerk-Endpunkten sicherstellt. Diese nicht-invasive Architektur erfüllt die KRITIS-Anforderung nach Backward-Kompatibilität: SSI-Agenten (ACA-Py), Blockchain-Knoten (Hyperledger Indy) und Wallet-Applikationen benötigen keine Modifikation ihrer Anwendungslogik, da die Quantensicherung transparent auf der Netzwerk-Infrastruktur-Ebene erfolgt.

Die zweite Schicht erweitert die Quantensicherheit auf die Applikationsebene (Data-At-Rest und Credential-Verarbeitung) durch native PQC-Integration in Hyperledger Aries Cloud Agent Python. Mittels einer Monkey-Patching-Strategie werden kritische Komponenten überschrieben: die didpeer4-Verarbeitung ermöglicht es, Decentralized Identifiers mit ML-DSA-65-Signaturen zu erzeugen und zu validieren; die DIDComm-Envelope-Verarbeitung unterstützt ML-KEM-768-basierte Schlüsselkapseln für die Nachrichtenverschlüsselung. Diese duale Implementierung adressiert das zentrale KRITIS-Sicherheitsprinzip der Defense-in-Depth: falls eine Schicht kompromittiert wird, bietet die andere weiterhin Schutz.

Die Validierung aller neun Compliance-Anforderungen (CR1 bis CR9) in Kapitel 6.2 demonstriert empirisch, dass die Architektur sowohl die kryptografischen Parameter des Bundesamts für Sicherheit in der Informationstechnik (ML-DSA-65 und ML-KEM-768 gemäß TR-02102) als auch die organisatorischen Anforderungen der Datenschutz-Grundverordnung erfüllt. Insbesondere werden die Anforderungen zur strikten Netzsegmentierung (CR6), zur Protokollierung sicherheitsrelevanter Ereignisse (CR5) und zur Datenminimierung durch Privacy-by-Design (CR7) vollständig operationalisiert. Das Artefakt demonstriert somit, dass quantenresistente Kryptografie nicht nur ein technisches Problem, sondern ein systemisches Problem ist, das Compliance-, Architektur- und Governance-Aspekte simultan adressieren muss.

\textbf{Forschungsfrage FF2}: \enquote{Welche Post-Quantum-Cryptography-Algorithmen eignen sich für die Integration in Self-Sovereign-Identity-Systemen hinsichtlich Sicherheit, Praktikabilität und Interoperabilität im Kontext kritischer Infrastrukturen?}

Diese Forschungsfrage wird durch die empirische Validierung der NIST-standardisierten Algorithmen ML-DSA-65 (Crystals-Dilithium mit Security-Kategorie 3) und ML-KEM-768 (Crystals-Kyber mit Security-Kategorie 3) beantwortet. Die bewusste Auswahl dieser Parameter-Sets entspricht der NIST Security Strength Category 3, die vom Bundesamt für Sicherheit in der Informationstechnik für Zielvorgaben im KRITIS-Kontext akzeptiert ist und gleichzeitig moderate Schlüsselgrößen ermöglicht: ML-DSA-65 erzeugt Signaturen von 2944 Bytes, ML-KEM-768 Ciphertexte von 1088 Bytes. Diese Größen sind für die Kodierung in didpeer4-Strukturen und DIDComm-v1-Envelopes praktikabel, ohne die Netzwerk-Overhead problematisch zu erhöhen.

Die empirische Evaluation in Kapitel 6.1 und 6.2 zeigt, dass diese Algorithmen nicht primär durch intrinsische Sicherheitsdefizite limitiert sind --- beide Verfahren werden von NIST empfohlen und ihre Sicherheitsreduktionen auf gut-verstandene mathematische Probleme sind rigoros --- sondern durch technische Integrationschallenges in bestehende Aries-Infrastrukturen, die originär auf Elliptic-Curve-Cryptography ausgelegt sind. Diese Infrastruktur-Kompatibilität adressiert die Arbeit durch drei komplementäre Mechanismen:

Erstens ermöglicht TLS 1.3-Algorithmen-Aushandlung es Endpunkten, zwischen hybriden Cipher-Suites und rein klassischen Verfahren zu fallback-en, was die Schrittweise-Migration für heterogene Deployments vereinfacht. Zweitens detektiert die DIDComm-v1-Envelope-Verarbeitung anhand von Schlüssellängen automatisch, ob PQC oder klassische Algorithmen erforderlich sind, wodurch PQC-fähige Agenten mit Legacy-Agenten interagieren können. Drittens schafft die Multikey-Kodierung mittels Multicodec-Registry die kryptographische Identifikation neuer Algorithmen ohne Schema-Änderungen im DID-Dokument.

Die Hybrid-Strategie, die klassische und quantenresistente Algorithmen kombiniert, ist ein Best-Practice für die aktuelle Übergangswelt: Die Kombination von X25519 (ECC, etabliert sicher) mit ML-KEM-768 (PQC, mathematisch neu) garantiert, dass die Gesamtsicherheit mindestens so stark wie der stärkere der beiden Algorithmen ist.

\textbf{Forschungsfrage FF3}: \enquote{Welche kryptografischen Agilität-Mechanismen sind erforderlich und praktikabel, um zukünftige Post-Quantum-Cryptography-Algorithmenupdates ohne Systemunterbrechung zu ermöglichen?}

Diese Forschungsfrage wird durch zwei orthogonale, schichtenspezifische Agilität-Implementierungen beantwortet. Auf der Transportebene realisiert TLS 1.3 Kryptoagilität durch eine fundamentale Separation of Concerns: Cipher-Suite, Schlüsselaustauschverfahren und Signaturalgorithmen werden unabhängig voneinander gehandelt. Eine konfigurationsbasierte Fallback-Chain in der nginx-Konfiguration ermöglicht es, neue Algorithmen durch einfache Textdatei-Änderung auszurollen, ohne nginx neu zu kompilieren oder Systemkomponenten zu neustarten. Dies erfüllt die Extensibility- und Removability-Kriterien und ist in Kapitel 6.3.1 empirisch demonstriert.

Auf der Applikationsebene implementiert das PQC-Plugin-System Agilität durch mehrere ineinander verschachtelte Abstraktions-Ebenen: Die Kryptografie-Abstraktionsschicht kapselt liboqs-Operationen hinter einer Python-FFI, wodurch zukünftige Versionen von liboqs ohne Änderung des Plugin-Interfaces integriert werden können. Die DID-Verarbeitungsschicht nutzt Multicodec-Präfixe für Algorithmen-Identifikation, die W3C-Standard konform erweiterbar sind. Die Monkey-Patching-Integration wird durch generische Patch-Methoden realisiert, die kritische ACA-Py-Funktionen überschreiben, ohne das Framework-Quellcode zu modifizieren.

Die empirische Validation in Kapitel 6.3.2 demonstriert exemplarisch einen kryptoagilen Fallback-Mechanismus: wenn ein X25519+ML-KEM-768-Schlüsselaustausch fehlschlägt, fallback-t der Server automatisch auf rein X25519, ohne die Verbindung abzubrechen.

\subsection{Kritische Reflexion} 
\label{sec:Kritische Reflexion}

\textbf{Methodische Stärken und Gewährleistung von Rigor:} Die Wahl des Design Science Research Frameworks kombiniert mit dem Framework for Evaluation in Design Science (FEDS) ermöglicht es, technische Risiken isoliert in einer kontrollierten Laborumgebung zu reduzieren, bevor naturalistischeFieldtests erwogen werden. Dies ist im KRITIS-Kontext nicht nur wissenschaftlich, sondern ethisch essentiell: experimentelle kryptographische Prototypen dürfen nicht ohne Risikovalidierung in produktiven Infrastrukturen mit Gemeinwohlverantwortung deployiert werden.

Die iterative Artefakt-Entwicklung mit zwei Iterationen ermöglicht es, frühe Design-Fehler in der first-order-Iteration zu identifizieren, bevor die second-order-Iteration aufgebaut wird. Dies realisiert eine sukzessive Komplexitätsreduktion und schafft frühzeitige Lerngewinne, die in die Designrefinements der zweiten Iteration einfließen. Die explizite Dokumentation des Evaluationsprozesses durch detaillierte Jupyter-Notebooks und strukturierte Log-Analysen erfüllt hohe Anforderungen an Reproduzierbarkeit und wissenschaftliche Transparenz. Das FEDS-Framework hat dabei insbesondere die bewusste Priorisierung auf Technical Risk erlaubt, wodurch die Evaluation nicht durch soziale oder organisatorische Konfounding-Faktoren überlagert wurde, die für diese frühe Forschungsphase noch verfrüht wären.

\textbf{Limitationen und deren Bedeutung für Generalisierbarkeit:} Der zentrale Limitation dieser Arbeit liegt in der bewussten Abgrenzung zwischen artificialem und naturalistischem Evaluationssetting. Die Evaluation erfolgt in einer isolierten docker-compose-Umgebung, die keine Netzwerk-Latenzen, Paketverluste oder asynchrone Fehlerszenarien simuliert, die in produktiven KRITIS-Netzen auftreten. Dies bedeutet, dass die gemessene Efficacy für das laborhafte Setup validiert ist, aber nicht notwendigerweise für dezentralisierte, geographisch verteilte KRITIS-Netzwerke mit WAN-Latenzen.

Die Blockchain-Infrastruktur wird mit vier lokalen Validator-Nodes betrieben, nicht mit dem typischen Produktions-Setting von zehn bis zwanzig geographisch verteilten Nodes. Dies beeinflusst die Konsensus-Latenz nicht fundamental, erlaubt aber nicht, realistische Blockierungszeiten oder Timeout-Szenarien zu evaluieren. Die SSI-Agenten laufen allesamt im selben Docker-Network mit lokal auflösbarem DNS, was die Netzwerk-Segmentierung-Evaluation vereinfacht. Ferner wurde die Evaluation ohne Last-Tests durchgeführt, weshalb Fragen zur Performance-Degradation unter Hochlast bewusst ausgegrenzt werden. Diese Limitationen sind Boundaries of the Evaluation, nicht Fehler.

\textbf{Trade-offs bei kritischen Design-Entscheidungen:} Die Wahl des Monkey-Patching-Ansatzes stellt einen bewussten Trade-off zwischen kurzfristiger Wartbarkeit und langfristiger Maintenance-Sicherheit dar. Monkey-Patching erlaubt es, das originale ACA-Py-Framework unverändert zu belassen, reduziert aber die Versionskompabilität mit zukünftigen ACA-Py-Releases. Ein offizieller Fork hätte langfristige Maintenance-Sicherheit geboten, erfordert aber erhebliche organisatorische Koordination. Für einen Research-Prototypen ist Monkey-Patching pragmatisch; für produktive Deployments wäre ein Upstream-Merge mittelfristig erstrebenswert.

Die Hybrid-Cryptography-Strategie erhöht die Komplexität des Vertrauensmodells erheblich. Ein Verifier muss entscheiden, welche Algorithmus-Kombinationen akzeptabel sind, was nicht-trivialen Governance-Overhead in KRITIS-Betreiberverbänden erzeugt.

Die Verwendung provisorischer Multicodec-Präfixe ist funktional, aber nicht W3C-standardisiert. Eine formale Registrierung würde eine sauberere, zukunftsfeste Lösung darstellen, benötigt aber externe Koordination mit der W3C Decentralized Identifiers Community Group.

Ein weiterer Trade-off betrifft die Signaturverifikation von Credentials: die Revocation Registry nutzt noch klassische ED25519-Signatures für CL-Accumulator-Operationen, um Abhängigkeiten zu minimieren. Eine vollständige PQC-Revocation-Registry hätte zusätzliche Entwicklung in Indy-Tails-Server erfordert.

\textbf{Einordnung in den Standardisierungs- und Technologie-Kontext:} Der zeitliche Kontext dieser Arbeit fällt in eine kritische Übergangsphase der Post-Quantum-Cryptography-Standardisierung. Die verwendeten Algorithmen wurden von NIST 2022 ausgewählt, 2024 formalisiert und als FIPS-Standards akzeptiert, sind aber noch nicht in allen industriellen Systemen vollständig integriert. Die in dieser Arbeit gewählten Versionen sind State-of-the-Art (Stand Dezember 2025), werden sich aber in kommenden Monaten ändern. Die Evaluation ist daher am Snapshot der Technologie-Maturität gültig; für Releases 2026 und darüber hinaus wird eine Wiederholung wertvoll sein.

Ferner befinden sich zentrale SSI-Standards noch in aktiver Entwicklung. Die Arbeit implementiert bewusst den stabilen DIDComm v1.0 Standard, was keine Breaking-Changes zu erwarten sind. Eine Migration zu DIDComm v2.0 wird bei dessen Finalisierung erforderlich sein.

\subsection{Wissenschaftliche und praktische Beiträge}
\label{sec:Wissenschaftliche und praktische Beiträge}

\textbf{Theoretischer Beitrag --- Konzeptuelle Integration dreier Forschungsdomänen:} Diese Arbeit leistet einen ersten umfassenden Beitrag zur Integration von drei bislang isoliert betrachteten Forschungsdomänen --- Self-Sovereign Identity, Post-Quantum Cryptography und kritische Infrastrukturen --- in einem kohärenten und empirisch validierten Systemdesign. Die existierende Literatur adressiert typischerweise zwei der drei Domänen; die Formulierung einer ganzheitlichen, quanten-sicheren SSI-Architektur speziell für KRITIS-Kontexte war bislang nicht konsistent durchgeführt.

Ein kernales Erkenntnisergebnis ist die Einsicht, dass quantenresistente Kryptografie nicht als eine bloße Cryptography-Substitution verstanden werden darf, sondern als ein Architektur-Problem, das fundamentale Fragen aufwirft: Wie integriert man größere Schlüssel ohne Netzwerk-Overhead? Wie gewährleistet man Interoperabilität zwischen PQC-fähigen und Legacy-Systemen? Wie realisiert man Kryptoagilität? Wie wahrt man Privacy-by-Design bei veränderten Signatur-Algorithmen? Diese Fragen erfordern spezifische technische Lösungsmuster: Sidecar Proxies für transparente Transport-Layer-Sicherung, Monkey-Patching oder Upstream-Integration für Applikations-Layer-Agilität, Multicodec-Extensibilität und sorgfältige Governance-Modelle.

Ein weiterer theoretischer Beitrag ist die Formulierung von schichtenspezifischer Kryptoagilität. Während Transport-Layer-Agilität durch TLS 1.3 relativ trivial ist, erfordert Applikations-Layer-Agilität komplexe Rückwärtskompatibilität in DIDComm-Envelope-Verarbeitung, Credential-Verifikation und Revocation-Mechanismen. Dies ist in klassischen Kryptosystemen nicht in diesem Ausmaß erforderlich.

\textbf{Praktischer Beitrag --- Reproduzierbarer Prototyp mit Migrations-Roadmap:} Das entwickelte Artefakt besteht aus Sidecar-Proxy-Templates, ACA-Py-Plugin-Code, vollständiger Docker-Compose-Orchestrierung und ausführlichen Jupyter-Notebooks und ist vollständig open-source zugänglich. Die Dokumentation basierend auf realistischen KRITIS-Szenarios ermöglicht es, die Technologie-Anwendbarkeit unabhängig zu bewerten.

Die modular aufgebaute Architektur erlaubt Partial-Adoptions: eine Organisation könnte nur die Sidecar-Proxy-Layer deployieren, um sofort ihre bestehende SSI-Infrastruktur quantensicher zu machen, und später zur Applikations-Layer-Integration übergehen. Dies adressiert eine zentrale Praktiker-Anforderung: die Vermeidung von Big-Bang-Deployments und die Ermöglichung gradueller, reversibel-testbarer Übergänge.

\textbf{Methodischer Beitrag --- DSR-Anwendung auf Cybersecurity-Probleme:} Die systematische Anwendung von Design Science Research auf komplexe Cybersecurity-Probleme ist noch nicht vollständig im deutschsprachigen Raum verankert. Diese Arbeit demonstriert durch konkrete Operationalisierung, dass DSR mit FEDS-Framework zu neuartigen, empirisch validierten Erkenntnissen führt, die durch reine theoretische Analyse oder ad-hoc-Prototyping nicht erreichbar sind.

Die Zwei-Iterationen-Zyklik illustriert, wie formative Evaluationen zu identifizierten Design-Defiziten und notwendigen Refinements führen, die dann summativ validiert werden --- ein evidenzbasiertes Designprinzip für kritische Infrastrukturen. Die iterative Entwicklung erfolgt nicht linear nach Wasserfall, sondern als epistemische Spirale mit Rücksprängen: formative Evaluationen der ersten Iteration identifizierten Inkompatibilitäten in der CL-Accumulator-Verarbeitung, die zur Design-Entscheidung für Monkey-Patching führte, was wiederum neue Evaluationsfragen aufwarf. Diese Zirkularität ist ein strukturelles Feature, das sicherstellt, dass Design und Evaluation sich gegenseitig kontinuierlich informieren.

\textbf{Konsequenzen für Standardisierung und Governance:} Die Erkenntnisse haben unmittelbare Implikationen für laufende Standardisierungsprozesse bei W3C Decentralized Identifiers und BSI-Richtlinien. Insbesondere wie Multicodec-Präfixe registriert werden, wie KRITIS-Governance-Modelle mit Hybrid-Algorithmen-Anforderungen umgehen, und wie Liability bei PQC-Zertifikaten organisiert wird, erfordert expliziter Industrie-Konsens, den diese Arbeit durch empirische Demonstrationen informiert.

\subsection{Beantwortung der Forschungsfragen} \label{sec:Beantwortung der Forschungsfragen}

- FF1: Systemarchitektur \& Compliance
- FF2: Algorithmenauswahl \& Sicherheitsbewertung
- FF3: Performance \& Skalierbarkeit
- FF4: Kryptografische Agilität
\subsection{Kritische Reflexion} \label{sec:Kritische Reflexion}

- Limitationen der Laborumgebung \\
- Übertragbarkeit auf reale KRITIS-Umgebungen \\
- Trade-offs zwischen Sicherheit und Performance
\subsection{Wissenschaftliche und praktische Beiträge} \label{sec:Wissenschaftliche und praktische Beiträge}

- Gestaltungsprinzipien für quantenresistente SSI
- Implementierungsrichtlinien

% 07_Fazit und Ausblick                             // 04-05 Seiten
\newpage
\section{Fazit und Ausblick} \label{sec:Fazit und Ausblick}




\fixme{Zusammengefasst leistet diese Arbeit einen wertvollen Pionierbeitrag, indem sie zeigt, dass PQC und Kryptoagilität in SSI-Systemen nicht nur wünschenswert, sondern bereits heute praktisch realisierbar sind. Die identifizierten Architekturgrenzen und Interoperabilitätsherausforderungen sind weniger als Fehler, sondern als produktive Indikatoren für Bereiche zu interpretieren, in denen Standardisierungsgremien und Framework-Entwickler ihre Arbeit vertiefen müssen. Der Prototyp fungiert dadurch als Katalysator für eine notwendige Diskussion über die technische Zukunftsfähigkeit dezentraler Identitätssysteme im Quantenzeitalter.}
\subsection{Zusammenfassung der Ergebnisse} \label{sec:Zusammenfassung der Ergebnisse}

- Zentrale Erkenntnisse und Zielerreichung
\subsection{Implikationen für Forschung und Praxis} \label{sec:Implikationen für Forschung und Praxis}

- Bedeutung und Anwendbarkeit der Resultate
\subsection{Zukünftige Forschungsrichtungen} \label{sec:Zukünftige Forschungsrichtungen}

- Offene Fragen und Forschungsperspektiven skizziert


%-----------------------------------
% Apendix / Anhang
%-----------------------------------
\newpage
\section*{\AppendixName}
\addcontentsline{toc}{section}{\AppendixName}

\begin{appendices}

\setcounter{manualListingCounter}{0}
\renewcommand{\thelstlisting}{Listing A-\arabic{manualListingCounter}}
\renewcommand{\theHlstlisting}{Listing A-\arabic{manualListingCounter}} % für hyperref


\setcounter{figure}{0}
\renewcommand{\thefigure}{A-\arabic{figure}}
\renewcommand{\theHfigure}{A-\arabic{figure}} % für hyperref


\renewcommand{\thetable}{A-\arabic{table}}
\renewcommand{\theHtable}{A-\arabic{table}} % für hyperref
\setcounter{table}{0}

\makeatletter
\renewcommand\@seccntformat[1]{\csname the#1\endcsname:\quad}
\makeatother

\addtocontents{toc}{\protect\setcounter{tocdepth}{0}}%
\renewcommand{\thesection}{\AppendixName\ \arabic{section}}
\renewcommand\thesubsection{\AppendixName\ \arabic{section}.\arabic{subsection}}
\renewcommand\thesubsubsection{\AppendixName\ \arabic{section}.\arabic{subsection}.\arabic{subsubsection}}

% >>> ANHANGS-TOC 
\listofappendices
\newpage

% WICHTIG: NUR IM APPENDIX die Befehle umleiten
\let\section\appsection
\let\subsection\appsubsection
\let\subsubsection\appsubsubsection

% Anhang einbinden
\section{Systematische Literaturrecherche}
\label{sec:Anhang_Dokumentation der ersten Iteration der systematischen Literaturrecherche (Exposé)}

\subsection{Erste Iteration (Exposé)}

\fixme{Texte anpassen? + Kapitelüberschriften ergänzen? Anhang 1.1 etc.?}

Die systematische Literaturrecherche zum Stand der Forschung orientiert sich am \enquote{iterative Review-Ansatz} nach \textcite[S. 208--209]{brocke_StandingShouldersGiantsChallengesRecommendationsLiteratureSearchInformationSystemsResearch_2015}, der mit einer initialen Recherche startet und sich iterativ vertieft. Für die Struktur und Dokumentation sind ausgewählte Methoden der \ac{PRISMA} 2020 Richtlinien zugrunde gelegt (\autoref{tab:Ausgewählte Methoden der PRISMA 2020 Richtlinien}). \ac{PRISMA} gewährleistet hierbei einen transparenten, reproduzierbaren Prozess und verbessert die Berichtqualität \parencite[S. 1, 6]{page_PRISMA2020Statementupdatedguidelinereportingsystematicreviews_2021}.

\begin{longtable}{L{0.3\textwidth}L{0.7\textwidth}}
    \caption{Ausgewählte Methoden der PRISMA 2020 Richtlinien}
    \label{tab:Ausgewählte Methoden der PRISMA 2020 Richtlinien} \\
    \toprule
    \textbf{Methode} & \textbf{Beschreibung} \\
    \midrule
    \endfirsthead
    \multicolumn{2}{l}{\textit{Tabelle \thetable\ (Fortsetzung)}} \\
    \toprule
    \textbf{Methode} & \textbf{Beschreibung} \\
    \midrule
    \endhead
    \midrule
    \multicolumn{2}{r}{\textit{Fortsetzung auf nächster Seite}} \\
    \endfoot
    \bottomrule
    \multicolumn{2}{p{\linewidth}}{\textit{Anmerkung.} In Anlehnung an \textcite[S. 4]{page_PRISMA2020Statementupdatedguidelinereportingsystematicreviews_2021}.} \\
    \endlastfoot
    Ein- und Ausschlusskriterien & Ein- und Ausschlusskriterien für die Überprüfung. \\
    \midrule
    Suchstrategie & Vollständigen Suchstrategie für alle Datenbanken, Websites einschließlich aller verwendeten Filter. \\
    \midrule
    Selektionsprozess & Methoden an, die verwendet wurden, um zu entscheiden, ob eine Studie die Einschlusskriterien der Überprüfung erfüllt. \\
\end{longtable}


\begin{figure}[H]
    \centering
    \includegraphics[width=\paperwidth]{PRISMA_2020_flow_diagram_new_SRs_v1_ITERATION_1.png}
    \caption{PRISMA 2020 Flussdiagramm - erste Iteration}
    \begin{flushleft}
    \textit{Anmerkung.} In Anlehnung an \textcite[S. 5]{page_PRISMA2020Statementupdatedguidelinereportingsystematicreviews_2021}.
    \end{flushleft}
    \label{fig:PRISMA_Flussdiagramm_iteration1}
\end{figure}

\fixme{\autoref{fig:PRISMA_Flussdiagramm_iteration1} beschreibt den Prozess der systematischen Literaturrecherche in der ersten Iteration.}

\pagebreak

Die in \autoref{tab:einausschlusskriterien} dargestellten Ein- und Ausschlusskriterien gewährleisten eine transparente, nachvollziehbare und zielgerichtete Auswahl relevanter wissenschaftlicher Quellen.

\begin{longtable}{L{3cm}L{4cm}L{4cm}L{3cm}}
    \caption{Ein- und Ausschlusskriterien für die systematische Literaturrecherche}
    \label{tab:einausschlusskriterien} \\
    \toprule
    \textbf{Kategorie} & \textbf{Einschluss} & \textbf{Ausschluss} & \textbf{Begründung} \\
    \midrule
    \endfirsthead
    \multicolumn{4}{l}{\textit{Tabelle \thetable\ (Fortsetzung)}} \\
    \toprule
    \textbf{Kategorie} & \textbf{Einschluss} & \textbf{Ausschluss} & \textbf{Begründung} \\
    \midrule
    \endhead
    \midrule
    \multicolumn{4}{r}{\textit{Fortsetzung auf nächster Seite}} \\
    \endfoot
    \bottomrule
    \multicolumn{4}{p{\linewidth}}{\textit{Anmerkung.} Eigene Darstellung.} \\
    \endlastfoot
    Thematischer Fokus &
    \ac{SSI} und dezentrale Identitätslösungen;
    Blockchain-basierte Identitätsmanagementsysteme;
    \ac{PQC} und quantensichere Algorithmen;
    Sicherheit und Compliance in \ac{KRITIS};
    \ac{DSR}-Methodik in IT/Informationssystemen;
    Kryptoagilität und kryptografische Migration & 
    Identitätsmanagement ohne Bezug zu \ac{SSI} oder Blockchain;
    Klassische \ac{PKI} ohne \ac{PQC}-Bezug;
    Kryptografie ohne Post-Quantum-Relevanz;
    Arbeiten ohne Bezug zu \ac{KRITIS} oder ohne sicherheitskritischen Kontext;
    Nicht-\ac{DSR}-basierte Entwicklungsansätze & 
    Fokussierung auf die Forschungsfragen und relevante technologische, methodische und regulatorische Aspekte. \\
    \midrule
    Zeitrahmen & 2015 bis heute & Vor 2015 & Berücksichtigung aktueller technologischer Entwicklungen (Blockchain, \ac{PQC}, \ac{SSI}) und regulatorischer Anforderungen. \\
    \midrule
    Publikationstypen & 
    Peer-reviewed Journalartikel;
    Konferenzbeiträge anerkannter Fachgesellschaften (z. B. IEEE, ACM, IFIP); Preprints;
    Offizielle Standards und Empfehlungen (z. B. \ac{NIST}, W3C, \ac{BSI});
    Whitepaper etablierter Organisationen;
    Dissertationen und anerkannte Fachbücher & 
    Blogposts, Forenbeiträge, Marketingmaterial;
    Populärwissenschaftliche Artikel ohne wissenschaftliche Fundierung;
    Unveröffentlichte Manuskripte ohne Peer-Review;
    Seminar- und Abschlussarbeiten ohne wissenschaftliche Begutachtung & 
    Sicherstellung wissenschaftlicher Qualität, Nachvollziehbarkeit und Relevanz der Quellen für die Masterarbeit; Da das Forschungsthema aktuell noch sehr neu ist und die einschlägige Fachliteratur teilweise noch nicht den Peer-Review-Prozess durchlaufen hat, werden auch Preprints in die Analyse einbezogen. Preprints ermöglichen eine zeitnahe Verfügbarkeit aktueller Forschungsergebnisse, was insbesondere bei diesem innovativen von großer Bedeutung ist. \\
    \midrule
    Sprache & Deutsch; Englisch & Andere Sprachen als Deutsch und Englisch & Gewährleistung der Verständlichkeit und Zugänglichkeit für den deutsch- und englischsprachigen Forschungskontext. \\
    \midrule
    Zugänglichkeit & Verfügbare Volltexte & Nicht verfügbare Volltexte & Ermöglichung einer gründlichen Analyse und Bewertung der Inhalte. \\
\end{longtable}

\autoref{tab:suchstrategie} stellt ausgewählte methodische Schritte zur Entwicklung einer Suchstrategie nach \textcite[S. 532]{bramer_SystematicApproachSearchingefficientcompletemethoddevelopliteraturesearches_2018} dar, an denen sich diese Seminararbeit orientiert.

\begin{longtable}{L{0.1\textwidth}L{0.9\textwidth}}
    \caption{Überblick über die Entwicklung der Suchstrategie}
    \label{tab:suchstrategie} \\
    \toprule
    \textbf{Schritt} & \textbf{Beschreibung} \\
    \midrule
    \endfirsthead
    \multicolumn{2}{l}{\textit{Tabelle \thetable\ (Fortsetzung)}} \\
    \toprule
    \textbf{Methode} & \textbf{Beschreibung} \\
    \midrule
    \endhead
    \midrule
    \multicolumn{2}{r}{\textit{Fortsetzung auf nächster Seite}} \\
    \endfoot
    \bottomrule
    \multicolumn{2}{p{\linewidth}}{\textit{Anmerkung.} In Anlehnung an \textcite[S. 532]{bramer_SystematicApproachSearchingefficientcompletemethoddevelopliteraturesearches_2018}.} \\
    \endlastfoot
    1  & Identifikation relevanter Schlüsselkonzepte \\
    \midrule
    2  & Identifikation relevanter Keywords \\
    \midrule 
    3  & Erstellung einer strukturierten Suchanfrage mit Booleschen Operatoren \\
    \midrule
    4  & Auswahl geeigneter Datenbanken \\
    \midrule
    5  & Übersetzung der Suchanfrage für verschiedene Datenbanken \\
\end{longtable}

\paragraph*{Identifikation relevanter Schlüsselkonzepte}

Im ersten Schritt wird eine präzise Identifikation und Abgrenzung zentraler Schlüsselkonzepte vorgenommen (\autoref{tab:Abgrenzung zentraler Schlüsselkonzepte}).

\begin{longtable}{L{4cm}L{10cm}}
    \caption{Abgrenzung zentraler Schlüsselkonzepte}
    \label{tab:Abgrenzung zentraler Schlüsselkonzepte} \\
    \toprule
    \textbf{Schlüsselkonzept} & \textbf{Erläuterung} \\
    \midrule
    \endfirsthead
    \multicolumn{2}{l}{\textit{Tabelle \thetable\ (Fortsetzung)}} \\
    \toprule
    \textbf{Schlüsselkonzept} & \textbf{Erläuterung} \\
    \midrule
    \endhead
    \midrule
    \multicolumn{2}{r}{\textit{Fortsetzung auf nächster Seite}} \\
    \endfoot
    \bottomrule
    \multicolumn{2}{p{\linewidth}}{\textit{Anmerkung.} Basierend auf \textcite[S. 2]{solavagione_TransitionSelfSovereignIdentityPostQuantumCryptography_2025}, \textcite{nationalinstituteofstandardsandtechnologyus_ModulelatticebasedKeyencapsulationMechanismstandard_2024,nationalinstituteofstandardsandtechnologyus_ModulelatticebasedDigitalSignaturestandard_2024,nationalinstituteofstandardsandtechnologyus_StatelessHashbasedDigitalsignaturestandard_2024}, \textcite{bundesministeriumderjustiz_GesetzUeberBundesamtfuerSicherheitInformationstechnikBSIGesetzBSIG_2009}, \textcite{hevner_DesignScienceInformationsystemsresearch_2004}.} \\
    \endlastfoot
    Self-Sovereign Identity & 
    Das Paradigma der selbstbestimmten digitalen Identität, das Nutzenden die Kontrolle über ihre Identitätsdaten und deren Weitergabe ermöglicht, basierend auf dezentralen Technologien wie Blockchain und interoperablen Standards wie \ac{DID} ,\ac{VC} und \ac{VP} \parencite[S. 2]{solavagione_TransitionSelfSovereignIdentityPostQuantumCryptography_2025}. \\
    \midrule
    Blockchain-Technologie & 
    Die Nutzung von \ac{DLT} zur Sicherstellung von Integrität, Transparenz und Manipulationssicherheit im Identitätsmanagement, insbesondere im Kontext von \ac{SSI}-Systemen \parencite[S. 2]{solavagione_TransitionSelfSovereignIdentityPostQuantumCryptography_2025}. \\
    \midrule
    Post-Quantum Kryptografie & 
    Kryptografische Verfahren, die auch gegen Angriffe durch leistungsfähige Quantencomputer resistent sind, einschließlich aktueller Standardisierungsansätze (z. B. \ac{NIST} FIPS 203, 204, 205) \parencite{nationalinstituteofstandardsandtechnologyus_ModulelatticebasedKeyencapsulationMechanismstandard_2024,nationalinstituteofstandardsandtechnologyus_ModulelatticebasedDigitalSignaturestandard_2024,nationalinstituteofstandardsandtechnologyus_StatelessHashbasedDigitalsignaturestandard_2024} und Empfehlungen zur kryptoagilen Systemgestaltung. \\
    \midrule
    Kritische Infrastrukturen & 
    Sektoren und Systeme, deren Funktionsfähigkeit essenziell für das Gemeinwesen ist und die daher besonders hohe Anforderungen an Sicherheit, Compliance und Resilienz stellen \parencite{bundesministeriumderjustiz_GesetzUeberBundesamtfuerSicherheitInformationstechnikBSIGesetzBSIG_2009}. \\
    \midrule
    Design Science Research & 
    Die methodische Grundlage zur systematischen Entwicklung, Evaluation und wissenschaftlichen Fundierung innovativer IT-Artefakte nach \textcite{hevner_DesignScienceInformationsystemsresearch_2004} im Kontext der genannten Technologien und Anwendungsdomänen. \\
\end{longtable}

\paragraph*{Identifikation relevanter Keywords}

Basierend auf den zuvor definierten Schlüsselkonzepten wurden die wichtigsten Suchbegriffe in Deutsch und Englisch sowie die gängigen Akronyme zusammengestellt (\autoref{tab:keywords der schlüsselkonzepte}), um die technologische, methodische und regulatorische Breite der Recherche optimal abzudecken.

\begin{longtable}{L{4cm}L{10cm}}
    \caption{Keywords der Schlüsselkonzepte}
    \label{tab:keywords der schlüsselkonzepte} \\
    \toprule
    \textbf{Schlüsselkonzept} & \textbf{Keywords} \\
    \midrule
    \endfirsthead
    \multicolumn{2}{l}{\textit{Tabelle \thetable\ (Fortsetzung)}} \\
    \toprule
    \textbf{Schlüsselkonzept} & \textbf{Keywords} \\
    \midrule
    \endhead
    \midrule
    \multicolumn{2}{r}{\textit{Fortsetzung auf nächster Seite}} \\
    \endfoot
    \bottomrule
    \multicolumn{2}{p{\linewidth}}{\textit{Anmerkung.} Eigene Darstellung.} \\
    \endlastfoot
    Self-Sovereign Identity &
    Self-Sovereign Identity, \ac{SSI}, dezentrale Identität, digitale Identität, decentralized identity, user-controlled identity, identity wallet, Verifiable Credentials, \ac{VC}, Decentralized Identifiers, \ac{DID}, Identity Management, \ac{IdM}, Identity Access Management, \ac{IAM} \\
    \midrule
    Blockchain-Technologie &
    Blockchain, Distributed Ledger Technology, \ac{DLT}, Distributed Ledger, Smart Contract, Ethereum, Hyperledger, IOTA, Permissioned Ledger, Consensus Mechanism, On-Chain Identity \\
    \midrule
    Post-Quantum Kryptografie &
    Post-Quantum Cryptography, \ac{PQC}, quantensichere Verschlüsselung, quantum-resistant cryptography, Lattice-based Cryptography, Hash-based Signature, Code-based Cryptography, Multivariate Cryptography, Module-Lattice-based Digital Signature, ML-DSA, Module-Lattice-based Key Encapsulation Mechanism, ML-KEM, Stateless Hash-based Digital Signature, SLH-DSA, CRYSTALS-Dilithium, SPHINCS+, Kryptoagilität, Cryptographic Agility, Algorithm Agility, Migration Strategy, Cryptographic Migration, Flexible Cryptography Update \\ 
    \midrule
    Kritische Infrastrukturen &
    Kritische Infrastrukturen, KRITIS, Critical Infrastructure Protection, CIP, sector-specific security requirements, Compliance, Privacy by Design, Privacy by Default, Data Protection, Regulatory Requirements, Bundesamt für Sicherheit in der Informationstechnik, BSI, European Union Agency for Cybersecurity, ENISA, eIDAS \\ 
    \midrule
    Design Science Research &
    Design Science Research, \ac{DSR}, Design Science Research Methodology, DSRM, Artefact Development, Evaluation of Artefacts, Method-driven Development, Iterative Design Process, Research Methodology, Framework for Evaluation in Design Science, \ac{FEDS}, Preferred Reporting Items for Systematic Reviews and Meta-Analyses, \ac{PRISMA} \\
\end{longtable}

\pagebreak

\paragraph*{Erstellung einer strukturierten Suchanfrage mit Booleschen Operatoren}

Listing~\ref{lst:boolesche_suchanfrage} stellt eine strukturierte Suchstrategie mit Booleschen Operatoren auf Basis der identifizierten Keywords dar. Die Suchanfrage verbindet zentrale Schlüsselkonzepte und nutzt gezielt Synonyme und Akronyme zur Abdeckung verschiedener Terminologien und Schreibweisen.
\newline

\refstepcounter{manualListingCounter}
\label{lst:boolesche_suchanfrage}
\begin{lstlisting}[
caption={Listing \arabic{lstlisting}: Strukturierte Suchanfrage mit Booleschen Operatoren},
basicstyle=\small\ttfamily,
breaklines=true,
frame=single,
language=SQL
]
(
(
"self-sovereign identity" OR SSI OR "decentralized identity" OR "dezentrale Identität" OR "digitale Identität" OR "user-controlled identity" OR "identity wallet" OR "verifiable credentials" OR VC OR "decentralized identifiers" OR DID OR "identity management" OR IdM OR "identity access management" OR IAM
)
AND
(
blockchain OR "distributed ledger technology" OR DLT OR "distributed ledger" OR "smart contract" OR Ethereum OR Hyperledger OR IOTA OR "permissioned ledger" OR "consensus mechanism" OR "on-chain identity"
)
AND
(
"post-quantum cryptography" OR PQC OR "quantensichere Verschlüsselung" OR "quantum-resistant cryptography" OR "lattice-based cryptography" OR "hash-based signature" OR "code-based cryptography" OR "multivariate cryptography" OR "module-lattice-based digital signature" OR ML-DSA OR "module-lattice-based key encapsulation mechanism" OR ML-KEM OR "stateless hash-based digital signature" OR SLH-DSA OR CRYSTALS-Dilithium OR SPHINCS+ OR kryptoagilität OR "cryptographic agility" OR "algorithm agility" OR "migration strategy" OR "cryptographic migration" OR "flexible cryptography update"
)
AND
(
"kritische Infrastrukturen" OR KRITIS OR "critical infrastructure protection" OR CIP OR "sector-specific security requirements" OR compliance OR "privacy by design" OR "privacy by default" OR "data protection" OR "regulatory requirements" OR "bundesamt für sicherheit in der informationstechnik" OR BSI OR "european union agency for cybersecurity" OR ENISA OR eIDAS
)
AND
(
"design science research" OR DSR OR "design science research methodology" OR DSRM OR "artefact development" OR "evaluation of artefacts" OR "method-driven development" OR "iterative design process" OR "research methodology" OR "framework for evaluation in design science" OR FEDS OR "preferred reporting items for systematic reviews and meta-analyses" OR PRISMA
)
)
\end{lstlisting}

\paragraph*{Auswahl einer geeigneten Datenbank}

Die Wahl auf \gls{EBSCO} als Datenbank für die systematische Literaturrecherche resultiert aus der Existenz von Campuslizenzen der FOM für diese Datenbank.

\paragraph*{Übersetzung der Suchanfrage für EBSCO}

Die strukturierten Suchanfrage mit den Booleschen Operatoren kann für
EBSCO ohne Anpassungen übernommen werden. Das Ergebnis der EBSCO Suchanfrage
ist in \autoref{fig:EBSCO Ergebnis} dargestellt. Insgesamt wurden mit dieser Abfrage 61 Quellen identifiziert.

\begin{figure}[H]
    \centering
    \includegraphics[width=\paperwidth, height=\paperheight, keepaspectratio, angle=90]{EBSCO.png}
    \caption{Ergebnis der EBSCO Suchanfrage}
    \begin{flushleft}
    \textit{Anmerkung.} Eigene Darstellung.
    \end{flushleft}
    \label{fig:EBSCO Ergebnis}
\end{figure}

\paragraph*{Bewertung der Ergebnisse}

\autoref{tab:quellenuebersicht} stellt eine Übersicht der Bewertung der 61 identifizierten Quellen dar, welche vollständig in \ref{sec:Bewertung der identifizierten Quellen hinsichtlich ihrer Relevanz} aufzufinden ist.

Die Einstufung basiert auf Titel und Abstract in Bezug auf den thematischen Fokus der Arbeit. Hohe Relevanz erhalten Quellen mit klaren Beiträgen zu \ac{SSI}, \ac{PQC}, \ac{KRITIS} oder dezentralen Identitätsarchitekturen. Mittlere Relevanz wird Arbeiten zugeordnet, die angrenzende Technologien wie Blockchain-Sicherheit im \ac{IoT} oder digitale Forensik behandeln. Niedrige Relevanz erhalten Quellen zu allgemeinen Technologietrends ohne direkten Bezug zum Thema.

\begin{longtable}{L{1.5cm}L{11cm}L{1cm}}
    \caption[]{Übersicht der Bewertung der identifizierten Quellen hinsichtlich ihrer Relevanz}
    \label{tab:quellenuebersicht} \\
    \toprule
    \textbf{Nr.} & \textbf{Quelle} & \textbf{Relevanz} \\
    \midrule
    \endfirsthead
    \multicolumn{3}{l}{\textit{Tabelle \thetable\ (Fortsetzung)}} \\
    \toprule
    \textbf{Nr.} & \textbf{Quelle} & \textbf{Relevanz} \\
    \midrule
    \endhead
    \midrule
    \multicolumn{3}{r}{\textit{Fortsetzung auf nächster Seite}} \\
    \endfoot
    \bottomrule
    \multicolumn{3}{p{\linewidth}}{\textit{Anmerkung.} Basierend auf \autoref{tab:quellenbewertung} und Titel und Abstracts von \textcite{szymanski_QuantumSafeSoftwareDefinedDeterministicInternetThingsIoTHardwareEnforcedCyberSecurityCriticalInfrastructures_2024,nouma_TrustworthyEfficientDigitalTwinsPostQuantumEraHybridHardwareAssistedSignatures_2024,sharif_EIDASRegulationSurveyTechnologicalTrendsEuropeanElectronicIdentitySchemes_2022,alam_PrivatelyGeneratedKeyPairsPostQuantumCryptographyDistributedNetwork_2024,radanliev_ReviewComparisonUSEUUKRegulationsCyberRiskSecurityCurrentBlockchainTechnologies_2023}.} \\
    \endlastfoot
    1 & Szymanski, T. H. (2024). A Quantum-Safe Software-Defined Deterministic Internet of Things (IoT) with Hardware-Enforced Cyber-Security for Critical Infrastructures. Information (2078-2489), 15(4), 173. \url{https://doi.org/10.3390/info15040173} & Hoch \\
    \midrule
    2 & Nouma, S. E., \& Yavuz, A. A. (2024). Trustworthy and Efficient Digital Twins in Post-Quantum Era with Hybrid Hardware-Assisted Signatures. ACM Transactions on Multimedia Computing, Communications \& Applications, 20(6), 1–30. \url{https://doi.org/10.1145/3638250} & Hoch \\
    \midrule
    3 & Sharif, A., Ranzi, M., Carbone, R., Sciarretta, G., Marino, F. A., \& Ranise, S. (2022). The eIDAS Regulation: A Survey of Technological Trends for European Electronic Identity Schemes. Applied Sciences (2076-3417), 12(24), 12679. \url{https://doi.org/10.3390/app122412679} & Hoch \\
    \midrule
    4 & Alam, M., Hoffstein, J., \& Cambou, B. (2024). Privately Generated Key Pairs for Post Quantum Cryptography in a Distributed Network. Applied Sciences (2076-3417), 14(19), 8863. \url{https://doi.org/10.3390/app14198863} & Hoch \\
    \midrule
    5 & Radanliev, P. (2023). Review and Comparison of US, EU, and UK Regulations on Cyber Risk/Security of the Current Blockchain Technologies: Viewpoint from 2023. Review of Socionetwork Strategies, 17(2), 105–129. \url{https://doi.org/10.1007/s12626-023-00139-x} & Hoch \\
    \midrule
    6--38 & Diverse & Mittel  \\
    \midrule
    39--61 & Diverse & Niedrig \\
\end{longtable}

\subsubsection{Ergebnis Iteration 1}
\label{sec:Bewertung der identifizierten Quellen hinsichtlich ihrer Relevanz}

Iteration 1 - Bewertung der identifizierten Quellen hinsichtlich ihrer Relevanz

\begin{longtable}{L{0.5cm}L{4cm}L{1.5cm}L{7cm}}
    \caption[]{Bewertung der identifizierten Quellen hinsichtlich ihrer Relevanz}
    \label{tab:quellenbewertung} \\
    \toprule
    \textbf{Nr.} & \textbf{Quelle} & \textbf{Relevanz} & \textbf{Kommentar} \\
    \midrule
    \endfirsthead
    \multicolumn{4}{l}{\textit{Tabelle \thetable\ (Fortsetzung)}} \\
    \toprule
    \textbf{Nr.} & \textbf{Quelle} & \textbf{Relevanz} & \textbf{Kommentar} \\
    \midrule
    \endhead
    \midrule
    \multicolumn{4}{r}{\textit{Fortsetzung auf nächster Seite}} \\
    \endfoot
    \bottomrule
    \multicolumn{4}{p{\linewidth}}{\textit{Anmerkung.} Basierend auf den Abstracts aller in Spalte zwei unter \enquote{Quelle} aufgeführten Quellenangaben.} \\
    \endlastfoot
1 & Szymanski, T. H. (2024). A Quantum-Safe Software-Defined Deterministic Internet of Things (IoT) with Hardware-Enforced Cyber-Security for Critical Infrastructures. Information (2078-2489), 15(4), 173. \url{https://doi.org/10.3390/info15040173} & Hoch & Fokussiert auf die Entwicklung quantensicherer Kommunikations- und Sicherheitssysteme im Kontext von \ac{KRITIS} und Industrial IoT; adressiert explizit \ac{PQC} durch Einsatz quantensicherer Verschlüsselungsmechanismen und QKD-Netzwerke; behandelt hardwarebasierte Zugriffskontrollen, Zero Trust Architekturen und AI-gestützte Sicherheit, die für Resilienz und Sicherheitsanforderungen in \ac{KRITIS} maßgeblich sind; zwar steht \ac{SSI} und Blockchain-Technologie nicht im Mittelpunkt, jedoch zeigen die vorgestellten innovativen Konzepte und die experimentelle Validierung einen sehr hohen Anwendungs- und Erkenntniswert für den methodischen und technologischen Fortschritt in mindestens zwei zentralen Domänen (\ac{PQC}, \ac{KRITIS}); damit bietet der Beitrag substanzielle Impulse für den Schutz hochsensibler digitaler Infrastrukturen. \\
\midrule
2 & Nouma, S. E., \& Yavuz, A. A. (2024). Trustworthy and Efficient Digital Twins in Post-Quantum Era with Hybrid Hardware-Assisted Signatures. ACM Transactions on Multimedia Computing, Communications \& Applications, 20(6), 1–30. \url{https://doi.org/10.1145/3638250} & Hoch & Betont die Notwendigkeit verlässlicher und effizienter digitaler Signaturen im Kontext von Digital Twins, die primär auf IoT-Infrastrukturen für hochsensible Daten abzielen und damit eine wichtige Schnittmenge zu \ac{KRITIS} darstellen; adressiert explizit \ac{PQC} durch die Entwicklung und Umsetzung quantensicherer und hybrider Signaturlösungen—einschließlich Forward Security und Aggregation, die auch für Blockchain-basierte und dezentrale Identitätsanwendungen (insb. mit Skalierungsbedarf) hoch relevant sind; der methodische Fortschritt im Bereich hardwareunterstützter, ressourcenschonender Kryptografie bietet substanzielle Innovationsimpulse für sicherheitskritische Systeme mit beschränkten Ressourcen, wie sie für \ac{SSI}-Lösungen und die sichere Verwaltung digitaler Identitäten in \ac{KRITIS} essenziell sind; Blockchain-Technologie und \ac{SSI} werden nicht explizit vertieft, jedoch ist die Übertragbarkeit der vorgestellten Konzepte—insbesondere hybride und aggregierbare Signaturen—auf beide Domänen methodisch und praxisnah gegeben. \\
\midrule
3 & Sharif, A., Ranzi, M., Carbone, R., Sciarretta, G., Marino, F. A., \& Ranise, S. (2022). The eIDAS Regulation: A Survey of Technological Trends for European Electronic Identity Schemes. Applied Sciences (2076-3417), 12(24), 12679. \url{https://doi.org/10.3390/app122412679} & Hoch & Adressiert zentrale Entwicklungen und Herausforderungen europäischer elektronischer Identitätssysteme im Kontext der eIDAS-Regulierung; analysiert technologische Trends und ihre Auswirkungen auf Sicherheit, Datenschutz und Interoperabilität nationaler eID-Lösungen, was unmittelbar an die Domäne \ac{SSI} und deren regulatorisches Umfeld anschließt; behandelt aktuelle Technologiestandards wie OAuth 2.0, SAML und OpenID Connect, ohne explizit Blockchain- oder \ac{PQC}-Lösungen zu integrieren, beleuchtet jedoch die (in eIDAS 2.0 antizipierte) Entwicklung hin zu dezentralisierten Identitätsarchitekturen, die als Grundlage künftiger \ac{SSI}-Lösungen dienen; liefert wesentliche Erkenntnisse für die Ausgestaltung sicherer und interoperabler digitaler Identitäten in \ac{KRITIS} und gibt Impulse für die technologische und methodische Weiterentwicklung nationalübergreifender Identitätsverwaltung. \\
\midrule
4 & Alam, M., Hoffstein, J., \& Cambou, B. (2024). Privately Generated Key Pairs for Post Quantum Cryptography in a Distributed Network. Applied Sciences (2076-3417), 14(19), 8863. \url{https://doi.org/10.3390/app14198863} & Hoch & Fokussiert auf die praktische Erzeugung, Verteilung und Verifikation privat generierter post-quanten-sicherer Schlüsselpaaren in verteilten Netzwerken, mit expliziter Anwendung von Crystals-Dilithium als \ac{PQC}-Algorithmus; adressiert wesentlich die Domäne \ac{PQC} durch Integration und Umsetzung eines aktuellen Standards, was sowohl für die Sicherheit verteilter Infrastrukturen als auch für zukünftige Identitätslösungen (z.B. im Kontext von \ac{SSI}) zentral ist; Berücksichtigung von Multi-Faktor-Authentifizierung, Challenge-Response-Mechanismen und biometrielosen Verfahren bietet substanzielle methodische Impulse für die Entwicklung von sicheren, dezentralen Schlüsselmanagement- und Authentifizierungslösungen; direkte Einbindung in Blockchain- oder spezifische \ac{KRITIS} wird nicht explizit diskutiert, ist aufgrund des Protokoll- und \ac{PKI}-Fokus jedoch technisch anschlussfähig und für die Domänen \ac{SSI} und \ac{KRITIS} innovativ und relevant. \\
\midrule
5 & Radanliev, P. (2023). Review and Comparison of US, EU, and UK Regulations on Cyber Risk/Security of the Current Blockchain Technologies: Viewpoint from 2023. Review of Socionetwork Strategies, 17(2), 105–129. \url{https://doi.org/10.1007/s12626-023-00139-x} & Hoch & Vergleichende Analyse der US-, EU- und UK-Regulierung im Bereich Cyber-Risiken und Sicherheit aktueller Blockchain-Technologien, basierend auf dem Stand 2023; systematische Prüfung und Gegenüberstellung führender Standardwerke wie \ac{NIST} (US), ISO27001 (international), und neueren Regularien wie MiCA (EU) und CPMI-IOSCO, unter Einbeziehung technologieübergreifender Aspekte (u.a. \ac{PQC}, Cloud Security, IoT); liefert bedeutende Einblicke und unmittelbaren Anwendungsbezug für die Bewertung, Weiterentwicklung und Integration internationaler Cybersecurity-Standards in neue Blockchain-Projekte—insbesondere mit Blick auf die vier Domänen moderner Identitäts- und Sicherheitssysteme \\
\midrule
6 & Enaya, A., Fernando, X., \& Kashef, R. (2025). Survey of Blockchain-Based Applications for IoT. Applied Sciences (2076-3417), 15(8), 4562. \url{https://doi.org/10.3390/app15084562} & Mittel & Betrachtet zentrale Aspekte von Blockchain-Technologien und Sicherheit; Schwerpunkt auf IoT-Anwendungen und branchenspezifische Implementierungen; explizite Bezüge zu \ac{SSI}, \ac{PQC} und \ac{KRITIS} fehlen; breiter Überblick, jedoch geringere Spezifizität hinsichtlich der vier Domänen der Masterarbeit ( \ac{SSI}, Blockchain, \ac{PQC}, \ac{KRITIS}). \\
\midrule
7 & Siam, M. K., Saha, B., Hasan, M. M., Hossain Faruk, M. J., Anjum, N., Tahora, S., Siddika, A., \& Shahriar, H. (2025). Securing Decentralized Ecosystems: A Comprehensive Systematic Review of Blockchain Vulnerabilities, Attacks, and Countermeasures and Mitigation Strategies. Future Internet, 17(4), 183. \url{https://doi.org/10.3390/fi17040183} & Mittel & Systematische Analyse von Schwachstellen, Angriffsszenarien und Gegenmaßnahmen in Blockchain-Ökosystemen; konzentriert sich auf Sicherheitsaspekte von Blockchain-Technologien ohne spezifische Betrachtung von \ac{SSI}, \ac{PQC} oder \ac{KRITIS}; hoher inhaltlicher Wert für das allgemeine Verständnis von Blockchain-Sicherheit, jedoch eingeschränkte Anwendbarkeit auf alle vier Domänen der Masterarbeit. \\
\midrule
8 & Ramirez Lopez, L. J., \& Morillo Ledezma, G. G. (2025). Employing Blockchain, NFTs, and Digital Certificates for Unparalleled Authenticity and Data Protection in Source Code: A Systematic Review. Computers (2073-431X), 14(4), 131. \url{https://doi.org/10.3390/computers14040131} & Mittel & Fokussiert auf Blockchain-basierte Technologien zur Sicherung von Authentizität und Zugriffskontrolle, jedoch im Anwendungskontext akademischer Quellcode-Sicherheit; behandelt primär NFTs und digitale Zertifikate, mit begrenztem Bezug zu \ac{SSI} und ohne Einbeziehung von \ac{PQC}; adressiert die Domäne „Blockchain“ und Aspekte der Datensicherheit, weist jedoch eine geringe Relevanz für die Domänen \ac{SSI}, \ac{PQC} und \ac{KRITIS} im Kontext der Masterarbeit auf. \\
\midrule
9 & Sebestyen, H., Popescu, D. E., \& Zmaranda, R. D. (2025). A Literature Review on Security in the Internet of Things: Identifying and Analysing Critical Categories. Computers (2073-431X), 14(2), 61. \url{https://doi.org/10.3390/computers14020061} & Mittel & Umfassende Betrachtung aktueller Sicherheitsthemen und Identitätsmanagement im IoT-Kontext, unter Rückgriff auf neue Technologien wie Blockchain; Integrationspotenzial hinsichtlich Blockchain erkennbar, jedoch keine explizite Behandlung von \ac{SSI}, \ac{PQC} oder \ac{KRITIS}; bietet einen breiten Überblick zu technologischen Lösungen und Herausforderungen, adressiert jedoch nur partiell die Anforderungen und Innovationspotenziale der vier Domänen der Masterarbeit. \\
\midrule
10 & Nambundo, J. M., de Souza Martins Gomes, O., de Souza, A. D., \& Machado, R. C. S. (2025). Cybersecurity and Major Cyber Threats of Smart Meters: A Systematic Mapping Review. Energies (19961073), 18(6), 1445. \url{https://doi.org/10.3390/en18061445} & Mittel & Konzentriert sich auf Cybersecurity-Bedrohungen und Schwachstellen im Kontext von Smart Metern als Teil kritischer Infrastruktur; adressiert relevante Sicherheitsfragen in Bezug auf Energieversorgung, thematisch verwandt mit der Schutzbedarfsanalyse für \ac{KRITIS}; der Einsatz von \ac{SSI}, Blockchain-Technologien oder \ac{PQC} wird nicht explizit thematisiert; bietet wichtige Einblicke in Bedrohungsszenarien und Mitigationsstrategien für smarte Energiesysteme, bleibt jedoch im Hinblick auf innovative Identitäts- oder Kryptografielösungen und deren methodischer Integration in Smart Metering-Systeme unspezifisch. \\
\midrule
11 & Yuan, F., Huang, X., Zheng, L., Wang, L., Wang, Y., Yan, X., Gu, S., \& Peng, Y.(2025). The Evolution and Optimization Strategies of a PBFT Consensus Algorithm for Consortium Blockchains. Information (2078-2489), 16(4), 268. \url{https://doi.org/10.3390/info16040268} & Mittel & Fokussiert auf die Optimierung des PBFT-Konsensalgorithmus, was grundlegende Bedeutung für Leistung, Sicherheit und Zuverlässigkeit von Consortium Blockchains hat; leistet methodischen Beitrag zur technologischen Weiterentwicklung im Blockchain-Bereich, jedoch ohne explizite Einbindung von \ac{SSI}, \ac{PQC} oder direktem Bezug zu \ac{KRITIS}; relevante Erkenntnisse zur Verbesserung von Konsensmechanismen, die potenziell für skalierbare und sichere Blockchain-basierte Infrastrukturen adaptierbar sind, aber inhaltlich primär auf Konsensalgorithmen begrenzt. \\
\midrule
12 & Radanliev, P. (2024). Digital security by design. Security Journal, 37(4), 1640–1679. \url{https://doi.org/10.1057/s41284-024-00435-3} & Mittel & Bietet eine umfassende, technologieübergreifende Analyse aktueller Herausforderungen im Bereich digitale Sicherheit; adressiert relevante Zukunftsthemen wie AI, Blockchain und Quantencomputing im Kontext sich wandelnder Sicherheitsparadigmen; keine explizite Schwerpunktsetzung auf \ac{SSI}, \ac{PQC} oder spezifisch \ac{KRITIS}; sektorübergreifende Betrachtung liefert wertvolle Einblicke zu regulatorischen und praxisbezogenen Aspekten, bleibt jedoch in Bezug auf die integrative Anwendung innovativer Sicherheitslösungen innerhalb der vier Domänen der Masterarbeit unspezifisch. \\
\midrule
13 & Miller, T., Durlik, I., Kostecka, E., Sokołowska, S., Kozlovska, P., \& Zwolak, R. (2025). Artificial Intelligence in Maritime Cybersecurity: A Systematic Review of AI-Driven Threat Detection and Risk Mitigation Strategies. Electronics (2079-9292), 14(9), 1844. \url{https://doi.org/10.3390/electronics14091844} & Mittel & Fokus liegt auf der Anwendung von KI-gestützten Verfahren zur Erkennung und Minderung von Cyber-Bedrohungen im maritimen Sektor, adressiert damit relevante Aspekte kritischer Infrastrukturen; Schnittstellen zu Blockchain und quantenkryptographischen Ansätzen werden als Forschungsperspektiven genannt, ohne im Review zentrale methodische oder anwendungsbezogene Ausarbeitung zu bieten; \ac{SSI} wird nicht behandelt, der primäre Schwerpunkt liegt auf KI und Cybersecurity, wodurch methodische und technische Details zu \ac{SSI} und \ac{PQC} im Kontext der vier Domänen der Masterarbeit fehlen. \\
\midrule
14 & Atlam, H. F., Ekuri, N., Azad, M. A., \& Lallie, H. S. (2024). Blockchain Forensics: A Systematic Literature Review of Techniques, Applications, Challenges, and Future Directions. Electronics (2079-9292), 13(17), 3568. \url{https://doi.org/10.3390/electronics13173568} & Mittel & Umfassende Analyse von Blockchain-Technologien im digitalen Forensik-Kontext mit Schwerpunkt auf Untersuchungsmethodik und Anwendungsfeldern; adressiert insbesondere die Herausforderungen beim Nachweis und der Verfolgung von Aktivitäten auf Blockchain-Systemen und beleuchtet regulatorische Aspekte, jedoch ohne explizite Behandlung von \ac{SSI}, \ac{PQC} oder den speziellen Anforderungen kritischer Infrastrukturen; liefert wertvolle Einblicke in forensische Anwendungen der Blockchain, bleibt jedoch mit Blick auf innovative Identitäts- und Kryptografielösungen sowie den Schutz kritischer Infrastrukturen im Rahmen der Masterarbeit begrenzt anschlussfähig. \\
\midrule
15 & Yakubu, M. M., Fadzil B Hassan, M., Danyaro, K. U., Junejo, A. Z., Siraj, M., Yahaya, S., Adamu, S., \& Abdulsalam, K. (2024). A Systematic Literature Review on Blockchain Consensus Mechanisms’ Security: Applications and Open Challenges. Computer Systems Science \& Engineering, 48(6), 1437–1481. \url{https://doi.org/10.32604/csse.2024.054556} & Mittel & Umfassende systematische Analyse der Sicherheitsaspekte und Herausforderungen von Blockchain-Konsensmechanismen mit Fokus auf deren Integrität, Zuverlässigkeit und praktische Anwendungen; adressiert die Domäne „Blockchain“ in methodischer Tiefe und liefert wertvolle Erkenntnisse hinsichtlich Sicherheit, Skalierbarkeit und Energieeffizienz von Konsensprotokollen; explizite Bezüge zu \ac{SSI}, \ac{PQC} und spezifisch \ac{KRITIS} fehlen, sodass die Übertragbarkeit auf die weiteren drei Domänen des Masterarbeit-Themas begrenzt ist. \\
\midrule
16 & Oude Roelink, B., El, H. M., \& Sarmah, D. (2024). Systematic review: Comparing zk‐SNARK, zk‐STARK, and bulletproof protocols for privacy‐preserving authentication. Security \& Privacy, 7(5), 1–59. \url{https://doi.org/10.1002/spy2.401} & Mittel & Fokussiert auf den Vergleich und die Analyse moderner Zero-Knowledge-Protokolle (zk-SNARKs, zk-STARKs, Bulletproofs) mit hoher Relevanz für die Domänen Privacy und Blockchain, insbesondere im Kontext von Authentifizierung und Datenschutz; liefert methodische und performancebezogene Einblicke, jedoch ohne explizite Berücksichtigung von \ac{SSI} oder \ac{PQC}; Anbindung an \ac{KRITIS} nicht direkt gegeben, bietet jedoch Potenzial für Integration innovativer Datenschutztechnologien in Blockchain-basierte Identitäts- und Authentifizierungssysteme. \\
\midrule
17 & Cherbal, S., Zier, A., Hebal, S., Louail, L., \& Annane, B. (2024). Security in internet of things: a review on approaches based on blockchain, machine learning, cryptography, and quantum computing. Journal of Supercomputing, 80(3), 3738–3816. \url{https://doi.org/10.1007/s11227-023-05616-2} & Mittel & Umfassende Analyse sicherheitsrelevanter Technologien im IoT-Kontext, darunter Blockchain, Kryptografie und Quantencomputing, mit breitem Überblick zu Ansätzen und Herausforderungen; adressiert insbesondere die Domänen Blockchain und \ac{PQC} durch die Einbeziehung quantenkryptographischer und klassischer kryptographischer Lösungen, ohne explizite Behandlung von \ac{SSI} oder dem spezifischen Anwendungsfeld kritischer Infrastrukturen; liefert wertvolle Vergleiche und Taxonomien zu modernen Sicherheitsmechanismen im IoT, bleibt jedoch hinsichtlich der methodischen und domänenspezifischen Vertiefung für \ac{SSI} und \ac{KRITIS} begrenzt. \\
\midrule
18 & Jagarlamudi, G. K., Yazdinejad, A., Parizi, R. M., \& Pouriyeh, S. (2024). Exploring privacy measurement in federated learning. Journal of Supercomputing, 80(8), 10511–10551. \url{https://doi.org/10.1007/s11227-023-05846-4} & Mittel & Fokussiert auf die Analyse von Privacy-Maßnahmen und deren Messbarkeit im Kontext von föderiertem Lernen, mit hohem methodischen Wert zur Bewertung von Datenschutz und Sicherheitsmetriken; Bezüge zu Blockchain oder \ac{SSI} werden nicht explizit hergestellt, und \ac{PQC} ist nur als Zukunftsperspektive am Rand angedeutet; die behandelten Konzepte und resultierenden Erkenntnisse zu Privacy-Measurement-Methoden können für sichere, dezentrale Systeme (einschließlich kritischer Infrastrukturen) grundsätzlich relevant sein, liefern jedoch keine spezifische oder anwendungsbezogene Vertiefung in den vier Kernbereichen der Masterarbeit. \\
\midrule
19 & Alzoubi, Y. I., Gill, A., \& Mishra, A. (2022). A systematic review of the purposes of Blockchain and fog computing integration: classification and open issues. Journal of Cloud Computing (2192-113X), 11(1), 1–36. \url{https://doi.org/10.1186/s13677-022-00353-y} & Mittel & Systematische Analyse der Integration von Blockchain-Technologien mit Fog Computing zur Lösung sicherheitsrelevanter Herausforderungen in IoT-Anwendungen; adressiert zentrale Aspekte wie Sicherheit, Datenschutz, Zugriffs- und Vertrauensmanagement, jedoch ohne explizite Behandlung von \ac{SSI} oder konkreten Umsetzungen von \ac{PQC}; verweist auf bestehende regulatorische und technologische Herausforderungen durch aufkommende Technologien wie Quantencomputing, aber ohne methodische oder anwendungsbezogene Vertiefung in Bezug auf \ac{KRITIS} oder innovative Identitätskonzepte; bietet wertvolle Klassifikation und Überblick zu Blockchain-Anwendungen im Kontext verteilter Edge-Infrastrukturen, bleibt jedoch bezüglich der vier Kernbereiche der Masterarbeit auf die Domäne „Blockchain“ und allgemeine Sicherheitsaspekte beschränkt. \\
\midrule
20 & Kazmi, S. H. A., Hassan, R., Qamar, F., Nisar, K., \& Ibrahim, A. A. A. (2023). Security Concepts in Emerging 6G Communication: Threats, Countermeasures, Authentication Techniques and Research Directions. Symmetry (20738994), 15(6), 1147. \url{https://doi.org/10.3390/sym15061147} & Mittel & Umfassende Analyse sicherheitsrelevanter Konzepte, Bedrohungen und Authentifizierungsmethoden im Kontext der aufkommenden 6G-Kommunikation; adressiert innovative Technologien wie Künstliche Intelligenz, Quantencomputing und Föderiertes Lernen, wodurch potenzielle Schnittstellen zu \ac{PQC} und Sicherheitsanforderungen für \ac{KRITIS} bestehen; explizite Bezüge zu \ac{SSI} und Blockchain-Technologien fehlen, ebenso eine methodische Vertiefung für spezielle \ac{SSI}- oder Blockchain-basierte Authentifizierungslösungen; bietet wertvolle Einblicke in zukünftige Forschungsrichtungen und technologische Herausforderungen, jedoch begrenzte direkte Anwendbarkeit auf alle vier Kernbereiche der Masterarbeit. \\
\midrule
21 & Attkan, A., \& Ranga, V. (2022). Cyber-physical security for IoT networks: a comprehensive review on traditional, blockchain and artificial intelligence based key-security. Complex \& Intelligent Systems, 8(4), 3559–3591. \url{https://doi.org/10.1007/s40747-022-00667-z} & Mittel & Fokussiert auf Authentifizierung und Schlüsselmanagement in IoT-Netzen unter Einbeziehung klassischer, Blockchain-basierter und KI-gestützter Sicherheitsmechanismen; adressiert im Wesentlichen die Domäne \enquote{Blockchain} und bietet innovative Perspektiven zur dezentralen Verwaltung von Session-Keys sowie zu KI-basierten Angriffserkennungsmethoden im IoT-Kontext; \ac{SSI} und \ac{PQC} werden nicht explizit behandelt, ebenso fehlt die gezielte Anwendung auf \ac{KRITIS}; der umfassende Überblick zu Authentifizierung und Schlüsselmanagement bietet methodische Anschlussmöglichkeiten für \ac{SSI}-basierte Systeme oder kritische Infrastruktur, bleibt im Kern jedoch breit und technologieorientiert ohne vertiefte Ausarbeitung der vier Domänen der Masterarbeit. \\
\midrule
22 & Ray, P. P. (2023). Web3: A comprehensive review on background, technologies, applications, zero-trust architectures, challenges and future directions. Internet of Things \& Cyber Physical Systems, 3, 213–248. \url{https://doi.org/10.1016/j.iotcps.2023.05.003} & Mittel & Bietet einen breiten Überblick über die technologischen und gesellschaftlichen Grundlagen sowie Anwendungsfelder von Web3 und dezentralen Plattformen, wobei die Domäne Blockchain als zentrales Element systematisch behandelt wird; explizite Bezüge zu \ac{SSI} und Zero-Trust-Architekturen existieren, jedoch fehlt eine detaillierte methodische Analyse spezifischer \ac{SSI}-Lösungen oder Implementierungen, und \ac{PQC} wird nicht thematisiert; der Beitrag verweist auf innovative Identitätskonzepte, Anwendungsintegration und technische Herausforderungen, bleibt jedoch hinsichtlich der anwendungsbezogenen Vertiefung zu \ac{PQC} sowie spezifischen Schutzmaßnahmen für \ac{KRITIS} auf einer konzeptionellen Ebene. \\
\midrule
23 & Stach, C., Gritti, C., Bräcker, J., Behringer, M., \& Mitschang, B. (2022). Protecting Sensitive Data in the Information Age: State of the Art and Future Prospects. Future Internet, 14(11), 302. \url{https://doi.org/10.3390/fi14110302} & Mittel & Umfassende Analyse aktueller Privacy-Mechanismen im Kontext datengetriebener Smart Services, mit Fokus auf die praktische Umsetzung nutzerfreundlicher Datenschutzlösungen; adressiert zentrale Herausforderungen und praktische Einsatzfelder moderner Datenschutzverfahren, ohne jedoch explizit auf \ac{SSI}, Blockchain-Technologien, \ac{PQC} oder spezielle Schutzanforderungen kritischer Infrastrukturen einzugehen; liefert wertvolle Einblicke zu datenorientierten Schutzmechanismen und deren Limitationen, bleibt jedoch hinsichtlich der methodischen und domänenspezifischen Vertiefung für \ac{SSI}, \ac{PQC} und \ac{KRITIS} konzeptionell und generisch. \\
\midrule
24 & Farooq, M. S., Riaz, S., \& Alvi, A. (2023). Security and Privacy Issues in Software-Defined Networking (SDN): A Systematic Literature Review. Electronics (2079-9292), 12(14), 3077. \url{https://doi.org/10.3390/electronics12143077} & Mittel & Systematische Analyse von Sicherheits- und Datenschutzproblemen in Software-Defined Networks mit Schwerpunkt auf Schwachstellen, Angriffen und Sicherungsmechanismen entlang der verschiedenen Netzwerkebenen; adressiert grundlegende Herausforderungen für den Schutz moderner Netzwerkarchitekturen, insbesondere durch die Trennung von Steuer- und Datenebene—ein Aspekt, der für den Betrieb kritischer Infrastrukturen relevante Einblicke und Methoden liefert; explizite Bezüge zu Blockchain-Technologien, \ac{SSI} und \ac{PQC} fehlen, jedoch kann die vorgestellte Taxonomie und die Diskussion zukünftiger Forschungsrichtungen methodische Impulse für sichere Integrationskonzepte in verteilten, kritischen oder identitätsgetriebenen Architekturen bieten—die inhaltliche Tiefe und Anwendbarkeit bleibt jedoch primär auf SDN-spezifische Herausforderungen fokussiert. \\
\midrule
25 & Chanal, P. M., \& Kakkasageri, M. S. (2020). Security and Privacy in IoT: A Survey. Wireless Personal Communications, 115(2), 1667–1693. \url{https://doi.org/10.1007/s11277-020-07649-9} & Mittel & Bietet einen umfassenden Überblick über grundlegende Sicherheits- und Datenschutzherausforderungen im Kontext des Internet of Things mit Fokus auf ressourcenbeschränkte Geräte; adressiert Kernaspekte wie Vertraulichkeit, Integrität, Authentifizierung und Verfügbarkeit, ohne jedoch explizit auf \ac{SSI}, Blockchain-Technologien, \ac{PQC} oder spezifische Anforderungen kritischer Infrastrukturen einzugehen; liefert wertvolle konzeptionelle Grundlagen zu Sicherheitsanforderungen und -architekturen im IoT, bleibt jedoch bezüglich anwendungs- oder methodenspezifischer Vertiefung zu den vier Kernbereichen der Masterarbeit generisch. \\
\midrule
26 & Choudhary, A. (2024). Internet of Things: a comprehensive overview, architectures, applications, simulation tools, challenges and future directions. Discover Internet of Things, 4(1), 1–41. \url{https://doi.org/10.1007/s43926-024-00084-3} & Mittel & Umfassende Übersicht und Analyse der IoT-Architektur, Anwendungen und Herausforderungen; adressiert technologische, soziale und funktionale Aspekte des Internet of Things, mit generischer Betrachtung von Architekturen und Simulationsumgebungen; explizite Bezüge zu \ac{SSI}, Blockchain-Technologien, \ac{PQC} oder spezifischen Schutzanforderungen kritischer Infrastrukturen fehlen; liefert wertvolle Grundlagen für das technologische Umfeld, bleibt jedoch hinsichtlich der vier Domänen der Masterarbeit konzeptionell und unspezifisch. \\
\midrule
27 & Sikiru, I. A., Kora, A. D., Ezin, E. C., Imoize, A. L., \& Li, C.-T. (2024). Hybridization of Learning Techniques and Quantum Mechanism for IIoT Security: Applications, Challenges, and Prospects. Electronics (2079-9292), 13(21), 4153. \url{https://doi.org/10.3390/electronics13214153} & Mittel & Systematische Analyse hybrider Sicherheitsansätze in der Industrial IoT (IIoT), insbesondere durch die Kombination klassischer Lernverfahren und quantenmechanistischer Ansätze; adressiert relevante Herausforderungen und Perspektiven der IIoT-Sicherheit mit Berücksichtigung von Blockchain-Technologien und Quantum Mechanisms, wobei \ac{PQC} eher implizit thematisiert wird; explizite Vertiefung von \ac{SSI} und spezifische Anwendungen in \ac{KRITIS} fehlen, liefert jedoch Impulse für die Integration moderner Kryptografie- und Sicherheitsverfahren im industriellen Umfeld. \\
\midrule
28 & RadRadanliev, P. (2024). Artificial intelligence and quantum cryptography. Journal of Analytical Science \& Technology, 14, 1–17. \url{https://doi.org/10.1186/s40543-024-00416-6} & Mittel & Thematisiert den aktuellen Stand und die Zukunftsperspektiven an der Schnittstelle von künstlicher Intelligenz und quantenkryptografischen Verfahren, wobei insbesondere der Einfluss von AI-Methoden auf Effizienz und Robustheit kryptografischer Systeme sowie die Herausforderungen durch das \enquote{Quantum Threat}-Szenario im Zentrum stehen; explizite Bezüge zu \ac{SSI}, Blockchain-Technologien oder spezifischen Anwendungsfällen in \ac{KRITIS} fehlen; liefert wertvolle Impulse zu methodischen Innovationen in der \ac{PQC}, bleibt jedoch hinsichtlich der vier Domänen der Masterarbeit primär konzeptionell und technologisch fokussiert auf AI und Quantenkryptografie. \\
\midrule
29 & O’Donoghue, O., Vazirani, A. A., Brindley, D., \& Meinert, E. (2019). Design Choices and Trade-Offs in Health Care Blockchain Implementations: Systematic Review. Journal of Medical Internet Research, 21(5), e12426. \url{https://doi.org/10.2196/12426} & Mittel & Systematische Analyse von Architektur- und Design-Entscheidungen bei der Implementierung von Blockchain-Technologie im Kontext elektronischer Gesundheitsakten (EMR), mit Schwerpunkt auf sicherheitsrelevanten und skalierbaren Systemanforderungen sowie Trade-offs zwischen verschiedenen technischen, organisatorischen und anwendungsbezogenen Merkmalen; behandelt die Domäne Blockchain umfassend und liefert wertvolle Erkenntnisse über sicherheitsrelevante Kompromisse im Gesundheitswesen, adressiert jedoch weder \ac{SSI} noch \ac{PQC} oder \ac{KRITIS} explizit; die Untersuchung des Spannungsfelds zwischen Sicherheit, Skalierbarkeit und Datenmanagement bildet eine methodisch relevante Grundlage, bleibt aber in Bezug auf die vier zentralen Domänen der Masterarbeit auf anwendungsbezogene Blockchain-Implementierungen beschränkt. \\
\midrule
30 & Mulholland, J., Mosca, M., \& Braun, J. (2017). The Day the Cryptography Dies. IEEE Security \& Privacy, 15(4), 14–21. \url{https://doi.org/10.1109/MSP.2017.3151325} & Mittel & Überblickartige Darstellung der Auswirkungen von Quantencomputern auf bestehende kryptografische Verfahren; adressiert explizit die Bedrohung aktueller Sicherheitstechnologien (\ac{PQC}), jedoch ohne methodische oder technologische Vertiefung zu Blockchain, \ac{SSI} oder \ac{KRITIS}; bietet konzeptionelle Einblicke in Risikoszenarien und Bedrohungsmodelle, bleibt jedoch hinsichtlich innovativer Lösungsansätze oder spezifischer Anwendungsgebiete der vier Domänen der Masterarbeit unspezifisch. \\
\midrule
31 & G, C. A., \& Basarkod, P. I. (2024). A survey on blockchain security for electronic health record. Multimedia Tools and Applications: An Journal, 1–35. \url{https://doi.org/10.1007/s11042-024-19883-5} & Mittel & Fokus auf Blockchain-basierte Sicherheitslösungen für elektronische Gesundheitsakten (EHR) mit Einbindung von Deep-Learning-Methoden; adressiert primär die Domäne Blockchain durch Analyse von Datenschutz, Datensicherheit und Zugriffskontrolle im Gesundheitswesen, bietet wertvolle methodische Einblicke zur Anwendung verteilter Technologien im Bereich sensibler Daten; explizite Bezüge zu \ac{SSI}, \ac{PQC} und spezifisch \ac{KRITIS} außerhalb des Gesundheitssektors fehlen, wodurch die Anwendbarkeit auf alle vier Domänen der Masterarbeit beschränkt bleibt. \\
\midrule
32 & Batta, P., Ahuja, S., \& Kumar, A. (2024). Future Directions for Secure IoT Frameworks: Insights from Blockchain-Based Solutions: A Comprehensive Review and Future Analysis. Wireless Personal Communications: An International Journal, 139(3), 1749–1781. \url{https://doi.org/10.1007/s11277-024-11694-z} & Mittel & Systematische Untersuchung sicherer IoT-Frameworks unter Verwendung von Blockchain-Technologien; detaillierte Analyse verschiedener Algorithmen (u.a. Konsensmechanismen, \ac{RSA}, Hashing) und Plattformen (Ethereum, CoSMOS, Hyperledger Fabric), mit Fokus auf die Verbesserung von Sicherheit und Performance in IoT-Systemen; explizite Bezüge zu \ac{SSI}, \ac{PQC} und den besonderen Anforderungen kritischer Infrastrukturen fehlen; adressiert vor allem die Domäne „Blockchain“ und liefert grundlegende Einblicke zur Anwendung verteilter Sicherheitsmechanismen im IoT, bleibt jedoch hinsichtlich der vier Kerndomänen der Masterarbeit ( \ac{SSI}, Blockchain, \ac{PQC}, \ac{KRITIS}) methodisch und domänenspezifisch eingeschränkt. \\
\midrule
33 & Radanliev, P. (2024). Integrated cybersecurity for metaverse systems operating with artificial intelligence, blockchains, and cloud computing. Frontiers in Blockchain, 1–14. \url{https://doi.org/10.3389/fbloc.2024.1359130} & Mittel & Umfassende Analyse der Cybersicherheitslandschaft im Kontext integrierter Metaverse-Systeme unter Einbezug von Artificial Intelligence, Blockchain und Cloud Computing; adressiert zentrale Risikofelder, regulatorische Herausforderungen und die Rolle moderner Sicherheitstechnologien für die digitale Ökonomie, wobei insbesondere Blockchain in seiner Bedeutung für selbstverwaltete Systeme und Netzwerkgovernance diskutiert wird; explizite Vertiefungen zu \ac{SSI}, \ac{PQC} oder deren spezieller Anwendung im Umfeld kritischer Infrastrukturen fehlen, ebenso bleibt die methodische und technologische Anbindung an innovative Identitäts- und Kryptografieansätze im Rahmen der vier Domänen der Masterarbeit auf konzeptionelle Ausblicke beschränkt. \\
\midrule
34 & Hajian Berenjestanaki, M., Barzegar, H. R., El Ioini, N., \& Pahl, C. (2024). Blockchain-Based E-Voting Systems: A Technology Review. Electronics (2079-9292), 13(1), 17. \url{https://doi.org/10.3390/electronics13010017} & Mittel & Systematische Analyse von Blockchain-basierten E-Voting-Systemen mit Fokus auf Sicherheits-, Transparenz- und Integritätsaspekte; zentrale Bewertung technologischer Herausforderungen und zukünftiger Forschungsfragen, insbesondere zu Skalierbarkeit und Datenschutz – thematisch eng an die Domäne „Blockchain“ angelehnt; explizite Vertiefung von \ac{SSI}, \ac{PQC} oder die Anwendung in besonders schützenswerten \ac{KRITIS} fehlt, bietet jedoch methodische Ansätze und technische Perspektiven, die für die Entwicklung sicherer und vertrauenswürdiger Abstimmungssysteme in digitalen Infrastrukturen relevant sein können. \\
\midrule
35 & Pirbhulal, S., Chockalingam, S., Shukla, A., \& Abie, H. (2024). IoT cybersecurity in 5G and beyond: a systematic literature review. International Journal of Information Security, 23(4), 2827–2879. \url{https://doi.org/10.1007/s10207-024-00865-5} & Mittel & Systematische Literaturübersicht zu Cybersicherheitsaspekten in 5G- und Next-Generation-IoT-Umgebungen, insbesondere hinsichtlich Threats, Authentifizierung, Zugriffskontrolle, Netzwerk- und Anwendungsschicht sowie Herausforderungen durch Softwarisierung und Virtualisierung der Netze; adressiert methodisch den aktuellen Forschungsstand, evaluiert genutzte Validierungsansätze (Praxis, Simulation, Theorie) und liefert ein Kategorienschema für existierende Sicherheitsmechanismen und offene Forschungsfragen; explizite Bezüge zu \ac{SSI}, Blockchain-Technologien oder \ac{PQC} fehlen, ebenso eine gezielte Betrachtung von \ac{KRITIS}—die Branchenbeispiele (z. B. Healthcare, Energie) lassen eine indirekte Bedeutung für KRITIS erkennen, ohne diese jedoch methodisch zu vertiefen; methodische und technologische Tiefe für die vier Masterarbeitsdomänen beschränkt sich auf generische Cybersicherheitsbedrohungen und Lösungsansätze in modernen IoT/5G-Systemen. \\
\midrule
36 & Asif, M., Abrar, M., Salam, A., Amin, F., Ullah, F., Shah, S., \& AlSalman, H. (2025). Intelligent two-phase dual authentication framework for Internet of Medical Things. Scientific Reports, 15(1), 1–19. \url{https://doi.org/10.1038/s41598-024-84713-5} & Mittel & Fokus liegt auf der Entwicklung und Evaluierung eines intelligenten Zwei-Phasen-Authentifizierungsframeworks für die Internet of Medical Things (IoMT) mit Ziel der effizienten und sicheren Kommunikation sensibler Gesundheitsdaten; zentrale technische Ansätze umfassen ECDH für Schlüsselaustausch und AES-GCM für Datenverschlüsselung, wobei signifikante Verbesserungen in Effizienz und Sicherheit gegenüber klassischen Authentifizierungsmethoden nachgewiesen werden; explizite Bezüge zu \ac{SSI} und Blockchain-Technologien sowie \ac{PQC} fehlen vollständig, sodass methodische und technologische Innovationen in diesen Bereichen für den Rahmen der Masterarbeit unberücksichtigt bleiben; adressiert Schutzanforderungen im Bereich kritischer Infrastrukturen exemplarisch am Gesundheitswesen, bleibt jedoch in der Tiefe auf klassische kryptografische Verfahren und Authentifizierungsprozesse limitiert. \\
\midrule
37 & Marengo, A., \& Santamato, V. (2025). Quantum algorithms and complexity in healthcare applications: a systematic review with machine learning-optimized analysis. Frontiers in Computer Science, 1–30. \url{https://doi.org/10.3389/fcomp.2025.1584114} & Mittel & Systematische Übersicht zur Anwendung von Quantenalgorithmen und quanten-inspirierten Komplexitätsanalysen im Gesundheitswesen, mit zwei Schwerpunkten: (1) Quantum Computing für KI-basierte Analysen biomedizinischer Daten und (2) quantenkryptografische Protokolle zur Absicherung medizinischer Daten. Expliziter Bezug zur Domäne \ac{PQC} durch Analyse quantensicherer und blockchain-basierter Sicherheitsmechanismen im Healthcare-Kontext. Keine explizite Behandlung von \ac{SSI} oder dezidierten Blockchain-Architekturen außerhalb sicherheitsrelevanter Frameworks; Anwendung auf \ac{KRITIS} implizit durch den Fokus auf sichere medizinische Systeme, jedoch nicht technologieübergreifend vertieft. Insgesamt methodisch relevant für die Domäne \ac{PQC} und für Sicherheitsthemen im medizinisch-kritischen Sektor, für die vier Themenbereiche der Masterarbeit aber primär im Bereich quantensicherer Daten- und KI-Anwendungen anschlussfähig. \\
\midrule
38 & Ahakonye, L. A. C., Nwakanma, C. I., \& Kim, D.-S. (2024). Tides of Blockchain in IoT Cybersecurity. Sensors (14248220), 24(10), 3111. \url{https://doi.org/10.3390/s24103111} & Mittel & Umfassende Übersicht zu Anwendungsmöglichkeiten und Herausforderungen von Blockchain-Technologie im Bereich der IoT-Cybersicherheit, insbesondere in Verbindung mit KI-unterstützten Intrusion Detection Systemen; adressiert die Domäne „Blockchain“ grundlegend sowie deren Potenzial für Transparenz, Dezentralität und Unveränderlichkeit im IoT-Kontext; Integration von AI und Blockchain als Innovationstreiber für sichere und skalierbare IDS-Lösungen im IoT/IIoT; \ac{SSI} und \ac{PQC} werden nicht explizit behandelt; spezifische Anforderungen und Anwendungsfälle für \ac{KRITIS} werden nur indirekt adressiert; liefert wertvolle Einblicke und methodische Ansätze für die Weiterentwicklung sicherer IoT-Systeme, bleibt jedoch hinsichtlich der vier Domänen der Masterarbeit vorwiegend auf Blockchain und allgemeine Sicherheitsthemen im IoT fokussiert. \\
\midrule
39 & Nain, A., Sheikh, S., Shahid, M., \& Malik, R. (2024). Resource optimization in edge and SDN-based edge computing: a comprehensive study. Cluster Computing, 27(5), 5517–5545. \url{https://doi.org/10.1007/s10586-023-04256-8} & Niedrig & Umfassende systematische Analyse aktueller Optimierungsansätze für Ressourcenmanagement in Edge-Computing-Umgebungen, insbesondere unter Integration von Software-Defined Networking (SDN); adressiert zentrale Herausforderungen der effizienten Ressourcennutzung, Kontrollarchitekturen und Netzwerkprogrammierbarkeit, was insbesondere für leistungsfähige, latenzarme Anwendungen und Systemarchitekturen an der Netzwerkkante relevant ist; explizite Bezüge zu \ac{SSI}, Blockchain-Technologien, \ac{PQC} oder den besonderen Anforderungen kritischer Infrastrukturen fehlen; liefert dennoch wertvolle methodische Impulse für das Design verteilter, dynamischer Infrastrukturen, bleibt aber in Bezug auf die vier Domänen der Masterarbeit überwiegend allgemein und technologieorientiert. \\
\midrule
40 & Netinant, P., Saengsuwan, N., Rukhiran, M., \& Pukdesree, S. (2023). Enhancing Data Management Strategies with a Hybrid Layering Framework in Assessing Data Validation and High Availability Sustainability. Sustainability (2071-1050), 15(20), 15034. \url{https://doi.org/10.3390/su152015034} & Niedrig & Betrachtet Methoden zur nachhaltigen und hochverfügbaren Datenmigration, insbesondere durch ein hybrides Layering-Framework im Kontext von Data Management und Data Validation; adressiert primär Herausforderungen und Optimierungsstrategien im Bereich Datenkonsistenz, Datenintegrität und Verfügbarkeitsmanagement bei Migration und Transformation—relevant für unternehmensweite IT-Systeme und Logistikdaten; explizite Bezüge zu \ac{SSI}, Blockchain-Technologien, \ac{PQC} oder besonderen Anforderungen kritischer Infrastrukturen fehlen vollständig; liefert wertvolle Erkenntnisse zur Bewertung und Ausgestaltung von Datenmigrationsprozessen, bleibt jedoch hinsichtlich der vier Domänen der Masterarbeit methodisch und inhaltlich unberührt. \\
\midrule
41 & Trautman, L. J., Shackelford, S., Elzweig, B., \& Ormerod, P. (2024). Understanding Cyber Risk: Unpacking and Responding to Cyber Threats Facing the Public and Private Sectors. University of Miami Law Review, 78(3), 840–916. \url{https://repository.law.miami.edu/umlr/vol78/iss3/5/} & Niedrig & Umfassende Analyse aktueller Cyberbedrohungen und deren Auswirkungen auf öffentliche und private Sektoren, mit Fokus auf Angriffsszenarien (u.a. Ransomware, Cyberwarfare, Datenlecks), regulatorische und unternehmensbezogene Steuerungsmechanismen sowie das Zusammenspiel von Recht, Unternehmensführung und geopolitischen Risiken; adressiert zentrale Aspekte der Cyber-Risikobewertung und Reaktion auf digitale Angriffe, insbesondere aus administrativer und juristischer Perspektive; explizite Bezüge zu \ac{SSI}, Blockchain-Technologien, \ac{PQC} sowie spezifische Schutzmaßnahmen für \ac{KRITIS} fehlen; bietet wertvolle konzeptionelle Grundlagen im Bereich Cybersicherheit, Governance und Compliance, bleibt jedoch hinsichtlich der vier Schwerpunktdomänen der Masterarbeit in methodischer und technologischer Tiefe eingeschränkt. \\
\midrule
42 & Hendaoui, F., Ferchichi, A., Trabelsi, L., Meddeb, R., Ahmed, R., \& Khelifi, M. K. (2024). Advances in deep learning intrusion detection over encrypted data with privacy preservation: a systematic review. Cluster Computing, 27(7), 8683–8724. \url{https://doi.org/10.1007/s10586-024-04424-4} & Niedrig & Systematische Analyse der Fortschritte im Bereich Deep-Learning-basierter Intrusion Detection über verschlüsselte Daten mit Fokus auf Privacy-Preservation; behandelt innovative Ansätze zur Anomalieerkennung in verschlüsselten Datenströmen durch tiefe neuronale Netze, ohne auf Datenentschlüsselung angewiesen zu sein; explizite Bezüge zu \ac{SSI}, Blockchain-Technologien, \ac{PQC} oder dem Schutz kritischer Infrastrukturen fehlen; bietet wertvolle methodische und technologische Impulse zur sicheren Datenverarbeitung und Angriffserkennung in datenschutzorientierten Systemen, bleibt aber hinsichtlich der vier Kerndomänen der Masterarbeit überwiegend auf den Bereich Deep Learning und Privacy-Preserving IDS fokussiert \\
\midrule
43 & Akartuna, E. A., Johnson, S. D., \& Thornton, A. E. (2023). The money laundering and terrorist financing risks of new and disruptive technologies: a futures-oriented scoping review: The money laundering and terrorist financing risks of new and disruptive technologies: a futures-oriented scoping review. Security Journal, 36(4), 615–650. \url{https://doi.org/10.1057/s41284-022-00356-z} & Niedrig & Systematische Analyse von Geldwäsche- und Terrorismusfinanzierungsrisiken im Zusammenhang mit neuen und disruptiven Technologien, insbesondere Distributed-Ledger-Technologien (inkl. Kryptowährungen), neue Zahlungswege und FinTech; behandelt umfassend die Risiken, Methoden sowie betroffene Akteure und skizziert daraus resultierende Trends und politische Implikationen – mit klarem Bezug zur Domäne \enquote{Blockchain} und angrenzender regulatorischer Herausforderungen; explizite Vertiefungen zu \ac{SSI} oder \ac{PQC} fehlen, ebenso eine gezielte Betrachtung kritischer Infrastrukturen im engeren technischen Sinne; liefert wichtige Einblicke für die Risiko- und Bedrohungsanalyse im Kontext innovativer Finanztechnologien, bleibt aber hinsichtlich der methodischen Tiefe und direkten Anwendbarkeit für die vier Domänen der Masterarbeit beschränkt. \\
\midrule
44 & Zboril, M., \& Svatá, V. (2025). Performance comparison of cloud virtual machines. Journal of Systems \& Information Technology, 27(2), 197–213. \url{https://doi.org/10.1108/JSIT-02-2022-0040} & Niedrig & Thematischer Fokus liegt auf der vergleichenden Performancemessung von Cloud-basierten virtuellen Maschinen (VMs) bei AWS, Microsoft Azure und Google Cloud Platform, basierend auf Benchmark-Tests unter Linux; adressiert ausschließlich Infrastruktur- und Leistungsaspekte von Cloud-Diensten sowie Auswahlkriterien für IT-Betriebsmodelle; keine Verbindung zu den vier Domänen der Masterarbeit, da keine sicherheitsrelevanten, kryptografischen oder identitätsbezogenen Aspekte behandelt werden; relevante Erkenntnisse für Cloud-Infrastrukturmanagement und Benchmarking, jedoch methodisch und inhaltlich außerhalb des Kernbereichs der Masterarbeit. \\
\midrule
45 & Radanliev, P. (2024). The rise and fall of cryptocurrencies: defining the economic and social values of blockchain technologies, assessing the opportunities, and defining the financial and cybersecurity risks of the Metaverse. Financial Innovation, 10, 1–34. \url{https://doi.org/10.1186/s40854-023-00537-8} & Niedrig & Umfassende Analyse der wirtschaftlichen, sozialen und technologischen Aspekte von Blockchain-Technologien, insbesondere im Kontext von Kryptowährungen und deren Rolle im Metaverse; untersucht wirtschaftliche Chancen, Investitionsstrategien und Cybersecurity-Risiken mit interdisziplinärem Ansatz, inklusive Risikobewertung und maschinellem Lernen im Finanzsektor; explizite Bezüge zu \ac{SSI}, \ac{PQC} und dem Schutz kritischer Infrastrukturen fehlen, ebenso eine methodische oder technologische Vertiefung zu innovativen Identitäts- oder Kryptografielösungen—fokussiert primär auf ökonomische und anwendungsbezogene Fragestellungen der Blockchain im Finanz- und Metaverse-Umfeld. \\
\midrule
46 & Bunescu, L., \& Vârtei, A. M. (2024). Modern finance through quantum computing—A systematic literature review. PLoS ONE, 19(7), 1–22. \url{https://doi.org/10.1371/journal.pone.0304317} & Niedrig & Systematische Analyse des Einsatzes von Quantencomputing im Finanzsektor, mit Fokus auf Simulation, Optimierung und maschinelles Lernen; adressiert Kernaspekte der \ac{PQC} im Hinblick auf die transformative Wirkung quantenbasierter Technologien, ohne dabei explizit auf Blockchain-Technologien oder \ac{SSI} einzugehen; liefert wertvolle Einblicke in die methodische und anwendungsbezogene Entwicklung von Quantum Finance, bleibt jedoch hinsichtlich der vier thematischen Domänen der Masterarbeit ( \ac{SSI}, Blockchain, \ac{PQC}, \ac{KRITIS}) primär auf Finanzanwendungen und damit nur partiell anschlussfähig. \\
\midrule
47 & Alzoubi, Y. I., Mishra, A., \& Topcu, A. E. (2024). Research trends in deep learning and machine learning for cloud computing security. Artificial Intelligence Review: An International Science and Engineering Journal, 57(5). \url{https://doi.org/10.1007/s10462-024-10776-5} & Niedrig & Fokussiert auf den Einsatz von Deep-Learning- und Machine-Learning-Technologien zur Identifikation und Bewältigung von Cloud-Sicherheitsbedrohungen; adressiert zentrale Herausforderungen wie Anomalieerkennung, Security Automation und die Integration neuer Technologien, ohne jedoch explizit \ac{SSI}, Blockchain-Ansätze oder \ac{PQC} systematisch einzubinden; hebt methodische, datenschutzbezogene und regulatorische Fragestellungen hervor, die für den Schutz kritischer Infrastrukturen relevant sind, bleibt jedoch bezüglich der vier Kernbereiche der Masterarbeit hauptsächlich auf Cloud Security und AI-gestützte Verfahren konzentriert und bietet nur indirekte Anschlussmöglichkeiten für innovative Kryptografie- oder Identitätslösungen \\
\midrule
48 & Williamson, S. M., \& Prybutok, V. (2024). Balancing Privacy and Progress: A Review of Privacy Challenges, Systemic Oversight, and Patient Perceptions in AI-Driven Healthcare. Applied Sciences (2076-3417), 14(2), 675. \url{https://doi.org/10.3390/app14020675} & Niedrig & Kritische Analyse von Datenschutz-, Ethik- und Compliance-Herausforderungen in AI-gestützten Gesundheitssystemen, mit Fokus auf Differential Privacy und patientenzentrierte Datenverarbeitung; adressiert relevante technologische Ansätze wie Verschlüsselung und Differential Privacy sowie organisatorische und regulatorische Rahmenbedingungen—beinhaltet zudem die Herausforderungen bei der Integration von Blockchain-Technologien im healthcare-spezifischen Kontext und deren Vereinbarkeit mit der DSGVO, wodurch ein übergreifender Bezug zur Domäne Blockchain gegeben ist; \ac{SSI} und \ac{PQC} werden nicht explizit behandelt, ebenso steht die Anbindung an \ac{KRITIS} außerhalb des engeren Fokus; bietet wertvolle Erkenntnisse zu datenschutzgerechter Systemgestaltung und Interdisziplinarität im Gesundheitswesen, bleibt aber hinsichtlich methodischer Tiefe und anwendungsbezogener Integration in allen vier Kernbereichen der Masterarbeit überblicksartig und konzeptionell. \\
\midrule
49 & Tukur, M., Schneider, J., Househ, M., Dokoro, A. H., Ismail, U. I., Dawaki, M., \& Agus, M. (2023). The metaverse digital environments: a scoping review of the challenges, privacy and security issues. Frontiers in Big Data, 1–25. \url{https://doi.org/10.3389/fdata.2023.1301812} & Niedrig & Umfassende Übersicht zu Herausforderungen, Datenschutz- und Sicherheitsfragen bei der Entwicklung und Implementierung von Metaverse-Umgebungen, insbesondere infolge der pandemiebedingten Digitalisierungsschübe; adressiert wirtschaftliche, technische, ethische und soziale Herausforderungen, darunter Hard- und Softwarekosten, digitale Ungleichheit sowie Regelwerks- und Datenmanagement-Fragen; konkrete Analyse und Klassifikation von Policy-, Privacy- und Security-Problemen, mit Fokus auf privatsphärenbezogene Risiken und Governance-Anforderungen im Metaverse. Explizite Bezüge zu \ac{SSI}, Blockchain-Technologien und \ac{PQC} fehlen; \ac{KRITIS} werden nur implizit durch den Verweis auf digitale Spaltungen und gesellschaftliche Implikationen berührt. \\
\midrule
50 & Hanafi, B., Ali, M., \& Singh, D. (2025). Quantum algorithms for enhanced educational technologies. Discover Education, 4(1), 1–33. \url{https://doi.org/10.1007/s44217-025-00400-1} & Niedrig & Fokus auf die Potenziale und Herausforderungen von Quantum Computing und Quantenkryptografie in der Bildungsbranche, z. B. für personalisiertes Lernen und sichere Datenübertragung; Bezug zu \ac{PQC} nur anwendungsbezogen im Bildungskontext, ohne technische Tiefe oder Bezug zu \ac{SSI}, Blockchain oder \ac{KRITIS}; aus Sicht der vier Masterarbeitsdomänen methodisch und thematisch nur sehr eingeschränkt anschlussfähig. \\
\midrule
51 & Pillai, S. E. V. S., Nadella, G. S., Meduri, K., Priyadharsini, N. A., Bhuvanesh, A., \& Kumar, D. (2025). A walrus optimization-enhanced long short-term memory model for credit fraud detection in banking. International Journal of Information Technology: An Official Journal of Bharati Vidyapeeth’s Institute of Computer Applications and Management, 1–17. \url{https://doi.org/10.1007/s41870-025-02574-1} & Niedrig & Beschreibung einer innovativen Framework-Kombination aus Autoencoder, Long Short-Term Memory Netzwerken und Walrus Optimization Algorithm zur Verbesserung der Betrugserkennung im Bankensektor; konzentriert sich ausschließlich auf Machine-Learning-gestützte Analyse, Datenvorverarbeitung und Hyperparameteroptimierung zur Echtzeit-Erkennung betrügerischer Transaktionen in großen Datenmengen; keinerlei Behandlung oder Integration der vier Domänen der Masterarbeit; relevante methodische Beiträge beschränken sich auf KI-basierte Fraud Detection, ohne Anschlusspunkte zu den Kernthemen der Masterarbeit. \\
\midrule
52 & Priya, S. S., Vijayabhasker, R., \& Rajaram, A. (2025). Advanced Security and Efficiency Framework for Mobile Ad-Hoc Networks Using Adaptive Clustering and Optimization Techniques. Journal of Electrical Engineering \& Technology (19750102), 20(3), 1815–1826. \url{https://doi.org/10.1007/s42835-024-02119-9} & Niedrig & Fokus auf ein innovatives Sicherheits- und Effizienz-Framework für Mobile Ad-Hoc Networks (MANETs) durch adaptive Clusterbildung, AI-unterstützte Vertrauensbewertung und quantenresistente PUF-Authentifizierung; explizite Relevanz für \ac{PQC} durch QR-PUF-Komponente; \ac{SSI} und Blockchain werden nicht behandelt, ebenso fehlt eine gezielte Betrachtung kritischer Infrastrukturen; für die vier Kerndomänen der Masterarbeit somit vor allem im Kontext quantensicherer Validierung/mobiler Netzwerksicherheit anschlussfähig, ansonsten methodisch und domänenspezifisch eingeschränkt. \\
\midrule
53 & Berkani, A.-S., Moumen, H., Benharzallah, S., Yahiaoui, S., \& Bounceur, A. (2024). Blockchain Use Cases in the Sports Industry: A Systematic Review. International Journal of Networked \& Distributed Computing, 12(1), 17–40. \url{https://doi.org/10.1007/s44227-024-00022-3} & Niedrig & Fokus auf branchenspezifische Anwendungen der Blockchain-Technologie im Sportsektor (Athleten-Datenmanagement, Fandaten, NFT-Sammlerstücke); methodische und technologische Vertiefung im Hinblick auf \ac{SSI}, \ac{PQC} oder \ac{KRITIS} fehlt; relevante Erkenntnisse nur für den Bereich Blockchain-Anwendungsfälle in Sport und Entertainment, für die vier Domänen der Masterarbeit jedoch insgesamt wenig anschlussfähig. \\
\midrule
54 & 2023 PNS Annual Meeting - Copenhagen, 17-20 June 2023. (2023). Journal of the Peripheral Nervous System: JPNS, 28 Suppl 4, S3–S254. \url{https://doi.org/10.1111/jns.12585} & Niedrig & Konferenzband ohne Bezug zu \ac{SSI}, Blockchain oder \ac{PQC}. \\
\midrule
55 & Posters. (2017). FEBS Journal, 284, 102–403. \url{https://doi.org/10.1111/febs.14174} & Niedrig & Posterband, kein Bezug zu \ac{SSI}, Blockchain oder \ac{PQC}. \\
\midrule
56 & Annotated Listing of New Books. (2024). Journal of Economic Literature, 62(4), 1696–1750. \url{https://doi.org/10.1257/jel.62.4.1696} & Niedrig & Buchliste, kein Bezug zum Thema. \\
\midrule
57 & PNS Abstracts 2023. (2023). Journal of the Peripheral Nervous System, 28, S3–S254. \url{https://doi.org/10.1111/jns.12585} & Niedrig & Abstractband, kein Bezug zu SSI, Blockchain oder \ac{PQC}. \\
\midrule
58 & Elendu, C., Omeludike, E. K., Oloyede, P. O., Obidigbo, B. T., \& Omeludike, J. C. (2024). Legal implications for clinicians in cybersecurity incidents: A review. Medicine, 103(39), 1–26. \url{https://doi.org/10.1097/MD.0000000000039887} & Niedrig & Fokus auf die rechtlichen Implikationen von Cybersecurity-Vorfällen im Gesundheitswesen, insbesondere für klinisch tätige Personen; Betrachtung technologischer Entwicklungen (u. a. künstliche Intelligenz und Quantencomputing) sowie internationaler regulatorischer Unterschiede; praxisnahe Empfehlungen und Fallstudien zu Cybersecurity-Management, ethischen und juristischen Aspekten im Gesundheitssektor; keine explizite Behandlung von \ac{SSI}, Blockchain oder \ac{PQC}, \ac{KRITIS} werden durch den Gesundheitsbereich berührt, Schwerpunkt liegt jedoch auf juristischen und ethischen Fragestellungen. \\
\midrule
59 & Albshaier, L., Almarri, S., \& Hafizur Rahman, M. M. (2024). A Review of Blockchain’s Role in E-Commerce Transactions: Open Challenges, and Future Research Directions. Computers (2073-431X), 13(1), 27. \url{https://doi.org/10.3390/computers13010027} & Niedrig & Fokus auf die Anwendung von Blockchain-Technologien zur Verbesserung von Sicherheit, Transparenz und Betrugserkennung in E-Commerce-Transaktionen; betont die Rolle verteilter, unveränderlicher digitaler Ledger für den Schutz sensibler Kundendaten und die Stärkung des Vertrauens in Online-Plattformen; adressiert die Domäne „Blockchain“ vorrangig, ohne explizite Bezüge zu \ac{SSI} oder \ac{PQC}; \ac{KRITIS} werden nicht thematisiert, da der Schwerpunkt auf E-Commerce liegt; methodische und technologische Tiefe für die vier Domänen der Masterarbeit ist auf Blockchain-Anwendungen im Bereich E-Commerce beschränkt. \\
\midrule
60 & Reddy, R. C., Bhattacharjee, B., Mishra, D., \& Mandal, A. (2022). A systematic literature review towards a conceptual framework for enablers and barriers of an enterprise data science strategy. Information Systems \& e-Business Management, 20(1), 223–255. \url{https://doi.org/10.1007/s10257-022-00550-x} & Niedrig & Fokus liegt auf der systematischen Analyse von Erfolgsfaktoren und Hindernissen bei der unternehmensweiten Einführung von Data-Science-Strategien; methodische Entwicklung eines Enabler-Barrier-Frameworks für die erfolgreiche Umsetzung datengetriebener Projekte in Unternehmen; adressiert dabei organisatorische, technologische und strategische Aspekte der digitalen Transformation im breiten Kontext, jedoch ohne explizite Behandlung oder Integration von \ac{SSI}, Blockchain-Technologien, \ac{PQC} oder spezifischen Schutzanforderungen kritischer Infrastrukturen; liefert wertvolle Erkenntnisse zur Implementierung von Data Science im Unternehmensumfeld, ist für die vier Domänen der Masterarbeit methodisch und thematisch jedoch nicht anschlussfähig. \\
\midrule
61 & Kumar, Y., Marchena, J., Awlla, A. H., Li, J. J., \& Abdalla, H. B. (2024). The AI-Powered Evolution of Big Data. Applied Sciences (2076-3417), 14(22), 10176. \url{https://doi.org/10.3390/app142210176} & Niedrig & Fokus auf die Weiterentwicklung von Big-Data-Analyse und Management durch künstliche Intelligenz, mit Betonung neuer Rahmenwerke für die Charakterisierung und Handhabung großer, komplexer Datensätze. Der Beitrag stellt innovative AI-gestützte Tools (wie RAG-basierte Analyse-Bots/ChatGPT) zur Verbesserung der Datenanalyse vor und hebt methodologische Fortschritte im Bereich datengetriebene Entscheidungsunterstützung hervor. Keine explizite Behandlung oder Integration von \ac{SSI}, Blockchain-Technologien, \ac{PQC} oder besonderen Anforderungen kritischer Infrastrukturen; methodische und technologische Beiträge beschränken sich auf Big-Data-Management und AI-basierte Analytics, ohne Verknüpfung zu den vier zentralen Domänen der Masterarbeit. \\
\end{longtable}

\subsection{Zweite Iteration}

\subsubsection{Ergebnis Iteration 2}
\label{sec:Bewertung der identifizierten Quellen hinsichtlich ihrer Relevanz 2}

Iteration 2 - Bewertung der identifizierten Quellen hinsichtlich ihrer Relevanz

\begin{longtable}{L{0.5cm}L{4cm}L{1.5cm}L{7cm}}
    \caption[]{Bewertung der identifizierten Quellen hinsichtlich ihrer Relevanz}
    \label{tab:quellenbewertung_iteration2} \\
    \toprule
    \textbf{Nr.} & \textbf{Quelle} & \textbf{Relevanz} & \textbf{Kommentar} \\
    \midrule
    \endfirsthead
    \multicolumn{4}{l}{\textit{Tabelle \thetable\ (Fortsetzung)}} \\
    \toprule
    \textbf{Nr.} & \textbf{Quelle} & \textbf{Relevanz} & \textbf{Kommentar} \\
    \midrule
    \endhead
    \midrule
    \multicolumn{4}{r}{\textit{Fortsetzung auf nächster Seite}} \\
    \endfoot
    \bottomrule
    \multicolumn{4}{p{\linewidth}}{\textit{Anmerkung.} Basierend auf den Abstracts aller in Spalte zwei unter \enquote{Quelle} aufgeführten Quellenangaben.} \\
    \endlastfoot
1 & Barrett-danes, F., \& Ahmad, F. (2025). Quantum computing and cybersecurity: a rigorous systematic review of emerging threats, post-quantum solutions, and research directions (2019-2024). Discover Applied Sciences, 7(10). \url{https://doi.org/10.1007/s42452-025-07322-5} & Hoch & Setzt sich systematisch, methodisch und interdisziplinär mit den Bedrohungen durch Quantum Computing für klassische Kryptosysteme auseinander und behandelt ausdrücklich Post-Quantum-Kryptografie (PQC), hybride Frameworks (u.a. QKD), Umsetzungsherausforderungen und deren Auswirkungen insbesondere für IoT-Umgebungen. Die Arbeit analysiert sowohl den Forschungsstand als auch Implementierungs- und Migrationspfade anhand konkreter Pilotstudien und adressiert Skalierbarkeit und Wirtschaftlichkeit; Die Nutzung eines PRISMA-basierten Review-Frameworks und der Fokus auf praktische Handlungs- und Politikempfehlungen verleihen der Arbeit hohe wissenschaftliche und praxisnahe Relevanz für die Entwicklung, Migration und Absicherung quantensicherer sowie zukunftsfähiger Systeme mit starker Anschlussfähigkeit an KRITIS, PQC und angrenzende sicherheitskritische Domänen. \\
\midrule
2 & Feng, Z., Li, Z., Cui, H., \& Whitty, M. T. (2025). Identity management systems: A comprehensive review. Information (Basel), 16(9), 778. \url{https://doi.org/10.3390/info16090778} & Hoch & Der Abstract beschreibt eine umfassende und systematische Übersicht zu Blockchain-basierten Identity Management Systems (IDMSs) mit explizitem Fokus auf Self-Sovereign Identity (SSI), dezentralen Identifikatoren (DIDs), Interoperabilität und Sicherheitsanalyse über den gesamten Identitätslebenszyklus hinweg. Die Arbeit adressiert technologische und organisatorische Herausforderungen, wie Revokation, Übertragbarkeit, Interoperabilität und Quantum-Resilienz für nutzerkontrollierte Identitätsmodelle. Die PRISMA-basierte Methodik sowie die sektorübergreifende Taxonomisierung und die systematische Ableitung aktuell ungelöster Probleme und künftiger Forschungsrichtungen machen die Quelle höchst wertvoll im Umfeld SSI, Blockchain, PQC und KRITIS. \\
\midrule
3 & Akkal, M., Cherbal, S., Annane, B., Lakhlef, H., \& Kharoubi, K. (2025). Quantum, post-quantum, and blockchain approaches for securing the internet of medical things: a systematic review. Cluster Computing, 28(10). \url{https://doi.org/10.1007/s10586-025-05481-z} & Mittel & Behandelt systematisch die Bedrohungen durch Quantum Computing für IoMT und fokussiert explizit auf den Schutz medizinischer Infrastrukturen mit quantensicherer Kryptografie und Blockchain-Technologie; analysiert sowohl PQC- als auch Blockchain-basierte Sicherheitslösungen und deren Anwendung in hochsensiblen medizinischen IoT-Umgebungen, adressiert Architektur, Herausforderungen und Lösungsansätze für KRITIS-nahe Gesundheitssektoren; Fokus zu stark auf IoMT ohne SSI- und ohne Implementierungsbezug. \\
\midrule
4 & Elkhodr, M. (2025). An AI-driven framework for integrated security and privacy in Internet of Things using quantum-resistant blockchain. Future Internet, 17(6), 246. \url{https://doi.org/10.3390/fi17060246} & Mittel & Der Abstract beschreibt ein ganzheitliches Framework, das KI-basierte Security-Orchestrierung, Blockchain-gestützte Identitätsverwaltung und quantenresistente Kryptografie explizit miteinander kombiniert und anwendungsnah für IoT-Umgebungen evaluiert. Die Arbeit bietet indirekten Bezug zur technischen Umsetzung (Evaluierung) von PQC, SSI und KRITIS. \\
\midrule
5 & Kameni Tcheumaga, F. N., Umba, K., Velupillaimeikandan, P., Haque, M. M., \& Ahamed, S. I. (2025). Security of quantum federated machine learning with blockchain for electronics health records. Cureus Journal of Computer Science. \url{https://doi.org/10.7759/s44389-025-03870-4} & Mittel & Der Abstract adressiert systematisch die Integration von Quantum Federated Machine Learning (QFML), Blockchain und den Schutz elektronischer Gesundheitsakten, mit explizitem Fokus auf Security und Privacy in hochsensiblen, dezentralen Architekturen. Die Kombination aus quantenresistenter Kryptografie, Blockchain-basierter Validierung, federiertem Lernen und gezieltem Schutz kritischer Infrastrukturen im Gesundheitswesen wird als innovativer Lösungsansatz diskutiert. Die Arbeit analysiert bestehende Lücken hinsichtlich Robustheit und Sicherheit und liefert damit auch Impulse für zukünftige Forschung und Entwicklung in den Feldern PQC, KRITIS und dezentraler Identitätsarchitekturen. \\
\midrule
6 & Kurt, K. K., Timurtaş, M., Pınar, S., Ozaydin, F., \& Türkeli, S. (2025). Smart contracts, blockchain, and health policies: Past, present, and future. Information (Basel), 16(10), 853. \url{https://doi.org/10.3390/info16100853} & Mittel & Der Abstract liefert eine methodisch fundierte, systematische Übersicht zu Blockchain- und Smart-Contract-basierten Lösungen für die Verwaltung von Gesundheitsdaten und das Policy Management im Gesundheitswesen. Explizit adressiert werden Sicherheit, Datenschutz, Interoperabilität und die Rolle von Smart Contracts in der Durchsetzung und Automatisierung digitaler Gesundheitsrichtlinien. Es fehlt ein expliziter Fokus auf Post-Quantum Kryptografie in Verbindung mit Self-Sovereign Identity. \\
\midrule
7 & Qatawneh, M. (2025). A framework for security risk assessment of blockchain-based applications. Indonesian Journal of Electrical Engineering and Computer Science, 39(2), 952. \url{https://doi.org/10.11591/ijeecs.v39.i2.pp952-962} & Mittel & Der Abstract beschreibt ein systematisch entwickeltes und praxisvalidiertes Framework zur Bewertung und Mitigation von Sicherheitsrisiken in Blockchain-Anwendungen über alle Schichten hinweg. Es adressiert explizit kritische Schwachstellen wie Smart-Contract-Exploits, Sybil-Attacken und Private-Key-Compromises, integriert quantitative und qualitative Risikoanalyse und demonstriert konkrete Wirksamkeit in Form signifikanter Risikoreduktionen in einem Ethereum-Case-Study-Szenario. Das resultierende BCRAM bietet einen standardisierten, adaptierbaren Bewertungsansatz, der für sichere Systemarchitektur in KRITIS-, SSI- und PQC-orientierten Blockchain-Projekten nutzbar ist. \\
\midrule
8 & Reddy, N. R., Suryadevara, S., Reddy, K. G. R., Umamaheswari, R., Guttula, R., \& Kotoju, R. (2025). Quantum secured blockchain framework for enhancing post quantum data security. Scientific Reports, 15(1), 31048. \url{https://doi.org/10.1038/s41598-025-16315-8} & Mittel & Der Abstract stellt ein hochinnovatives, ganzheitliches Framework („QuantumShield-BC“) für die Absicherung von Blockchain-Architekturen gegen die Bedrohungen durch Quantencomputer vor: Es integriert explizit Post-Quantum-Kryptografie (z.B. Dilithium, Falcon), Quantum Key Distribution (QKD) und einen Quantum Byzantine Fault Tolerance (Q-BFT) Konsens-Mechanismus auf Basis von Quantum Random Number Generation (QRNG) für Leader- und Validator-Auswahl. Das System zeigt in experimenteller Validierung sehr geringe Latenz, hohe Durchsatzraten und vollständige Immunität gegenüber typischen Quantum-Angriffen (Shor, Grover), einschließlich effektiver Sybil-, Replay- und MITM-Abwehr. Die Architektur demonstriert praktisch den Weg für interoperable, skalierbare, hochsichere Blockchains ohne direkten Bezug zu KRITIS und SSI. \\
\midrule
9 & Tawfik, A. M., Al-Ahwal, A., Eldien, A. S. T., \& Zayed, H. H. (2025). Blockchain-based access control and privacy preservation in healthcare: a comprehensive survey. Cluster Computing, 28(8). \url{https://doi.org/10.1007/s10586-025-05308-x} & Mittel & Die Arbeit bietet eine systematische Übersicht über Blockchain-basierte Ansätze zur Zugriffssteuerung und Wahrung der Privatsphäre im Gesundheitswesen, mit besonderem Fokus auf permissioned und permissionless Frameworks, Smart Contracts, kryptographische Verfahren und Plattformen wie Hyperledger Fabric und Ethereum. Die Analyse der untersuchten Lösungen zeigt, wie feingranularer Zugriff, automatisierte Autorisierung und revisionssichere Auditierung durch Blockchain und Privacy-Preserving-Techniken realisiert werden können. Es fehlen Bezüge zu SSI. \\
\midrule
10 & Aboshosha, B. W., Zayed, M. M., Khalifa, H. S., \& Ramadan, R. A. (2025). Enhancing Internet of Things security in healthcare using a blockchain-driven lightweight hashing system. Beni-Suef University Journal of Basic and Applied Sciences, 14(1). \url{https://doi.org/10.1186/s43088-025-00644-8} & Niedrig & Fokussiert auf die Verbesserung der Datensicherheit und Integrität in IoT-basierten Gesundheitsanwendungen durch Blockchain und leichtgewichtige Hashverfahren; adressiert wesentliche Aspekte dezentraler Datenverwaltungsmodelle und kryptographischer Effizienz für ressourcenbeschränkte Geräte; PQC und SSI werden jedoch nicht behandelt, und der methodische Fokus liegt auf klassischen leichten Hashfunktionen statt quantensicheren Primitive. \\
\midrule
11 & Addula, S. R. (2025). Mobile banking adoption: A multi-factorial study on social influence, compatibility, digital self-efficacy, and perceived cost among generation Z consumers in the United States. Journal of Theoretical and Applied Electronic Commerce Research, 20(3), 192. \url{https://doi.org/10.3390/jtaer20030192} & Niedrig & Konzentriert sich auf nutzerpsychologische und soziotechnische Faktoren zur Akzeptanz mobiler Banking-Anwendungen im FinTech-Bereich mit Fokus auf Konsumentenverhalten; weder Kryptografie, Blockchain, noch dezentrale Identitätssysteme werden thematisiert; auch fehlen technische oder sicherheitsbezogene Bezüge zu PQC, SSI oder KRITIS. \\
\midrule
12 & Al Jasem, M. S., De Clark, T., \& Shrestha, A. K. (2025). Toward decentralized intelligence: A systematic literature review of blockchain-enabled AI systems. Information (Basel), 16(9), 765. \url{https://doi.org/10.3390/info16090765} & Niedrig & Die Arbeit liefert einen systematischen Überblick zu Blockchain-gestützten dezentralen KI-Systemen und adressiert damit relevante Aspekte dezentraler Architekturen, insbesondere durch die Untersuchung von Governance, Integrität, skalierbaren Konsensmechanismen sowie Sicherheits- und Datenschutzherausforderungen. Explizite technologische Bezüge zu Self-Sovereign Identity, Post-Quantum-Kryptografie oder kritischen Infrastrukturen fehlen allerdings; stattdessen steht das Zusammenwirken von Blockchain, Smart Contracts und KI im Fokus. \\
\midrule
13 & Almazroi, A. A., Alqarni, M. A., Al-Shareeda, M. A., Alkinani, M. H., Almazroey, A. A., \& Gaber, T. (2025). A bilinear pairing-based anonymous authentication scheme for 5G-assisted vehicular fog computing. Arabian Journal for Science and Engineering, 50(15), 11757–11778. \url{https://doi.org/10.1007/s13369-024-09617-y} & Niedrig & Die Arbeit konzentriert sich auf Authentifizierungsverfahren für 5G-unterstützte Vehicular Fog Computing-Systeme unter Nutzung bilinearer Parings; Im Mittelpunkt stehen effiziente Überprüfungsmechanismen, die Anonymität, Authentizität und praktische Sicherheit im Fahrzeugnetz verbessern; dezentrale Identitätsarchitekturen, Blockchain, PQC oder spezielle Konzepte für Self-Sovereign Identity oder den Schutz kritischer Infrastrukturen werden jedoch nicht behandelt. \\
\midrule
14 & Alsadie, D. (2025). Cybersecurity and artificial intelligence in unmanned aerial vehicles: Emerging challenges and advanced countermeasures. IET Information Security, 2025(1). \url{https://doi.org/10.1049/ise2/2046868} & Niedrig & Stellt eine fundierte und breit gefächerte Übersicht zu aktuellen Bedrohungen und fortschrittlichen Gegenmaßnahmen im Bereich AI-gestützter UAV-Systeme bereit und greift dabei gleich mehrere Schlüsseltechnologien des Bewertungsschemas direkt und explizit auf: Es werden konkrete Methoden der Post-Quantum-Kryptografie (PQC) und blockchain-basierte Sicherheitsmechanismen analysiert und hinsichtlich ihrer Wirksamkeit im Kontext hochsensibler und autonom agierender Drohnennetze bewertet; Kein Bezug zu KRITIS \& SSI; \\
\midrule
15 & Cavus, M., Ayan, H., Bell, M., \& Dissanayake, D. (2025). Advances in energy storage, AI optimisation, and cybersecurity for electric vehicle grid integration. Energies, 18(17), 4599. \url{https://doi.org/10.3390/en18174599} & Niedrig & Die Arbeit liefert einen integrativen und interdisziplinären Überblick zu drei zentralen Säulen hochrelevanter Technologien: Fortschritte bei sicheren dezentralen Energiespeichern, KI-basierter Optimierung für Echtzeit-Energiemanagement und den Einsatz post-quantum Kryptografie und Blockchain-Systemen für die Absicherung von Vehicle-to-Grid (V2G)-Transaktionen im Smart-Grid-Kontext. Konkret fehlt es an praktischen Beiträgen bzgl. Implementierung von SSI, PQC und Blockchain für KRITIS. \\
\midrule
16 & A, C., \& Basarkod, P. I. (2024). A survey on blockchain security for electronic health record. Multimedia Tools and Applications. \url{https://doi.org/10.1007/s11042-024-19883-5} & Niedrig & Der Abstract beschreibt eine systematische Übersicht zu Blockchain-basierten Sicherheitsansätzen für elektronische Gesundheitsakten (EHR), mit Schwerpunkt auf Datenschutz, Zugriffskontrolle, Integritätsprüfung und Effizienz, insbesondere auch mit Blick auf Deep-Learning-Ansätze im Healthcare-Umfeld. Zwar adressiert die Arbeit relevante Aspekte dezentraler vernetzter Datenstrukturen und gibt einen fundierten Vergleich vorhandener Blockchain-basierten Technologien für EHR-Sicherheit, doch stehen SSI, PQC oder KRITIS-spezifische Anwendungen nicht explizit im Fokus. \\
\midrule
17 & Guayasamín, A., Fuertes, W., Carrera, N., Tello-Oquendo, L., \& Suango, V. (2025). Blockchain-empowered e-ticket distribution system for secure and efficient transactions, validation, and audits. Annals of Telecommunications - Annales Des Télécommunications. \url{https://doi.org/10.1007/s12243-025-01125-w} & Niedrig & Der Abstract beschreibt ein Blockchain-basiertes System zur sicheren und effizienten Verwaltung, Prüfung und Auditierung von E-Tickets im Unterhaltungsbereich, insbesondere für Raffles und Veranstaltungsmanagement. Der Fokus liegt auf betrugsresistenter Ticketvergabe, Transparenz, digitaler Signierung, Hash-Funktionen und Proof-of-Work-basiertem Mining, was für die Optimierung von Fälschungsschutz und Automatisierung in öffentlichen Systemen relevant ist. Es besteht kein direkter fachlicher Bezug zu den Kerndomänen PQC, SSI, oder KRITIS. \\
\midrule
18 & He, X., Xu, G., Han, X., Wang, Q., Zhao, L., Shen, C., … Feng, D. (2025). Artificial intelligence security and privacy: a survey. Science China Information Sciences, 68(8). \url{https://doi.org/10.1007/s11432-025-4388-5} & Niedrig & Der Abstract bietet eine breite, systematische Übersicht über Sicherheits- und Datenschutzherausforderungen von KI-Systemen, inklusive Bedrohungen für Datenintegrität, Trainings-, Inferenzphasen und verteilter Umgebungen. Zentral sind klassische KI-Attacken wie Datenvergiftung, Backdoor und Adversarial Attacks, jedoch fehlt ein expliziter Bezug zu Self-Sovereign Identity, Blockchain-Integration, Post-Quantum Kryptografie oder KRITIS-spezifischen Sicherheitsarchitekturen. \\
\midrule
19 & Khagga, V., Priya., S., \& Prasad, A. M. (2025). Enhanced QoS-aware secure routing protocol for WAHNs using advanced fast double decker new binary archimedes kepler pure convolutional transformer network and cryptographic techniques. Peer-to-Peer Networking and Applications, 18(4). \url{https://doi.org/10.1007/s12083-025-02035-3} & Niedrig & Der Abstract stellt eine klassische netzwerktechnische Studie dar, die ein innovatives Routing-Protokoll für Wireless Ad-Hoc Networks (WAHNs) vorstellt und dabei verschiedene Optimierungsverfahren, Deep-Learning-Modelle und komplexe kryptografische Techniken integriert. Die Arbeit adressiert primär technische Fragen der Effizienzsteigerung, Clusterbildung, Verzögerungsminimierung und Zugangssicherheit auf Netzwerkebene, beschränkt sich aber methodisch auf klassische und KI-gestützte kryptografische Mechanismen ohne expliziten Bezug zu PQC, Blockchain, SSI oder KRITIS-spezifischen Identitäts- beziehungsweise Infrastrukturarchitekturen. \\
\midrule
20 & Kumar, M., Kaur, G., \& Rana, P. S. (2025). Performance, portability, productivity, and security in HPC cloud: a systematic literature review. The Journal of Supercomputing, 81(11). \url{https://doi.org/10.1007/s11227-025-07685-x} & Niedrig & Der Abstract liefert eine umfassende systematische Übersicht aktueller Entwicklungstrends und Herausforderungen im Bereich Cloud-basierter Hochleistungsrechner (HPC), strukturiert entlang der Aspekte Performance, Portabilität, Produktivität und Security. Während das Security-Kapitel innovative Technologien wie Trusted Execution Environments, Verschlüsselung und feingranulare Zugangskontrolle behandelt und so auf zentrale Anforderungen in sensiblen Anwendungsdomänen eingeht, fehlt ein expliziter Fokus auf Post-Quantum Kryptografie, Self-Sovereign Identity oder dezentrale Identitätsarchitekturen im KRITIS-Kontext. \\
\midrule
21 & Lubis, M., Safitra, M. F., Fakhrurroja, H., \& Muttaqin, A. N. (2025). Guarding our vital systems: A metric for critical infrastructure cyber resilience. Sensors (Basel, Switzerland), 25(15), 4545. \url{https://doi.org/10.3390/s25154545} & Niedrig & Dieses Paper adressiert direkt die Entwicklung, Messung und Erhöhung der Cyber-Resilienz für kritische Infrastrukturen anhand des InfraGuard Cybersecurity Frameworks, welches etablierte Reifegradmodelle (ISO/IEC 15504, NIST CSF, COBIT) einbindet und die gesamte Breite an Abwehr-, Verteidigungs- und Wiederherstellungsmaßnahmen strukturiert abbildet. Im Mittelpunkt stehen situative Awareness, aktive Verteidigung, Risikomanagement und Incident Recovery. Es fehlt ein expliziter Fokus auf Post-Quantum Kryptografie in Verbindung mit Self-Sovereign Identity. \\
\midrule
22 & Marengo, A., \& Santamato, V. (2025). Quantum algorithms and complexity in healthcare applications: a systematic review with machine learning-optimized analysis. Frontiers in Computer Science, 7(1584114). \url{https://doi.org/10.3389/fcomp.2025.1584114} & Niedrig & Die Studie bietet eine systematische Übersicht zu quantencomputergestützten Algorithmen, deren algorithmischer Komplexität und Anwendungsbereichen im Gesundheitswesen, insbesondere für KI-gestützte Diagnostik und die Sicherheit medizinischer Daten. Sie kombiniert fortgeschrittene Literaturanalyse (PSO, LDA, LIME) mit einer empirisch validierten Kategorisierung der Forschung in „Quantum für KI in Healthcare“ und „Quantum für Datensicherheit im Gesundheitswesen“. Besonderes Gewicht liegt auf quantenkryptografischen Protokollen, Blockchain-basierten Sicherheitsarchitekturen und hybriden Quantum-KI-Ansätzen, die innovative Lösungsansätze für den Schutz sensibler Gesundheitsdaten, die Beschleunigung diagnostischer Prozesse und die Entwicklung resilienter Infrastrukturmodelle liefern. Es fehlt ein konkreter Bezug zu SSI und KRITIS. \\
\midrule
23 & Meka, C., Palakollu, K. R., Azees, M., Rajasekaran, A. S., Das, A. K., \& Hölbl, M. (2025). A comprehensive survey on integration of machine learning with secure blockchain-based applications. Cluster Computing, 28(10). \url{https://doi.org/10.1007/s10586-025-05330-z} & Niedrig & Der Abstract beschreibt eine systematische Übersicht zu Anwendungen, Herausforderungen und offenen Problemen an der Schnittstelle von Machine Learning und Blockchain, insbesondere hinsichtlich Sicherheit, Automatisierung, Auditing und Skalierung in IoT- und Industrieszenarien. Der Beitrag legt den Fokus auf die beidseitigen Synergien beider Technologien—z.B. Fraud Detection, Schutz der Modellintegrität, sichere Prozessautomatisierung per Smart Contracts und Anomalieerkennung—und skizziert aktuelle Forschung, technische Integrationsaspekte und sektorübergreifende Use Cases. Explizite Bezüge zu PQC, Self-Sovereign Identity oder KRITIS-spezifischen Identitäts- beziehungsweise Infrastrukturarchitekturen werden jedoch nicht vertieft adressiert. \\
\midrule
24 & Mustafa, R., Sarkar, N. I., Mohaghegh, M., Pervez, S., \& Vohra, O. (2025). Cross-layer analysis of machine learning models for secure and energy-efficient IoT networks. Sensors (Basel, Switzerland), 25(12), 3720. \url{https://doi.org/10.3390/s25123720} & Niedrig & Die Arbeit verfolgt einen innovativen Ansatz zur Harmonisierung von IoT-Sicherheit und Energieeffizienz durch eine umfassende, schichtübergreifende Architektur, die spezialisierte Machine-Learning-Modelle (z. B. LSTM-Netzwerke zur Anomalieerkennung, Entscheidungsbäume zur Validierung) mit leichtgewichtiger, adaptiver Kryptografie (Speck) koppelt. Durch rollenspezifische Zugriffskontrolle (RBAC) und energieorientierte Sicherheitspolitiken adressiert die Studie zentrale Herausforderungen für ressourcenbeschränkte IoT-Geräte und demonstriert signifikante Verbesserungen bei Fehlalarmraten, Zugangssicherheit und Energieverbrauch. Explizite Bezüge zu PQC, Self-Sovereign Identity oder KRITIS-spezifischen Identitäts- beziehungsweise Infrastrukturarchitekturen werden jedoch nicht vertieft adressiert. \\
\midrule
25 & Mutahhar, A., Khanzada, T. J. S., \& Shahid, M. F. (2025). Enhanced scalability and security in blockchain-based transportation systems for mass gatherings. Information (Basel), 16(8), 641. \url{https://doi.org/10.3390/info16080641} & Niedrig & Der Abstract beschreibt ein Blockchain-basiertes System zur Erhöhung von Effizienz und Sicherheit in intelligenten Transportsystemen, insbesondere für Massengroßveranstaltungen und städtische Mobilitätsnetzwerke. Die Lösung integriert State Channels und Rollups zur Skalierungsoptimierung, erreicht hohe Transaktionsgeschwindigkeiten und adressiert die Herausforderungen der Datenmanipulation, Integrität und Verschlüsselung in urbanen Verkehrsnetzen. Die Konzeption ist praxisnah für dynamische, datenschutzsensitive Transportinfrastrukturen, adressiert aber weder explizit Self-Sovereign Identity, Post-Quantum Kryptografie noch Anwendungsfälle für KRITIS-nahe Domänen. \\
\midrule
26 & Pillai, S. E. V. S., Nadella, G. S., Meduri, K., Priyadharsini, N. A., Bhuvanesh, A., \& Kumar, D. (2025). A walrus optimization-enhanced long short-term memory model for credit fraud detection in banking. International Journal of Information Technology. \url{https://doi.org/10.1007/s41870-025-02574-1} & Niedrig & Der Abstract beschreibt ein fortgeschrittenes Framework zur Nutzung von Autoencoder, LSTM-Netzwerken und Walrus Optimization Algorithm (WOA) für die Optimierung von Kreditkartenbetrugserkennung in Bankdaten. Die Kombination von modernen Machine-Learning-Techniken und bio-inspirierten Metaheuristiken erhöht die Performance und Skalierbarkeit im Datenmanagement klassischer Finanzsysteme, bleibt aber ohne expliziten Bezug zu Blockchain, Self-Sovereign Identity, Post-Quantum Kryptografie, kritischer Infrastruktur oder dezentralen Identitätsarchitekturen. \\
\midrule
27 & Rao, C. K., Sahoo, S. K., \& Yanine, F. F. (2025). A review of IoT-based smart energy solutions for photovoltaic systems. Electrical Engineering (Berlin. Print), 107(12), 15049–15068. \url{https://doi.org/10.1007/s00202-025-03312-3} & Niedrig & Der Abstract gibt einen breiten Überblick über den Stand und die Rolle von IoT-basierten Monitoring- und Managementsystemen für Photovoltaik-Anlagen, insbesondere zur Optimierung von Energieeffizienz, Datenanalyse, Cloud-Integration und Betriebsplanung. Der Schwerpunkt liegt auf technologischen Innovationen rund um IoT, Smart Grids, Energieverwaltung und Echtzeitdatenerfassung für industrielle und wissenschaftliche Anwendungen. Methodisch und thematisch fehlt jedoch ein Bezug zu Self-Sovereign Identity, Blockchain, Post-Quantum Kryptografie oder zur Absicherung kritischer digitaler Versorgungsinfrastrukturen. \\
\midrule
28 & Sefati, S. S., Arasteh, B., Halunga, S., \& Fratu, O. (2025). A comprehensive survey of cybersecurity techniques based on quality of service (QoS) on the Internet of Things (IoT). Cluster Computing, 28(12). \url{https://doi.org/10.1007/s10586-025-05449-z} & Niedrig & Der Abstract liefert einen systematischen Überblick zu aktuellen Cybersecurity-Techniken zur QoS-bewussten Absicherung von IoT-Systemen, mit besonderem Fokus auf die Trade-offs zwischen Sicherheitsmechanismen (etwa Anomalieerkennung, Hybrid- und KI-basierte Methoden, Blockchain, Federated Learning) und kritischen Leistungsparametern (Latenz, Durchsatz, Energieverbrauch) in ressourcenbeschränkten Umgebungen. Der Beitrag beleuchtet vielseitige Angriffsszenarien (DDoS, MITM, Datenextraktion), validiert aktuelle Detektionsparadigmen und hebt offene Forschungsfelder hervor, etwa adaptive und echtzeitfähige Security-Modelle sowie Recovery-Maßnahmen nach Angriffen. Explizite Bezüge zu PQC, SSI oder KRITIS-spezifischen Architekturparadigmen werden nicht gesetzt. \\
\midrule
29 & Shahzad, M., Rizvi, S., Khan, T. A., Ahmad, S., \& Ateya, A. A. (2025). An exhaustive parametric analysis for securing SDN through traditional, AI/ML, and blockchain approaches: A systematic review. International Journal of Networked and Distributed Computing, 13(1). \url{https://doi.org/10.1007/s44227-024-00055-8} & Niedrig & Das Paper bietet eine umfassende, systematische Analyse und Vergleich der wichtigsten Sicherheitsansätze für Software-defined Networking (SDN), inklusive klassischer, Machine-Learning- und Blockchain-basierter Methoden. Die Literaturauswertung demonstriert, dass Blockchain-basierte Mechanismen für Flussregelung, Datenvalidierung, Parser und Controller-Authentifizierung die Robustheit und Angriffsresistenz von SDN signifikant steigern und Aspekte wie Integrität, Trust und Dezentralität adressieren. Machine-Learning-Technologien (CNN, SVM, KNN) liefern praxisnahe Mehrwerte bei der Angriffserkennung. Jedoch bleibt ein expliziter Bezug zu Post-Quantum Kryptografie, Self-Sovereign Identity oder KRITIS-spezifischen Identitätsarchitekturen aus. \\
\midrule
30 & Singh, B., Indu, S., \& Majumdar, S. (2025). Comparative analysis of intrusion detection models using quantum machine learning techniques. Circuits, Systems, and Signal Processing. \url{https://doi.org/10.1007/s00034-025-03256-w} & Niedrig & Der Abstract und die zitierten Studien zeigen, dass Quantum Machine Learning-Techniken bei der Intrusion Detection gegenüber klassischen Machine Learning-Ansätzen, insbesondere in großen Netzwerken und komplexen Datenlagen, signifikante Vorteile in Genauigkeit und Performanz bieten können. Die experimentellen Vergleiche (z.B. QSVM, QCNN, VQC) über mehrere bekannte Datensätze belegen, dass quantenunterstützte Modelle wie das QML-IDS eine robustere Erkennung von Angriffsmustern (z.B. DDoS, BruteForce, Reconnaissance) und eine bessere Generalisierbarkeit im Kontext von Post-Quantum Cryptography und modernen Cyberbedrohungen erreichen. Auch werden praktische Herausforderungen und die Skalierbarkeit hybrider Modelle thematisiert Ein expliziter Bezug zu Post-Quantum Kryptografie, Self-Sovereign Identity oder KRITIS-spezifischen Identitätsarchitekturen bleibt aus.\\
\midrule
31 & Tom, A. K., Khraisat, A., Jan, T., Whaiduzzaman, M., Nguyen, T. D., \& Alazab, A. (2025). Survey of federated learning for cyber threat intelligence in industrial IoT: Techniques, applications and deployment models. Future Internet, 17(9), 409. \url{https://doi.org/10.3390/fi17090409} & Niedrig & Die Arbeit liefert einen detaillierten und breit gefächerten Überblick zu aktuellen Methoden für cyber threat intelligence (CTI) im Kontext industrieller IoT-Umgebungen (IIoT) unter expliziter Berücksichtigung von föderiertem Lernen (FL) als Schlüsseltechnologie für datenschutzwahrende, skalierbare und dezentrale Bedrohungsanalyse. Die systematische Betrachtung von FL-Architekturen, Aggregationsstrategien (z.B. FedAvg, FedProx, Krum) und deren Anwendungen auf Intrusion Detection, Malware-Analyse, Botnet-Mitigation und Anomalieerkennung demonstriert praxisrelevante Fortschritte für KRITIS-nahe und industriell ausgerichtete Sicherheitsarchitekturen. Es fehlen Bezüge zu SSI, Blockchain und PQC. \\
\midrule
32 & Zhang, Y., Zhao, K., Yang, Y., \& Zhou, Z. (2025). Real-time service migration in edge networks: A survey. Journal of Sensor and Actuator Networks, 14(4), 79. \url{https://doi.org/10.3390/jsan14040079} & Niedrig & Der Abstract stellt eine umfassende systematische Übersicht zu Echtzeit-Service-Migration in Edge-Netzwerken bereit, einschließlich Architekturen, Modellen, Motivationen, Techniken und Anwendungsszenarien (z.B. Smart Cities, Smart Homes, Smart Manufacturing). Der Schwerpunkt liegt auf Algorithmen, Modellen und Methoden zur Reduktion von Latenz, Lastverteilung und dynamischer Ressourcenzuteilung für zeitkritische, verteilte Dienste. Sicherheit, Privacy, PQC, Blockchain oder dezentrale Identitätsarchitekturen werden nicht explizit adressiert. Der Beitrag ist für Netzwerkarchitektur und Echtzeitfähigkeit relevant, liefert aber keinen direkten methodischen oder konzeptionellen Beitrag zu SSI/PQC/KRITIS. \\
\midrule
33 & Zinabu, N. G., Marye, Y. W., Tune, K. K., \& Demilew, S. A. (2025). Comprehensive analysis of lightweight cryptographic algorithms for battery‐limited Internet of Things devices. International Journal of Distributed Sensor Networks, 2025(1). \url{https://doi.org/10.1155/dsn/9639728} & Niedrig & Die Arbeit liefert eine systematische und breit angelegte Analyse zu aktuellen Lightweight-Kryptografie-Algorithmen speziell für energie- und ressourcenbeschränkte IoT-Geräte. Verglichen werden dabei unter anderem Algorithmen wie ASCON, SPECK, PRINCE, TWINE und modifizierte Varianten von AES-128 hinsichtlich Effizienz, Sicherheit, Energieverbrauch und Implementierungskomplexität in konkreten Hardware- und Softwareszenarien. Besonders hervorgehoben werden die trade-offs zwischen Durchsatz, Speicher- und Energiebedarf sowie Angriffstoleranz. Der systematische Review zeigt, dass etwa ASCON als sichere, performante Allround-Lösung für Authenticated Encryption anerkannt ist, während SPECK in Szenarien mit höchsten Durchsatzanforderungen durch Einfachheit punktet, jedoch unter größerer kritischer Scrutiny in puncto langfristiger Sicherheit steht. PQC- oder SSI-Verfahren werden in diesem Kontext nicht umfassend adressiert. \\
\midrule
34 & Zreikat, A. I., AlArnaout, Z., Abadleh, A., Elbasi, E., \& Mostafa, N. (2025). The integration of the Internet of Things (IoT) applications into 5G networks: A review and analysis. Computers, 14(7), 250. \url{https://doi.org/10.3390/computers14070250} & Niedrig & Die Arbeit bietet einen fundierten Review und eine Analyse zur Integration von IoT-Anwendungen in 5G-Netzwerke mit Schwerpunkt auf Konnektivität, Datenrate, Latenz, Interoperabilität und Anwendungsvielfalt (Smart Cities, Industrie 4.0, Healthcare etc.). Methodisch werden technische Potenziale (wie Network Slicing, Edge Computing, massive Machine-Type-Kommunikation) und Herausforderungen (Sicherheit, Energie, Netzmanagement) systematisch aufgearbeitet. Zwar werden Security und Privacy angesprochen, jedoch gibt es keinerlei dezidierte Analyse zu PQC, Self-Sovereign Identity, Blockchain, dezentrale Identitätsarchitekturen oder KRITIS-spezifische Schutzmaßnahmen. \\
\end{longtable}


\newpage
\section{Artefaktentwicklung Iteration 1} \label{sec:Anhang_Artefaktentwicklung Iteration 1}

\subsection{Zertifikatserstellungsworkflow}

\refstepcounter{manualListingCounter}
\label{lst:Zertifikatserstellungsworkflow}
\begin{lstlisting}[language=bash, caption={Listing \arabic{lstlisting}: Zertifikatserstellungsworkflow}, numbers=left, frame=single]
# Openssl 3.5.4 LTS (version mit PQC support)
OpenSSL 3.5.4 30 Sep 2025 (Library: OpenSSL 3.5.4 30 Sep 2025)

# ML-DSA-87 Private Key (höchste Sicherheit für Root)
openssl genpkey -algorithm mldsa87 -out rootCA.key

# Self-signed Root Certificate (10 Jahre)
openssl req -x509 -new -key rootCA.key -out rootCA.crt \
  -days 3650 -subj "/CN=My PQC Root CA/O=MyOrg/C=DE" \
  -addext "basicConstraints=critical,CA:TRUE" \
  -addext "keyUsage=critical,keyCertSign,cRLSign"

# ML-DSA-65 Private Keys für Sidecar Proxy
openssl genpkey -algorithm mldsa65 -out server.key

# Certificate Signing Requests (CSRs) für Sidecar Proxy mit SAN
openssl req -new -key server.key -out server.csr \
    -subj "/CN=server proxy/O=FM/C=DE" \
    -addext "subjectAltName=DNS:issuer,DNS:pqc-sidecarproxy-issuer,DNS:host.docker.internal,DNS:localhost,IP:127.0.0.1"

# Signierung mit ML-DSA-65 (Balance zwischen Sicherheit und Performance)
openssl x509 -req -in von-webserver-pqc-proxy.csr \
  -CA rootCA.crt -CAkey rootCA.key -CAcreateserial \
  -out von-webserver-pqc-proxy.crt -days 365 -sha3-256 \
  -copy_extensions copy
\end{lstlisting}

\subsection{Dockerfile: Sidecar Proxy nginx}

\refstepcounter{manualListingCounter}
\label{lst:Dockerfile-Sidecar-Proxy-nginx}
\begin{lstlisting}[language=bash, caption={Listing \arabic{lstlisting}: Dockerfile - Sidecar Proxy nginx}, numbers=left, frame=single]
# Nginx with OpenSSL 3.5.4 (LTS) + OQS Provider for Post-Quantum Cryptography
# Based on: https://github.com/open-quantum-safe/oqs-demos/blob/main/nginx/Dockerfile
#
# Customizations:
# - OpenSSL 3.5.4 (LTS with Security Fixes)
# - Uses custom certificates (mounted via volume)
# - ML-KEM-768 Key Exchange enabled

# Define build arguments for version tags, installation paths, and configurations
ARG ALPINE_VERSION=3.21
ARG OPENSSL_TAG=openssl-3.5.4
ARG LIBOQS_TAG=0.13.0
ARG OQSPROVIDER_TAG=0.9.0
ARG NGINX_VERSION=1.28.0
ARG BASEDIR="/opt"
ARG INSTALLDIR=${BASEDIR}/nginx

# Specify supported signature and key encapsulation mechanisms (KEM) algorithms
ARG SIG_ALG="mldsa65"
ARG DEFAULT_GROUPS=X25519MLKEM768:mlkem768:x25519:mlkem1024

# Stage 1: Build - Compile and assemble all necessary components and dependencies
FROM alpine:${ALPINE_VERSION} AS intermediate
ARG OPENSSL_TAG
ARG LIBOQS_TAG
ARG OQSPROVIDER_TAG
ARG NGINX_VERSION
ARG BASEDIR
ARG INSTALLDIR
ARG SIG_ALG
ARG DEFAULT_GROUPS
ARG OSSLDIR=${BASEDIR}/openssl/.openssl

# Install required build tools and system dependencies
RUN apk update && apk --no-cache add \
    build-base linux-headers libtool \
    automake autoconf make cmake ninja \
    openssl openssl-dev git wget pcre-dev

# Download and prepare source files needed for the build process
WORKDIR /opt
RUN git clone --depth 1 --branch ${LIBOQS_TAG} https://github.com/open-quantum-safe/liboqs \
    && git clone --depth 1 --branch ${OQSPROVIDER_TAG} https://github.com/open-quantum-safe/oqs-provider.git \
    && git clone --depth 1 --branch ${OPENSSL_TAG} https://github.com/openssl/openssl.git \
    && wget -q nginx.org/download/nginx-${NGINX_VERSION}.tar.gz \
    && tar -zxf nginx-${NGINX_VERSION}.tar.gz \
    && rm nginx-${NGINX_VERSION}.tar.gz

# Build and install OpenSSL with shared libraries (for curl and nginx)
WORKDIR /opt/openssl
RUN ./Configure --prefix=${OSSLDIR} \
                --openssldir=${OSSLDIR}/ssl \
                shared \
                enable-fips \
    && make -j"$(nproc)" \
    && make install_sw install_ssldirs

# Configure OpenSSL to support the oqs-provider
RUN cp /opt/openssl/apps/openssl.cnf ${OSSLDIR}/ssl/ && \
    sed -i "s/default = default_sect/default = default_sect\noqsprovider = oqsprovider_sect/g" ${OSSLDIR}/ssl/openssl.cnf && \
    sed -i "s/\[default_sect\]/\[default_sect\]\nactivate = 1\n\[oqsprovider_sect\]\nactivate = 1\n/g" ${OSSLDIR}/ssl/openssl.cnf && \
    sed -i "s/providers = provider_sect/providers = provider_sect\nssl_conf = ssl_sect\n\n\[ssl_sect\]\nsystem_default = system_default_sect\n\n\[system_default_sect\]\nGroups = \$ENV\:\:DEFAULT_GROUPS\n/g" ${OSSLDIR}/ssl/openssl.cnf && \
    sed -i "s/HOME\t\t\t= ./HOME\t\t= .\nDEFAULT_GROUPS\t= ${DEFAULT_GROUPS}/g" ${OSSLDIR}/ssl/openssl.cnf

# Build and install liboqs
WORKDIR /opt/liboqs/build
RUN cmake -G"Ninja"  \
    -DOQS_DIST_BUILD=ON  \
    -DBUILD_SHARED_LIBS=OFF  \
    -DCMAKE_INSTALL_PREFIX="${INSTALLDIR}" ..  \
    && ninja -j"$(nproc)" && ninja install

# Build and install Nginx with shared OpenSSL
WORKDIR /opt/nginx-${NGINX_VERSION}
RUN ./configure --prefix=${INSTALLDIR} \
    --with-debug --with-http_ssl_module  \
    --with-cc-opt="-I${OSSLDIR}/include" \
    --with-ld-opt="-L${OSSLDIR}/lib64 -Wl,-rpath,${OSSLDIR}/lib64" \
    --without-http_gzip_module && \
    make -j"$(nproc)" && make install

# Build and install OQS provider
WORKDIR /opt/oqs-provider
RUN ln -s "/opt/nginx/include/oqs" "${OSSLDIR}/include" && \
    rm -rf build && \
    cmake -DCMAKE_BUILD_TYPE=Debug \
          -DOPENSSL_ROOT_DIR="${OSSLDIR}" \
          -DCMAKE_PREFIX_PATH="${INSTALLDIR}" \
          -S . -B build && \
    cmake --build build && \
    MODULESDIR=$(find "${OSSLDIR}" -name ossl-modules -type d | head -1) && \
    export MODULESDIR && \
    cp build/lib/oqsprovider.so "${MODULESDIR}" && \
    rm -rf "${INSTALLDIR:?}/lib64"

# Build curl with shared OpenSSL 3.5.4
ARG CURL_VERSION=8.11.1
WORKDIR /opt
RUN wget -q https://curl.se/download/curl-${CURL_VERSION}.tar.gz && \
    tar -zxf curl-${CURL_VERSION}.tar.gz && \
    rm curl-${CURL_VERSION}.tar.gz

WORKDIR /opt/curl-${CURL_VERSION}
RUN LDFLAGS="-Wl,-rpath,${OSSLDIR}/lib64 -L${OSSLDIR}/lib64" \
    PKG_CONFIG_PATH="${OSSLDIR}/lib64/pkgconfig" \
    ./configure \
    --prefix=${INSTALLDIR} \
    --with-openssl=${OSSLDIR} \
    --with-ca-bundle=/etc/ssl/certs/ca-certificates.crt \
    --disable-manual \
    --disable-ldap \
    --disable-ldaps \
    --without-libpsl \
    --without-zlib \
    --without-brotli \
    --without-zstd && \
    make -j"$(nproc)" && \
    make install

# Minimize image size by stripping binaries
WORKDIR ${INSTALLDIR}
ENV PATH="${INSTALLDIR}/sbin:${INSTALLDIR}/bin:${OSSLDIR}/bin:${PATH}"

RUN set -ex && \
    strip "${OSSLDIR}/lib64/"*.a \
          "${OSSLDIR}/lib64/ossl-modules/oqsprovider.so" \
          "${INSTALLDIR}/sbin/"* \
          "${INSTALLDIR}/bin/curl" \
          "${OSSLDIR}/bin/openssl" && \
    mkdir -p certs

# Stage 2: Runtime - Create a lightweight image with essential binaries and configurations
FROM alpine:${ALPINE_VERSION}
ARG INSTALLDIR
ARG BASEDIR
ARG OSSLDIR=${BASEDIR}/openssl/.openssl

# Install runtime dependencies
RUN apk update && apk --no-cache add pcre-dev ca-certificates

# Copy compiled artifacts and configuration from the intermediate stage
COPY --from=intermediate ${INSTALLDIR} ${INSTALLDIR}
COPY --from=intermediate ${OSSLDIR} ${OSSLDIR}

# Link logs to Docker collector
RUN set -ex && \
    mkdir -p "${INSTALLDIR}/logs" && \
    ln -sf /dev/stdout "${INSTALLDIR}/logs/access.log" && \
    ln -sf /dev/stderr "${INSTALLDIR}/logs/error.log"

# Expose HTTPS port
# EXPOSE 443

# Set OpenSSL configuration environment
# From Nginx 1.25.2: "nginx does not try to load OpenSSL configuration if the
# --with-openssl option was used to build OpenSSL and the OPENSSL_CONF
# environment variable is not set." Hence we must explicitly set OPENSSL_CONF.
ENV PATH="${INSTALLDIR}/sbin:${INSTALLDIR}/bin:${OSSLDIR}/bin:${PATH}" \
    OPENSSL_CONF="${OSSLDIR}/ssl/openssl.cnf" \
    DEFAULT_GROUPS="X25519MLKEM768:mlkem768:x25519:mlkem1024"

# Create non-root user and update permissions
RUN addgroup -g 1000 -S oqs \
 && adduser --uid 1000 -S oqs -G oqs \
 && chown -R oqs:oqs "${INSTALLDIR}"

# Run as non-root user
USER oqs
WORKDIR ${INSTALLDIR}

STOPSIGNAL SIGTERM
CMD ["nginx", "-c", "nginx-conf/nginx.conf", "-g", "daemon off;"]
\end{lstlisting}



\refstepcounter{manualListingCounter}
\label{lst:nginx_holder.conf}
\begin{lstlisting}[language=bash, caption={Listing \arabic{lstlisting}: nginx\_holder.conf}, numbers=left, frame=single]

# OQS Nginx Configuration for VON Network Webserver Reverse Proxy
# Post-Quantum Cryptography enabled with ML-KEM

worker_processes auto;
error_log /opt/nginx/logs/error.log info;
pid /opt/nginx/logs/nginx.pid;

events {
    worker_connections 1024;
}

http {
    include /opt/nginx/conf/mime.types;
    default_type application/octet-stream;

    log_format main '$remote_addr - $remote_user [$time_local] "$request" '
                    '$status $body_bytes_sent "$http_referer" '
                    '"$http_user_agent" "$http_x_forwarded_for"';

    access_log /opt/nginx/logs/access.log main;

    sendfile on;
    keepalive_timeout 65;

    # Upstream: Holder Agent Inbound Transport (Port 8030)
    upstream holder_inbound {
        server holder:8030;
    }

    # Upstream: Holder Agent Admin API (Port 8031)
    upstream holder_admin {
        server holder:8031;
    }

    # HTTPS Server for Holder Inbound Transport (Port 8030)
    server {
        listen 8030 ssl;
        server_name pqc-sidecarproxy-holder;

        # SSL Certificates (custom ML-DSA-65 certificates)
        ssl_certificate /opt/nginx/certs/holder.crt;
        ssl_certificate_key /opt/nginx/certs/holder.key;

        # TLS 1.3 with Post-Quantum Cryptography
        ssl_protocols TLSv1.3;
        ssl_ecdh_curve X25519MLKEM768;
        # Quantum-Safe Key Exchange Groups (ML-KEM from NIST FIPS-203)
        # Groups are set via DEFAULT_GROUPS environment variable
        # Default: mlkem768:x25519:mlkem1024
        # TLS 1.3 cipher suites are automatically selected

        ssl_prefer_server_ciphers off;

        # Reverse Proxy to Holder Inbound Transport
        location / {
            proxy_pass http://holder_inbound;
            proxy_set_header Host $host;
            proxy_set_header X-Real-IP $remote_addr;
            proxy_set_header X-Forwarded-For $proxy_add_x_forwarded_for;
            proxy_set_header X-Forwarded-Proto https;

            # Timeouts
            proxy_connect_timeout 60s;
            proxy_send_timeout 60s;
            proxy_read_timeout 60s;
        }
    }

    # HTTPS Server for Holder Admin API (Port 8031)
    server {
        listen 8031 ssl;
        server_name pqc-sidecarproxy-holder-admin;

        # SSL Certificates (ML-DSA-65)
        ssl_certificate /opt/nginx/certs/holder.crt;
        ssl_certificate_key /opt/nginx/certs/holder.key;

        # TLS 1.3 with Post-Quantum Cryptography
        ssl_protocols TLSv1.3;
        ssl_ecdh_curve X25519MLKEM768;
        # Quantum-Safe Key Exchange Groups (ML-KEM from NIST FIPS-203)
        # Groups are set via DEFAULT_GROUPS environment variable
        # Default: X25519MLKEM768:mlkem768:x25519:mlkem1024

        ssl_prefer_server_ciphers off;

        # Reverse Proxy to Holder Admin API
        location / {
            proxy_pass http://holder_admin;
            proxy_set_header Host $host;
            proxy_set_header X-Real-IP $remote_addr;
            proxy_set_header X-Forwarded-For $proxy_add_x_forwarded_for;
            proxy_set_header X-Forwarded-Proto https;

            # Timeouts
            proxy_connect_timeout 60s;
            proxy_send_timeout 60s;
            proxy_read_timeout 60s;
        }

        # Health Check
        location /health {
            access_log off;
            return 200 "Holder Admin PQC Proxy OK\n";
            add_header Content-Type text/plain;
        }
    }
}
\end{lstlisting}

\subsection{docker-compose.yml: DLT-Infrastruktur}

\refstepcounter{manualListingCounter}
\label{lst:docker-compose.yml-DLT-Infrastruktur}
\begin{lstlisting}[language=bash, caption={Listing \arabic{lstlisting}: docker-compose.yml: DLT-Infrastruktur}, numbers=left, frame=single]
version: '3'
services:
  #
  # Client
  #
  client:
    image: von-network-base
    command: ./scripts/start_client.sh
    environment:
      - IP=${IP}
      - IPS=${IPS}
      - DOCKERHOST=${DOCKERHOST}
      - RUST_LOG=${RUST_LOG}
    networks:
      - von
    volumes:
      - client-data:/home/indy/.indy_client
      - ./tmp:/tmp

  #
  # Webserver
  #
  webserver:
    image: von-network-base
    command: bash -c 'sleep 10 && ./scripts/start_webserver.sh'
    container_name: von-webserver
    environment:
      - IP=${IP}
      - IPS=${IPS}
      - DOCKERHOST=${DOCKERHOST}
      - LOG_LEVEL=${LOG_LEVEL}
      - RUST_LOG=${RUST_LOG}
      - GENESIS_URL=${GENESIS_URL}
      - LEDGER_SEED=${LEDGER_SEED}
      - LEDGER_CACHE_PATH=${LEDGER_CACHE_PATH}
      - MAX_FETCH=${MAX_FETCH:-50000}
      - RESYNC_TIME=${RESYNC_TIME:-120}
      - POOL_CONNECTION_ATTEMPTS=${POOL_CONNECTION_ATTEMPTS:-5}
      - POOL_CONNECTION_DELAY=${POOL_CONNECTION_DELAY:-10}
      - REGISTER_NEW_DIDS=${REGISTER_NEW_DIDS:-True}
      - ENABLE_LEDGER_CACHE=${ENABLE_LEDGER_CACHE:-True}
      - ENABLE_BROWSER_ROUTES=${ENABLE_BROWSER_ROUTES:-True}
      - DISPLAY_LEDGER_STATE=${DISPLAY_LEDGER_STATE:-True}
      - LEDGER_INSTANCE_NAME=${LEDGER_INSTANCE_NAME:-localhost}
      - LEDGER_DESCRIPTION=${LEDGER_DESCRIPTION}
      - WEB_ANALYTICS_SCRIPT=${WEB_ANALYTICS_SCRIPT}
      - INFO_SITE_TEXT=${INFO_SITE_TEXT}
      - INFO_SITE_URL=${INFO_SITE_URL}
      - INDY_SCAN_URL=${INDY_SCAN_URL}
      - INDY_SCAN_TEXT=${INDY_SCAN_TEXT}
    networks:
      - von
    # ports:
    #   - ${WEB_SERVER_HOST_PORT:-9000}:8000
    volumes:
      - ./config:/home/indy/config
      - ./server:/home/indy/server
      - webserver-cli:/home/indy/.indy-cli
      - webserver-ledger:/home/indy/ledger

  #
  # Synchronization test
  #
  synctest:
    image: von-network-base
    command: ./scripts/start_synctest.sh
    environment:
      - IP=${IP}
      - IPS=${IPS}
      - DOCKERHOST=${DOCKERHOST}
      - LOG_LEVEL=${LOG_LEVEL}
      - RUST_LOG=${RUST_LOG}
    networks:
      - von
    ports:
      - ${WEB_SERVER_HOST_PORT:-9000}:8000
    volumes:
      - ./config:/home/indy/config
      - ./server:/home/indy/server
      - webserver-cli:/home/indy/.indy-cli
      - webserver-ledger:/home/indy/ledger

  #
  # Nodes
  #
  nodes:
    image: von-network-base
    command: ./scripts/start_nodes.sh
    networks:
      - von
    ports:
      - 9701:9701
      - 9702:9702
      - 9703:9703
      - 9704:9704
      - 9705:9705
      - 9706:9706
      - 9707:9707
      - 9708:9708
    environment:
      - IP=${IP}
      - IPS=${IPS}
      - DOCKERHOST=${DOCKERHOST}
      - LOG_LEVEL=${LOG_LEVEL}
      - RUST_LOG=${RUST_LOG}
    volumes:
      - nodes-data:/home/indy/ledger

  node1:
    image: von-network-base
    command: ./scripts/start_node.sh 1
    networks:
      - von
    ports:
      - 9701:9701
      - 9702:9702
    container_name: von-node1
    environment:
      - IP=${IP}
      - IPS=${IPS}
      - DOCKERHOST=${DOCKERHOST}
      - LOG_LEVEL=${LOG_LEVEL}
      - RUST_LOG=${RUST_LOG}
    volumes:
      - node1-data:/home/indy/ledger

  node2:
    image: von-network-base
    command: ./scripts/start_node.sh 2
    networks:
      - von
    ports:
      - 9703:9703
      - 9704:9704
    container_name: von-node2
    environment:
      - IP=${IP}
      - IPS=${IPS}
      - DOCKERHOST=${DOCKERHOST}
      - LOG_LEVEL=${LOG_LEVEL}
      - RUST_LOG=${RUST_LOG}
    volumes:
      - node2-data:/home/indy/ledger

  node3:
    image: von-network-base
    command: ./scripts/start_node.sh 3
    networks:
      - von
    ports:
      - 9705:9705
      - 9706:9706
    container_name: von-node3
    environment:
      - IP=${IP}
      - IPS=${IPS}
      - DOCKERHOST=${DOCKERHOST}
      - LOG_LEVEL=${LOG_LEVEL}
      - RUST_LOG=${RUST_LOG}
    volumes:
      - node3-data:/home/indy/ledger

  node4:
    image: von-network-base
    command: ./scripts/start_node.sh 4
    networks:
      - von
    ports:
      - 9707:9707
      - 9708:9708
    container_name: von-node4
    environment:
      - IP=${IP}
      - IPS=${IPS}
      - DOCKERHOST=${DOCKERHOST}
      - LOG_LEVEL=${LOG_LEVEL}
      - RUST_LOG=${RUST_LOG}
    volumes:
      - node4-data:/home/indy/ledger

  # Post-Quantum Nginx Reverse Proxy für VON Network Webserver
  pqc-sidecarproxy-webserver:
    build:
      context: ./pqc_sidecarproxy_nginx
      dockerfile: Dockerfile
      args:
        OPENSSL_TAG: openssl-3.5.4
        LIBOQS_TAG: 0.13.0
        OQSPROVIDER_TAG: 0.9.0
        NGINX_VERSION: 1.28.0
        SIG_ALG: mldsa65
        DEFAULT_GROUPS: X25519MLKEM768:mlkem768:x25519:mlkem1024
    container_name: von-pqc-sidecarproxy-webserver
    environment:
      # OpenSSL Configuration
      - OPENSSL_CONF=/opt/openssl/.openssl/ssl/openssl.cnf
      # Post-Quantum Key Exchange Groups
      - DEFAULT_GROUPS=X25519MLKEM768:mlkem768:x25519:mlkem1024
    networks:
      - von
      - sidecarproxy
    ports:
      - 8000:8000  # HTTPS with Post-Quantum Cryptography (ML-KEM-768)
    volumes:
      # Custom nginx configuration for reverse proxy
      - ./pqc_sidecarproxy_nginx/nginx-conf/nginx_webserver.conf:/opt/nginx/nginx-conf/nginx.conf:ro
      # Custom ML-DSA-65 certificates
      - ./pqc_sidecarproxy_nginx/certs:/opt/nginx/certs:ro
      # Logs
      - nginx-logs:/opt/nginx/logs
    depends_on:
      - webserver
    restart: unless-stopped
    healthcheck:
      test: ["CMD", "curl", "-k", "-f", "https://localhost:8000/health"]
      interval: 30s
      timeout: 10s
      retries: 5
      start_period: 10s

networks:
  von:
  sidecarproxy:

volumes:
  client-data:
  webserver-cli:
  webserver-ledger:
  node1-data:
  node2-data:
  node3-data:
  node4-data:
  nodes-data:
  nginx-logs:
\end{lstlisting}

\subsection{Dockerfile: acapy-base}

\refstepcounter{manualListingCounter}
\label{lst:Dockerfile-acapy-base}
\begin{lstlisting}[language=bash, caption={Listing \arabic{lstlisting}: Dockerfile - acapy-base: Revocation Registry}, numbers=left, frame=single]
ARG python_version=3.12
FROM python:${python_version}-slim-bookworm AS build

RUN pip install --no-cache-dir poetry==2.1.1

WORKDIR /src

COPY ./pyproject.toml ./poetry.lock ./
RUN poetry install --no-root

COPY ./acapy_agent ./acapy_agent
COPY ./README.md /src
RUN poetry build

FROM python:${python_version}-slim-bookworm AS main

ARG uid=1001
ARG user=aries
ARG acapy_name="acapy-agent"
ARG acapy_version
ARG acapy_reqs=[didcommv2]

ENV HOME="/home/$user" \
    APP_ROOT="/home/$user" \
    LC_ALL=C.UTF-8 \
    LANG=C.UTF-8 \
    PIP_NO_CACHE_DIR=off \
    PYTHONUNBUFFERED=1 \
    PYTHONIOENCODING=UTF-8 \
    RUST_LOG=warn \
    SHELL=/bin/bash \
    SUMMARY="$acapy_name image" \
    DESCRIPTION="$acapy_name provides a base image for running acapy agents in Docker. \
    This image layers the python implementation of $acapy_name $acapy_version. Based on Debian Buster."

LABEL summary="$SUMMARY" \
    description="$DESCRIPTION" \
    io.k8s.description="$DESCRIPTION" \
    io.k8s.display-name="$acapy_name $acapy_version" \
    name=$acapy_name \
    acapy.version="$acapy_version" \
    maintainer=""

# Add aries user
RUN useradd -U -ms /bin/bash -u $uid $user

# Install environment
RUN apt-get update && \
    apt-get install -y --no-install-recommends \
    apt-transport-https \
    ca-certificates \
    curl \
    git \
    libffi-dev \
    libgmp10 \
    libncurses5 \
    libncursesw5 \
    openssl \
    sqlite3 \
    zlib1g && \
    apt-get autopurge -y && \
    apt-get clean -y && \
    rm -rf /var/lib/apt/lists/* /usr/share/doc/*

WORKDIR $HOME

# Add local binaries and aliases to path
ENV PATH="$HOME/.local/bin:$PATH"

# - In order to drop the root user, we have to make some directories writable
#   to the root group as OpenShift default security model is to run the container
#   under random UID.
RUN usermod -a -G 0 $user

# Create standard directories to allow volume mounting and set permissions
# Note: PIP_NO_CACHE_DIR environment variable should be cleared to allow caching
RUN mkdir -p \
    $HOME/.acapy_agent \
    $HOME/.cache/pip/http \
    $HOME/.indy_client \
    $HOME/ledger/sandbox/data \
    $HOME/log

# The root group needs access the directories under $HOME/.indy_client and $HOME/.acapy_agent for the container to function in OpenShift.
RUN chown -R $user:root $HOME/.indy_client $HOME/.acapy_agent && \
    chmod -R ug+rw $HOME/log $HOME/ledger $HOME/.acapy_agent $HOME/.cache $HOME/.indy_client

# Create /home/indy and symlink .indy_client folder for backwards compatibility with artifacts created on older indy-based images.
RUN mkdir -p /home/indy
RUN ln -s /home/aries/.indy_client /home/indy/.indy_client

# Install ACA-py from the wheel as $user,
# and ensure the permissions on the python 'site-packages' and $HOME/.local folders are set correctly.
USER $user
COPY --from=build /src/dist/acapy_agent*.whl .
RUN acapy_agent_package=$(find ./ -name "acapy_agent*.whl" | head -n 1) && \
    echo "Installing ${acapy_agent_package} ..." && \
    pip install --no-cache-dir --find-links=. ${acapy_agent_package}${acapy_reqs} && \
    rm acapy_agent*.whl && \
    chmod +rx $(python -m site --user-site) $HOME/.local

ENTRYPOINT ["aca-py"]
\end{lstlisting}

\subsection{docker-compose.yml: Revocation Registry}

\refstepcounter{manualListingCounter}
\label{lst:docker-compose.yml-Revocation-Registry}
\begin{lstlisting}[language=bash, caption={Listing \arabic{lstlisting}: docker-compose.yml: Revocation Registry}, numbers=left, frame=single]
services:
  ngrok-tails-server:
    image: ngrok/ngrok
    networks:
      - tails-server
    ports:
      - 4044:4040
    command: start --all
    environment:
      - NGROK_CONFIG=/etc/ngrok.yml
      - NGROK_AUTHTOKEN=${NGROK_AUTHTOKEN}
    volumes:
      - ./ngrok.yml:/etc/ngrok.yml
  tails-server:
    build:
      context: ..
      dockerfile: docker/Dockerfile.tails-server
    networks:
      - tails-server
    command: >
      tails-server
        --host 0.0.0.0
        --port 6543
        --storage-path $STORAGE_PATH
        --log-level $LOG_LEVEL
        --log-config $LOGGING_CONFIG
  tester:
    build:
      context: ..
      dockerfile: docker/Dockerfile.test

  # Post-Quantum Nginx Reverse Proxy für Tails Server
  pqc-sidecarproxy-tails-server:
    build:
      context: ./pqc_sidecarproxy_nginx
      dockerfile: Dockerfile
      args:
        OPENSSL_TAG: openssl-3.5.4
        LIBOQS_TAG: 0.13.0
        OQSPROVIDER_TAG: 0.9.0
        NGINX_VERSION: 1.28.0
        SIG_ALG: mldsa65
        DEFAULT_GROUPS: X25519MLKEM768:mlkem768:x25519:mlkem1024
    container_name: pqc-sidecarproxy-tails-server
    environment:
      # OpenSSL Configuration
      - OPENSSL_CONF=/opt/openssl/.openssl/ssl/openssl.cnf
      # Post-Quantum Key Exchange Groups
      - DEFAULT_GROUPS=X25519MLKEM768:mlkem768:x25519:mlkem1024
    networks:
      - tails-server
      - von_sidecarproxy
    ports:
      - 6543:6543  # HTTPS with Post-Quantum Cryptography (ML-KEM-768)
    volumes:
      # Custom nginx configuration for reverse proxy
      - ./pqc_sidecarproxy_nginx/nginx-conf/nginx_tails-server.conf:/opt/nginx/nginx-conf/nginx.conf:ro
      # Custom ML-DSA-65 certificates
      - ./pqc_sidecarproxy_nginx/certs:/opt/nginx/certs:ro
      # Logs
      - nginx-logs:/opt/nginx/logs
    depends_on:
      - tails-server
    restart: unless-stopped
    healthcheck:
      test: ["CMD", "curl", "-k", "-f", "https://localhost:6543/health"]
      interval: 30s
      timeout: 10s
      retries: 5
      start_period: 10s

networks:
  tails-server:
  von_sidecarproxy:
    external: true

volumes:
  nginx-logs:
\end{lstlisting}

\subsection{docker-compose.yml: SSI-Agenten}

\refstepcounter{manualListingCounter}
\label{lst:docker-compose.yml-SSI-Agenten}
\begin{lstlisting}[language=bash, caption={Listing \arabic{lstlisting}: docker-compose.yml: SSI-Agenten}, numbers=left, frame=single]

version: '3.8'

services:
  issuer:
    build:
      context: ..
      dockerfile: hopE/Dockerfile.acapy-base
    image: acapy-base
    container_name: issuer-agent
    environment:
      - DOCKERHOST=${DOCKERHOST:-host.docker.internal}
      - GENESIS_URL=${GENESIS_URL:-https://host.docker.internal:8000/genesis}
      - LEDGER_URL=${LEDGER_URL:-https://host.docker.internal:8000}
      - PUBLIC_TAILS_URL=https://host.docker.internal:6543
      - TAILS_FILE_COUNT=100
    networks:
      - hope-issuer
    volumes:
      - issuer-data:/home/aries/.acapy_agent
    extra_hosts:
      - "host.docker.internal:host-gateway"
    command: >
      start
      --label "Issuer Agent"
      --inbound-transport http 0.0.0.0 8020
      --outbound-transport http
      --endpoint https://host.docker.internal:8020
      --admin 0.0.0.0 8021
      --admin-insecure-mode
      --auto-provision
      --wallet-type askar
      --wallet-name issuer_wallet
      --wallet-key issuer_wallet_key_000000000000
      --genesis-url https://host.docker.internal:8000/genesis
      --log-level info
      --auto-accept-invites
      --auto-accept-requests
      --auto-ping-connection
      --auto-respond-credential-proposal
      --auto-respond-credential-offer
      --auto-respond-credential-request
      --auto-verify-presentation
      --public-invites
      --preserve-exchange-records
      --tails-server-base-url https://host.docker.internal:6543
      --notify-revocation
    healthcheck:
      test: ["CMD", "curl", "-f", "http://localhost:8021/status/ready"]
      interval: 30s
      timeout: 10s
      retries: 5
      start_period: 30s

  holder:
    build:
      context: ..
      dockerfile: hopE/Dockerfile.acapy-base
    image: acapy-base
    container_name: holder-agent
    environment:
      - DOCKERHOST=${DOCKERHOST:-host.docker.internal}
      - GENESIS_URL=${GENESIS_URL:-https://host.docker.internal:8000/genesis}
      - LEDGER_URL=${LEDGER_URL:-https://host.docker.internal:8000}
      - PUBLIC_TAILS_URL=https://host.docker.internal:6543
    networks:
      - hope-holder
    volumes:
      - holder-data:/home/aries/.acapy_agent
    extra_hosts:
      - "host.docker.internal:host-gateway"
    command: >
      start
      --label "Holder Agent"
      --inbound-transport http 0.0.0.0 8030
      --outbound-transport http
      --endpoint https://host.docker.internal:8030
      --admin 0.0.0.0 8031
      --admin-insecure-mode
      --auto-provision
      --wallet-type askar
      --wallet-name holder_wallet
      --wallet-key holder_wallet_key_000000000000
      --genesis-url https://host.docker.internal:8000/genesis
      --log-level info
      --auto-accept-invites
      --auto-accept-requests
      --auto-ping-connection
      --auto-respond-credential-offer
      --auto-store-credential
      --public-invites
      --tails-server-base-url https://host.docker.internal:6543
      --preserve-exchange-records
    healthcheck:
      test: ["CMD", "curl", "-f", "http://localhost:8031/status/ready"]
      interval: 30s
      timeout: 10s
      retries: 5
      start_period: 30s

  verifier:
    build:
      context: ..
      dockerfile: hopE/Dockerfile.acapy-base
    image: acapy-base
    container_name: verifier-agent
    environment:
      - DOCKERHOST=${DOCKERHOST:-host.docker.internal}
      - GENESIS_URL=${GENESIS_URL:-https://host.docker.internal:8000/genesis}
      - LEDGER_URL=${LEDGER_URL:-https://host.docker.internal:8000}
      - PUBLIC_TAILS_URL=https://host.docker.internal:6543
    networks:
      - hope-verifier
    volumes:
      - verifier-data:/home/aries/.acapy_agent
    extra_hosts:
      - "host.docker.internal:host-gateway"
    command: >
      start
      --label "Verifier Agent"
      --inbound-transport http 0.0.0.0 8040
      --outbound-transport http
      --endpoint https://host.docker.internal:8040
      --admin 0.0.0.0 8041
      --admin-insecure-mode
      --auto-provision
      --wallet-type askar
      --wallet-name verifier_wallet
      --wallet-key verifier_wallet_key_00000000000
      --genesis-url https://host.docker.internal:8000/genesis
      --log-level info
      --auto-accept-invites
      --auto-accept-requests
      --auto-ping-connection
      --auto-verify-presentation
      --public-invites
      --tails-server-base-url https://host.docker.internal:6543
      --preserve-exchange-records
    healthcheck:
      test: ["CMD", "curl", "-f", "http://localhost:8041/status/ready"]
      interval: 30s
      timeout: 10s
      retries: 5
      start_period: 30s

  pqc-sidecarproxy-issuer:
    build:
      context: ./pqc_sidecarproxy_nginx
      dockerfile: Dockerfile
      args:
        OPENSSL_TAG: openssl-3.5.4
        LIBOQS_TAG: 0.13.0
        OQSPROVIDER_TAG: 0.9.0
        NGINX_VERSION: 1.28.0
        SIG_ALG: mldsa65
        DEFAULT_GROUPS: X25519MLKEM768:mlkem768:x25519:mlkem1024
    container_name: pqc-sidecarproxy-issuer
    environment:
      # OpenSSL Configuration
      - OPENSSL_CONF=/opt/openssl/.openssl/ssl/openssl.cnf
      # Post-Quantum Key Exchange Groups
      - DEFAULT_GROUPS=X25519MLKEM768:mlkem768:x25519:mlkem1024
    networks:
      - von_sidecarproxy
      - hope-issuer
    ports:
      - "8020:8020"  # Issuer Inbound Transport HTTPS (ML-KEM-768)
      - "8021:8021"  # Issuer Admin API HTTPS (ML-KEM-768)
    volumes:
      # Custom nginx configuration for reverse proxy
      - ./pqc_sidecarproxy_nginx/nginx-conf/nginx_issuer.conf:/opt/nginx/nginx-conf/nginx.conf:ro
      # Custom ML-DSA-65 certificates
      - ./pqc_sidecarproxy_nginx/certs:/opt/nginx/certs:ro
      # Logs
      - nginx-logs:/opt/nginx/logs
    depends_on:
      - issuer
      - holder
      - verifier
    restart: unless-stopped
    healthcheck:
      test: ["CMD", "curl", "-k", "-f", "https://localhost:8021/health"]
      interval: 30s
      timeout: 10s
      retries: 5
      start_period: 10s

  pqc-sidecarproxy-holder:
    build:
      context: ./pqc_sidecarproxy_nginx
      dockerfile: Dockerfile
      args:
        OPENSSL_TAG: openssl-3.5.4
        LIBOQS_TAG: 0.13.0
        OQSPROVIDER_TAG: 0.9.0
        NGINX_VERSION: 1.28.0
        SIG_ALG: mldsa65
        DEFAULT_GROUPS: X25519MLKEM768:mlkem768:x25519:mlkem1024
    container_name: pqc-sidecarproxy-holder
    environment:
      # OpenSSL Configuration
      - OPENSSL_CONF=/opt/openssl/.openssl/ssl/openssl.cnf
      # Post-Quantum Key Exchange Groups
      - DEFAULT_GROUPS=X25519MLKEM768:mlkem768:x25519:mlkem1024
    networks:
      - von_sidecarproxy
      - hope-holder
    ports:
      - "8030:8030"  # Holder Inbound Transport HTTPS (ML-KEM-768)
      - "8031:8031"  # Holder Admin API HTTPS (ML-KEM-768)
    volumes:
      # Custom nginx configuration for reverse proxy
      - ./pqc_sidecarproxy_nginx/nginx-conf/nginx_holder.conf:/opt/nginx/nginx-conf/nginx.conf:ro
      # Custom ML-DSA-65 certificates
      - ./pqc_sidecarproxy_nginx/certs:/opt/nginx/certs:ro
      # Logs
      - nginx-logs:/opt/nginx/logs
    depends_on:
      - issuer
      - holder
      - verifier
    restart: unless-stopped
    healthcheck:
      test: ["CMD", "curl", "-k", "-f", "https://localhost:8031/health"]
      interval: 30s
      timeout: 10s
      retries: 5
      start_period: 10s

  pqc-sidecarproxy-verifier:
    build:
      context: ./pqc_sidecarproxy_nginx
      dockerfile: Dockerfile
      args:
        OPENSSL_TAG: openssl-3.5.4
        LIBOQS_TAG: 0.13.0
        OQSPROVIDER_TAG: 0.9.0
        NGINX_VERSION: 1.28.0
        SIG_ALG: mldsa65
        DEFAULT_GROUPS: X25519MLKEM768:mlkem768:x25519:mlkem1024
    container_name: pqc-sidecarproxy-verifier
    environment:
      # OpenSSL Configuration
      - OPENSSL_CONF=/opt/openssl/.openssl/ssl/openssl.cnf
      # Post-Quantum Key Exchange Groups
      - DEFAULT_GROUPS=X25519MLKEM768:mlkem768:x25519:mlkem1024
    networks:
      - von_sidecarproxy
      - hope-verifier
    ports:
      - "8040:8040"  # Verifier Inbound Transport HTTPS (ML-KEM-768)
      - "8041:8041"  # Verifier Admin API HTTPS (ML-KEM-768)
    volumes:
      # Custom nginx configuration for reverse proxy
      - ./pqc_sidecarproxy_nginx/nginx-conf/nginx_verifier.conf:/opt/nginx/nginx-conf/nginx.conf:ro
      # Custom ML-DSA-65 certificates
      - ./pqc_sidecarproxy_nginx/certs:/opt/nginx/certs:ro
      # Logs
      - nginx-logs:/opt/nginx/logs
    depends_on:
      - issuer
      - holder
      - verifier
    restart: unless-stopped
    healthcheck:
      test: ["CMD", "curl", "-k", "-f", "https://localhost:8041/health"]
      interval: 30s
      timeout: 10s
      retries: 5
      start_period: 10s

networks:
  hope-issuer:
  hope-holder:
  hope-verifier:
  von_sidecarproxy:
    external: true

volumes:
  issuer-data:
  holder-data:
  verifier-data:
  nginx-logs:
\end{lstlisting}

\subsection{Eigenkompilation eines Chromium-Browsers mit PQC-Unterstützung}
\label{Eigenkompilation eines Chromium-Browsers mit PQC-Unterstützung}

Zur experimentellen Evaluation von Verfahren der Post-Quanten-Kryptographie (PQC) im Kontext realer Webbrowser wurde ein Chromium-basierter Browser mit erweiterten TLS-Fähigkeiten selbst kompiliert. Grundlage bildete das von Open Quantum Safe (OQS) bereitgestellte Chromium-Demoprojekt, das eine Integration der \emph{liboqs}-Bibliothek in die TLS-Implementierung von Chromium (BoringSSL) demonstriert und hybride sowie rein PQ-basierte Schlüsselaustausch- und Signaturverfahren bereitstellt \parencite{open-quantum-safe_OqsdemosChromium643ef99297fe8c6ebd3587b5dd238d5e7a457037openquantumsafeoqsdemos_,open-quantum-safe_OqsdemosChromiumREADMELinuxmd643ef99297fe8c6ebd3587b5dd238d5e7a457037openquantumsafeoqsdemos_}. Ziel war es, eine lauffähige Build-Umgebung unter Linux aufzusetzen, den OQS-angepassten Quellcode zu beziehen, die notwendigen Abhängigkeiten zu installieren und anschließend ein reproduzierbares Build-Artefakt des Browsers mit PQC-Unterstützung zu erzeugen.

Die Einrichtung der Build-Umgebung erfolgte weitgehend entsprechend der offiziellen Linux-Build-Dokumentation des Chromium-Projekts \parencite{_ChromiumDocsCheckingoutbuildingChromiumLinux_}. Hierzu wurden zunächst die von Chromium bereitgestellten \texttt{depot\_tools} geklont und in den \texttt{PATH} eingebunden, um Werkzeuge wie \texttt{fetch}, \texttt{gclient} und \texttt{gn} verwenden zu können. Anschließend wurde ein separates Arbeitsverzeichnis angelegt und über \texttt{fetch} ein Chromium-Checkout inklusive aller benötigten Abhängigkeiten durchgeführt. Unter einer aktuellen Ubuntu-Linux-Distribution wurden im nächsten Schritt die von Chromium empfohlenen Systemabhängigkeiten installiert, etwa über das Skript \texttt{build/install-build-deps.sh}, welches Compiler, Entwicklungsbibliotheken und Laufzeitbibliotheken für das spätere Linken der Browser-Binärdateien bereitstellt \parencite{_ChromiumDocsCheckingoutbuildingChromiumLinux_}. Nach Abschluss dieser Vorbereitungen wurden mittels \texttt{gclient runhooks} die Chromium-spezifischen Hooks ausgeführt, um zusätzliche Werkzeuge und vorkompilierte Komponenten nachzuladen.

Aufbauend auf dieser Standard-Umgebung wurde der von OQS bereitgestellte Chromium-Zweig eingebunden, der Anpassungen an BoringSSL sowie die Einbindung von \texttt{liboqs} enthält \parencite{open-quantum-safe_OqsdemosChromium643ef99297fe8c6ebd3587b5dd238d5e7a457037openquantumsafeoqsdemos_,open-quantum-safe_OqsdemosChromiumREADMELinuxmd643ef99297fe8c6ebd3587b5dd238d5e7a457037openquantumsafeoqsdemos_}. Dazu wurde das entsprechende Repository aus dem OQS-Demoprojekt geklont und gemäß der dort beschriebenen Struktur so in die bestehende Chromium-Arbeitsumgebung integriert, dass die PQC-Erweiterungen anstelle der unveränderten Upstream-Kryptographiebibliothek verwendet werden. Zentral war dabei die Übernahme der in der OQS-Dokumentation beschriebenen Build-Konfigurationen, insbesondere GN-Argumente, die das Linken gegen \texttt{liboqs} aktivieren und die experimentellen PQ- bzw. Hybrid-Ciphersuites in der TLS-Konfiguration von Chromium einschalten \parencite{open-quantum-safe_OqsdemosChromiumREADMELinuxmd643ef99297fe8c6ebd3587b5dd238d5e7a457037openquantumsafeoqsdemos_}. Diese Konfiguration wurde in einer eigenen Build-Directory, etwa \texttt{out/oqs-Default}, über den Aufruf \texttt{gn gen} mit den projektspezifischen Argumenten erzeugt.

Im Anschluss daran erfolgte der eigentliche Kompiliervorgang des Browsers mit dem von Chromium vorgesehenen Build-Werkzeug \texttt{autoninja}, das die GN-Konfiguration nutzt, um alle notwendigen Targets effizient zu bauen \parencite{_ChromiumDocsCheckingoutbuildingChromiumLinux_}. Durch den Aufruf von \texttt{autoninja -C out/oqs-Default chrome} wurde eine Browser-Binärdatei erzeugt, die die OQS-Erweiterungen in der TLS-Schicht enthält. Der resultierende Browser konnte direkt aus dem Build-Verzeichnis gestartet und gegen PQC-fähige Testserver genutzt werden, um TLS-Verbindungen mit hybriden oder rein PQ-basierten Schlüsselaustauschmechanismen zu etablieren. Die so aufgebaute Umgebung ermöglicht eine kontrollierte experimentelle Analyse der praktischen Auswirkungen von PQC im Browserkontext, etwa hinsichtlich Kompatibilität, Performance und Protokollhandshake, auf Basis eines realen Chromium-Builds mit explizit aktivierter Post-Quanten-Kryptographie.


\subsection{Issuer Agent Boot Logs}

\refstepcounter{manualListingCounter}
\label{lst:Issuer-Agent-Boot-Logs}
\begin{lstlisting}[language=bash, caption={Listing \arabic{lstlisting}: Issuer Agent Boot Logs}, numbers=left, frame=single]
Executing task in folder ferris: docker logs --tail 1000 -f 179e43b336fa16b399efa7326cdc0b8bfa6ab24c8f86a2b4b630fe57fc382064 

2025-11-28 23:49:38,921 acapy_agent.config.default_context INFO Registering default plugins
2025-11-28 23:49:39,083 acapy_agent.config.default_context INFO Registering askar plugins
2025-11-28 23:49:39,308 acapy_agent.config.ledger INFO Fetching genesis transactions from: https://host.docker.internal:8000/genesis
2025-11-28 23:49:46,340 acapy_agent.core.profile INFO Create profile manager: askar
2025-11-28 23:49:46,827 acapy_agent.config.wallet INFO Created new profile - Profile name: issuer_wallet, backend: askar
2025-11-28 23:49:46,829 acapy_agent.config.wallet INFO No public DID created
2025-11-28 23:49:46,885 acapy_agent.config.ledger INFO Ledger configuration complete
2025-11-28 23:49:46,885 acapy_agent.core.conductor INFO Ledger configured successfully.
2025-11-28 23:49:46,893 acapy_agent.core.conductor INFO Wallet type record not found.
2025-11-28 23:49:46,894 acapy_agent.core.conductor INFO New agent. Setting wallet type to askar.
2025-11-28 23:49:47,028 acapy_agent.config.banner INFO 
::::::::::::::::::::::::::::::::::::::::::::::
::               Issuer Agent               ::
::                                          ::
::                                          ::
:: Inbound Transports:                      ::
::                                          ::
::   - http://0.0.0.0:8020                  ::
::                                          ::
:: Outbound Transports:                     ::
::                                          ::
::   - http                                 ::
::   - https                                ::
::                                          ::
:: Administration API:                      ::
::                                          ::
::   - http://0.0.0.0:8021                  ::
::                                          ::
::                               ver: 1.3.2 ::
::::::::::::::::::::::::::::::::::::::::::::::

2025-11-28 23:49:47,028 acapy_agent.config.banner INFO 
::::::::::::::::::::::::::::::::::::::::::::::::::::::::::::::::::::::::::::::::::::::
::                             DEPRECATION NOTICE:                                  ::
:: -------------------------------------------------------------------------------- ::
:: Receiving a core DIDComm protocol with the `did:sov:BzCbsNYhMrjHiqZDTUASHg;spec` ::
:: prefix is deprecated. All parties sending this prefix should be notified that    ::
:: support for receiving such messages will be removed in a future release. Use     ::
:: https://didcomm.org/ instead.                                                    ::
:: -------------------------------------------------------------------------------- ::
:: Aries RFC 0160: Connection Protocol is deprecated and support will be removed in ::
:: a future release; use RFC 0023: DID Exchange instead.                            ::
:: -------------------------------------------------------------------------------- ::
:: Aries RFC 0036: Issue Credential 1.0 is deprecated and support will be removed   ::
:: in a future release; use RFC 0453: Issue Credential 2.0 instead.                 ::
:: -------------------------------------------------------------------------------- ::
:: Aries RFC 0037: Present Proof 1.0 is deprecated and support will be removed in a ::
:: future release; use RFC 0454: Present Proof 2.0 instead.                         ::
::::::::::::::::::::::::::::::::::::::::::::::::::::::::::::::::::::::::::::::::::::::

2025-11-28 23:49:47,029 acapy_agent.core.conductor INFO Wallet version storage record not found.
2025-11-28 23:49:47,030 acapy_agent.core.conductor INFO No upgrade from version was found from wallet or via --from-version startup argument. Defaulting to v0.7.5.
2025-11-28 23:49:47,031 acapy_agent.core.conductor INFO Upgrade configurations available. Initiating upgrade.
2025-11-28 23:49:47,033 acapy_agent.commands.upgrade INFO No ACA-Py version found in wallet storage.
2025-11-28 23:49:47,033 acapy_agent.commands.upgrade INFO Selecting v0.7.5 as --from-version from the config.
2025-11-28 23:49:47,033 acapy_agent.commands.upgrade INFO Running upgrade process for v0.8.1
2025-11-28 23:49:47,034 acapy_agent.commands.upgrade INFO No records of <class 'acapy_agent.connections.models.conn_record.ConnRecord'> found
2025-11-28 23:49:47,040 acapy_agent.commands.upgrade INFO acapy_version storage record set to v1.3.2
2025-11-28 23:49:47,042 acapy_agent.core.conductor INFO Listening...
2025-11-28 23:49:52,008 aiohttp.access INFO 127.0.0.1 [28/Nov/2025:23:49:52 +0000] "GET /status/ready HTTP/1.1" 200 138 "-" "curl/7.88.1"
2025-11-28 23:50:22,041 aiohttp.access INFO 127.0.0.1 [28/Nov/2025:23:50:22 +0000] "GET /status/ready HTTP/1.1" 200 138 "-" "curl/7.88.1"
\end{lstlisting}


\newpage
\section{Artefaktentwicklung Iteration 2}
\label{sec:Anhang_Artefaktentwicklung Iteration 2}

\end{appendices}
\addtocontents{toc}{\protect\setcounter{tocdepth}{2}}



%-----------------------------------
% Literaturverzeichnis
%-----------------------------------
\newpage

% Die folgende Zeile trägt ALLE Werke aus literatur.bib in das
% Literaturverzeichnis ein, egal ob sie zietiert wurden oder nicht.
% Der Befehl ist also nur zum Test der Skripte sinnvoll und muss bei echten
% Arbeiten entfernt werden.
%\nocite{*}

%\addcontentsline{toc}{section}{Literatur}

% Die folgenden beiden Befehle würden ab dem Literaturverzeichnis wieder eine
% römische Seitennummerierung nutzen.
% Das ist nach dem Leitfaden nicht zu tun. Dort steht nur dass 'sämtliche
% Verzeichnisse VOR dem Textteil' römisch zu nummerieren sind. (vgl. S. 3)
%\pagenumbering{Roman} %Zähler wieder römisch ausgeben
%\setcounter{page}{4}  %Zähler manuell hochsetzen

% Ausgabe des Literaturverzeichnisses

% Keine Trennung der Werke im Literaturverzeichnis nach ihrer Art
% (Online/nicht-Online)
%\begin{RaggedRight}
%\printbibliography
%\end{RaggedRight}

% Alternative Darstellung, die laut Leitfaden genutzt werden sollte.
% Dazu die Zeilen auskommentieren und folgenden code verwenden:

% Literaturverzeichnis getrennt nach Nicht-Online-Werken und Online-Werken
% (Internetquellen).
% Die Option nottype=online nimmt alles, was kein Online-Werk ist.
% Die Option heading=bibintoc sorgt dafür, dass das Literaturverzeichnis im
% Inhaltsverzeichnis steht.
% Es ist übrigens auch möglich mehrere type- bzw. nottype-Optionen anzugeben, um
% noch weitere Arten von Zusammenfassungen eines Literaturverzeichnisse zu
% erzeugen.
% Beispiel: [type=book,type=article]
\printbibliography[nottype=online,heading=bibintoc,title={\langde{Literaturverzeichnis}\langen{Bibliography}}]

% neue Seite für Internetquellen-Verzeichnis
\newpage

% Laut Leitfaden 2018, S. 14, Fussnote 44 stehen die Internetquellen NICHT im
% Inhaltsverzeichnis, sondern gehören zum Literaturverzeichnis.
% Die Option heading=bibintoc würde die Internetquelle als eigenen Eintrag im
% Inhaltsverzeicnis anzeigen.
%\printbibliography[type=online,heading=bibintoc,title={\headingNameInternetSources}]
\printbibliography[type=online,heading=subbibliography,title={\headingNameInternetSources}]



%-----------------------------------
% KI-Hilfsmittelverzeichnis
%-----------------------------------
\newpage
\section*{KI-Hilfsmittelverzeichnis}
\addcontentsline{toc}{section}{KI-Hilfsmittelverzeichnis}

\renewcommand{\thetable}{A-\arabic{table}}
\renewcommand{\theHtable}{A-\arabic{table}} % für hyperref

Die in \autoref{tab:ki-hilfsmittelverzeichnis} gelisteten Generative KI-Tools wurden in einem iterativen Prozess als kritische Analyseinstrumente in interaktiver, dialogischer Weise eingesetzt. Ausgehend von selbst verfassten Textentwürfen und literaturgestützten Kernaussagen wurden mehrere Optimierungszyklen durchlaufen, in denen die KI zur Schärfung von Formulierungen, Präzisierung der Argumentationslogik und Verbesserung der sprachlichen Kohärenz beitrug. Die KI fungierte dabei als Qualitätssicherungsinstanz für bereits konzipierte Inhalte, nicht als Inhaltsgenerator. Im technischen Teil unterstützte die KI durch Code-Validierung, der Identifikation logischer Fehler, Effizienzoptimierung und Syntaxkonformitätsprüfung während der iterativen PQC-SSI-Prototyp-Entwicklung.

\begin{longtable}{L{3cm}L{5cm}L{5cm}}
    \caption{KI-Hilfsmittelverzeichnis}
    \label{tab:ki-hilfsmittelverzeichnis} \\
    \toprule
    \textbf{KI-Tool} & \textbf{Beschreibung} & \textbf{Einsatzbereich} \\
    \midrule
    \endfirsthead
    \multicolumn{3}{l}{\textit{Tabelle \thetable\ (Fortsetzung)}} \\
    \toprule
    \textbf{KI-Tool} & \textbf{Beschreibung} & \textbf{Einsatzbereich} \\
    \midrule
    \endhead
    \midrule
    \multicolumn{3}{r}{\textit{Fortsetzung auf nächster Seite}} \\
    \endfoot
    \bottomrule
    \multicolumn{3}{p{\linewidth}}{\textit{Anmerkung.} Eigene Darstellung der eingesetzen KI-Hilfsmittel.} \\
    \endlastfoot
    Claude \newline 4.5 Sonnet & Vielseitiges Sprachmodell von Anthropic mit erweiterten Reasoning- und Computer-Use-Fähigkeiten. & Optimierung und Syntaxkontrolle des selbst verfassten Source-Codes und LaTeX-Korrektur \\
    \midrule
    Perplexity.ai \newline Sonar Reasoning & Reasoning-Modell mit integrierter Web-Suche und Chain-of-Thought-Fähigkeiten. & Recherche, Textvalidierung, Faktencheck \\
    \midrule
    Github Copilot \newline 1.0 & KI-gestützte Programmierhilfe zur Analyse und Optimierung der LaTeX-Syntax & Textvalidierung, Rechtschreib- und Grammatik-Prüfung innerhalb der LaTeX-Dokumente \\
\end{longtable}


\input{kapitel/anhang/erklaerung}
\end{document}